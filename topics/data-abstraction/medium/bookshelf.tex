
    In the following problem, we will represent 
    a bookshelf object using dictionaries.
    \newline
    \newline
    In the first section, we will set up the format. Here, we will directly work with the internals of the
    Bookshelf, so don't worry about abstraction barriers for now. Fill in the following functions based
    on their descriptions (the constructor is given to you):
    
    \begin{lstlisting}
    def Bookshelf(capacity):
        """ Creates an empty bookshelf with a certain max capacity. """
        return {'size': capacity, 'books': {}}

    def add_book(bookshelf, author, title):
        """
        Adds a book to the bookshelf. If the bookshelf is full,
        print "Bookshelf is full!" and do not add the book.
        >>> books = Bookshelf(2)
        >>> add_book(books, 'Jane Austen', 'Pride and Prejudice')
        >>> add_book(books, 'Daniel Kleppner', 'An Introduction to Mechanics 5th Edition')
        >>> add_book(books, 'Kurt Vonnegut', 'Galapagos')
        Bookshelf is full!
        """
        if _______________________________:
            print('Bookshelf is full!')
        else:
            if author in bookshelf['books']:
                ____________________________________
            else:
                ____________________________________
    \end{lstlisting}


    \begin{solution}
        \begin{lstlisting}
        if len(bookshelf['books']) == bookshelf['size']:
            print('Bookshelf is full!')
        else:
            if author in bookshelf['books']:
                bookshelf['books'][author].append(title)
            else:
                bookshelf['books'][author] = [title]
        \end{lstlisting}
    \end{solution}

    \newpage
    \begin{lstlisting}
    def get_all_authors(bookshelf):
        """
        Returns a list of all authors who have at least one book in the bookshelf.
        >>> books = Bookshelf(10)
        >>> add_book(books, 'Jane Austen', 'Pride and Prejudice')
        >>> add_book(books, 'Sheldon Axler', 'Linear Algebra Done Right')
        >>> add_book(books, 'Kurt Vonnegut', 'Galapagos')
        >>> get_all_authors(books)
        ['Jane Austen', 'Sheldon Axler', 'Kurt Vonnegut']
        """
        return ___________________________________
    \end{lstlisting}

    \begin{solution}[1in]
        \begin{lstlisting}
            return list(bookshelf['books'].keys())
        \end{lstlisting}
    \end{solution}

    \begin{lstlisting}
    def get_author_books(bookshelf, author):
        """
        Given an author name, returns a list with
        all books on the bookshelf written by that author.
        >>> books = Bookshelf(100)
        >>> add_book(books, 'Orson Scott Card', "Ender's Game")
        >>> add_book(books, 'Orson Scott Card', 'Speaker for the Dead')
        >>> add_book(books, 'J.R.R. Tolkien', 'The Hobbit')
        >>> get_author_books(books, 'Orson Scott Card')
        ['Ender's Game', 'Speaker for the Dead']
        """
        return _________________________________________________
    \end{lstlisting}

    \begin{solution}[1in]
        \begin{lstlisting}
            return bookshelf['books'][author] 
        \end{lstlisting}
    \end{solution}

    \newpage
    Now, complete the function \lstinline{most_popular_author} \textbf{without breaking the abstraction barrier}.
    In other words, you are not allowed to assume anything about the implementation of a Bookshelf object, or
    use the fact that it is a dictionary. You can only use the methods above and their stated return values.

    \begin{lstlisting}
    def most_popular_author(bookshelf):
        """
        Returns the author with the greatest number of books on this bookshelf.
        You can assume that the bookshelf is not empty.
        >>> books = Bookshelf(100)
        >>> add_book(books, 'Orson Scott Card', 'Xenocide')
        >>> add_book(books, 'Orson Scott Card', 'Children of the Mind')
        >>> add_book(books, 'J.R.R. Tolkien', 'The Hobbit')
        >>> most_popular_author(bookshelf)
        'Orson Scott Card'
        """
        return max(________________________________________________, key=_______________________________________________________)
    \end{lstlisting}


    \begin{solution}[1in]
        \begin{lstlisting}
            return max(get_all_authors(bookshelf), key=lambda x: len(get_author_books(x))
        \end{lstlisting}
    \end{solution}

    \begin{questionmeta}
        This is a hard question! Only do it if your students are absolutely assured in their definition of data abstraction and the abstraction barrier.

        Feel free to spend even more time on this. Lists are more important for students generally, but understanding data abstraction deeply is a great setup for OOP!
    \end{questionmeta}
