\textbf{Data Abstraction Overview:}

Abstraction allows us to create and access data through a controlled, restricted programming interface, hiding implementation details and encouraging programmers to focus on how data is used, rather than how data is organized. The two fundamental components of a programming interface are a constructor and selectors.
\begin{enumerate}
	\item Constructor: The interface that creates a piece of data; e.g. calling \texttt{(define p (pair 1 2))} creates a new pair object and assigns it to the variable \texttt{p}.
	\item Selectors: The interface by which we access attributes of a piece of data; e.g. calling \lstinline{(first p)} should return \lstinline{1}.
\end{enumerate}

Through constructors and selectors, abstract data can hide its implementation, and a programmer doesn’t need to {\it know} its implementation to {\it use} it.