\textbf{Data Abstraction Overview:}

Abstraction allows us to create and access different types of data through a controlled, restricted programming interface, hiding implementation details and encouraging programmers to focus on how data is used, rather than how data is organized. The two fundamental components of a programming interface are a constructor and selectors.
\begin{enumerate}
	\item Constructor: The interface that creates a piece of data; e.g. calling `t = tree(3)` creates a new tree object and assigns it to `t`. `tree()` is a constructor.
	\item Selectors: The interface by which we access attributes of a piece of data; e.g. calling `branches(t)` and `is\_leaf(t)` return different attributes of a tree (a list of branches and whether the tree is a leaf, respectively). `branches()` and `is\_leaf()` are both selectors.
\end{enumerate}

Through constructors and selectors, a data type can hide its implementation, and a programmer doesn’t need to {\it know} its implementation to {\it use} it.