\begin{blocksection}
    \question The following is an abstract data type that represents Pokemon.
    Each Pokemon keeps track of its name, type, and friends. Given
    our provided constructor, fill out the selectors:
    
    \begin{lstlisting}
def pokemon(name, p_type, friends):
    """
    Constructs a Pokemon with the given attributes. 
    >>> cyndaquil = pokemon('Cyndaquil', 'Fire', ['Chikorita', 'Totodile'])
    >>> p_name(cyndaquil)
    'Cyndaquil'
    >>> p_type(cyndaquil)
    'Fire'
    >>> p_friends(cyndaquil)
    ['Chikorita', 'Totodile']
    """
    return [name, p_type, friends]

    \end{lstlisting}
    
    \end{blocksection}
    \begin{blocksection}
    
    \begin{lstlisting}
def p_name(p):
    \end{lstlisting}
    \begin{solution}[0.25in]
    \begin{lstlisting}
    return p[0]
    \end{lstlisting}
    \end{solution}
    \end{blocksection}
    \begin{blocksection}
    
    \begin{lstlisting}
def p_type(p):
    \end{lstlisting}
    \begin{solution}[0.25in]
    \begin{lstlisting}
    return p[1]
    \end{lstlisting}
    \end{solution}
    \end{blocksection}
    \begin{blocksection}
    
    \begin{lstlisting}
def p_friends(p):
    \end{lstlisting}
    \begin{solution}[0.25in]
    \begin{lstlisting}
    return p[2]
    \end{lstlisting}
    \end{solution}
    \end{blocksection}
    
\begin{questionmeta}
This problem is a gentle introduction to ADTs. We tried to keep it as simple as possible while giving those majority of students who
have experience with Pokemon something interesting to play around with. 

Students may be confused on how exactly to figure out what the selectors do. Here the doctests are really useful for helping them 
figure out what they should do. If they're stuck, you can try nudging them in the right direction by asking them to consider what
the relevant functions take as input and give as output. For example, \lstinline{pokemon} takes in a Pokemon's attributes and 
returns a Pokemon ADT instance, represented as a list. \lstinline{p_friends} takes a Pokemon ADT instance (internally represented
as a list) and returns its friends. By recognizing that the argument to \lstinline{p_friends} must be a list of a specified form,
they should be able to come to the correct answer relatively easily. 

It's really important that students understand this part before moving on to the rest of the problems in this section, since they
all build on the Pokemon ADT. 

A cheeky thing about this problem is that the amount of space we give them to complete the selectors could be a clue for the problem.
Oh well. 

Please take the time to go over each of the functions with the doctests so students understand how abstraction works in the context of this problem
in order to build up their knowledge for the later parts of using these selector functions
\end{questionmeta}
    %%% Question %%%
    
    \begin{blocksection}
    \question This function returns the correct result, but there's something wrong
    with its implementation. What's the issue, and how can we fix it?
    
    \begin{lstlisting}
def are_friends(p1, p2):
    """
    Returns True iff the Pokemon p1 and p2 are each other's friends.
    """
    return p1[0] in p2[2] and p2[0] in p1[2]
    \end{lstlisting}
    \begin{solution}[0.5in]
    Treating the \lstinline{p1} and \lstinline{p2} are lists is a Data Abstraction Violation (DAV).
    We should use a selector instead.
    The corrected function looks like:
    \begin{lstlisting}
def are_friends(p1, p2):
    return p_name(p1) in p_friends(p2) and p_name(p2) in p_friends(p1)
    \end{lstlisting}
    \end{solution}
    \end{blocksection}

    \begin{questionmeta}
    The purpose of this problem is to introduce the idea of the abstraction barrier. Instead of teaching the abstraction barrier
    to your students as an unbreakable rule that is to be obeyed without question, I encourage you to discuss \textit{why} abstraction
    barriers exist. For example, which version of \lstinline{are_friends} is more readable? (The revised version, because it uses the interface,
    which has named selectors.) If we decided to change the underlying implementation of the ADT, how would we have to change the different
    versions of \lstinline{are_friends}? (We'd have to update the old version but not the revised version.) It's much more valuable if 
    students understand this reasoning than if they just understand that abstraction barriers shouldn't be broken. 
    
    After going over this problem, some students might be confused about why the previous problem is not a violation of the abstraction 
    barrier. After all, in that problem, we were also dealing with the internal representation of the ADT. The difference is, of course,
    that we need to deal with the internal representation of the ADT in order to make the interface. An analogy I might use is that the
    car factory needs to tinker with the internal functioning of the engine because while constructing the car you do need to deal with 
    those details; however, once the car is on the road, you don't need to mess with the engine anymore. In the same sense, after we make
    the interface for an ADT, it is no longer necessary for us to deal with the internal representation and the abstraction barriers can
    ``go into effect''. 

    If data abstraction is new to your students or they don't feel very confident in the topic, \textbf{consider walking them through this problem}.

    This part may seem easy/trivial, but emphasize how the selector interface allows you to easily use the ADT without violating abstraction barriers.
    \end{questionmeta}

    \begin{blocksection}
    \question Write the function \lstinline{cross_type_friends}, which takes in a Pokemon \lstinline{p} and
    a list of Pokemon \lstinline{pokemon_list} and returns a list of the names of \lstinline{p}'s cross-type
    friends in \lstinline{pokemon_list}. (A cross-type friend is a friend of a different type.)
    You may assume that the \lstinline{are_friends} function has been correctly implemented. 
    \begin{lstlisting}
def cross_type_friends(p, pokemon_list): 
    """
    >>> c = pokemon('Charmander', 'Fire', ['Torchic', 'Squirtle', 'Bulbasaur'])
    >>> t = pokemon('Torchic', 'Fire', ['Charmander', 'Squirtle'])
    >>> s = pokemon('Squirtle', 'Water', ['Torchic', 'Bulbasaur'])
    >>> b = pokemon('Bulbasaur', 'Grass', ['Charmander', 'Squirtle'])
    >>> cross_type_friends(c, [t, s, b])
    ['Bulbasaur']
    >>> cross_type_friends(b, [c, s, b])
    ['Charmander', 'Squirtle']
    """
    \end{lstlisting}

    \begin{solution}[2in]
    \begin{lstlisting}
    friend_list = []
    for other in pokemon_list:
        if are_friends(p, other) and p_type(p) != p_type(other):
            friend_list += [p_name(other)]
    return friend_list

    # Alternative solution

    return [p_name(o) for o in pokemon_list if are_friends(p, o) and p_type(p) != p_type(o)]
    \end{lstlisting}
    \end{solution}

    \end{blocksection}
    \begin{questionmeta}
    This is the only code-writing question in this section where we ask students to utilize an ADT with the full abstraction barrier intact. 
    I believe that this is a relatively important skill for students to have, so I think this problem is not one to skip. 

    There are a large number of alternate solutions to this problem. To give students more of a challenge, I elected to not give them a 
    skeleton for this problem. 

    Some potential leading questions: 
    \begin{itemize}
        \item If I have two Pokemon instances, how can I determine whether they are cross-type friends or not? 
        \item What's a typical way we can count up something over a list? 
        \item Did we define any functions that can help us here? 
    \end{itemize}

    \end{questionmeta}
    \newpage
    \question In this problem, you'll change the implementation of the Pokemon ADT while keeping the interface the same.
    \begin{parts}
    \begin{blocksection}
    \part
    
    Complete the constructor for the given selectors. 
    
    \begin{lstlisting}
def pokemon(name, p_type, friends):
    """
    >>> lil_guy = pokemon('Pikachu', 'Electric', ['Mewtwo', 'Lucario'])
    >>> p_name(lil_guy)
    'Pikachu'
    >>> p_type(lil_guy)
    'Electric'
    >>> p_friends(lil_guy)
    ['Mewtwo', 'Lucario']
    """
    \end{lstlisting}
    
    \begin{solution}[1.8in]
    \begin{lstlisting}
    def select(command):
        if command == 'name':
            return name
        elif command == 'type':
            return p_type
        elif command == 'friends':
            return friends
    return select
    \end{lstlisting}

    Alternate solution: 
    \begin{lstlisting}
    return lambda sel: {'name': name, 'type': p_type, 'friends': friends}[sel]
    \end{lstlisting}
    \end{solution}
    \end{blocksection}

    \begin{blocksection}
    \begin{lstlisting}
def p_name(p):
    return p('name')
    \end{lstlisting}

    \end{blocksection}
    \begin{blocksection}
    
    \begin{lstlisting}
def p_type(p):
    return p('type')
    \end{lstlisting}

    \end{blocksection}
    \begin{blocksection}

    \begin{lstlisting}
def p_friends(p):
    return p('friends')
    \end{lstlisting}
    
    \end{blocksection}
    
    \begin{questionmeta}
        The purpose of this problem is to hammer home the bedrock principles of implementation independence. I like 
        how it kind of ties everything together. You can probably skip it if you don't have enough time, but there's a 
        certain closure to this part that I feel would be missing if it were skipped. 

        This problem is similar to the first Pokemon problem, where students were given a constructor and were
        asked to write selectors. However, it is significantly more challenging because here they are given
        selectors and have to reverse-engineer a constructor. Students will need to be detectives and use
        the clues given to them in the selectors to figure this out. If they're stuck, I'd recommend looking 
        at specific problems and helping them deduce the answer. For example, if we see the line \lstinline{p('friends')},
        what does \lstinline{p} have to be? How can we ensure that \lstinline{p} returns the correct value when
        \lstinline{'friends'} is provided? If we need to store data, does it need to be in a sequence or container,
        or perhaps is there another (sneaky) place it can be stored? 

        If students are confused by the different terminology---``interface'' vs. ``implementation''---make sure to
        clarify this for them. Implementation refers to the ``behind the scenes'' work that makes an ADT work. an
        interface, on the other hand, is a set of potential interactions with a datatype, each defined by
        what inputs we provide and what outputs the interface produces given those inputs. 
    \end{questionmeta}

    \begin{blocksection}
    \part What do we need to change about the implementations of \lstinline{are_friends} (as revised) and 
    \lstinline{cross_type_friends} now that we've changed the implementation of the Pokemon ADT? Why?
    \begin{solution}
    Nothing. Because we relied on the implementation-independent interface of the Pokemon ADT, changing
    the underlying implementation does not affect the correctness of these functions.
    \end{solution}
    \end{blocksection}

    \begin{questionmeta}
        The purpose of this (trick) question is to underscore the incredible value of implementation independence. That we do not
        have to change any of our existing code even though we fundamentally change the underlying implementation of the ADT
        is very useful. Tell your students about how cool this is. You can also note that the value of implementation
        independence scales with complexity; if we wrote thousands of lines of code that all depended on this ADT, not having 
        to change them would be even more valuable then the small savings we're seeing in this problem. 
    \end{questionmeta}
\end{parts}