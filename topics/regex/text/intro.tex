Regular expressions help you match "patterns" to strings. We use the \lstinline{re} module.
For example, the \lstinline{re.search()} function returns a Match object, which tells you if a certain "pattern" can be found within a given string.

\begin{lstlisting}
>>> import re
>>> bool(re.search(r"hello", "Why hello!"))
True
>>> bool(re.search(r"bye", "Why hello!"))
False
\end{lstlisting}

We use the raw string syntax, r"regex",to specify a pattern because
it stops any backslash characters from being interpreted as special characters such as newlines.

\begin{lstlisting}
>>> print("my\nstring") # \n is interpreted as newline
my
string
>>> print(r"my\nstring")
my\nstring
\end{lstlisting}

There are many components to a pattern.

\textbf{Character Classes:}
Character classes are one of the ways we can specify what characters your pattern matches.
For each example, we match characters from the string, "Hello World! Today is 10/15/2021".
Boxed characters are matched.

% TODO: Fix Margins
\begin{figure}[h]
\begin{tabular}{ | c | p{0.2\linewidth} | c |} \hline
    Character class & Description & Example \\ \hline
    \lstinline$[oa]$ & A singular o or singular a & "Hell\tboxed{o} W\tboxed{o}rld! T\tboxed{o}d\tboxed{a}y is 10/15/2021" \\ \hline
    \lstinline$[0-9]$, \lstinline$\d$ & Any digit & "Hello World! Today is \tboxed{1}\tboxed{0}/\tboxed{1}\tboxed{5}/\tboxed{2}\tboxed{0}\tboxed{2}\tboxed{1}" \\ \hline
    \lstinline$[^a-z]$ & Anything except a lowercase letter & "\tboxed{H}ello\tphan{I}\tboxed{W}orld\tboxed{!}\tphan{I}\tboxed{T}oday\tphan{I}is\tphan{I}\tboxed{1}\tboxed{0}\tboxed{/}\tboxed{1}\tboxed{5}\tboxed{/}\tboxed{2}\tboxed{0}\tboxed{2}\tboxed{1}" \\ \hline
    \lstinline$\s$ & Any whitespace & "Hello\tphan{I}World!\tphan{I}Today\tphan{I}is\tphan{I}10/15/2021" \\ \hline
    \lstinline$[a-zA-z0-9_]$, \lstinline$\w$ & Any letter, digit, or underscore & \tboxed{Hello} \tboxed{World}! \tboxed{Today} \tboxed{is} \tboxed{10}/\tboxed{15}/\tboxed{2021} \\ \hline
    \lstinline$.$ & Anything except newline & \tboxed{Hello World! Today is 10/15/2021} \\ \hline
    \lstinline$\b$ & Word boundary (this is an anchor class!) & \tphan{]}Hello\tphan{]} \tphan{]}World!\tphan{]} \tphan{]}Today\tphan{]} \tphan{]}is\tphan{]} \tphan{]}10\tphan{]}/\tphan{]}15\tphan{]}/\tphan{]}2021\tphan{]}\\ \hline
\end{tabular}
\end{figure}
We can combine character classes in the following two ways.
\begin{itemize}
    \item \lstinline$[a-z][A-Z]$ matches a lowercase letter AND then an uppercase letter
    \item \lstinline$([a-z]|[A-Z])$ matches a lowercase letter OR an uppercase letter like \lstinline{[a-zA-Z]} does. 
    \lstinline{|} splits the entire pattern to match either the left-hand side or the right-hand side.
\end{itemize}
There are also two special characters: \textasciicircum (used outside of a character class) and \textdollar. These are called anchors and
match the beginning and ends of strings respectively.

\textbf{Quantifiers:}
Quantifiers tell you how many of something you will match.
\begin{itemize}
    \item \lstinline$[a-z]+$ matches one or more lowercase letters
    \item \lstinline$[0-9]*$ matches zero or more digits
    \item \lstinline$[A-Z]?$ matches zero or one uppercase letters
    \item \lstinline$a{1,3}$ matches 1, 2, or 3 a's
    \item \lstinline$a{2,}$ matches 2 or more a's
    \item \lstinline$a{,2}$ matches 0, 1, or 2 a's
    \item \lstinline$a{2}$ matches exactly 2 a's
\end{itemize}

\textbf{Python Functions:}
The Python \lstinline$re$ module has a lot of functions for matching patterns.

\begin{lstlisting}
import re
ptn, stn = r"...", "..."     # ptn = pattern, stn = string
# note: regex patterns use raw strings!
re.search(ptn, stn)         # does `string` contain `pattern`?
re.fullmatch(ptn, stn)      # does all of `string` follow `pattern`?
re.match(ptn, stn)          # does `string` start with `pattern`?
re.findall(ptn, stn)        # list of all matches of `pattern
re.sub(ptn, "<nope>", stn)  # replace `pattern` with `"<nope>"`
\end{lstlisting}

\textbf{Groups:}
Groups allow you to group character classes.
\begin{lstlisting}
>>> re.findall(r"ab{1,3}", "abab abb abbb aaab aabb")
['ab', 'ab', 'abb', 'abbb', 'ab', 'abb']
>>> re.findall(r"(ab){1,3}", "abab abb abbb aaab aabb")
['ab', 'ab', 'ab', 'ab', 'ab']
\end{lstlisting}
% They also allow you to "capture" part of a string and return it instead of the entire match.
% \begin{lstlisting}
% >>> re.findall(r"CSM(\w*)", "CSM61A, CSM61B, CSM70")
% ['61A', '61B', '70']
% \end{lstlisting}

\begin{guide}
\begin{blocksection}
\textbf{Teaching Tips}
    \begin{itemize}
        \item Make sure to take the time to carefully go over the examples in the table. Regex is best learned with examples - walking students through the given examples will help them tackle the other regex problems on the worksheet.
        \begin{itemize}
            \item \lstinline{'\b'} may be a particularly confusing example in the table. Highlight how it matches the zero-width boundaries at the start and end of words.
        \end{itemize}
        \item It may be useful to come up with a couple short strings that match the given examples of character groups and qualifiers in order to give students tangible examples to work with.
        \item The second example of \lstinline{re.findall} with \lstinline$r"(ab){1,3}"$ is a bit challenging, and highlights many important nuances of regex and \lstinline{re.findall}. Take some time to explain this example in particular.
        \begin{itemize}
            \item Since regex is greedy, this expression matches the entire 'abab' at the start of the string, which is why there are only 5 elements in the output and not 6.
            \item It returns "ab" rather than "abab" for that match since the group is "ab", not "abab". 
        \end{itemize}
        \item regex101.com is a great resource to get a better understanding of regex! Recommend students to try out different examples on the site, and consider using it in section to visualize some of the problems and how the matches work.
    \end{itemize}
\end{blocksection}
\end{guide}
    