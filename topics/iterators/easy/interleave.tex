\begin{blocksection}
\question Write a generator that will take in two iterators and compares the
first element of each iterator and yields the smaller of the two values.

\begin{lstlisting}
def interleave(iter1, iter2):
    """
    >>> gen = interleave(iter([1, 3, 5, 7, 9]),
                         iter([2, 4, 6, 8, 10]))
    >>> for elem in gen:
    ...     print(elem)
    ...
    1
    2
    3
    4
    5
    6
    7
    8
    9
    """
\end{lstlisting}

\begin{solution}[1in]
\begin{lstlisting}
    t1, t2 = next(iter1), next(iter2)
    while True:
        if t1 > t2:
            yield t2
            t2 = next(iter2)
        else:
            yield t1
            t1 = next(iter1)
\end{lstlisting}
\end{solution}
\end{blocksection}

\begin{guide}
\begin{blocksection}
    \textbf{Teaching Tips}
    \begin{itemize}
    \item We can get the values of an iterator by calling next(iter) repeatedly. The first time it'll yield the first one, second the second, etc.
    \item We should store these values yielded by the iterator in variables to compare them in the future.
    \item Once a value is yielded (depending on the comparison), the next value of the respective iterator should be loaded into the variable.
    \end{itemize}
\end{blocksection}
\end{guide}