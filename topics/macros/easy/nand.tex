%Requires meta-apply
\begin{blocksection}
\question NAND (not and) is a logical operation that returns false if all of its operands are true, and true otherwise. That is, it returns the opposite of AND. Implement the \lstinline{nand} macro procedure below, which takes in a list of expressions and returns the NAND of their values. Similar to \lstinline{and}, \lstinline{nand} should short circuit and return true as soon as it encounters a false operand, evaluating from left to right. 

Hint: You may use \lstinline{meta-apply} in your implementation. 

\begin{lstlisting}
;Doctests
scm> (nand (#t #t #t #t #t #t))
#f
scm> (nand (#t #f #t))
#t
scm> (nand (#f (/ 1 0)))
#t
(define-macro (nand operands))


)
\end{lstlisting}

\begin{solution}[1.5in]
\begin{lstlisting}
(define-macro (nand operands)
    `(not (meta-apply and ,operands)))
\end{lstlisting}
\end{solution}
\end{blocksection}

\begin{questionmeta}
This problem must be completed after \lstinline{meta-apply} because the solution involves \lstinline{meta-apply}. 

If students are having trouble with this problem, first try to make sure that they have a good understanding of how \lstinline{nand} works. You can walk though the problem, or draw them a table to show them the values of \lstinline{nand} with different arrangements of true and false. 

A common issue with macro problems is that our Scheme interpreter does not allow us to define procedures and macros that take arbitrary numbers of elements. Therefore, \lstinline{nand} must take a list of operands rather than taking the operands directly. This is a difference that you should note for your students (though you don't have to explain to them why.)

A good way to think of this problem is to ask your students: ``How could you logically NAND something without defining a function? What sequence of operations would allow you to do that?'' The answer, of course, is that you would take the \lstinline{and} of the expressions and the apply \lstinline{not} to the result, i.e., \lstinline{(not (and a b))}. Therefore, the body of your macro procedure should probably evaluate to an expression that looks something like that. 

The next hiccup is how you apply \lstinline{and} to a list of operands; hopefully the hint will guide them to the conclusion that they should employ the previously defined \lstinline{meta-apply}, but if not you can ask them a leading question to that effect. 
\end{questionmeta}