\begin{blocksection}
\question The built-in \lstinline{apply} procedure in Scheme applies a procedure to a given list of arguments. For example, \lstinline{(apply f '(1 2 3))} is equivalent to \lstinline{(f 1 2 3)}. Write a macro procedure \lstinline{meta-apply}, which is similar to \lstinline{apply}, except that it works not only for procedures, but also for macros and special forms. That is, \lstinline{(meta-apply operator (operand1 ... operandN))} should be equivalent to \lstinline{(operator operand1 ... operandN)} for any operator and operands. See doctests for examples. 

\begin{lstlisting}
; Doctests
scm> (meta-apply + (1 2)) 
3
scm> (meta-apply or (#t (/ 1 0) #f))
#t
(define-macro (meta-apply operator operands)
    

)
\end{lstlisting}

\begin{solution}
\begin{lstlisting}
(define-macro (meta-apply operator operands)
    (cons operator operands))
\end{lstlisting}
\end{solution}
\end{blocksection}

\begin{questionmeta}
This is supposed to be a relatively gentle introduction to the process of writing macros. Nothing much to see here. Though there may be some students who want to just write \lstinline{(operator operands)}. The issue with this, of course, is that it is treated as a call expression in the body of the macro, which causes an error. An alternate solution with quasiquote is also possible: 
\begin{lstlisting}
(define-macro (meta-apply operator operands)
    `(,operator ,operands))
\end{lstlisting}
\end{questionmeta}