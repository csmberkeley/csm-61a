\begin{blocksection}

\question Define a macro, \texttt{eval-and-check} that takes in three expressions and evaluates each expression in order. 
If the last expression evaluates to a truth-y value, return the symbol ok. Otherwise, return fail.

\begin{lstlisting}
;Doctests
scm> (eval-and-check #f #f #t)
ok
scm> (eval-and-check (+ 2 3) (print 2) (> 2 3))
2
fail
scm> (eval-and-check (define x 1) (print x) (> x 0))
1
ok


(define-macro (eval-and-check expr1 expr2 expr3)
     
     _______________________________________

     _______________________________________

     _______________________________________)
\end{lstlisting}
\end{blocksection}
\begin{blocksection}
\begin{solution}[.25in]
\begin{lstlisting}
(define-macro (eval-and-check expr1 expr2 expr3)
      `(if (begin ,expr1 ,expr2 ,expr3)
            'ok
            'fail))
\end{lstlisting}
\end{solution}

\end{blocksection}

\begin{blocksection}
\begin{guide}
\textbf{Teaching Tips}
\begin{itemize}
	\item If they offer a solution using \lstinline{cdr}, remind them of the \lstinline{begin} function, which returns the last expression.
	\item Remind them of the need to quote \lstinline{ok} and \lstinline{fail}.
\end{itemize}
\end{guide}
\end{blocksection}

\begin{blocksection}
\question Now expand \texttt{eval-and-check} to take in any number of expressions (as long as there is at least one).

\begin{lstlisting}
;Doctests
scm> (eval-and-check #f #f #f #f #t)
ok
scm> (eval-and-check (print 2) (> 2 3))
2
fail


(define-macro (eval-and-check expr1 . args)
     
     _______________________________________

     _______________________________________

     _______________________________________)
\end{lstlisting}
\begin{solution}[.25in]
\begin{lstlisting}
(define-macro (eval-and-check expr1 . args)
        `(if ,(cons 'begin (cons expr1 args))
                'ok
                'fail))
\end{lstlisting}
\end{solution}
\end{blocksection}

\begin{blocksection}
\begin{guide}
\textbf{Teaching Tips}
\begin{itemize}
	\item Remind them of the usage of \lstinline{. args} and have them consider how we can require at least one expression.
	\item Ask them to try to make as few edits as possible to the previous solution.
\end{itemize}
\end{guide}
\end{blocksection}