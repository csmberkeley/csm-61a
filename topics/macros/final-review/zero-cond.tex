\question Write a macro, \texttt{zero-cond} that takes in a list of condition-value pairs where each pair contains only numbers or arithmetic expressions that evaluate to numbers. It should evaluate each condition, \emph{treating expressions that evaluate to 0 as false-y} and then return the value corresponding to the first truth-y value.

\begin{lstlisting}
;Doctests
scm> (zero-cond
              ((0 'wrong)
               ((- 1 1) 'wrong)
               ((* 1 1) 'correct!)
               (2 'wrong)))
correct!

(define-macro (zero-cond conditions)
\end{lstlisting}
\begin{solution}
\begin{lstlisting}
(define-macro (zero-cond conditions)
     (cons 'cond 
              (map (lambda (pair) (cons (not (= 0 (eval (car pair)))) (cdr pair))) 
                       conditions)))
\end{lstlisting}
\end{solution}

\begin{blocksection}
\begin{guide}
\textbf{Teaching Tips}
\begin{itemize}
	\item Ask them to consider which built-in function might be useful and guide them to the conclusion that they should use the built in \lstinline{cond} function.
	\item Highlight the difference between a normal cond call and the zero-cond call. Get them to conclude that we need to evaluate each expression according to the zero is false rule.
	\item Have them begin by performing the custom true-false evaluation on a single condition. Then, bring up the 
\lstinline{map} and ask them to apply the evaluation to every condition.
\end{itemize}
\end{guide}
\end{blocksection}