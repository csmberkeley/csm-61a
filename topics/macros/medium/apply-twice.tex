\begin{blocksection}

\question Implement \texttt{apply-twice}, which is a macro that takes in a call expression with
a single argument. It should return the result of applying the operator to the operand twice.

\begin{lstlisting}
;Doctests
scm> (define add-one (lambda (x) (+ x 1)))
add-one
scm> (apply-twice (add-one 1))
3
scm> (apply-twice (print 'hi)) 
hi
undefined

(define-macro (apply-twice call-expr)
    
    `(let ((operator _______________________)

            (operand  _______________________))
            
            (___________________________________________)))
\end{lstlisting}

\begin{solution}[0.5in]
\begin{lstlisting}
(define-macro (apply-twice call-expr)
    `(let ((operator ,(car call-expr))
            (operand ,(car (cdr call-expr))))
          (operator (operator operand))))
\end{lstlisting}
Recall that the \lstinline{let} special form allows us to define temporary variables— in this case, 
we have \lstinline{operator} and \lstinline{operand}. Since we are working with macros, we do not evaluate the 
operands when we call \lstinline{apply-twice}. So, we will treat our input as an unevaluated 
expression, more specifically, a list. By Scheme syntax standards, we know that the 
first element of this expression must be the operator, and the second element the 
operand, and we can get these using \lstinline{car} and \lstinline{cadr}, respectively. Once we have these, 
we simply have to apply the operator to the operand twice in the body of our let 
statement.


The order of evaluation of the actual macro can get extremely tricky. 
Here’s a breakdown of what happens when we create a macro:

\begin{itemize}
  \item When a macro is created, none of the body evaluated. (This is analogous to a 
  function body, where it is only evaluated when called.)
\end{itemize}
\end{solution}
\end{blocksection}

\begin{blocksection}
\begin{solution}
\begin{itemize}
  \item When we apply the macro on some inputs, the macro’s body is then evaluated. 
  (Note that the macro’s input does not get evaluated before the macro is evaluated! 
  For example, a call on \lstinline{(apply-twice (add-one 1))} causes the lstinline{(add-one 1)} list to be 
  parsed by \lstinline{apply-twice}.) This means that we create a list (because of the quasiquote 
  operator). Inside the list, we will need to unquote some things (ie. substitute some 
  values in). Here, the values are the operator and the operand, both bound in the let 
  statement.
  \item When we bind this, this transforms the body into a "new body" that will cause a 
  function to be applied to its input.
\end{itemize}
\end{solution}

\end{blocksection}