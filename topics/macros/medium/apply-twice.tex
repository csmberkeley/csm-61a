\begin{blocksection}

\question Implement \texttt{apply-twice}, which is a macro that takes in a call expression with
a single argument. It should return the result of applying the operator to the operand twice.

\begin{lstlisting}
;Doctests
scm> (define add-one (lambda (x) (+ x 1)))
add-one
scm> (apply-twice (add-one 1))
3
scm> (apply-twice (print 'hi))
hi
undefined

(define-macro (apply-twice call-expr)
		_______________________
)

\end{lstlisting}

\begin{solution}[0.5in]
\begin{lstlisting}
(define-macro (apply-twice call-expr)
  (list (car call-expr) call-expr)
)

\end{lstlisting}
\end{solution}
\end{blocksection}

\begin{guide}
\begin{blocksection}
\textbf{Teaching Tips}
\begin{itemize}
  \item The first question to ask your students would be, "What are the inputs to our function?" In this case, we are accepting a scheme list called call-expr containing an operator and an operand.
  \item The second question to ask your students would be, "How should our function behave?" We want to apply the operator onto the result of applying the operator to the operand. So applying our function twice, so for an arbitrary function \lstinline{f}, and input \lstinline{x}, \lstinline{f(f(x))}
  \item So how do we apply both of these. We create the list! The expression should have two elements, the first one being the operator of \lstinline{call-expr}, and the second being another list, with the operator and operand of \lstinline{call-expr}
  \item if your students ask why this has to be a macro procedure, consider running through the function normally to see that you would have to initially evaluate the operands of the function, which wouldn't be particularly useful. For example if we had the call \lstinline{(apply-twice (add-one 3))} with a function \lstinline{add-one} that adds one to our input, we would simply pass in 4 into our function \lstinline{apply-twice} which isn't that helpful
\end{itemize}

\end{blocksection}
\end{guide}
