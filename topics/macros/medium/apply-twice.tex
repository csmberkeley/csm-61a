\begin{blocksection}

\question Implement \texttt{apply-twice}, which is a macro that takes in a call expression with
a single argument. It should return the result of applying the operator to the operand twice.

\begin{lstlisting}
;Doctests
scm> (define add-one (lambda (x) (+ x 1)))
add-one
scm> (apply-twice (add-one 1))
3
scm> (apply-twice (print 'hi))
hi
undefined

(define-macro (apply-twice call-expr)

    `(let ((operator _______________________)

            (operand  _______________________))

            (___________________________________________)))
\end{lstlisting}

\begin{solution}[0.5in]
\begin{lstlisting}
(define-macro (apply-twice call-expr)
    `(let ((operator ,(car call-expr))
            (operand ,(car (cdr call-expr))))
          (operator (operator operand))))
\end{lstlisting}
\end{solution}
\end{blocksection}

\begin{guide}
\begin{blocksection}
\textbf{Teaching Tips}
\begin{itemize}
  \item Recall that the \lstinline{let} special form allows us to define temporary variables— in this case,
    we have \lstinline{operator} and \lstinline{operand}
  \item The order of evaluation of a macro can get extremely tricky- here is a brief overview:
  \item When a macro is created, none of the body evaluated. (This is analogous to a
  function body, where it is only evaluated when called.)
  \item When we apply the macro on some inputs, the macro’s body is then evaluated.
  \item Note that the macro’s input does not get evaluated before the macro is evaluated!
  \item For example, a call on \lstinline{(apply-twice (add-one 1))} causes the \lstinline{(add-one 1)} list to be
  parsed by \lstinline{apply-twice}.
  \item This means that we create a list (because of the quasiquote
  operator), and inside the list, we will need to unquote some things (ie. substitute some
  values in)
  \item Here, the values are the operator and the operand, both bound in the let
  statement
  \item When we bind this, it transforms the body into a "new body" that will cause a
  function to be applied to its input.
\end{itemize}

\end{blocksection}
\end{guide}