\begin{blocksection}
\question Recall that in Scheme, all expressions are truthy except for the boolean \lstinline{#f}. In contrast, Python considers empty lists, \lstinline{None}, and \lstinline{0} falsy as well. Implement the macro \lstinline{python-if}, which acts just like the \lstinline{if} special form, but it treats empty lists, \lstinline{undefined}, \lstinline{0}, and \lstinline{#f} as falsy and all other values as truthy. 

\begin{lstlisting}
;Doctests
scm> (python-if (- 1 1) (/ 1 0) 1)
1
scm> (python-if (not #t) (/ 1 0) 2)
2
scm> (python-if (cdr '(1)) (/ 1 0) 3)
3
scm> (python-if (print 4) (/ 1 0) 5)
4
5
scm> (python-if (- 4 3) 6 (/ 1 0))
6

(define-macro (python-if pred if-true if-false)











)
\end{lstlisting}
\end{blocksection}

\begin{solution}
\begin{lstlisting}
(define-macro (python-if pred if-true if-false)
    (let ((pred-val (eval pred)))
         (if (or (not pred-val) (null? pred-val) (equal? 0 pred-val) (equal? undefined pred-val))
            if-false
            if-true)))
\end{lstlisting}
\end{solution}

\begin{questionmeta}
This is a relatively straightforward problem, so we offered no skeleton to encourage students to think through the design process a little bit more. If you students are having trouble, this is the logical process they would hopefully work through. 

\begin{itemize}
\item Evaluate the predicate. We really only want to evaluate the predicate once, so in this case, we use a \lstinline{let} expression to store it, but you could also use \lstinline{define} for the same purpose. 
\item Determine if the value of the predicate is truthy or falsy. The predicate is falsy if it is 
\begin{itemize}
    \item \lstinline{#f},
    \item \lstinline{nil},
    \item \lstinline{0}, or
    \item \lstinline{undefined}.
\end{itemize}
\end{itemize}

There are a few tricky technical details here. For this problem, students should use \lstinline{equal?} or \lstinline{eq?} because \lstinline{=} only works with numbers (and we must be able to check if everything, not just numbers, is falsy). Students may also be unfamiliar with \lstinline{undefined}; you can just tell them it is like Python's \lstinline{None} and that it is returned by functions that don't evaluate to anything (like \lstinline{print}).  
\end{questionmeta}