\begin{blocksection}

\question Implement \texttt{if-macro}, which behaves similarly to the \texttt{if} special form in Scheme
but has some additional properties. Here's how the \texttt{if-macro} is called:
\newline
\texttt{if <cond1> <expr1> elif <cond2> <expr2> else <expr3>}
\newline
If cond1 evaluates to a truth-y value, expr1 is evaluated and returned. Otherwise, if cond2 evaluates
to a truth-y value, expr2 is evaluated and returned. If neither condition is true, expr3 is evaluted and returned.

\begin{lstlisting}
;Doctests
scm> (if-macro (= 1 0) 1 elif (= 1 1) 2 else 3)
2
scm> (if-macro (= 1 1) 1 elif (= 2 2) 2 else 3)
1
scm> (if-macro (= 1 0) (/ 1 0) elif (= 2 0) (/ 1 0) else 3)
3

(define-macro (if-macro cond1 expr1 elif cond2 expr2 else expr3)













)
\end{lstlisting}
\end{blocksection}
\begin{blocksection}
\begin{solution}[0.5in]
\begin{lstlisting}
(define-macro (if-macro cond1 expr1 elif cond2 expr2 else expr3)
    (list 'cond (list cond1 expr1)
             (list cond2 expr2)
             (list 'else expr3)))
\end{lstlisting}
Alternate solution with nested ifs:
\begin{lstlisting}
(define-macro (if-macro cond1 expr1 elif cond2 expr2 else expr3)
    (list 'if cond1 expr1 (list 'if cond2 expr2 expr3)))
\end{lstlisting}
Alternate solution with quasiquoting:
\begin{lstlisting}
(define-macro (if-macro cond1 expr1 elif cond2 expr2 else expr3)
    `(cond (,cond1 ,expr1)
             (,cond2 ,expr2)
             (else ,expr3)))
\end{lstlisting}
\end{solution}

\end{blocksection}

\begin{blocksection}
\question Could we have implemented \texttt{if-macro} using a function instead of a macro? Why or why not?
\begin{solution}
Without using macros, the inputs would be evaluated when we evaluated the function call. This is problematic for two reasons:
\newline
First, we only want to evaluate the expressions under certain conditions. If cond1 was false, we would not want to evaluate expr1. This might lead to errors!
\newline
Secondly, some of the inputs to the call would be names which have no binding in the global frame. Elif, for example, is not supposed to be interpreted as a name
but rather as a symbol. This would cause our code to error if we ran it as is!
\newline
Of course, we could have written out a \texttt{cond} or nested \texttt{if} expression instead of defining an \texttt{if-macro}. But the syntax for \texttt{if-macro} 
is more familiar, which is why we might want to do something like this!
\end{solution}
\end{blocksection}