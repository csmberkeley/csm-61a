\question
\begin{parts}
\part
Write a function \lstinline{contains-stream}, which takes in a finite stream of integers \lstinline{s} and an integer \lstinline{elem}, and returns true if \lstinline{elem} is in \lstinline{s}; otherwise, return false.

\begin{lstlisting}
(define (contains-stream s elem)
    (cond

        (_________________________________________________)

        (_________________________________________________)

        (else ____________________________________________)
    )
)
\end{lstlisting}

\begin{solution}
\begin{lstlisting}
(define (contains-stream s elem)
    (cond
        ((null? s) #f)
        ((= (car s) elem) #t)
        (else (contains-stream (cdr-stream s) elem))
    )
)
\end{lstlisting}
\end{solution}
\newpage

\part
Now, let's write a function \lstinline{pairer}, which takes in a finite stream of integers \lstinline{s} and a two-argument boolean function \lstinline{condition}. \lstinline{pairer} should return a new stream, where each element is a length 2 list of elements of \lstinline{s}; call these elements \lstinline{X} and \lstinline{Y}. \lstinline{(X Y)} should appear in the returned stream if \lstinline{(condition X Y)} returns true and \lstinline{Y} is either in the same position or appears later than \lstinline{X} in \lstinline{s}. Finally, \lstinline{X} can be the first element of a list in the returned stream at most once, including duplicates.

You may use \lstinline{contains-stream} from the previous part in your answer.

\begin{lstlisting}
; doctests
scm> (define (mult-four x y) (= (remainder (+ x y) 4) 0))
mult-four
scm> (define s (make-stream '(1 2 1 3 5)))
s
scm> (define final-stream (pairer s mult-four))
final-stream
scm> (car final-stream)
(1 3)
scm> (car (cdr-stream final-stream))
(2 2)
scm> (car (cdr-stream (cdr-stream final-stream)))
(3 5)
; note: final-stream omits the 2nd (1 3) because 1 already appears as the 1st element of a list
\end{lstlisting}

Problem continues on the next page. You may not need to use all of the provided lines.
\newpage

\begin{lstlisting}
(define (pairer s condition)
    (define (pair-helper s copy firsts)
        (cond
            ((null? s) nil)

            ((null? copy) ________________________________)

            ((____________________________________________)
            
                _________________________________________ \

                _________________________________________ \

                __________________________________________)

            ((____________________________________________)

                _________________________________________ \

                __________________________________________)

            (else ________________________________________)
        )
    )

    (pair-helper ___________________________________)
)
\end{lstlisting}

\begin{solution}
\begin{lstlisting}
(define (pairer s condition)
    (define (pair-helper s copy firsts)
        (cond
            ((null? s) nil)
            ((null? copy) (pair-helper (cdr-stream s) (cdr-stream s) firsts))
            ((and (not (contains-stream firsts (car s))) (condition (car s) (car copy))) 
                (cons-stream (list (car s) (car copy)) 
                    (pair-helper (cdr-stream s) 
                        (cdr-stream s) 
                        (cons-stream (car s) firsts))))
            ((not (contains-stream firsts (car s))) 
                (pair-helper s (cdr-stream copy) firsts))
            (else (pair-helper (cdr-stream s) (cdr-stream s) firsts))
        )
    )
    (pair-helper s s nil)
)
\end{lstlisting}
\end{solution}
\newpage

\part
Finally, write a function \lstinline{filter-stream}, which takes in a finite stream of integers \lstinline{s}, a list of integers \lstinline{lst}, and a two-argument boolean function \lstinline{condition}. \lstinline{filter-stream} should return a stream of two-element lists, as in part (b), except that a list \lstinline{(X Y)} should only be included in the returned stream if \lstinline{X} is in \lstinline{lst}. The lists in the returned stream do not need to be in the same order as they appear in the stream returned by a call to \lstinline{pairer}.

You may use \lstinline{pairer} from the previous part in your answer. You may not need to use all of the provided lines.

\begin{lstlisting}
; doctests
scm> (define (mult-four x y) (= (remainder (+ x y) 4) 0))
mult-four
scm> (define s (make-stream '(1 2 1 3 5)))
s
scm> (print-stream (pairer s mult-four))
((1 3) (2 2) (3 5))
scm> (print-stream (filter-stream s '(1 2) mult-four))
((1 3) (2 2))

(define (filter-stream s lst condition)
    (define (filter-helper s copy lst)
        (cond

            ((____________________________) ______________)

            ((null? copy) _______________________________ \

                __________________________________________)

            ((____________________________________________)

                _________________________________________ \

                __________________________________________)

            (else _______________________________________ \

                __________________________________________)
        )
    )
    (filter-helper _______________________________________)
)
\end{lstlisting}

\begin{solution}
\begin{lstlisting}
(define (filter-stream s lst condition)
    (define (filter-helper s copy lst)
        (cond
            ((or (null? s) (null? lst)) nil)
            ((null? copy) 
                (filter-helper (cdr-stream s) 
                    (cdr-stream s) lst))
            ((= (car (car copy)) (car lst)) 
                (cons-stream (car copy) 
                    (filter-helper (cdr-stream s) 
                        (cdr-stream s) (cdr lst))))
            (else (filter-helper s (cdr-stream copy) lst))
        )
    )
    (filter-helper (pairer s condition) 
        (pairer s condition) lst)
)
\end{lstlisting}
\end{solution}

\end{parts}