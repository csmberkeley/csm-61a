\question
\begin{parts}
\part
Write a function \lstinline{add-streams}, which takes in two infinite streams of integers and returns a stream where the \lstinline{i}-th element is the sum of the \lstinline{i}-th elements of the input streams.

\begin{lstlisting}
(define (add-streams s1 s2)

    ____________________________________________________ \

    ____________________________________________________
)
\end{lstlisting}

\begin{solution}
\begin{lstlisting}
(define (add-streams s1 s2)
    (cons-stream (+ (car s1) (car s2))
        (add-streams (cdr-stream s1) (cdr-stream s2)))
)
\end{lstlisting}
\end{solution}

\part
Now complete the following \lstinline{define} statement for \lstinline{fib}, which should give us the Fibonacci sequence as a stream. You may use \lstinline{add-streams} from the previous part in your answer.

As a reminder, the Fibonacci sequence is as follows: 0, 1, 1, 2, 3, 5, 8, ...

\begin{lstlisting}
(define fib 

    _____________________________________________________ \

    _____________________________________________________
)
\end{lstlisting}

\begin{solution}
\begin{lstlisting}
(define fib
    (cons-stream 0
        (cons-stream 1
            (add-streams fib (cdr-stream fib))))
)
\end{lstlisting}
\end{solution}
\end{parts}