\begin{blocksection}
\question Find the $\Theta(\cdot)$ runtime bound for \lstinline$hiya(n)$.  Remember that Python strings are immutable: when we add two strings together, we need to make a copy.

\begin{lstlisting}
def hiii(m):
    word = "h"
    for i in range(m):
        word += "i"
    return word

def hiya(n):
    i = 1
    while i < n:
        print(hiii(i))
        i *= 2
\end{lstlisting}
\end{blocksection}

\begin{solution}
$\Theta(n^2)$.

Solution: 
We can determine the efficiency by approximately counting the number of characters we have to store upon a call to \lstinline{hiya(n)}. First, let us determine the efficiency of a call \lstinline{hiii(m)}. Within \lstinline{hiii}'s for loop: 
\begin{itemize}
    \item When \lstinline{i} is \lstinline{1}, we store the string "hi", which is 2 characters.
    \item When \lstinline{i} is \lstinline{2}, we store the string "hii", which is 3 characters. 

    ...
    \item When \lstinline{i} is \lstinline{m}, we store \lstinline{m + 1} characters. 
\end{itemize}

Adding up these values, we see that calling \lstinline{hiii(m)} causes us to store on the order of $m^2$ characters. (The exact value is $\frac{m(m+3)}{2} = \frac{m^2}2 + \frac{3}2 m$, but we really only care about the highest order term.)

Now, when we make a call \lstinline{hiya(n)}, we will make calls to \lstinline{hiii(1)}, \lstinline{hiii(2)}, \lstinline{hiii(4)}, ..., \lstinline{hiii(4)}. This will store approximately $1^2 + 2^2 + 4^2 + 8^2 + ... + n^2$ characters. Calculating out the partial sums of this sequence shows that
$$1^2 = 1$$
$$1^2 + 2^2 = 5 < 2\cdot 2^2$$
$$1^2 + 2^2 + 4^2 = 21 < 2\cdot 4^2$$
$$1^2 + 2^2 + 4^2 + 8^2 = 85 < 2\cdot 8^2$$

At some point, we are reasonably convinced that this pattern holds. Thus the value of $1^2 + 2^2 + 4^2 + 8^2 + ... + n^2$ is approximately $n^2$, within a constant factor. So we store about $n^2$ characters upon a call to \lstinline{hiya(n)}, which means the efficiency is $\Theta(n^2)$. 
\end{solution}