\begin{blocksection}
\question What is the order of growth in time for the following functions? Use Θ notation.

\begin{parts}

\part
\begin{lstlisting}
def bar(n):
    if n == 3:
        return 'three!'
    for i in range(n / 2):
        bar(3)
\end{lstlisting}
\begin{solution}[0.25in]
$O(n)$. bar(3) takes constant time (even if we did not have the base case, bar(3) takes time independent from n). We do bar(3) n times.
\end{solution}

\part
\begin{lstlisting}
def repeat_digits(n):
    last, rest = n % 10, n // 10
		if rest == 0:
        return last * 11
    return repeat_digits(rest) * 100 + last * 11
\end{lstlisting}
What is the runtime in terms of d, the number of digits in n?
\begin{solution}[0.25in]
$O(d)$. We make one recursive call with constant work per digit.
\end{solution}

\end{parts}
\end{blocksection}
