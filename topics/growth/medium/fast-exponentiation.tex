\question \textbf{Fast Exponentiation:} in this problem, we will examine a
real-world algorithm used to improve the speed of calculating exponents.
You can assume for the purposes of this problem that multiplication is a
constant time operation.

\begin{parts}

\begin{blocksection}
\part First, determine the runtime efficiency of the naive exponentiation algorithm.
\begin{lstlisting}
def exp(b, n):
    if n == 0:
        return 1
    else:
        return b * exp(b, n - 1)
\end{lstlisting}
\begin{solution}[0.25in]
Linear runtime. $n$ decreases by 1 each call, so there are naturally $n$ calls.
\end{solution}
\end{blocksection}

\begin{blocksection}
\part Now, express the runtime of the fast exponentiation algorithm.
\begin{lstlisting}
def fast_exp(b, n):
    if n == 0:
        return 1
    elif n % 2 == 0: # Assume square runs in constant time
        return square(fast_exp(b, n // 2))
    else:
        return b * fast_exp(b, n - 1)
\end{lstlisting}
\begin{solution}[0.25in]
$\theta(\log n)$. $n$ is halved each call, so the number of calls is the number of
times $n$ must be halved to get to 1. This is $\log n$, or logarithmic.
\end{solution}
\end{blocksection}

\begin{blocksection}
\part What about this slightly modified version of \texttt{fast\char`_exp}?
\begin{lstlisting}
def fast_exp(b, n):
    for _ in range(50 * n):
        print("Killing time")
    if n == 0:
        return 1
    elif n % 2 == 0:
        return square(fast_exp(b, n // 2))
    else:
        return b * fast_exp(b, n - 1)
\end{lstlisting}
\begin{solution}[0.25in]
$\theta(n)$. Ignore the constant term. The first call will perform $n$ operations,
the second call will perform $n / 2$ operations, the third will perform $n / 4$
operations, etc. Using geometric series, we see this adds up to $2n$, which is
$n$ (linear) if we ignore constant terms.
\end{solution}
\end{blocksection}

\end{parts}

\begin{guide}    
    \textbf{Teaching Tips:}
    \begin{itemize}
        \item If students are having trouble, make sure to run them through a couple of simple inputs first.
        \item An important distinction to make is that we are looking at the \textbf{asymptotic} behavior of these programs (what happens as inputs get arbitrarily large).
        % \item Students might also be unclear about the differences between big-O and big-$\theta$
        \item You can also give them examples of common problems such as fibonacci to talk about exponential runtime.
    \end{itemize}
\end{guide}
