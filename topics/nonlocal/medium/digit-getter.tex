\begin{blocksection}
\question Write a function \texttt{make\char`_digit\char`_getter} that, given a positive integer \texttt{n}, returns a new function that returns the digits in the integer  
one by one, starting from the rightmost digit. Once all digits have been removed, subsequent calls to the function should return the sum of all the digits in the original integer.

\begin{lstlisting}
def make_digit_getter(n):
    """ Returns a function that returns the next digit in n
    each time it is called, and the total value of all the integers
    once all the digits have been returned.
    >>> year = 8102
    >>> get_year_digit = make_digit_getter(year)
    >>> for _ in range(4):
    ...     print(get_year_digit())
    2
    0
    1
    8
    >>> get_year_digit()
    11
    """
    ____________________________________

    def get_next():
        
       ________________________________
        
       if ______________________________:

                ______________________________
        
       ________________________________
        
       ________________________________
			
       ________________________________
       
       return  __________________________

    return ________________________________
\end{lstlisting}

\begin{solution}
\begin{lstlisting}
def make_digit_getter(n):
    total = 0
    def get_next():
        nonlocal n, total
        if n == 0:
             return total
        val = n % 10
        n = n // 10
        total += val
        return val
    return get_next
\end{lstlisting}
\end{solution}

\end{blocksection}
