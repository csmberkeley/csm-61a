To evaluate a variable name, we find its value from the first frame (current frame, parent, parent’s parent, …) the variable is defined. The first time we assign (bind) a value to a variable, we declare a new variable in the current frame and bind to the respective value.

The \lstinline{nonlocal} keyword introduces more control over this process. The first time we assign a value to a \lstinline{nonlocal} variable, rather than declare a new variable in the current frame, we bind the value to the variable of the same name found in the first parent frame that contains such a variable. This means that this variable does not exist in the current frame. Note: you cannot declare variables in the global frame as \lstinline{nonlocal}.
\newline
\begin{lstlisting}
def example_without_nonlocal():
    grade = 1.0
    def gpa_boost():
        grade = 4.0 # creates a variable named grade in the 
                    # gpa_boost frame
    gpa_boost()
    print(grade)
>>> example_without_nonlocal()
1.0
\end{lstlisting}
\pagebreak
\begin{lstlisting}
def example_with_nonlocal():
    grade = 1.0
    def gpa_boost():
        nonlocal grade
        grade = 4.0 # modifies the variable in the
                    # example_with_nonlocal frame instead
                    # of creating a new variable
    gpa_boost()
    print(grade)
>>> example_with_nonlocal()
4.0
\end{lstlisting}