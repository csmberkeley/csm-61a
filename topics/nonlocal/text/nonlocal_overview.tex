% To evaluate a variable name, we find its value from the first frame the variable is defined in. The first time we assign a value to a variable, we declare this new variable in the current frame and bind its value there.
%
% The \lstinline{nonlocal} keyword introduces more control over this process.
\\The first time we assign a value to a \lstinline{nonlocal} variable, rather than declare a new variable in the current frame, we bind the value to the variable in the first parent frame that contains such a variable. The variable does not exist in the current frame!
\\Note: you cannot declare variables in the global frame as \lstinline{nonlocal}.
\newline
\begin{lstlisting}
def example_without_nonlocal():
    grade = 1.0
    def gpa_boost():
        grade = 4.0 # creates a variable named grade
    gpa_boost()
    print(grade)
>>> example_without_nonlocal()
1.0

def example_with_nonlocal():
    grade = 1.0
    def gpa_boost():
        nonlocal grade
        grade = 4.0 # modifies the variable in the
                    # example_with_nonlocal frame
    gpa_boost()
    print(grade)
>>> example_with_nonlocal()
4.0
\end{lstlisting}
