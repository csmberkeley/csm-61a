\begin{blocksection}
\question Write a function \lstinline{partial_summer}, which takes in a list of integers \lstinline{lst} and returns a function. The returned function takes in a non-negative integer \lstinline{n}. It prints a sum derived from the first \lstinline{n} elements of \lstinline{lst}: if element \texttt{X} is even, divide \texttt{X} by \lstinline{2} before adding it to the sum, and if \texttt{X} is odd. add \lstinline{1} to \texttt{X} before adding it to the sum. If \lstinline{n > len(lst)}, then sum as many elements of \lstinline{lst} as you can. After printing the sum, the returned function returns another function, that when called, will perform the same procedure on the remaining \lstinline{len(lst) - n} elements of \lstinline{lst}. \\

\begin{lstlisting}
def partial_summer(lst):
    """
    >>> lst = [1, 2, 3, 4, 5, 6, 7, 8, 9]
    >>> f = partial_summer(lst)(3)
    7 # 7 = (1+1) + (2/2) + (3+1)
    >>> g = f(4)
    19 # 19 = (4/2) + (5+1) + (6/2) + (7+1)
    >>> h = g(3)
    14 # 14 = (8/2) + (9+1)
    >>> i = h(1)
    0	
    """
    def helper(n):

        total, i = ________, ________

        while ________________ and ___________________:

            if _______________________:

                total += ____________________________
            else:
                total += lst[i] + 1
            
            _________________________________________
        print(total)

        return ______________________________________
    return helper
\end{lstlisting}
\end{blocksection}
\begin{blocksection}
\begin{solution}
\begin{lstlisting}
def partial_summer(lst):
    def helper(n):
        total, i = 0, 0
        while i < n and i < len(lst):
            if lst[i] % 2 == 0:
                total += lst[i] // 2
            else:
                total += lst[i] + 1
            i += 1
        print(total)
        return partial_summer(lst[n:])
    return helper
\end{lstlisting}
\end{solution}
\end{blocksection}