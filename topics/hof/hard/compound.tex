\begin{blocksection}
\question Implement \texttt{compound}, which takes in a single-argument function \texttt{base\_func} and returns a two-argument compounder function \texttt{g}. The function \texttt{g} takes in an integer \texttt{x} and positive integer \texttt{n}.

Each call to \texttt{g} will print the result of calling \texttt{f} repeatedly 0,1,.., n-1 times on \texttt{x}. That is, \texttt{g(x, 2)} prints \texttt{x}, then \texttt{f(x)}. Then, \texttt{g} will return the next two-argument compounder function.

\begin{lstlisting}
def compound(base_func, prev_compound=lambda x: x):
    """
    >>> add_one = lambda x: x + 1
    >>> adder = compound(add_one)
    >>> adder = adder(3, 2)
    3       # 3
    4       # f(3)
    >>> adder = adder(4, 4)
    6       # f(f(4))
    7       # f(f(f(4)))
    8       # f(f(f(f(4))))
    9       # f(f(f(f(f(4)))))
    """
    def g(x, n):
      new_comp = ____________________________

      while n > 0:
        print(____________________________)

        new_comp = (lambda save_comp: \
                    _______________________)(____________)

        ___________________________________________________

      return ______________________________________________
      
    return ___________________________________  
\end{lstlisting}

\begin{solution}[1in]
\begin{lstlisting}
def compound(base_func, prev_compound=lambda x : x):
  def g(x, n):
    new_comp = prev_compound
    while n > 0:
      print(new_comp(x))
      new_comp = (lambda save_comp: \
                  lambda x: base_func(save_comp(x)))(new_comp)
      n -= 1
    return compound(base_func, new_comp)
  return g
\end{lstlisting}
\end{solution}
\end{blocksection}