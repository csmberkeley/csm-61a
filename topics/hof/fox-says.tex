\begin{blocksection}
\question (Fall 2013 MT1 Q3D) The CS61A staff has developed a formula for determining what a fox might say.
Given three strings, a start, a middle, and an end, a fox will say the start string, followed by the 
middle string repeated a number of times, followed by the end string. These parts are all separated by hyphens.

Complete the definition of \lstinline$fox_says$, which takes the three string parts of the fox's statement
(\lstinline$start$, \lstinline$middle$, and \lstinline$end$) and a positive integer \lstinline$num$ indicating
how many times to repeat \lstinline$middle$. It returns a string.

You cannot use any \lstinline$for$ or \lstinline$while$ statements. Use recursion in \lstinline$repeat$. Moreover,
you cannot use string operations other than the + operator to concatenate strings together.

\begin{lstlisting}
def fox_says(start, middle, end, num):
    """    
    >>> fox_says('wa', 'pa', 'pow', 3)
    'wa-pa-pa-pa-pow'
    >>> fox_says('fraka', 'kaka', 'kow', 4)
    'fraka-kaka-kaka-kaka-kaka-kow'
    """
    def repeat(k):
        
        
        
        
        
        
        
        
    return start + '-' + repeat(num) + '-' + end
    
\end{lstlisting}

\begin{solution}[1.5in]
\begin{lstlisting}
def fox_says(start, middle, end, num):
    def repeat(k):
        if k == 1:
            return middle
        else:
            return middle + '-' + repeat(k - 1)
    return start + '-' + repeat(num) + '-' + end
\end{lstlisting}
\end{solution}
\end{blocksection}
