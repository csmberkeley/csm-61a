\begin{blocksection}
\question Write a function, \lstinline{make_digit_remover}, which takes in a single digit \texttt{i}. It returns another function that takes in an integer and, scanning from right to left, removes all digits from the integer up to and including the first occurrence of \texttt{i}, starting from the ones place. If \texttt{i} does not occur in the integer, the original number is returned. 

\begin{lstlisting}
def make_digit_remover(i):
    """
    >>> remove_two = make_digit_remover(2)
    >>> remove_two(232018)
    23
    >>> remove_two(23)
    0
    >>> remove_two(99)
    99
    """
    def remove(_______):

    	removed = _______________________

        while _______________________ > 0:

            _____________________________

            removed = removed // 10

            if __________________________:

                _________________________

        return __________________________

    return __________________________
\end{lstlisting}

\begin{solution}
\begin{lstlisting}
def make_digit_remover(i):
    def remove(n):
        removed = n
        while removed > 0:
            digit = removed % 10
            removed = removed // 10
            if digit == i:
                return removed
        return n
    return remove
\end{lstlisting}
\end{solution}
\end{blocksection}
