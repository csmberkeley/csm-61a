\begin{blocksection}
\question Write a function, \texttt{kill\_the\_n}, which takes in an integer \texttt{n}. It returns another function that takes in integer \texttt{x}, and removes the smallest (right-most) occurrence of the appearance of x in n.

\begin{lstlisting}
def kill_the_n(n):
  """
  >>> kill_the_n(99)(10990990)
  109900
  >>> kill_the_n(99)(900999)
  9009
  >>> kill_the_n(99)(989)
  989
  """
 def kill(x):
 
   _________________________________________________
   
   for i in range(___________ - ___________, 0, -1):
   
     if ______[______:_________+num_digits(n)] == ___________:
     
        return int(_________[:i] + ________[i+num_digits(n):])
       
   return int(___________)
   
 return ___________

def num_digits(n):
    #Finds the number of digits in the positive number n
    if n == 0:
        return 0
    return 1 + num_digits(n//10)

\end{lstlisting}

\begin{solution}
\begin{lstlisting}
def kill_the_n(n):
 #removes smallest occurence of the digits of x in n.
 #ex: kill_the_n(99)(10990990) ==> 109900
 #ex: kill_the_n(99)(900999) ==> 9009
 #ex: kill_the_n(99)(989) ==> 989
 def kill(x):
   x_str = str(x)
   for i in range(len(x_str) - num_digits(n), 0, -1):
     if x_str[i:i+num_digits(n)] == str(n):
       return int(x_str[:i] + x_str[i+num_digits(n):])
   return int(x_str)
 return kill

def num_digits(n):
 #Finds the number of digits in the positive number n
 if n == 0:
   return 0
 return 1 + num_digits(n//10)



\end{lstlisting}
\end{solution}
\end{blocksection}

