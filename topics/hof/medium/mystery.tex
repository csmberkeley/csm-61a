\begin{blocksection}
\question Write a higher-order function that passes the following doctests.

\emph{Challenge:} Write the function body in one line.

\begin{lstlisting}
def mystery(f, x):
    """
    >>> from operator import add, mul
    >>> a = mystery(add, 3)
    >>> a(4) # add(3, 4)
    7
    >>> a(12)
    15
    >>> b = mystery(mul, 5)
    >>> b(7) # mul(5, 7)
    35
    >>> b(1)
    5
    >>> c = mystery(lambda x, y: x * x + y, 4)
    >>> c(5)
    21
    >>> c(7)
    23
    """
\end{lstlisting}

\begin{solution}[1in]
\begin{lstlisting}
    def helper(y):
        return f(x, y)
    return helper
\end{lstlisting}

Challenge solution:

\begin{lstlisting}
    return lambda y : f(x, y)
\end{lstlisting}
\end{solution}
\end{blocksection}

\begin{questionmeta}
    Using doctests to understand how a function should work is a fundamental part of CS 61A. The goal of this question is to force students to exercise that muscle by removing any other description of \lstinline{mystery}. 

    % Note: see questionmeta on xyz.tex
    The advice noted before about acting as a ``detective'' applies again to this problem. 
\end{questionmeta}