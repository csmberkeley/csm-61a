\begin{blocksection}
\question Complete the definition of the function \texttt{flip\char`_flop} which takes in one-argument functions \texttt{f} and \texttt{g}, and returns the result of alternatively applying \texttt{f} and \texttt{g} to \texttt{base} \texttt{n} times. See the doctest for examples.
\begin{lstlisting}
def flip_flop(f, g, n, base):
	"""
	>>> mul_3 = lambda x: x * 3
	>>> add_1 = lambda y: y + 1
	>>> flip_flop(mul_3, add_1, 2, 4)
	# ((4 + 1) * 3) * 3) + 1
	46
	>>> flip_flop(add_1, mul_3, 2, 4)
	# ((4 * 3) + 1) + 1) * 3
	42
	"""

	while ________________:
	
		___________________
		
		___________________
		
		___________________
		
	return _______________

\end{lstlisting}

\begin{solution}
\begin{lstlisting}
def flip_flop(f, g, n, base):
	"""returns the result of calling f(g(...)) 
	then g(f(...)) on base alternatively, n times.
	>>> mul_3 = lambda x: x * 3
	>>> add_1 = lambda y: y + 1
	>>> flip_flop(mul_3, add_1, 2, 4)
	# ((4 + 1) * 3) * 3) + 1
	46
	>>> flip_flop(add_1, mul_3, 2, 4)
	# ((4 * 3) + 1) + 1) * 3
	42
	"""

	while n > 0:
	
		base = f(g(base))
		
		f, g = g, f
		
		n -= 1
		
	return base

\end{lstlisting}
\end{solution}
\end{blocksection}