\begin{blocksection}
\question Write a function, \texttt{whole\_sum}, which takes in an integer, \texttt{n}. It returns another function which takes in an integer, and returns \texttt{True} if the digits of that integer sum to \texttt{n} and \texttt{False} otherwise.

\begin{lstlisting}
def whole_sum(n): 
    """ 
    >>> whole_sum(21)(777)
    True
    >>> whole_sum(142)(10010101010)
    False
    """
    def check(x):

        ___________________
		
        while _____________:
		
            last = __________________
				
            _________________________
				
            _________________________
				
        return __________________
		
    return _________________

\end{lstlisting}

\begin{solution}
\begin{lstlisting}
def whole_sum(n):
    def check(x):
        total = 0
        while x > 0:
            last = x % 10
            x = x // 10
            total += last
        return total == n
    return check
\end{lstlisting}
\end{solution}
\end{blocksection}

\begin{questionmeta}

    Great time to emphasize that whenever you write a HOF,
    you must return the inner function in order to use it properly
    (i.e. in order to use function \lstinline{check} we must return on it the last line). 
    This is quite an ``easy'' thing to check and it's a relatively accessible point on
    exams to students who are familiar with it. 

    We suggest that you go through this problem step by step
    in order to solidify digit manipulation concepts (i.e. x // 10, x \% 10).
\end{questionmeta}
