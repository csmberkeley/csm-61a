\begin{blocksection}
\question 
For our purposes, an order-n function returns an order-$n-1$ function if $n > 1$. An $order-1$ function can return any value. 
\newline
Therefore, if \texttt{f} is an order-3 numeric function, we could call \texttt{f} like \lstinline{f(1)(2)(3)}. 
\newline
Implement the function \texttt{apply\_n}, which takes an integer \texttt{n}, a one argument function \texttt{base\_fn},
and a value \texttt{x}. This function returns an order-n function which takes in one argument. This order-n function takes in
a single function. If this order-n function is called on $f_1, f_2, \ldots, f_n$, it should return the value equivalent to calling
$f_n(f_{n - 1}(\ldots (f_1(base_fn(x)))))$. Since an order-0 function is not defined, if \texttt{n == 0}, \texttt{apply\_n} should return the result of calling
\texttt{base\_fn} on \texttt{x}
\begin{lstlisting}
def apply_n(n, base_fn, x): 
  """ 
  >>> incr = lambda x: x + 1
  >>> double = lambda x: x * 2
	>>> identity = lambda x: x
  >>> apply_n(2, identity, 1)(double)(incr) # incr(double(identity(1)))
  3
  >>> apply_n(0, identity, 2) # identity(2)
	2
	>>> apply_n(1, double, 2)(incr) #incr(double(2))
	5
  """
	
	def take_fn(f):
	    
			return apply_n(_______________________________________)
			
	if n == 0:
	
	    return ___________________________
			
	else:
	
	    return ___________________________
	    
			

				
\end{lstlisting}

\begin{solution}[1in]
\begin{lstlisting}:
  def take_fn(f):
	    return apply_n(n - 1, lambda x: f(base_fn(x)), x)
	if n == 0:
	    return base_fn(x)
	else:
	    return take_fn
\end{lstlisting}
\end{solution}
\end{blocksection}
