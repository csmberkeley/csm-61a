\begin{blocksection}
\question Fill in the blanks (\emph{without using any numbers in the first blank}) such that the entire expression evaluates to \texttt{9}.

\ifprintanswers\else
\begin{lstlisting}
(lambda x: lambda y: ________________)(_____)(lambda z: z*z)()
\end{lstlisting}
\vspace{40 mm}
\fi

\begin{solution}[0in]
\begin{lstlisting}
(lambda x: lambda y: lambda: y(x))(3)(lambda z: z*z)()
\end{lstlisting}
\end{solution}
\end{blocksection}

\begin{questionmeta}
  Notice the arguments passed into each function call. Use these to help students break down the nested lambdas (i.e. the variable y will be assigned to the lambda function with parameter z). For tricky skeleton problems like this, I like to tell students that they are ``detectives'' and that their mission is to 1) use the available evidence to come up with a hypothesis for how a function is supposed to work and 2) test that hypothesis by attempting to implement the function, returning to step 1 if they find their hypothesis to be incorrect. 
\end{questionmeta}
