\begin{blocksection}
\question Make a lambda function, \lstinline$make_interval()$, that takes in the upper and lower bound of an interval, and returns a function that takes in a value \lstinline$x$ and checks whether \lstinline$x$ is in the interval [\lstinline$lower$, \lstinline$upper$], inclusive.

\begin{lstlisting}
>>> make_interval = _____________________________________
>>> in_interval = make_interval(-1, 2)
>>> in_interval(0)
True
>>> in_interval(61)
False

\end{lstlisting}

\begin{solution}[1in]
\begin{lstlisting}
>>> make_interval = lambda lower, upper: lambda x: x <= upper and x >= lower
>>> in_interval = make_interval(-1, 2)
>>> in_interval(0)
True
>>> in_interval(61)
False

\end{lstlisting}
\end{solution}

\begin{blocksection}
 \begin{guide}
   \textbf{Teaching Tips}
   \begin{itemize}
   	   \item Question students exactly the function's purpose is- if they are stuck, walk them through the doctest and ask leading questions based on each one. 
   	   \item It is often easier for students to understand currying if you show the equivalent standard Python higher-order function (with def and nested functions).
   	   \item It can help to explain the first functional arguments as "setting the bounds", and then subsequent arguments as checking those bounds. 
   \end{itemize}
 \end{guide}
\end{blocksection}
 
\end{blocksection}