\begin{blocksection}
\question What are higher-order functions? Why and where do we use lambda and higher-order functions? Can you give a practical example of where we would use a HOF?

\begin{solution}[0.5in]
Higher-order functions are functions that does at least one of the following: take at least one or more functions as arguments and returns a function. 
In practice, we use lambda functions to pass code as data in a concise manner. One specific example to illustrate the use of lambdas is the optional \lstinline{key} parameter for \lstinline{min} and \lstinline{max} functions. Lambda functions can be passed as arguments to higher-order functions.
Higher order functions serve as a tool of abstraction, allowing us to simplify repeated actions into one function that we can use over and over again. 
Students can have varying answers for practice uses of HOFs, though here are some suggestions for the average student coming across this worksheet:
\begin{itemize}
    \item Our method signature is composed of one parameter, but we wish to use a higher order function with more parameters to abstract extra steps.
    \item When our function is long and complex; easier to read code when it's organized into several different higher order functions.
\end{itemize}

\end{solution}
\end{blocksection}