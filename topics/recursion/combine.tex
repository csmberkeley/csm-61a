\begin{blocksection}
\question (Spring 2015 MT1 Q3C) Implement the \lstinline$combine$ function, which takes a non-negative integer
\lstinline$n$, a two-argument function \lstinline$f$, and a number \lstinline$result$. It applies
\lstinline$f$ to the first digit of \lstinline$n$ and the result of combining the rest of the digits of \lstinline$n$
by repeatedly applying \lstinline$f$ (see the doctests). If \lstinline$n$ has no digits (because it is zero),
\lstinline$combine$ returns \lstinline$result$.

\begin{lstlisting}
def combine(n, f, result):
    """
    Combine the digits in non-negative integer n using f.
    
    >>> combine(3, mul, 2) # mul(3, 2)
    6
    >>> combine(43, mul, 2) # mul(4, mul(3, 2))
    24
    >>> combine(6502, add, 3) # add(6, add(5, add(0, add(2, 3))))
    16
    >>> combine(239, pow, 0) # pow(2, pow(3, pow(9, 0))))
    8
    """
    if n == 0:
        return result
    else:
        return combine(_______ , _______ , __________________________)
    
\end{lstlisting}

\begin{solution}[1.5in]
\begin{lstlisting}
def combine(n, f, result):
    if n == 0:
        return result
    else:
        return combine(n // 10, f, f(n % 10, result))
\end{lstlisting}
\end{solution}
\end{blocksection}
