\begin{blocksection}
\textbf{Tree Recursion vs Recursion}


In most recursive problems we've seen so far, the solution function contains only one call to itself. However, some problems will require multiple recursive calls -- we colloquially call this type of recursion "tree recursion," since the propagation of function frames reminds us of the branches of a tree. "Tree recursive" or not, these problems are still solved the same way as those requiring a single function call: a base case, the recursive leap of faith on a subproblem, and solving the original problem with the solution to our subproblems. The difference? We simply may need to use multiple subproblems to solve our original problem.

Tree recursion will often be needed when solving counting problems (how many ways are there of doing something?) and optimization problems (what is the maximum or minimum number of ways of doing something?), but remember there are all sorts of problems that may need multiple recursive calls! Always come back to the recursive leap of faith.

\end{blocksection}

\begin{guide}
\begin{blocksection}
\textbf{Teaching Tips}
\begin{itemize}
    \item Stress the power of tree recursion: it lets us find a single solution among 14,000,605 futures.
    \item Try dividing tree recursion questions into three parts: base cases, recursive calls, and combining recursive calls.
    \begin{enumerate}
        \item What are the simplest possible arguments for the function?
        \begin{itemize}
            \item There may be hints for base cases in doc tests. Run through simple examples!
        \end{itemize}
        \item What options should be recursively explored?
        \begin{itemize}
            \item Drawing tree diagrams can help a lot for this section.
        \end{itemize}
        \item How should the answers of subproblems be combined?
        \begin{itemize}
            \item Trust recursive calls to return the correct values (recursive leap of faith!) and combine them with mathematical or logical operators.
        \end{itemize}
    \end{enumerate}
\end{itemize}
\end{blocksection}
\end{guide}
