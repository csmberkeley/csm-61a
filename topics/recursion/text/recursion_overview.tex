\textbf{There are three steps to writing a recursive function:}
\begin{enumerate}
    \item Create base case(s)
    \item Reduce your problem to a smaller subproblem and call your function recursively on the smaller subproblem
    \item Figure out how to get from the smaller subproblem back to the larger problem
\end{enumerate}

\vspace{3mm}

\textbf{Real World Analogy for Recursion}


Imagine that you’re in line for boba, but the line is really long, so you want to know what position you’re in. You decide to ask the person in front of you how many people are in front of them. That way, you can take their response and add 1 to it. Now, the person in front of you is faced with the same problem that you were trying to solve, with one less person in front of them than you.  They decide to take the same approach that you did, by asking the person in front of them. This continues until the very first person in line is asked. At this point, the person at the front knows that there are 0 people in front of them, so they can tell the person behind them that there are 0 people in front.  Now, the second person can figure out that there is 1 person in front of them, and can relay that back to the person behind them, and so on, until the answer reaches you.


Looking at this example, we can see that the "recursive function" is trying to figure out how many people are in front of you.  The base case is if you are in position 1 (since then there are 0 people in front of you, and you don’t have anyone else). The recursive case is if you are in any position greater than 1, since then you can "recursively ask" the person in front of you (which is our smaller subproblem!) Finally, we can get from the smaller subproblem back to the big problem by adding 1 to the number from the person in front of you, since you have to include them in your count.