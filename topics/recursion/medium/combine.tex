\begin{blocksection}
\question Implement the function \texttt{combine}, which takes in a positive number \texttt{n}, two argument function \texttt{f}, and integer \texttt{base}. It returns the result of combining \texttt{base} and each digit of \texttt{n} using \texttt{f}. You may assume \texttt{f} is commutative (that is, it doesn't matter in which order you combine each value). \\

\begin{lstlisting}
def combine(n, f, result):
    """
    >>> combine(3, mul, 2) # mul(3, 2)
    6
    >>> combine(43, mul, 2) # mul(4, mul(3, 2))
    24
    >>> combine(650, add, 3) # add(6, add(5, add(0, 3)))
    16
    """
    if _____________________________________:
        ____________________________________
    else:
        ____________________________________

				
\end{lstlisting}

\begin{solution}[1in]
\begin{lstlisting}:
    if n == 0:
        return result
    else:
        return combine(n // 10, f, f(n % 10, result))
\end{lstlisting}
\end{solution}
\end{blocksection}
