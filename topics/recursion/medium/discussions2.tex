\begin{blocksection}
    \question Realizing the need for improvement, Tyler has recruited you to help them make 61A sections more flexible! 
    Tyler would like discussion sections to have 20 $\leq$ \lstinline$x$ $\leq$ 30 students each, 
    and tutoring sections to have 6 $\leq$ \lstinline$y$ $\leq$ 8 students. 
    Additionally, it's okay to have up to \lstinline$upper$ total slots, as long as we have at least \lstinline$lower$ amount to accomodate all students.
    Is it possible to assign each student a section? (Note: In this alternate universe, students can choose either
    a tutoring section or a discussion section, but not both.)
    
    \begin{lstlisting}
    def sum_range(lower, upper):
        """
        >>> sum_range(25, 30) # Everyone can go into one discussion section
        True
        >>> sum_range(9, 10) # If we make a tutoring section, there will be 1-4 extra students
        False
        >>> sum_range(56, 64) # 2 discussion sections, 2 discussions 1 tutoring section, etc. all work
        True
        """
        def discussions(pmin, pmax):
            if ____________________________________________:
    
                return ____________________________________
    
            elif __________________________________________:
    
                return ____________________________________
    
            return ________________________________________
    
        return discussions(0, 0)
    
    \end{lstlisting}
    \end{blocksection}
    
    \begin{solution}[1.5in]
    \begin{blocksection}
    \begin{lstlisting}
    def sum_range(lower, upper):
        def discussions(pmin, pmax):
            if (pmin >= lower and pmin <= upper) or (pmax >= lower and pmax <= upper):
                return True
            elif pmin > upper:
                return False
            else:
                return discussions(pmin + 6, pmax + 8) or discussions(pmin + 20, pmax + 30)
        return discussions(0, 0)
    \end{lstlisting}
    
    % \textbf{(Solution continues on the next page)}
    \end{blocksection}
    \end{solution}
    
    \begin{solution}
    \begin{blocksection}
    
    This question is similar to the last one but now we have to deal with two parameters \lstinline{pmin} and \lstinline{pmax}. 
    Let's start with our recursive calls. Each call represents using either a discussion or tutoring section and 
    returns a boolean value that indicates whether or not we can include the number of students in our desired range (set by lower and upper).
    If we use a discussion section, then our current minimum is + 20 and the maximum is whatever our current maximum is + 30. We need to take the two recursive calls, 
    which each represents a choice we have (either add a tutoring or discussion section) and combine them in some way. 
    If one of these choices led to \lstinline{True} then we should return \lstinline{True} so we use an \lstinline{or} to combine the recursive calls.
    
    Our first base case represents successfully accomodating the number of students within our range in which case we can stop the recursion because we 
    know it is possible to print that number of pages within the range. Our second base case handles the case where we know for sure that we cannot 
    fit the amount of students within our range so we return \lstinline{False} and stop doing the recursion.
    \end{blocksection}
    \end{solution}
    
    \begin{guide}
    \begin{blocksection}
    \textbf{Teaching Tips}
    \begin{itemize}
        \item Only do this problem if you have plenty of time (it takes a long time).
        \item Go through the doctests thoroughly before attempting the problem.
        \item Suggested sequence for teaching:
        \begin{enumerate}
            \item Explain the difference between \lstinline{lower} and \lstinline{pmin}, and \lstinline{higher} and \lstinline{pmax}. In the same vein, consider what \lstinline{discussions} should return (what is its purpose?).
            \item Find the base cases (we need a True case and a False case).
            \item Think about what the helper function call means- what does seeing 0 and 0 as arguments tell us? Should we recurse downward or upward?
            \item How do we factor in all the different possible combinations of the two discussion types?
            \begin{itemize}
                \item Try to put students into 1 section.
                \begin{itemize}
                    \item If it is a discussion section, we can accomodate [20, 30] copies.
                    \item If it is a tutoring section, we can accomodate [6, 8] copies.
                \end{itemize}
                \item Try to put students into 2 section.
                \begin{itemize}
                    \item (1, 1): [20+20, 30+30].
                    \item (2, 2): [6+6, 8+8].
                    \item (1, 2) = (2, 1): [20+6, 30+8].
                \end{itemize}
                \item And so on. Guide students to find the recursive pattern.
    
            \end{itemize}
        \end{enumerate}
    \end{itemize}
    \end{blocksection}
    \end{guide}
    