\begin{blocksection}
\question
Fill in \lstinline{collapse}, which takes in a non-negative integer \lstinline{n} and returns the number resulting from removing all digits that are equal to an adjacent digit, i.e. the number has no adjacent digits that are the same.

\begin{lstlisting}
def collapse(n):
    """
    >>> collapse(12234441)
    12341
    >>> collapse(11200000013333)
    12013
    """
    rest, last = n // 10, n % 10

    if ___________________________________:

        ____________________________________

    elif _________________________________:

        ____________________________________
    else:

        ____________________________________
\end{lstlisting}
\end{blocksection}

\begin{blocksection}
\begin{solution}
\begin{lstlisting}
def collapse(n):
    rest, last = n // 10, n % 10
    if rest == 0:
        return last
    elif last == rest % 10:
        return collapse(rest)
    else:
        return collapse(rest) * 10 + last
\end{lstlisting}
\end{solution}
\end{blocksection}

\begin{blocksection}
\begin{guide}
\textbf{Teaching Tips}
\begin{itemize}
  \item Demonstrating with the doc tests is very important- the problem description can be confusing.
  \item How are we going to traverse the number?
  \begin{itemize}
    \item As always, we traverse right to left, since traversing left to right will only produce the same results for more work.
    \item Knowing this, we can figure \lstinline{rest} stands for the remaining number and \lstinline{last} stands for the last digit
  \end{itemize}
  \item What are the base cases?
  \begin{itemize}
    \item The simplest possible case is if \lstinline{n} is one digit, at which there is nothing to compare it to.
  \end{itemize}
  \item What are the recursive cases?
  \begin{itemize}
    \item Either we remove a digit or we do not.
    \item How do we structure the recursive call when we \textbf{don't} want to get rid of \lstinline{last}? We need to add it back on \textbf{after} the recursive call.
  \end{itemize}
  \item Why does this work?
  \begin{itemize}
    \item Remind students that part of the recursive leap of faith is to trust that calling \lstinline{collapse(left)} will remove identical left adjacent numbers.
    \item All they need to care about in the moment is whether or not \lstinline{last} should be removed or not.
  \end{itemize}
\end{itemize}
\end{guide}
\end{blocksection}
