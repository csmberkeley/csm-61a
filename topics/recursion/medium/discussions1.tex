% This problem is the same as copy-machine1, but adapted for online teaching (where copy machines are a figment of the past)...
\begin{blocksection}
    \question In an alternate universe, 61A software is not that good (inconceivable!). 
    Tyler is in charge of assigning students to discussion sections, but sections.cs61a.org 
    only knows how to list sections with either \lstinline{m} or \lstinline{n} number of students (the two most popular sizes). 
    Given a \lstinline{total} number of students, can Tyler create sections with only sizes of either \lstinline{m} or \lstinline{n} and not have any leftover students? 
    Return \lstinline{True} if he can and \lstinline{False} otherwise.
    
    \begin{lstlisting}
    def fit_sections(total, n, m):
        """
        >>> fit_sections(1, 3, 5)
        False
        >>> fit_sections(5, 3, 5) # 0 * 3 + 1 * 5 = 5
        True
        >>> fit_sections(11, 3, 5) # 2 * 3 + 1 * 5 = 11
        True
        >>> fit_sections(61, 11, 15) # can't express 61 as a * 11 + b * 15
        False
        """
        if _______________________________________________:
    
            return True
    
        elif __________________________________________________:
    
            return False
    
        return ___________________________________________
    
    \end{lstlisting}
    \end{blocksection}
    
    \begin{solution}[1.5in]
    \begin{blocksection}
    \begin{lstlisting}
    def fit_sections(total, n, m):
        if total == 0:
            return True
        elif total < 0: # you could also put total < min(m, n)
            return False
        return fit_sections(total - n, n, m) or fit_sections(total - m, n, m)
    \end{lstlisting}
    An alternate solution you could write that may be slightly faster in certain cases:
    \begin{lstlisting}
    def fit_sections(total, n, m):
        if total == 0 or total % n == 0 or total % m == 0:
            return True
        elif total < 0: # you could also put total < min(m, n)
            return False
        return fit_sections(total - n, n, m) or fit_sections(total - m, n, m)
    \end{lstlisting}
    
    \textbf{(Solution continues on the next page)}
    \end{blocksection}
    \end{solution}
    
    \begin{solution}
    \begin{blocksection}
    When thinking about the recursive calls, we need to think about how each step of the problem works. Tree recursion allows us to explore the two options we have: 
    either create a new \lstinline{m}-person discussion at this step or create a new \lstinline{n}-person discussion at this step and can combine the results after exploring both options. 
    Inside the recursive call for \lstinline{fit_sections(total - n, n, m)}, which represents accommodating \lstinline{n} students, we again consider adding either
    \lstinline{n} or \lstinline{m} students to the next section.
    
    Once we have these recursive calls we need to think about how to put them together. We know the return should be a boolean so we want to use either \lstinline{and} or \lstinline{or} to combine the values for a final result. Given that we only need one of the calls to work, we can use \lstinline{or} to reach our final answer.
    
    In the base cases we also need to make sure we return the correct data type. Given that the final return should be a boolean we want to return booleans in the base cases.
    
    Another alternate base case would be: \lstinline{total == 0 or total % n == 0 or total % m == 0}. This solution would also work! You would just be stopping the recursion early, since the total can be a multiple of \lstinline{n} or \lstinline{m} in order to trigger the base case - it doesn't have to be 0 anymore. Just be sure to still include the \lstinline{total == 0} check, just in case someone inputs 0 as the total into the function.
    \end{blocksection}
    \end{solution}
    
    \begin{guide}
    \begin{blocksection}
    \textbf{Teaching Tips}
    \begin{itemize}
        \item Some leading questions:
        \begin{itemize}
            \item What are the base cases (when to return True/False)?
            \item How can we reduce this problem into smaller subproblems (recursive step)?
            \item What does the value of a recursive call tell us?
            \item How can we put recursive calls together to get a final answer?
        \end{itemize}
        \item Ask students about the simplest possible cases to identify base cases; make sure they realize \lstinline{(total == n)} or \lstinline{(total == m)}  is incorrect because \lstinline{(total == 0)} is a simpler True case.
        \item Point out the fact that tree recursion problems usually have you consider multiple ``options'' or ``possibilities'', and they should all be explored when you are writing your recursive cases.
        \item Out of all possible combinations of \lstinline{n} and \lstinline{m}, we only need 1 way for \lstinline{n} and \lstinline{m} to sum to the total for the function to return True, which implies \lstinline{or} is an appropriate way to aggregate our recursive calls.
    \end{itemize}
    \end{blocksection}
    \end{guide}
    