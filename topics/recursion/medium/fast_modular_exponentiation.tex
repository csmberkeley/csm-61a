\begin{blocksection}
    \question \textbf{Fast Modular Exponentiation:} In many computing applications, we need to quickly compute $n^{x} \mod z $ where $n > 0$, and $x$ and $z$ are arbitary whole numbers. Computing $n^{x} \mod z$ for large numbers can get extremely slow if we repeatedly multiply $n$ for $x$ times. We can implement the following recursive algorithm to help us speed up the exponentiation operation.

    \begin{gather*}
        x^{n} \mod z = 
        \begin{cases}
            x * (x^{2})^{(n-1)/2} \, \% \, z & \text{if $n$ is odd}\\
            (x^{2})^{(n/2)} \, \% \, z & \text{if $n$ is even}
        \end{cases}
    \end{gather*}

    This is an example of a "divide \& conquer" algorithm and follows the same train of thought as tree-recursion problems (you are dividing some complex problem into smaller parts and performing both options).
    
    \begin{lstlisting}
def modular_exponentiation(base, exponent, modulus):
    """
    >>> modular_exponentiation(2, 2, 2)
    0
    >>> modular_exponentiation(4, 2, 3)
    1
    """
    if _____________________:

        return ____________________________________

    if _____________________:
            
        half_power = ____________________________________
        # Hint: Which math formula above has exponent *just* divided by half?

        return ____________________________________ % modulus

    else:  

        half_power = ____________________________________

        return _________________________________________ % modulus
    \end{lstlisting}

    Note: The algorithm you just implemented is a key part of modern day cryptography techniques such as RSA and Diffie-Hellman key exchange. In some cases, the exact operations you just implemented is used in modern day, state of the art, programs (if you are curious, Google "Right-to-left binary method"). You will learn more about RSA in CS70. If you want to learn more about computer security, consider taking CS161 after CS61C.
    \end{blocksection}
    
    \begin{blocksection}
    \begin{solution}
    \begin{lstlisting}
    def modular_exponentiation(base, exponent, modulus):
    # Base case: exponent is 0
    if exponent == 0:
        return 1
    
    # Recursive case
    if exponent % 2 == 0: # If exponent is even
        half_power = modular_exponentiation(base, exponent // 2,modulus)
        return (half_power * half_power) % modulus
    else: # If exponent is odd
        half_power = modular_exponentiation(base, (exponent - 1) // 2, modulus)
        return (base * half_power * half_power) % modulus
    \end{lstlisting}
    \end{solution}
    \end{blocksection}
    
    \begin{blocksection}
    \begin{guide}
    \begin{itemize}
      \item This problem doesn't exactly follow the scheme of tree recursion discussed in lecture, but it does follow the general philosophy behind tree recursion. You probably want to reiterate this.
      \subitem I think it is important for students to relate "tree recursion" which is a SICP concept to something in the real world (whether that'd be a type of \st{leetcode} algorithms problem or a real world use). Here real world use made more sense as a lot of algorithms problems with tree recursion involved binary trees (and trees in general), which hasn't been introduced in lecture yet. 
      \item This problem is not meant to be tricky or as much of a "drill" problem as the other ones. This is more of a "here is what tree recursion can do" type -- so feel free to walk through this problem if you want.
    \end{itemize}
    \end{guide}
    \end{blocksection}
    