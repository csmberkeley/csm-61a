\begin{blocksection}
    \question Write a function \lstinline{selective_sum}, which takes in an integer \lstinline{n} and a predicate function \lstinline{cond}. \lstinline{selective_sum} returns the sum of all positive integers up to \lstinline{n} for which \lstinline{cond(n)} is true. 
    
    \begin{lstlisting}
def selective_sum(n, cond):
    """
    >>> is_odd = lambda x: x % 2 == 1
    >>> selective_sum(5, is_odd) # 5 + 3 + 1 = 9
    9
    >>> bigger_than_10 = lambda x: x > 10
    >>> selective_sum(13, bigger_than_10) # 13 + 12 + 11 = 36
    36
    >>> selective_sum(-1, is_odd) # no positive integers <= 1
    0
    """
    if _______________________________________________:

        return _______________________________________________

    if _______________________________________________:

        return _______________________________________________

    return _______________________________________________
    \end{lstlisting}

    \begin{solution}[0.5in]
    \begin{lstlisting}
def selective_sum(n, cond):
    if n == 0:
        return 0
    if cond(n):
        return n + selective_sum(n - 1, cond)
    return selective_sum(n - 1, cond)
    \end{lstlisting}
    \end{solution}
    \begin{questionmeta}
        \textbf{Teaching Tips}
        \begin{itemize}
            \item Make sure to explain what predicate functions are (functions that return either True or False).
            \item Start with the base case! If there are no numbers that satisfy \lstinline{cond} what do we return?
            \item Then move to the recursive step. How do we keep only the numbers that satisfy \lstinline{cond}?
        \end{itemize}
      \end{questionmeta}

    \end{blocksection}
    