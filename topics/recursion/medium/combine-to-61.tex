\begin{blocksection}
\question
Fill in \lstinline{combine_to_61}, which takes in a list of positive integers and returns \lstinline{True} if a contiguous sublist (i.e. a sublist of adjacent elements) combine to 61. You can \textbf{combine} two adjacent elements by either summing them or multiplying them together. If there is no combination of summing and multiplying that equals 61, return \lstinline{False}.

\begin{lstlisting}
def combine_to_61(lst):
    """
    >>> combine_to_61([3, 4, 5])
    False # no combination will produce 61
    >>> combine_to_61([2, 6, 10, 1, 3])
    True # 61 = 6 * 10 + 1
    >>> combine_to_61([2, 6, 3, 10, 1])
    False # elements must be contiguous
    """

    def helper(lst, num_so_far):

        if _______________________________:
            return True

        elif _____________________________:
            return False

        with_sum = _________________________ and \

            helper(________________, __________________)

        with_mul = _________________________ and \

            helper(________________, __________________)
        return with_sum or with_mul

    return _____________________________
\end{lstlisting}
\end{blocksection}

\begin{blocksection}
\begin{solution}
\begin{lstlisting}
def combine_to_61(lst): 
    def helper(lst, num_so_far): 	
        if num_so_far == 61: 
            return True 
        elif not lst: 
            return False 
        with_sum = num_so_far + lst[0] <= 61 and helper(lst[1:], num_so_far + lst[0]) 
        with_mul =  num_so_far * lst[0] <= 61 and helper(lst[1:], num_so_far * lst[0]) 
        return with_sum or with_mul 
    return helper(lst, 0)
\end{lstlisting}
\end{solution}
\end{blocksection}