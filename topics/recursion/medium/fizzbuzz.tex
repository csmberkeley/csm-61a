\begin{blocksection}
\question Implement a recursive version of \lstinline$fizzbuzz$.

\begin{lstlisting}
def fizzbuzz(n):
    """Prints the numbers from 1 to n. If the number is divisible by 3, it
    instead prints 'fizz'. If the number is divisible by 5, it instead prints
    'buzz'. If the number is divisible by both, it prints 'fizzbuzz'. You must do this recursively!

    >>> fizzbuzz(15)
    1
    2
    fizz
    4
    buzz
    fizz
    7
    8
    fizz
    buzz
    11
    fizz
    13
    14
    fizzbuzz
    """
\end{lstlisting}

\begin{solution}[1.5in]
\begin{lstlisting}
    if n == 1:
        print(n)
    else:
        fizzbuzz(n - 1)
        if n % 3 == 0 and n % 5 == 0:
            print('fizzbuzz')
        elif n % 3 == 0:
            print('fizz')
        elif n % 5 == 0:
            print('buzz')
        else:
            print(n)
\end{lstlisting}
\end{solution}
\end{blocksection}

\begin{questionmeta}
    It may be beneficial to reiterate the recursive leap of faith! Hypothetically, if fizzbuzz works for n, what would fizzbuzz(n - 1) output?
  \end{questionmeta}
