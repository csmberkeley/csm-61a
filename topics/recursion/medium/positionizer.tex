\question
Suppose the \textbf{position} of the rightmost digit of a number is defined as 1. The position of a digit increases from right to left.
\begin{parts}
\part
Fill in the \lstinline{positionizer} function below. It takes in a number, and for each digit, either changes digit \lstinline{d} to be the remainder of \lstinline{d} divided by its position, or leaves \lstinline{d} the same if it is equal to its position. 

\begin{lstlisting}
def positionizer(n):
   """
   >>> positionizer(12)
   10
   >>> positionizer(23)
   20
   >>> positionizer(12345)
   12300
   """
   def helper(n, pos):
      if ___________________________________:
        return _____________________________
      rest = _______________________________
      if n % 10 == pos:
        return rest + ______________________
      else:
        return rest + ______________________
   return helper(_________, __________)
\end{lstlisting}
\begin{solution}
\begin{lstlisting}:
def positionizer(n):
  def helper(n, pos):
    if n == 0:
      return 0
    rest = helper(n // 10, pos + 1) * 10
    if n % 10 == pos:
      return rest + n % 10
    else:
      return rest + (n % 10) % pos
  return helper(n, 1)
\end{lstlisting}
\end{solution}
\newpage

\part
Now fill in the \lstinline{max_positionizer} function below. Given a list of digits, it returns the k-digit number with the highest value when positionized. The digits must stay in the same order. (ex: if \lstinline{k=2}, and \lstinline{lst = [1,2,3]}, you can't form 21). You may use \lstinline{positionizer} in your solution.

\textit{Hint: Use the helper to create a list of all possible k digit numbers formed from \lstinline{lst}.}

\begin{lstlisting}
def max_positionizer(k, lst):
  """
  >>> max_positionizer(2, [1, 2, 3]) # positionized version 
                                     # of 12, 13, 23 are 
                                     # 10, 10, 20 
  23
  >>> max_positionizer(3, [2, 5, 3, 1])
  251
  """
  def make_nums(k,lst):
    if ________________________________:
      return __________________________
    elif ______________________________:
      return []
    a =  [_______________________ \
              for rest in ____________________]
    b = _______________________________________
    return a + b
  return _______(make_nums(____, ____), ____________)
\end{lstlisting}
\begin{solution}
\begin{lstlisting}
def max_positionizer(k, lst):
  def make_nums(k,lst):
    if k == 0: # Note that the check for k must come first 
               # (what should be returned if k == 0 and 
               # len(lst) == 0?)
      return [0]
    elif len(lst) == 0:
      return []
    a =  [lst[0] * 10**(k - 1) + rest \
            for rest in make_nums(k - 1, lst[1:])]
    b = make_nums(k, lst[1:])
    return a + b
  return max(make_nums(k, lst), key=positionizer)
\end{lstlisting}
\end{solution}
\end{parts}