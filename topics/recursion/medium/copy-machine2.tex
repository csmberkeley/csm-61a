\begin{blocksection}
\question The next day, the printers break down even more! Each time they are used, the first printer
prints a random \lstinline$x$ copies 50 $\leq$ \lstinline$x$ $\leq$ 60, and the second printer prints a
random \lstinline$y$ copies 130 $\leq$ \lstinline$y$ $\leq$ 140. James also relaxes his expectations: he's
satisfied as long as there's at least \lstinline$lower$ copies so there are enough for everyone, but no
more than \lstinline$upper$ copies to prevent waste. \\

\begin{lstlisting}
def sum_range(lower, upper):
    """
    >>> sum_range(45, 60) # Printer 1 prints within this range
    True
    >>> sum_range(40, 55) # Printer 1 can print a number 50-60
    False
    >>> sum_range(170, 201) # Printer 1 + 2 will print between 180 and 200 copies total
    True
    """
    def copies(pmin, pmax):
        if ________________________________________________:

            return ____________________________________

        elif ______________________________________________:

            return ____________________________________

        return ____________________________________________

    return copies(0, 0)

\end{lstlisting}

\begin{solution}[1.5in]
\begin{lstlisting}
def sum_range(lower, upper):
    def copies(pmin, pmax):
        if lower <= pmin and pmax <= upper:
            return True
        elif upper < pmin:
            return False
        return copies(pmin + 50, pmax + 60) or copies(pmin + 130, pmax + 140)
    return copies(0, 0)
\end{lstlisting}

\textbf{(Solution continues on the next page)}
\end{solution}
\end{blocksection}

\begin{blocksection}
\begin{solution}

This question is similar to the last one but now we have to deal with two parameters \lstinline{pmin} and \lstinline{pmax}. Let's start with our recursive calls. Each call represents using one printer and returns a boolean value that indicates whether or not we can create the number of copies in our desired range (set by lower and upper) if we decide to use that speicfic printer at this point: if we use the first printer then the minimum pages we can print is whatever our current minimum is + 50 and the maximum pages we can print is whatever our current maximum is + 60. We need to take the two recursive calls, which each represents a choice we have (either use the first printer or use the second printer) and combine them in some way. If one of these choices led to \lstinline{True} then we should return \lstinline{True} so we use an \lstinline{or} to combine the recursive calls.

Our first base case represents successfully printing the number of pages within our range in which case we can stop the recursion because we know it is possible to print that number of pages within the range. Our second base case handles the case where we know for sure that we cannot print the number of pages within our range so we return \lstinline{False} and stop doing the recursion.
\end{solution}
\end{blocksection}

\begin{blocksection}
\begin{guide}
\textbf{Teaching Tips}
\begin{itemize}
    \item Only do this problem if you have plenty of time (it takes a long time).
    \item Go through the doctests thoroughly before attempting the problem.
    \item Suggested sequence for teaching:
    \begin{enumerate}
        \item Explain the difference between \lstinline{lower} and \lstinline{pmin}, and \lstinline{higher} and \lstinline{pmax}. In the same vein, consider what \lstinline{copies} should return (what is its purpose?).
        \item Find the base cases (we need a True case and a False case).
        \item Think about what the helper function call means- what does seeing 0 and 0 as arguments tell us? Should we recurse downward or upward?
        \item How do we factor in all the different possible combinations of using the two printers?
        \begin{itemize}
            \item We use 1 printer.
            \begin{itemize}
                \item We use printer 1 once: it prints [40, 50] copies.
                \item We use printer 2 once: it prints [130, 140] copies.
            \end{itemize}
            \item We use 2 printers:
            \begin{itemize}
                \item (1, 1): [40+40, 50+50].
                \item (2, 2): [130+130, 140+140].
                \item (1, 2) = (2, 1): [40+130, 50+140].
            \end{itemize}
            \item And so on. Guide students to find the recursive pattern.

        \end{itemize}
    \end{enumerate}
\end{itemize}
\end{guide}
\end{blocksection}
