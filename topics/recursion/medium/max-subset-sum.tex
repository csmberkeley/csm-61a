\question
Given a list of integers \lstinline{lst}, return the maximum sum of a subset of size \lstinline{n}. If \lstinline{n} is greater than or equal to the length of \lstinline{lst}, just return the sum of the elements \lstinline{lst}.

\begin{lstlisting}
def max_subset_sum(lst, n):
    """
    >>> max_subset_sum([1, 2, 3, 4], 2)
    7
    >>> max_subset_sum([1, 4, 2, 0, 6], 3)
    12
    """

    if _________________________:

        __________________________

    elif ________________________:

        __________________________

    with_elem = _____________________________ + _________

    without_elem = ____________________________________

    return ____________________________________________
\end{lstlisting}

\begin{solution}
\begin{lstlisting}
def max_subset_sum(lst, n):
    if n == 0:
        return 0
    elif len(lst) <= n:
        return sum(lst)
    with_elem = max_subset_sum(lst[1:], n - 1) + lst[0]
    without_elem = max_subset_sum(lst[1:], n)
    return max(with_elem, without_elem)
\end{lstlisting}
\end{solution}

\begin{blocksection}
\begin{guide}
  \textbf{Teaching Tips}
  \begin{itemize}
    \item The solution to this problem is not the smartest nor the most elegant. It's a brute force approach using tree recursion.
    \item Tell students not to overthink the problem and approach it with tree recursion in mind.
    \item Drawing a tree diagram will be very helpful for understanding recursive calls.
    \item What are the base cases?
    \begin{itemize}
      \item The simplest possible case is if \lstinline{n = 0}, and there is nothing to sum.
      \item Another case (read the spec carefully!) is if \lstinline{n >= len(lst)}, and we return the sum of the whole list.
    \end{itemize}
    \item What are the recursive cases?
    \begin{itemize}
      \item We either choose to use an element in the subset or do not.
      \item If we use the element, then add it to the sum and subtract \lstinline{n}.
      \item Either way, we have to shrink the list by the element we're considering, since elements can only be used at most once.
    \end{itemize}
    \item How do we combine the recursive cases?
    \begin{itemize}
      \item Make sure students understand that the recursive cases are numbers by the recursive leap of faith- \lstinline{with_elem} is a number, not a function.
      \item It's in the name- we use \lstinline{max} to find the largest \lstinline{subset_sum}.
    \end{itemize}
  \end{itemize}
\end{guide}
\end{blocksection}
