\textbf{Linked lists} are a recursive data structure for representing sequences. They consist of a series of ``links,'' each of which has two attributes: \lstinline{first} and \lstinline{rest}. The \lstinline{first} attribute contains the value of the link (which can be an integer, string, list, even another linked list!). The \lstinline{rest} attribute, on the other hand, is a pointer to another link or \lstinline{Link.empty}, which is just an empty linked list. 

For example, \lstinline{Link(1, Link(2, Link(3)))} is a linked list representation of the sequence $1, 2, 3$. 

\begin{blocksection}
Like trees, linked lists naturally lend themselves to recursive problem solving. Consider the following example, in which we double every value in linked list. We mutate the current link and then recursively double the rest. 
\begin{lstlisting}
def double_values(link): 
     if link is not Link.empty:
        link.first *= 2 # we mutate the value inside of the link
        double_val(link.rest) # we mutate the values in the rest 
                              # of the linked list
    # if the link is empty then do nothing
\end{lstlisting}
\end{blocksection}

\begin{guide}
    \textbf{Teaching Tips}
    \begin{itemize}
       \item Try to draw box and pointer diagrams.
       \item Make clear that the pointer *points* to a linked list if we have nested linked lists.
       \item Try to experiment with going over various ways to mutate and create linked lists. 
       \item We have a great visualizer on \url{https://code.cs61a.org/} where you can call draw(lst) to visualize a list! 
       \item Try using PythonTutor as well!
    \end{itemize}
 \end{guide}