\begin{blocksection}
\question What will Python output? Draw box-and-pointer diagrams to help determine this.

\begin{lstlisting}
>>> a = Link(1, Link(2, Link(3)))
\end{lstlisting}
\begin{solution}[0in]
\begin{lstlisting}
+---+---+  +---+---+  +---+---+
| 1 | --|->| 2 | --|->| 3 | / |
+---+---+  +---+---+  +---+---+
\end{lstlisting}
\end{solution}

\begin{lstlisting}
>>> a.first
\end{lstlisting}
\begin{solution}[.5in]
\begin{lstlisting}
1
\end{lstlisting}
\end{solution}

\begin{lstlisting}
>>> a.first = 5
\end{lstlisting}
\begin{solution}[0in]
\begin{lstlisting}
+---+---+  +---+---+  +---+---+
| 5 | --|->| 2 | --|->| 3 | / |
+---+---+  +---+---+  +---+---+
\end{lstlisting}
\end{solution}

\begin{lstlisting}
>>> a.first
\end{lstlisting}
\begin{solution}[.5in]
5
\end{solution}

\begin{lstlisting}
>>> a.rest.first
\end{lstlisting}
\begin{solution}[.5in]
2
\end{solution}

\begin{lstlisting}
>>> a.rest.rest.rest.rest.first
\end{lstlisting}
\begin{solution}[.5in]
Error: tuple object has no attribute rest (Link.empty has no rest)
\end{solution}
\end{blocksection}

\begin{blocksection}
\begin{lstlisting}
>>> a.rest.rest.rest = a
\end{lstlisting}
\begin{solution}[0in]
\begin{lstlisting}
   +---+---+  +---+---+  +---+---+
+->| 5 | --|->| 2 | --|->| 3 | --|--+
|  +---+---+  +---+---+  +---+---+  |
|                                   |
+-----------------------------------+
\end{lstlisting}
\end{solution}

\begin{lstlisting}
>>> a.rest.rest.rest.rest.first
\end{lstlisting}
\begin{solution}[.5in]
\begin{lstlisting}
2
\end{lstlisting}
\end{solution}

\begin{lstlisting}
>>> repr(Link(1, Link(2, Link(3, Link.empty))))
\end{lstlisting}
\begin{solution}[.5in]
\begin{lstlisting}
"Link(1, Link(2, Link(3)))"
\end{lstlisting}
\end{solution}

\begin{lstlisting}
>>> Link(1, Link(2, Link(3, Link.empty)))
\end{lstlisting}
\begin{solution}[.5in]
\begin{lstlisting}
Link(1, Link(2, Link(3)))
\end{lstlisting}
\end{solution}

\begin{lstlisting}
>>> str(Link(1, Link(2, Link(3))))
\end{lstlisting}
\begin{solution}[.5in]
\begin{lstlisting}
'<1 2 3>'
\end{lstlisting}
\end{solution}

\begin{lstlisting}
>>> print(Link(Link(1), Link(2, Link(3))))
\end{lstlisting}
\begin{solution}[.25in]
\begin{lstlisting}
<<1> 2 3>
\end{lstlisting}
\end{solution}

\end{blocksection}

\begin{guide}
   \textbf{Teaching Tips}
   \begin{itemize}
      \item For assignment statements, Python will not print anything but still have them draw out what the linked list will look like
      \item Note that we are doing mutation here, so we are actually altering the object that was created in the first assignment. 
      \begin{itemize}
          \item Some students may have minimal exposure to mutating objects so try to emphasize this and make it obvious through diagrams.
      \end{itemize}
      \item For the error, walk-through how to keep track of which rest corresponds to which object in the box and pointer diagram. **Make sure they understand why calling rest a fourth time will give us an error (look back at the class definition)**
      \begin{itemize}
          \item Abstraction:
          \begin{itemize}
               \item our last .rest is set to Link.empty
               \item Link.empty is not a Link objects — they do not have a .rest attribute
           \end{itemize}
           \item Actual implementation:
           \begin{itemize}
               \item our last .rest is set to Link.empty
               \item Link.empty is not a Link objects — they do not have a .rest attribute
           \end{itemize}
      \end{itemize}
      \item Reassigning the last .rest to point back at the front always trips students up. 
      \begin{itemize}
       \item Make it clear that a is a pointer that points to the linked list. So we are trying to assign the last rest of a to point at what a points to, which is the beginning of the list. **To test their understanding ask what would be different if we instead had**:
       \begin{itemize}
            \item a.rest.rest.rest = a.rest
       \end{itemize}
       \item a way to explain the assignment for this problem is to emphasize the “evaluation” of the RHS and the LHS
       \item what is the value of a (a pointer). Really emphasize the implications of pointers here.        
       \item where are we putting a into? (the box that represents a.rest.rest.rest)
       \item same for a.rest.rest.rest = a.rest. what is the value of a.rest? (still a pointer!)
       \item Mention that this creates a cycle in the list
   \end{itemize}
   \end{itemize}
\end{guide}