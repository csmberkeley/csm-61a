\begin{blocksection}
\question Now write function \texttt{skip} by mutating the original list, instead of returning a new list. Do NOT call the \texttt{Link} constructor.

\begin{lstlisting}
def skip(lst):
    """
    >>> a = Link(1, Link(2, Link(3, Link(4))))
    >>> skip(a)
    >>> a
    Link(1, Link(3))
    """
\end{lstlisting}

\begin{solution}[0in]
\begin{lstlisting}
def skip(lst): # Recursively
    if lst is Link.empty or lst.rest is Link.empty:
        return
    lst.rest = lst.rest.rest
    skip(lst.rest)

def skip(lst): # Iteratively
    while lst is not Link.empty and lst.rest is not Link.empty:
        lst.rest = lst.rest.rest
        lst = lst.rest
\end{lstlisting}
Because this problem is mutative, we should never be creating a new list - we should never have \lstinline{Link(x)}, or the creation of a new Link instance, anywhere in our code! Instead, we’ll be reassigning \lstinline{lst.rest}.

In order to skip a node, we can assign \lstinline{lst.rest = lst.rest.rest}. If we have lst assigned to a link list that looks like the following:
\begin{lstlisting}
1 -> 2 -> 3 -> 4 -> 5
\end{lstlisting}
Setting \lstinline{lst.rest = lst.rest.rest} will take the arrow that points form 1 to 2 and change it to point from 1 to 3. We can see this by evaluating \lstinline{lst.rest.rest}. \lstinline{lst.rest} is the arrow that comes from 1, and \lstinline{lst.rest.rest} is the link with 3.

Once we’ve created the following list:
\begin{lstlisting}
1 -> 3 -> 4 -> 5
\end{lstlisting}
we just need to call skip on the rest of the list. If we call skip on the list that starts at 3, we’ll skip over the link with 4 and set the pointer from 3 to point to the link with 5. This is the behavior that we want! Therefore, our recursive call is \lstinline{skip(lst.rest)}, since \lstinline{lst.rest} is now the link that contains 3.

\end{solution}
\end{blocksection}

\begin{blocksection}
\begin{guide}
    \textbf{Teaching Tips}
    \begin{itemize}
       \item Make sure they understand when we are mutating and when we are creating a new linked list
       \item Draw box-and-pointer diagrams!
       \item Look for “patterns” or repeated work while you work with your box-and-pointer diagram that you can abstract away with your recursive call.
       \item Sometimes it is easier to write the recursive call before doing the base cases
       \item I usually write the recursive call and then see what could "break"
       \begin{itemize}
           \item If we access lst.first at any point, we have to make sure that lst exists
           \item If we access lst.rest.rest at any point we have to make sure that lst.rest exists
           \item What errors would we get if we didn't ensure these conditions?
       \end{itemize}
    \end{itemize}
 \end{guide}
\end{blocksection}
