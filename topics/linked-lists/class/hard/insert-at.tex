\begin{blocksection}
    \question Write a recursive function \lstinline{insert_at} that takes as input three parameters, two linked lists, \lstinline{s}  and \lstinline{x}, and an index \lstinline{index}.  \lstinline{insert_at} should return a new Linked list with \lstinline{x} inserted at \lstinline{x} of \lstinline{s}. Assume that \lstinline{index} will be a non-negative integer.

    
    \begin{lstlisting}
    def insert_at(s, x, index):
        """
        >>> insert = Link(3, Link(4))
        >>> original = Link(1, Link(2, Link(5)))
        >>> insert_at(original, insert, 2)
        Link(1, Link(2, Link(3, Link(4, Link(5)))))
        >>> start = Link(1)
        >>> insert_at(original, start, 0)
        Link(1, Link(1, Link(2, Link(5))))
        """
        if ___________________ and ___________________:

            ___________________________________________

        if ___________________ and ___________________:

            ___________________________________________

        _______________________________________________
    
    \end{lstlisting}
\end{blocksection}
\begin{solution}[0.6in]
\begin{lstlisting}
    def insert_at(s, x, index):
        """
        >>> insert = Link(3, Link(4))
        >>> original = Link(1, Link(2, Link(5)))
        >>> insert_at(original, insert, 2)
        Link(1, Link(2, Link(3, Link(4, Link(5)))))
        >>> start = Link(1)
        >>> insert_at(original, start, 0)
        Link(1, Link(1, Link(2, Link(5))))
        >>> insert_at(original, insert, 3)
        Link(1, Link(2, Link(5, Link(3, Link(4)))))
        """
        if s is Link.empty and x is Link.empty:
            return Link.empty
        if x is not Link.empty and index == 0:
            return Link(x.first, insert_at(s, x.rest, 0))
        return Link(s.first, insert_at(s.rest, x, index - 1))
\end{lstlisting}
    Because this problem is returning a new Linked List, all of our recursive calls should be a new Linked List. 

    We know that we are done inserting our Linked List \lstinline{x} into \lstinline{s} when both of the Linked Lists are empty. The reason why we have to check both of them is because we could potentially insert \lstinline{x} at the end of our original Linked List, \lstinline{s}, so \lstinline{s} would \lstinline{s} be \lstinline{Link.empty}, but we would still have the procedure of inserting \lstinline{x}. There may also be a time where we insert \lstinline{x} into the middle of \lstinline{s}, and although \lstinline{x} may be empty, we still want to keep the remaining elements of \lstinline{s}

    If we are inserting \lstinline{x} at an \lstinline{index} that is not 0, then we know we have to get to the \lstinline{index} before we start inserting \lstinline{x}. In order to do that, we can maintain the current \lstinline{Link.first} element of \lstinline{s}, and decrement \lstinline{index}, which explains the default recursive call.

    Once we have reached our desired \lstinline{index}, then we know we can start to insert \lstinline{x}. At this point our Linked List should start including elements of \lstinline{x}, and we can ensure that we hit this case, by keeping \lstinline{index} as 0 in the next recursive call. However, once \lstinline{x} is empty, we want to default back to our last recursive call.
    
\end{solution}

\begin{questionmeta}
    Despite being a few lines and being a familiar concept with lists, I've found that this problem is quite difficult, so one thing I would emphasize is to draw out this problem with a box-and-pointer diagram and illustrate the different steps of our function. Illustrate how the function works for the doctests, which should cover all possible cases of inserting a new Linked List into the beginning, middle, and end of the original Linked List
    
    If students are lost, which they most likely will be, here are some leading questions you could ask: 
    \begin{itemize}
        \item When do we know that we are done inserting items into the list?
        \item What should the parameters be equal to if we are going to start inserting \lstinline{x}, what if we are not currently inserting \lstinline{x}?
        \item How do ensure to add all elements of \lstinline{x} into \lstinline{x}?
    \end{itemize}
    
    \end{questionmeta}
    
    