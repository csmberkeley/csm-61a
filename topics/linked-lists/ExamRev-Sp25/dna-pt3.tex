%Linked List --> Class --> Hard


\begin{blocksection}
    \question A frameshift mutation causes a DNA strand to shift by n nucleotides. For example, if the original DNA strand is ATTGCGA, the strand mutated by two nucleotides would be TGCGA. 
    
    
    Implement \textit{findFrameShift}, which takes in two linked lists \textit{original} and \textit{mutated} that each represent a DNA strand. It returns the number of nucleotides that \textit{original} has been shifted by. You can use the \textit{isEqual} function. Assume the length of \textit{original} is greater than the length of \textit{mutated}. 

    
    
    \begin{lstlisting}
        def findFrameShift(original, mutated):
        """Return the number of nucleotides that original has been shifted by after being mutated
        >>> o = Link("C", Link("A", Link("C", Link("G", Link("T", Link ("A")))))) 
         <C A C G T A>
        >>> m = Link("C", Link("G", Link("T", Link ("A")))) 
         <C G T A>
        >>> n = findFrameshift(o,m)
        >>> print(n)
        2
        """
        assert isinstance(original, Link)
        assert isinstance(mutated, Link)
        	
        
    \end{lstlisting}
    
    \begin{solution}
        def findFrameShift(original, mutated):
            shift = 0
            while mutated is not Link.empty:
                if isEqual(original, mutated):
                    return shift
                mutated = mutated.rest
                shift += 1
            return 0

    \end{solution}
    \end{blocksection}
    
    \begin{questionmeta}
        This question mirrors a lot of exam questions that use a helper function from a previous question part. If students are struggling, point them towards using an iterative approach. 
    \end{questionmeta}
    