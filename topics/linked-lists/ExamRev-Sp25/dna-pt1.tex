\begin{blocksection}
\question DNA carries the genetic instructions that enable the functioning of many living creatures, including us. The bases of a DNA sequence include adenine (A), guanine (G), cytosine (C), and thymine (T). Adenine (A) pairs with thymine (T), and guanine (G) pairs with cytosine (C).

Let us represent DNA as a linked list with values representing A, G, C, and T.

Implement \texttt{reverse}, which takes in a linked list \texttt{strand} that represents a DNA strand. It destructively alters the linked list to reverse it. This function does not return anything.

\begin{lstlisting}
def reverse(strand):
    """Reverses a DNA strand 
    >>> d = Link("C", Link("A", Link("C", Link("G")))) \# <C A C G>
    >>> reverse(d)
    >>> print(d)
    <G C A C>
    """
    assert isinstance(strand, Link)
    if ______________:
        return _____
    reverse(___________)
    __________________
    __________________
    return strand
\end{lstlisting}

\begin{solution}
\begin{lstlisting}
def reverse(strand):
    """Reverses a DNA strand 
    >>> d = Link("C", Link("A", Link("C", Link("G"))))  \# <C A C G>
    >>> reverse(d)
    >>> print(d)
    <G C A C>
    """
    assert isinstance(strand, Link)
    if strand is Link.empty or strand.rest is Link.empty:
        return strand
    reverse(strand.rest)
    strand.rest.rest = strand
    strand.rest = Link.empty
    return strand
\end{lstlisting}
\end{solution}

\begin{questionmeta}
Through this question, I hope to first introduce the theme of the next few questions of the worksheet. I used DNA as I thought it is a nice way of modeling real-world phenomena with a concept that people have learned.

This question requires students to have a good understanding of working with linked list pointers, for example the difference between .rest.rest and rest or when you can set a Link.empty.

This also shows how recursion can be integrated into a linked list question. I tried to add blanks in a manner that mirrors the 61a exams, as the blanks try to guide the student towards a particular solution.
\end{questionmeta}
\end{blocksection}