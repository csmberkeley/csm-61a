\question 
The total weight of a tree is defined as the sum of the labels of all its nodes. A tree is defined to be \lstinline{equally_weighted} if the total weight of each of its branches are equal. A leaf is assumed to be equally weighted.

Complete the following functions for \lstinline{equally_weighted} and \lstinline{num_eq_weight}. You may use the below definition of the \lstinline{total_weight} function in your answer.

\begin{lstlisting}
def total_weight(t):
  """
  Return the total weight of a tree, i.e. the sum of all its labels.
  >>>total_weight(Tree(1, [Tree(2), Tree(3,[Tree(4)])]))
  10
  """
  weight =  t.label + sum([total_weight(branch) for branch in t.branches])
  return weight
\end{lstlisting}
\vspace{1\baselineskip}
\begin{parts}
\part
\begin{lstlisting}
def equally_weighted(t):
  """
  Return whether a tree is equally weighted.
  >>>equally_weighted(Tree(1))
  True
  >>>equally_weighted(Tree(1,[Tree(2), Tree(1, [Tree(1)])]))
  True
  >>>equally_weighted(Tree(0, [Tree(3), Tree(2, [Tree(3)])]))
  False
  """
  _______________________ = [_________________________]

  for _______________ in _____________________________:

    if _____________________________:

      return ______________________________________

  return __________________________________________
\end{lstlisting}

\begin{solution}
\begin{lstlisting}
def equally_weighted(t):
  all_weights = [total_weight(b) for b in t.branches]
  for weight in all_weights[1:]:
    if weight != all_weights[0]:
      return False
  return True
\end{lstlisting}
\end{solution}
\begin{guide}
\textbf{Teaching Tips}
  \begin{itemize}
    \item Important point: this function is not recursive. A tree t is considered equally weighted if and only if every branch in t.branches has equal weight.
    \item If your students are stuck, lead them with some guiding questions:
    \begin{itemize}
      \item When is the first point at which we know that a tree is/is not equally weighted?
      \item When determinining whether a tree is equally weighted or not, what values do we need to compare?
    \end{itemize}
  \end{itemize}
\end{guide}
\newpage

\part
\textbf{Note:} You are allowed to use \texttt{equally\_weighted} in this part.


\begin{lstlisting}
def num_eq_weight(t):
  """
  Return the number of equally weighted subtrees of t. Note that t is considered a subtree of itself.
  >>> num_eq_weight(Tree(1, [Tree(4), Tree(3,[Tree(1)])]))
  4
  >>> num_eq_weight(Tree(1, [Tree(9), 
                             Tree(1, [Tree(4), 
                                      Tree(3,[Tree(1)])])]))
  6
  >>> num_eq_weight(Tree(1, [Tree(8, [Tree(1)]), 
                             Tree(1, [Tree(4), 
                                      Tree(3,[Tree(1)])])]))
  7
  """
  val = ______________________________________________

  if ______________________________________:

    return ___________________________________________
  else:
    return val
\end{lstlisting}

\begin{solution}
\begin{lstlisting}
def num_eq_weight(t):
  val = sum([num_eq_weight(b) for b in t.branches])
  if equally_weighted(t):
    return 1 + val
  else:
    return val
\end{lstlisting}
\end{solution}
\begin{guide}
\textbf{Teaching Tips}
  \begin{itemize}
    \item How can we call our equally\_weighted function on all subtrees?
  \end{itemize}
\end{guide}
\end{parts}