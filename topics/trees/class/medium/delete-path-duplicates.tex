
\question
Define \texttt{delete\_path\_duplicates}, which takes in \texttt{t}, a tree
with non-negative labels. If there are any duplicate labels on any path
from root to leaf, the function should mutate the label of the occurrences
deeper in the tree (i.e. farther from the root) to be the value \texttt{-1}.

\begin{lstlisting}
def delete_path_duplicates(t):
   """
   >>> t = Tree(1, [Tree(2, [Tree(1), Tree(1)])])
   >>> delete_path_duplicates(t)
   >>> t
   Tree(1, [Tree(2, [Tree(-1), Tree(-1)])])
   >>> t2 = Tree(1, [Tree(2), Tree(2, [Tree(2, [Tree(1, Tree(5))])])])
   >>> delete_path_duplicates(t2)
   >>> t2
   Tree(1, [Tree(2), Tree(2, [Tree(-1, [Tree(-1, [Tree(5)])])])])
   """
    def helper(_______________, _______________):

        if ________________________________:

            ________________________________

        else:

            ________________________________

        for _______ in ____________________:

            ________________________________

    ________________________________________

\end{lstlisting}

\begin{blocksection}
\begin{solution}
\begin{lstlisting}
    def helper(t, seen_so_far):
        if t.label in seen_so_far:
          t.label = -1
        else:
            seen_so_far = seen_so_far + [t.label]
        for b in t.branches:
            helper(b, seen_so_far)
    helper(t, [])
\end{lstlisting}
\end{solution}
\end{blocksection}

\begin{guide}
    \textbf{Teaching Tips}
    \begin{itemize}
       \item Draw out the doctest Tree and walk through how you would delete path duplicates by hand. Then, ask your students, "how would we write this in code?"
       \item Recap with your students the core functions for trees such as label() and branches().
       \item We don't need to use the is\_leaf() function because our for loop will not run if there are no branches (which only occurs if the tree is a leaf).
    \end{itemize}
 \end{guide}

