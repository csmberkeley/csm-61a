\vspace{2mm}
\begin{itemize}
\item The constructor constructs and returns a new instance of \lstinline{Tree}
    \subitem \lstinline{t = Tree(1) # creates a Tree instance with label 1 and no branches.}
\item The \lstinline{label} and \lstinline{branches} are variables, and \lstinline{is_leaf()} is a method of the class.
    \subitem \lstinline{t.label # returns the label of the tree}
    \subitem \lstinline{t.branches # returns the branches of the tree, which is a list of trees}
    \subitem \lstinline{t.is_leaf() # returns True if the tree is a leaf}
\item A tree object is mutable
    \subitem To modify a \lstinline{Tree} object, simply reassign its attributes. For example, \lstinline{t.label = 2}.
    \subitem This means we can mutate values in the tree object instead of making a new tree that we return. In other words, we can solve tree class problems non-destructively and destructively.
\end{itemize}