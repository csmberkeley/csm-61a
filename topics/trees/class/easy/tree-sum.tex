\begin{blocksection}
\question Implement \lstinline$tree_sum$ which takes in a Tree
object and replaces the label of the tree with the sum of all the values
in the tree. \lstinline$tree_sum$ should also return the new label.

\begin{lstlisting}
def tree_sum(t):
    """
    >>> t = Tree(1, [Tree(2, [Tree(3)]), Tree(4)])
    >>> tree_sum(t)
    10
    >>> t.label
    10
    >>> t.branches[0].label
    5
    >>> t.branches[1].label
    4
    """
\end{lstlisting}

\begin{solution}[1in]
\begin{lstlisting}
    for b in t.branches:
        t.label += tree_sum(b)
    return t.label
\end{lstlisting}
\end{solution}
\end{blocksection}

\begin{guide}
    \textbf{Teaching Tips}
    \begin{itemize}
        \item Make sure students understand why an explicit is\_leaf() base case is unnecessary. If the function is called on a leaf, the for loop does not run, and it simply returns the label.
        \item The recursion occurs as part of the expression updating the label, which may confuse students at first. Explain how the returning of the label makes this work.
        \begin{itemize}
            \item It may also help to show how the code would be written without tree\_sum(b) on the right hand side of the expression to make the recursion clearer.
        \end{itemize}
        \item Consider first drawing the Tree out and running through a doctest, showing how you would sum the labels in subtrees first before updating the root label.
    \end{itemize}
\end{guide}
    
