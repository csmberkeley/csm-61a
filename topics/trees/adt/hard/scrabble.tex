\begin{blocksection}
\question A character tree is a tree where the characters along a path of the tree form a word (as defined in the English dictionary). A path through a tree is a list of adjacent node values that starts from any node and ends with a leaf value. 

Imagine you're playing a version of Scrabble and you really want to win. Implement \texttt{scrabble\char`_tree} which takes in a character tree. The function will then find all words in the character tree and return the word with the highest value. You can assume that all characters in the character tree are lower cased.

You may use the pre-defined functions \texttt{is\char`_word(word)} and \texttt{score(word)}. The function \texttt{is\char`_word(word)} returns True if word is a valid dictionary word and The function \texttt{score(word)} returns the score of word in a game of Scrabble. You do not need to worry about how these functions are implemented.

\textbf{Note}: If all characters have a weight of 1, then this problem is the same as finding the longest string of the character tree.
\end{blocksection}

\begin{parts}
\item First, implement the function \texttt{word\char`_exists}, which takes in a word \texttt{word} and a character tree \texttt{t}. The function will return \texttt{True} if characters along a path from the root of \texttt{t} to a leaf spells \texttt{word}. Otherwise, it returns \texttt{False}.

\begin{lstlisting}
def word_exists(word, t):
    if len(word) == 1:
        return _____________________________________
    elif ____________________:
        return False
    return _____(_____________________________________________)

\end{lstlisting}

\item Now, implement the function \texttt{scrabble\char`_tree}. You may use the function you defined in part a, as well as the provided functions \texttt{is\char`_word(word)} and \texttt{score}. You may also want to use the built-in Python function \texttt{filter}.

The function \texttt{filter} takes in a single argument function as its first parameter and a sequence as its second parameter. The function will then test which elements of the sequence is \texttt{True} using the provided function.

\begin{lstlisting}
>>> lst = [1, 2, 3, 4, 5, 6, 7, 8, 9, 10]
>>> evens = list(filter(lambda x: x % 2 == 0, lst))
>>> evens
[2, 4, 6, 8, 10]
\end{lstlisting}

Note: We have to call list on the output of filter because filter returns an object (which will be covered in a later part of this course).

\newpage
\begin{lstlisting}
def scrabble_tree(t):
    """
    We assume that all characters have a score of 1.

    >>> t1 = tree('h', [tree('j', [tree('i')])])
    >>> scrabble_tree(t1)
    'hi'
    >>> t2 = tree('i', [tree('l', [tree('l')])])
    >>> t3 = tree('h', [tree('i'), t2])
    >>> scrabble_tree(t3)
    'hill'
    """
    def find_all_words(t):
        if _______________:
            return _________________
        all_words = []
        for b in branches(t):
            words_in_branch = _____________________
            words_from_t = [___________________________________]
            filter_from_t = ___________________________________
            all_words = _______________________________________
        return _________________
    clean_words = [____________________________________________]
    return max(_______________, key=_________________________)

\end{lstlisting}
\end{parts}

\begin{solution}
\begin{lstlisting}
def word_exists(word, t):
    if len(word) == 1:
        return is_leaf(t) and word[0] == label(t)
    if word[0] != label(t):
        return False
    return any([word_exists(word[1:], b) for b in branches(t)])


def scrabble_tree(t):
    def find_all_words(t):
        if is_leaf(t):
            return [label(t)]
        all_words = []
        for b in branches(t):
            words_in_branch = find_all_words(b)
            words_from_t = [label(t) + w for w in words_in_branch]
            filter_from_t = list(filter(lambda w: word_exists(w, t), words_from_t))
            all_words = all_words + words_in_branch + filter_from_t
        return all_words
    clean_words = [word for word in find_all_words(t)if is_word(word)]
    return max(clean_words, key=lambda w: score(w))

\end{lstlisting}
\end{solution}
