\begin{blocksection}
\question Challenge: Write a function that returns true only if there exists a path from root to leaf that contains at least \lstinline$n$ instances of \lstinline$elem$ in a tree \lstinline$t$.

\begin{lstlisting}
def contains_n(elem, n, t):
    """
    >>> t1 = tree(1, [tree(1, [tree(2)])])
    >>> contains(1, 2, t1)
    True
    >>> contains(2, 2, t1)
    False
    >>> contains(2, 1, t1)
    True
    >>> t2 = tree(1, [tree(2), tree(1, [tree(1), tree(2)])])
    >>> contains(1, 3, t2)
    True
    >>> contains(2, 2, t2) # Not on a path
    False
    """
    if n == 0:

        return True

    elif ___________________________________________:

        return _____________________________________

    elif label(t) == elem:

        return _____________________________________

    else:

        return _____________________________________
\end{lstlisting}
\end{blocksection}

\begin{blocksection}
\begin{solution}
\begin{lstlisting}
    if n == 0:
        return True
    elif is_leaf(t):
        return n == 1 and label(t) == elem
    elif label(t) == elem:
        return True in [contains_n(elem, n - 1, b) for b in     
          branches(t)]
    else:
        return True in [contains_n(elem, n, b) for b in 
          branches(t)]
\end{lstlisting}
\end{solution}
\end{blocksection}



\begin{guide}
\begin{blocksection}
\textbf{Teaching Tips}

\begin{itemize}
\item Consider the simplest cases first
\begin{itemize} 
\item What would happen if \lstinline{n == 0}?
\item What if the tree is a leaf? What would we need to know?
\end{itemize}
\item Guide them through the recursive call.
\begin{itemize}
	\item What's the best way to iterate through all the branches? (Answer: for loop)
	\item What's difference does it make if the current value is \lstinline{elem}? How should we update \lstinline{n} in our recursive call?
	\item Make the "leap of faith": we want know if any of the subtrees contain the path we want, how can we return that?
\end{itemize}
\end{blocksection}
\end{guide}
