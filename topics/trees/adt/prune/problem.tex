\begin{blocksection}
\question Implement \lstinline$prune$, which takes in a tree \lstinline$t$ and a depth
\lstinline$k$, and should return a new tree that is a copy of only the first \lstinline$k$
levels of \lstinline$t$. Suppose \lstinline$t$ is the tree shown to the right. Then
\lstinline$prune(t, 1)$ returns nodes up to a depth of level 1.

\begin{solution}[1in]
\begin{lstlisting}
def prune(t, k):
    if k == 0:
        return tree(label(t))
    else:
        return tree(label(t), [prune(b, k - 1) for b in branches(t)])
\end{lstlisting}
\end{solution}
\end{blocksection}

\begin{guide}
    \begin{blocksection}
    \textbf{Teaching Tips}
    \begin{itemize}
        \item Emphasize what the base case is/draw it out.
        \item Emphasize how we can use a tree constructor to create the recursive case. What do we increment down by in the arguments of \lstinline$prune$ to get the base case?
        \item Tree problems lend themselves to list comprehensions. How are list comprehensions utilized in this problem?
    \end{itemize}
    \end{blocksection}
\end{guide}