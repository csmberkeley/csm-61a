\textbf{What are trees?}

A tree has a root label and a sequence of branches. Each branch of a tree is a tree. A tree with no branches is called a leaf. Any tree contained within a tree is called a sub-tree of that tree (such as a branch of a branch). The root of each sub-tree of a tree is called a node in that tree.
Trees are a recursive data abstraction, since trees have branches that are trees themselves

Because of this, it often makes sense to solve tree problems using recursion:
\begin{enumerate}
	\item Base case is often when we reach a leaf node
	\item Recursive case is often when we still need to recurse down, e.g. we haven’t hit a leaf yet. Recursive calls need to break the problem into smaller parts, which for trees often means passing in each branch as an input.
\end{enumerate}

When trying to understand and solve tree problems, it is helpful to draw out the tree.

\begin{blocksection}
	\begin{guide}
	\textbf{Teaching Tips}
	\begin{itemize}
			\item Common Misconceptions:
			\begin{itemize}
				\item Students often have trouble with the idea that branches is a list of trees.
				\begin{itemize}
					\item Try using the tree functions to build up different trees.
					\item Write out all the functions on the board and clearly define what type of thing is returned and what types of things are inputted.
				\end{itemize}
				\item Data Abstraction and Trees
				\begin{itemize}
					\item Although t[0] returns the label from the tree, students should be using label(t). This is because t is not a list, it is a tree which is an ADT!
					\item It’s important to explain why branches(t)[0] for example is not violating an abstraction barrier (because branches returns a list of trees).
				\end{itemize}
			\end{itemize}
			\item The objectives for students are to:
			\begin{itemize}
				\item Draw trees as graphical representations given Python code
				\begin{itemize}
					\item Mention to students that empty branches [] is a default argument so it doesn’t have to be written out, i.e. tree(5) is the same as tree(5, []).
					\item Emphasize variable types.
					\begin{itemize}
						\item Branches is a function that returns a list of trees.
						\item Label values are numbers.
					\end{itemize}
				\end{itemize}
				\item Construct Python code given a graphical representation of a tree
			\end{itemize}
	\end{itemize}
	\end{guide}
\end{blocksection}
