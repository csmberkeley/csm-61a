\begin{blocksection}
\question For each of the expressions in the table below, write the output displayed by the interactive Python interpreter when the expression is evaluated. The output may have multiple lines. If an error occurs, write Error. Assume that you have started python3 and executed the following statements:


\begin{lstlisting}
def quick_maths(man, free):    
    print(man)
    if man % 2 == 0 and free:
        return quick_maths(man + 2, not free)
    else:
        man -= 1
        if man == 3:
            print(man)
        else:
            return man
						
big_shaq = 2
    
\end{lstlisting}

\begin{center}
\begin{tabular}{|p{10cm}|p{5cm}|} 
\hline
\textbf{Expression} & \textbf{Interactive Output} \\ 
\hline
\rule{0pt}{3ex}
\begin{lstlisting}
print(quick_maths(big_shaq, True))
\end{lstlisting}
&  \\ 
\hline
\end{tabular}
\end{center}


\begin{solution}[1.5in]
\begin{center}
\begin{tabular}{|p{8cm}|p{6cm}|} 
\hline
\textbf{Expression} & \textbf{Interactive Output} \\ 
\hline
\rule{0pt}{4ex}
\begin{lstlisting}
print(quick_maths(big_shaq, True))
\end{lstlisting}
&  
\begin{lstlisting}
2
4
3
None
\end{lstlisting}\\ 
\hline
\end{tabular}
\end{center}

\end{solution}
\end{blocksection}