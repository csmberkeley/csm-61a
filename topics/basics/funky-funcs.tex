%%%%%%%%%%%%%%%%%%%%%%%%%%%%%%%%%%%%%%%%%%%%%%%%%%%%%%%%
% REVIEW OF PROBLEM S23
% REVIEWER: ALYSSA SMITH
% Easy question that familiarizes students with how the order of operations of functions works.
% Arguably the easier of the two basics questions.
% For future content mentors, I encourage you to add a degree of complexity to this problem, perhaps even working
% in print statements? It's up to you, but overall this problem should not be a major brain tickler as it's a "warm-up" of sorts.
%%%%%%%%%%%%%%%%%%%%%%%%%%%%%%%%%%%%%%%%%%%%%%%%%%%%%%%%
\question For the following expressions, simplify the operands in the order of evaluation of the entire expression

Example: \lstinline{add(3, mul(4, 5))}

Order of Evaluation: \lstinline{add(3, mul(4, 5))} $\rightarrow$ \lstinline{add(3, 20)} $\rightarrow$ \lstinline{23}

\begin{parts}
\part \begin{lstlisting}
add(1, mul(2, 3))
\end{lstlisting}
\begin{solution}[.5in]
\begin{lstlisting}
add(1, mul(2, 3))
add(1, 6)  
7
\end{lstlisting}
\end{solution}

\part \begin{lstlisting}
add(mul(2, 3), add(1, 4))
\end{lstlisting}
\begin{solution}[.5in]
\begin{lstlisting}
add(mul(2, 3), add(1, 4))  
add(6, add(1, 4))  
add(6, 5)  
11
\end{lstlisting}
\end{solution}

\part \begin{lstlisting}
max(mul(1, 2), add(5, 6), 3, mul(mul(3, 4), 1), 7)
\end{lstlisting}
\begin{solution}[.85in]
\begin{lstlisting}
max(mul(1, 2), add(5, 6), 3, mul(mul(3, 4), 1), 7)
max(2, add(5, 6), 3, mul(mul(3, 4), 1), 7)
max(2, 11, 3, mul(mul(3, 4), 1), 7)
max(2, 11, 3, mul(12, 1), 7)
max(2, 11, 3, 12, 7)
12
\end{lstlisting}
\end{solution}
\end{parts}
