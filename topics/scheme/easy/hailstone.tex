\begin{blocksection}
\question Define a procedure called \lstinline$hailstone$,
which takes in two numbers \lstinline$seed$ and \lstinline$n$ and returns the
\lstinline$n$th number in the hailstone sequence starting at \lstinline$seed$.
Assume the hailstone sequence starting at \lstinline$seed$ has a length of at least
\lstinline$n$. As a reminder, to get the next number in the sequence, divide by $2$ if the current number is even. Otherwise, multiply by $3$ and add $1$.

\textbf{Useful procedures}

\begin{itemize}
\item \lstinline$quotient$: floor divides, much like \lstinline$//$ in python

\lstinline$(quotient 103 10)$ outputs 10

\item \lstinline$remainder$: takes two numbers and computes the remainder of dividing the first number by the second

\lstinline$(remainder 103 10)$ outputs 3
\end{itemize}

\vspace{\baselineskip}
\begin{lstlisting}
; The hailstone sequence starting at seed = 10 would be
; 10 => 5 => 16 => 8 => 4 => 2 => 1

; Doctests
> (hailstone 10 0)
10
> (hailstone 10 1)
5
> (hailstone 10 2)
16
> (hailstone 5 1)
16

(define (hailstone seed n)










)
\end{lstlisting}
\end{blocksection}

% Fix spacing for new page
\begin{solution}[-32pt]
\begin{blocksection}
\begin{lstlisting}
(define (hailstone seed n)
    (if (= n 0)
        seed
        (if (= 0 (remainder seed 2))
            (hailstone
            (quotient seed 2)
            (- n 1))
          (hailstone
          (+ 1 (* seed 3))
          (- n 1)))))

; Alternative solution with cond

(define (hailstone seed n)
    (cond 
        ((= n 0) seed)
        ((= 0 (remainder seed 2))
          (hailstone
          (quotient seed 2)
          (- n 1)))
        (else 
          (hailstone
          (+ 1 (* seed 3))
          (- n 1)))))
\end{lstlisting}
\end{blocksection}
\end{solution}


\begin{guide}
\begin{blocksection}
Students have seen hailstone before. The goal with this problem is to get the students comfortable with Scheme by having them solve a familiar problem in an unfamiliar language. However, students may find this to be boring because they have seen it before. If this is the case, you can feel free to skip this problem. 
\vspace{10px} \\

\textbf{Teaching Tips} \\
Python version:
\begin{lstlisting}
def hailstone(seed, n):
    if n == 0:
        return seed
    if seed % 2 == 0:
        return hailstone(seed//2, n - 1)
    else:
        return hailstone(3*seed + 1, n - 1)
\end{lstlisting}
\begin{itemize}
    \item If they’re confused, point them towards the \% in Python and how they can get the same value back in Scheme (answer: remainder function) 
    \item Remind them to be careful about parentheses
    \item If you don’t use cond, how might you write an equivalent function? 
    \item In problems like this, I like to emphasize to my students how valuable it is to indent their code so that they can keep everything well organized. Scheme can be very confusing when not properly formatted because the language seems to just be an endless stream of parentheses. 
\end{itemize}
\end{blocksection}
\end{guide}