%%% Question %%%
% \begin{blocksection}
\begin{nonsol}
You can make the following analogy:
\begin{center}
\begin{tabular}{ |l|l| }
\hline
 \texttt{Link(1, Link.empty)} & \texttt{(cons 1 nil)} \\
 \texttt{a = Link(1, Link(2, Link.empty))} & \texttt{(define a (cons 1 (cons 2 nil)))}  \\
 \texttt{a.first} & \texttt{(car a)} \\
 \texttt{a.rest} & \texttt{(cdr a)} \\
 \hline
\end{tabular}

\end{center}
Draw box and pointers when appropriate. Ask your mentor if you're unsure what's going on. You aren't expected to understand this completely on your own.
\question What will Scheme output? Draw box-and-pointer diagrams to help determine this.
\end{nonsol}

\begin{lstlisting}
scm> (cons 1 (cons 2 nil))
\end{lstlisting}
\begin{solution}[0.25in]
\texttt{(1 2)}
\begin{center}
\includegraphics[scale=0.7]{scheme_lists_2}
\end{center}
\end{solution}

\begin{lstlisting}
scm> (cons 1 '(2 3 4 5))
\end{lstlisting}
\begin{solution}[0.25in]
\texttt{(1 2 3 4 5)}
\begin{center}
\includegraphics[scale=0.7]{scheme_lists_3}
When we use the quote before the list, we are saying that we should put the literal list (2 3 4 5) in the cdr of this list. So in this case we create a list where the first element (car) is 1, and the cdr is the list (2 3 4 5).
\end{center}
\end{solution}

\begin{lstlisting}
scm> (cons 1 '(2 (cons 3 nil)))
\end{lstlisting}
\begin{solution}[0.25in]
\texttt{(1 2 (cons 3 ()))}
\begin{center}
\includegraphics[scale=0.7]{scheme_lists_5}
Since we also used a quote here, we do not evaluate the \texttt{(cons 3 nil)}. We keep everything inside the quotes the same so the \texttt{cdr} of this list is the list \texttt{(2 (cons 3 nil))}. That means that we add the element 2, and then the nested list \texttt{(cons 3 nil)}.
\end{center}
\end{solution}

\begin{lstlisting}
scm> (cons 1 (2 (cons 3 nil)))
\end{lstlisting}
\begin{solution}[.25in]
\begin{lstlisting}
eval: bad function in : (2 (cons 3 nil))
\end{lstlisting}
While evaluating the operands, Scheme will try to evaluate the expression \texttt{(2 (cons 3 nil))}. Since 2 is not a valid operator, this expression Errors.
\end{solution}

\begin{lstlisting}
scm> (cons 3 (cons (cons 4 nil) nil))
\end{lstlisting}

\begin{solution}[.5in]
\lstinline$(3 (4))$
\end{solution}
% \end{blocksection}

% \begin{blocksection}
\begin{lstlisting}
scm> (define a '(1 2 3))
\end{lstlisting}
\begin{solution}[.25in]
\begin{lstlisting}
a
\end{lstlisting}
Defines a list of elements of \texttt{(1 2 3)} and binds the list to the variable \texttt{a}. Recall that define returns the name of the symbol.
\end{solution}

\begin{lstlisting}
scm> a
\end{lstlisting}
\begin{solution}[.25in]
\begin{lstlisting}
(1 2 3)
\end{lstlisting}
\includegraphics[scale=0.7]{scheme_lists_6}
\end{solution}

\begin{lstlisting}
scm> (car a)
\end{lstlisting}
\begin{solution}[.25in]
\begin{lstlisting}
1
\end{lstlisting}
\end{solution}

\begin{lstlisting}
scm> (cdr a)
\end{lstlisting}
\begin{solution}[.25in]
\begin{lstlisting}
(2 3)
\end{lstlisting}
\end{solution}

\begin{lstlisting}
scm> (cadr a)
\end{lstlisting}
\begin{solution}[.25in]
\begin{lstlisting}
2
\end{lstlisting}
Recall that cadr is short for \texttt{(car (cdr a))}. From above, we know that \texttt{(cdr a)} is \texttt{(2 3)}. From that, we can evaluate \texttt{(car (cdr a))} to 2.
\end{solution}

How can we get the 3 out of a?
\begin{solution}[.25in]
\begin{lstlisting}
(car (cdr (cdr a)))
\end{lstlisting}
To get to the pair that contains 3, we need to call \texttt{(cdr (cdr a))}. To get the element 3, we need the \texttt{car} of \texttt{(cdr (cdr a))}.
\end{solution}
% \end{blocksection}

\begin{guide}
\begin{blocksection}
\textbf{Teaching Tips}
\begin{itemize}
  \item Make sure students know \lstinline{(cadr a)} is short for \lstinline{(car (cdr a))} and \lstinline{(cddr a)} is short for \lstinline{(cdr (cdr a))}
  \item Draw diagrams or use the \href{https://code.cs61a.org/}{61A Scheme Web interpreter} for visualizing lists
  \item Encourage students to ask questions and experiment with extra \lstinline{cons}, \lstinline{car}, and \lstinline{cdr} statements to see how they change the outputs of statments!
  \item While unrelated to the problem, it may be helpful to teach students these keywords:
  \begin{itemize}
    \item\lstinline{(pair? arg)}, which checks if arg has a first and rest
    \item\lstinline{(list? arg)}, which returns true if arg is a well-formed list
  \end{itemize}
\end{itemize}
\end{blocksection}
\end{guide}
