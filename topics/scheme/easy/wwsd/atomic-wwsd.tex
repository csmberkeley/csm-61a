\begin{blocksection}
Let's practice identifying atomic expressions in Scheme.

\begin{lstlisting}
    True or False: 3.14 is an atomic expression.
\end{lstlisting}
\begin{solution}[0.25in]
\texttt{True. Floats are atomic expressions.}
\end{solution}

\begin{lstlisting}
    True or False: pi is an atomic expression.
\end{lstlisting}
\begin{solution}[0.25in]
    \texttt{False. Pi in most cases isn't naturally defined within the programming language -- rather, it's a variable name. As we haven't defined pi in this case, Scheme doesn't know what to interpret pi as, so we get an Error if we try to evaluate this on its own.}
\end{solution}

\begin{lstlisting}
    True or False: - is an atomic expression.
\end{lstlisting}
\begin{solution}[0.25in]
    \texttt{True. - is an operator for subtraction.}
\end{solution}

\begin{lstlisting}
    True or False: // is an atomic expression.
\end{lstlisting}
\begin{solution}[0.25in]
    \texttt{False. Scheme can only interpret the basic symbols of +, -, *, /. We'll have to define our own floor division function!}
\end{solution}

\begin{lstlisting}
    True or False: 'b' is an atomic expression.
\end{lstlisting}
\begin{solution}[0.25in]
    \texttt{True. b is a character, which is an atomic data type!}
\end{solution}

\begin{lstlisting}
    True or False: "is this atomic?" is an atomic expression.
\end{lstlisting}
\begin{solution}[0.25in]
    \texttt{True. Even though a string is a list of characters, Scheme still treats them as an atomic. It's important to note that in Scheme's case, strings can only be denoted with ".}
\end{solution}
\end{blocksection}