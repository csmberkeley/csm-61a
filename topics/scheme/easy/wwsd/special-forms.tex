\begin{blocksection}
\question What will Scheme output? Draw box-and-pointer diagrams to help determine this.

\begin{lstlisting}
scm> (if 1 1 (/ 1 0))
\end{lstlisting}
\begin{solution}[0.25in]
\begin{lstlisting}
1
\end{lstlisting}
\end{solution}

\begin{lstlisting}
scm> (and 1 #f (/ 1 0))
\end{lstlisting}
\begin{solution}[0.25in]
\begin{lstlisting}
#f
\end{lstlisting}
\end{solution}

\begin{lstlisting}
scm> (or #f #f 0 #f (/ 1 0))
\end{lstlisting}
\begin{solution}[0.25in]
\begin{lstlisting}
0
\end{lstlisting}
\end{solution}

\begin{lstlisting}
scm> (define a 4)
\end{lstlisting}
\begin{solution}[0.25in]
\begin{lstlisting}
a
\end{lstlisting}
\end{solution}

\begin{lstlisting}
scm> ((lambda (x y) (+ a x y)) 1 2)
\end{lstlisting}
\begin{solution}[0.25in]
\begin{lstlisting}
7
\end{lstlisting}
\end{solution}

\begin{lstlisting}
scm> ((lambda (x y z) (y x z)) 2 / 2)
\end{lstlisting}
\begin{solution}[0.25in]
\begin{lstlisting}
1
\end{lstlisting}
\end{solution}

\begin{lstlisting}
scm> ((lambda (x) (x x)) (lambda (y) 4))
\end{lstlisting}
\begin{solution}[0.25in]
4
\end{solution}
\end{blocksection}

\begin{blocksection}
\begin{lstlisting}
scm> (define boom1 (/ 1 0))
\end{lstlisting}
\begin{solution}[0.25in]
Error: Zero Division
\end{solution}

\begin{lstlisting}
scm> (define boom2 (lambda () (/ 1 0)))
\end{lstlisting}
\begin{solution}[0.25in]
boom2
\end{solution}

\begin{lstlisting}
scm> (boom2)
\end{lstlisting}
\begin{solution}[0.25in]
Error: Zero Division
\end{solution}
\end{blocksection}

\begin{blocksection}
Why/How are the two ``boom'' definitions above different?
\begin{solution}[1in]
The first line is setting boom1 to be equal to the value \texttt{(/ 1 0)}, which
turns out to be an error. On the other hand, boom2 is defined as a lambda that
takes in no arguments that, when called, will evaluate \texttt{(/ 1 0)}.
\end{solution}

How can we rewrite boom2 without using the lambda operator?
\begin{solution}[0.5in]
\begin{lstlisting}
(define (boom2) (/ 1 0))
\end{lstlisting}
\end{solution}
\end{blocksection}
