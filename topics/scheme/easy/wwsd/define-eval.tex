\begin{blocksection}
\question What will Scheme output?
	
\begin{lstlisting}
scm> (define c 4)
\end{lstlisting}
\begin{solution}[0.25in]
\begin{lstlisting}
c
\end{lstlisting}
\end{solution}
	
\begin{lstlisting}
scm> ((define (x) 1))
\end{lstlisting}
\begin{solution}[0.25in]
\begin{lstlisting}
Error: str is not callable: x
\end{lstlisting}
\end{solution}
	
\begin{lstlisting}
scm> (x)
\end{lstlisting}
\begin{solution}[0.25in]
\begin{lstlisting}
1
\end{lstlisting}
\end{solution}
	
\begin{lstlisting}
scm> ((lambda (x y) (+ c)) 1 2)
\end{lstlisting}
\begin{solution}[0.25in]
\begin{lstlisting}
4
\end{lstlisting}
\end{solution}
	
\begin{lstlisting}
scm> (eval 'c)
\end{lstlisting}
\begin{solution}[0.25in]
\begin{lstlisting}
2
\end{lstlisting}
\end{solution}
	
\begin{lstlisting}
scm> '(cons 1 nil)
\end{lstlisting}
\begin{solution}[0.25in]
\begin{lstlisting}
(cons 1 nil)
\end{lstlisting}
\end{solution}
	
\begin{comment}
	\begin{lstlisting}
	scm> (eval '(cons 1 nil))
	\end{lstlisting}
	\begin{solution}[0.25in]
	\begin{lstlisting}
	(1)
	\end{lstlisting}
	\end{solution}
\end{comment}
	
\begin{lstlisting}
scm> (eval (list 'if '(even? c) 1 2))
\end{lstlisting}
\begin{solution}[0.25in]
\begin{lstlisting}
1
\end{lstlisting}
\end{solution}
	
\begin{lstlisting}
scm> (let ((a (+ 3 1)) (b 3)) (+ a b) (/ a b))
\end{lstlisting}
\begin{solution}[0.25in]
\begin{lstlisting}
1.333333333333333
\end{lstlisting}
\end{solution}
\end{blocksection}
	
\begin{blocksection}
\begin{guide}
\textbf{Teaching Tips}
\begin{itemize}
	\item In the first four questions, we can see that defining a function using the define keyword and then calling it on the same line does not work but defining a function using the lambda keyword and then calling it on the same line does work. This is because \lstinline{(define (x) 1)} evaluates to x and then Scheme is trying to apply/call the string x, however you cannot apply a string, you have to apply a procedure. This question teaches us essentially how Scheme works for call expressions. All the expressions in the list are evaluated and then Scheme tries to call the evaluated expression of the first expression with the evaluated expressions of the rest of the expressions in the list as parameters. 
	\item Quotation marks are easily one of the trickiest concepts in Scheme, so spend a lot of time making sure your students understand these problems thoroughly! It would help to do a quick mini lecture or review of quotation marks if your students need.		
	\item In some cases it is easier to think of the single quotation mark as double quotes in Python that encompass a string, such as in "standalone" expressions like the third one. In such cases the exact "string" is returned
	\item Encourage students to make the connection between Scheme lists and Scheme expressions as often as they can. Being able to read Scheme expressions as lists will be very helpful in the future.
	\item Whenever there is an eval with a quote, you go down one level of evaluation and essentially pretend the quote is not there.
	\item Going off of the third point, it could be beneficial to model Scheme expressions on a pyramid of evaluation: exact string, evaluating the string, etc.
	\item While doing the one with the list operator, go to the Scheme Specification and the other links and show them how to navigate through the pages. 
\end{itemize}
\end{guide}
\end{blocksection}