\begin{blocksection}
\question
\textbf{let} is a special form in Scheme which allows you to create local bindings. Consider the example 
\newline
\newline
\texttt{(let ((x 1)) (+ x 1))}
\newline
\newline
Here, we assign \texttt{x} to \texttt{1}, and then evaluate the expression \texttt{(+ x 1)} using that binding, returning \texttt{2}. However, outside of this expression, \texttt{x} would not be bound to anything. \\
Each \texttt{let} special form has a corresponding lambda equivalent. The equivalent lambda expression for the above example is 
\newline
\newline
\texttt{((lambda (x) (+ x 1)) 1)}
\newline
\newline
The following line of code does not work. Why? Write the lambda
equivalent of the \texttt{let} expressions.

\begin{lstlisting}
(let ((foo 3)
      (bar (+ foo 2)))
    (+ foo bar))
\end{lstlisting}
\begin{solution}[0.5in]
The above function will error because it is equivalent to:
\begin{lstlisting}
((lambda (foo bar) (+ foo bar)) 3 (+ foo 2))
\end{lstlisting}

In other words, foo has not been defined in the global frame. When bar is being
assigned to \texttt{(+ foo 2)}, it will error. The assignment of foo to 3
happens in the lambda?s frame when it's called, not the global frame (this is
relevant to the Scheme project -- when the interpreter sees \texttt{lambda}, it
will call a function to start a new frame).

If we had the line \texttt{(define foo 3)} before the call to let, then it would
return 8, because within let, foo would be 3 and bar would be \texttt{(+ 3 2)},
since it would use the foo in the Global frame.
\end{solution}

\end{blocksection}
