\begin{blocksection}
\question
Fill in \lstinline{backwards-sum} such that it takes in a list of numbers \lstinline{lst} and returns a new list with each element being the sum of itself and all elements to the right of it in \lstinline{lst}. 

\textbf{Sidebar:} the word "sum" being bolded has no significance, it is an auto-formatting issue.

\begin{lstlisting}
; doctests
scm> (backwards-sum '(1 2 3 4))
(10 9 7 4)
scm> (backwards-sum '(2 -1 3 7))
(11 9 10 7)

(define (backwards-sum lst)














)
\end{lstlisting}
\end{blocksection}

\begin{blocksection}
\begin{solution}
\begin{lstlisting}
(define (backwards-sum lst)
    (cond 
        ((null? lst) nil)
        ((null? (cdr lst)) (list (car lst)))
        (else (cons
            (+ (car lst) (car (backwards-sum (cdr lst))))
            (backwards-sum (cdr lst))))
    )   
)
\end{lstlisting}
\end{solution}
\end{blocksection}



\begin{blocksection}
\begin{guide}
  \textbf{Teaching Tips}
  \begin{itemize}
      \item Walk through the base cases. Is it enough to only consider the case with one element left? What if we pass in an empty list?
	\item Point out the information we recieve from the recursive call (partial backwards sum) and have them consider how we can use that information (get the car).
	\item When constructing the recursive call, suggest to them that we can repeat the same recursive call twice.
  \end{itemize}
\end{guide}
\end{blocksection}