\begin{blocksection}
\question
The \lstinline{map} function takes in a two-argument function and a list of elements, and applies that function to each element in that list. We want to define our own version of the \lstinline{map} function EXCEPT instead of applying a function to a list of elements, we want to pass in a single element and apply each function in a list of functions to that element.

Define a function \lstinline{reverse-map}, which takes in a list of functions, \lstinline{operators}, and a single argument, \lstinline{arg}, and returns a list that results from applying all of the functions in \lstinline{operators} on \lstinline{arg}. You may assume that all functions in \lstinline{operators} will work properly with the single input \lstinline{arg}.

\begin{lstlisting}
; doctests
scm> (define funcs (list (lambda (x) (- x 10)) (lambda (x) (* x 2)) (lambda (x) (integer? x))))
funcs
scm> (reverse-map funcs 2)
(-8 4 #t)
scm> (reverse-map funcs 16)
(6 32 #t)

(define (reverse-map operators args)

    (if (______________________________)

        _____________________________________________ \

        ______________________________________________)
)
\end{lstlisting}

\begin{solution}
\begin{lstlisting}
(define (reverse-map operators arg)
    (if (null? operators)
        nil
        (cons ((car operators) arg) 
            (reverse-map (cdr operators) arg)))
)
\end{lstlisting}
\end{solution}
\end{blocksection}

\begin{blocksection}
\begin{guide}
  \textbf{Teaching Tips}
  \begin{itemize}
      \item Begin by suggesting recursion and guiding them to a base case.
	\item Consider reviewing the syntax for a single-argument function call. Contrast the implementation of map with the implementation of reverse-map.
  \end{itemize}
\end{guide}
\end{blocksection}