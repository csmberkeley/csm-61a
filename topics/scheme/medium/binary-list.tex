\begin{blocksection}
\question
You are creating a computer from scratch. In their rawest form, computers use 0s and 1s to compose commands and data. Fill in a function that takes a list of boolean values representing an \textbf{unsigned binary number} and returns its \textbf{decimal representation}. Each \lstinline{#t} in the list represents a 1 and each \lstinline{#f} represents a 0, with the \textbf{first} element in the list being the \textbf{rightmost} (smallest) binary digit and the \textbf{last} element being the \textbf{leftmost} (largest) binary digit.
\\
\begin{lstlisting}
;Doctests
scm> (binary (list #f #t)) ; 10
2
scm> (binary (list #t #f #t #t)) ; 1101
13
scm> (binary (list #t #t #f #f #t)) ; 10011
19
scm> (binary (list #f)) ; 0
0

(define (binary bin-list)
  (cond
    ((null? ____________)
      __________________
    )
    ((__________________)
      __________________________________
    )
    (else
      __________________________________
    )
  )
)
\end{lstlisting}
\end{blocksection}

\begin{solution}
\begin{blocksection}
\begin{lstlisting}
(define (binary bin-list)
  (cond
    ((null? bin-list)
      0
    )
    ((car bin-list)
      (+ 1 (* 2 (binary (cdr bin-list))))
    )
    (else
      (* 2 (binary (cdr bin-list)))
    )
  )
)
\end{lstlisting}
\end{blocksection}
\end{solution}


\begin{blocksection}
\question
Now, write the binary to decimal function, but in tail recursive form. Note that the \lstinline{expt} function takes in a base and an exponent. For example, \lstinline{(expt 2 3)} raises 2 to the third power, returning 8.
\\
\begin{lstlisting}
;Doctests
scm> (binary-tail (list #f #t)) ; 10
2
scm> (binary-tail (list #t #f #t #t)) ; 1101
13
scm> (binary-tail (list #t #t #f #f #t)) ; 10011
19
scm> (binary-tail (list #f)) ; 0
0

(define (binary-tail bin-list)
  (define (helper bin-list i sum)
    (cond
      ((null? ____________)
        __________________
      )
      ((__________________)
        ______________________________________________
      )
      (else
        ______________________________________________
      )
    )
  )
  (helper ________________)
)
\end{lstlisting}
\end{blocksection}

\begin{solution}
\begin{blocksection}
\begin{lstlisting}
(define (binary-tail bin-list)
  (define (helper bin-list i sum)
    (cond
      ((null? bin-list)
        sum
      )
      ((car bin-list)
        (helper
          (cdr bin-list) (+ 1 i) (+ sum (expt 2 i))
        )
      )
      (else
        (helper
          (cdr bin-list) (+ 1 i) sum
        )
      )
    )
  )
  (helper bin-list 0 0)
)
\end{lstlisting}
\end{blocksection}
\end{solution}
