\begin{blocksection}
\section{Challenge Question}
\question \textbf{(Optional)} The quicksort sorting algorithm is an efficient and com
monly used algorithm to
order the elements of a list. We choose one element of the list to be the pivot
element and partition the remaining elements into two lists: one of elements less
than the pivot and one of elements greater than the pivot. We recursively sort the
two lists, which gives us a sorted list of all the elements less than the pivot and all
the elements greater than the pivot, which we can then combine with the pivot for
a completely sorted list.
\newline
\newline
Implement \texttt{quicksort} in Scheme. Choose the first element of the list
as the pivot. You may assume that all elements are distinct. Hint: you may want to use a helper function.

\begin{lstlisting}
scm> (quicksort (list 5 2 4 3 12 7))
(2 3 4 5 7 12)
\end{lstlisting}

\begin{solution}[0.5in]
\begin{lstlisting}
(define (quicksort lst)
    (define (helper lst pivot less greater )
        (cond
            ((null? lst) (append (qs less) (list pivot) (qs greater)))
                ((> pivot (car lst)) (helper (cdr lst) pivot (append (list (car lst)) less) greater))
                ((< pivot (car lst)) (helper (cdr lst) pivot less (append (list (car lst)) greater)))
        )
    )
    (if (or (null? lst) (null? (cdr lst))) lst (helper (cdr lst) (car lst) nil nil))
)
\end{lstlisting}

\end{solution}
\end{blocksection}
