\question \begin{parts}
\part Define \lstinline$append$, which takes in two lists and
returns a new list with all the elements of the first list followed by all the elements of the second. Do not use the built-in \lstinline$append$ function.
%%%%%%% This guide section is specific to su21%%%%%%%%
\newline
\begin{guide}
Students have most likely implemented the append function in discussion last week, so don't spend too much time on this part.
If you are running behind schedule, consider just reminding them of the implementation and skipping to the next part.
\newline
\end{guide}
\begin{blocksection}
\begin{lstlisting}
> (append '(1 2 3) '(4 5 6))
(1 2 3 4 5 6)

(define (append lst1 lst2)






)
\end{lstlisting}

\begin{solution}%[0.5in]
\begin{lstlisting}
(define (append lst1 lst2)
    (if (null? lst1) lst2        
        (cons (car lst1) (append (cdr lst1) lst2)))))
\end{lstlisting}
\end{solution}

\part Define \lstinline$reverse$. Hint: use \lstinline$append$.

\begin{lstlisting}
> (reverse '(1 2 3))
(3 2 1)

(define (reverse lst)






)
\end{lstlisting}
\begin{solution}%[1in]
\begin{lstlisting}
(define (reverse lst)
    (if (null? lst) lst
        (append (reverse (cdr lst)) (list (car lst)))))
\end{lstlisting}
\end{solution}
\part Define \lstinline$reverse$ tail-recursively. Hint: use a helper function and
\lstinline$cons$.

\begin{lstlisting}
(define (reverse lst)






)
\end{lstlisting}
\end{blocksection}
\begin{blocksection}
\begin{solution}%[1in]
\begin{lstlisting}
(define (reverse lst)
    (define (helper lst reversed)
        (if (null? lst) reversed
            (helper (cdr lst) (cons (car lst) reversed ))))
    (helper lst '()))
\end{lstlisting}
\end{solution}
\end{blocksection}
\end{parts}
\begin{blocksection}
	\begin{guide}
	\textbf{Teaching Tips}
	\begin{enumerate}
			\item Board tip: this question requires creating several frames so save enough room on the board to draw this one out 
            \item The difficult part here is the fact that slicing or concatenating a list makes a new list. This is a good time to go over shallow copies.
            \item Talk about the recursive leap of faith for this problem since it shows how it works!
            \item To check students’ understanding it’s good to check whether the list returned is the same original object as the list passed in. i.e. why doesn’t lst change after calling reverse on it?
	\end{enumerate}
	\end{guide}
\end{blocksection}