\question
In Python, the \lstinline{reduce} function takes in a two-argument function and an iterable, and applies the function to all elements in the iterable, using each result as an argument for the next application. It "reduces" an iterable into a single value.

For example, calling \lstinline{reduce(lambda x, y: x + y, [1, 2, 3, 4])} will first compute \lstinline{1+2=3}, then use \lstinline{3} in the next step, \lstinline{3+3=6}, and then \lstinline{6+4=10}. Thus, the result will be \lstinline{10}.

Write a function \lstinline{skip-reduce} that implements the \lstinline{reduce} function on a Scheme list, but skips every other element.

\begin{lstlisting}
; doctests
scm> (skip-reduce (lambda (x y) (+ x y)) '())
0
scm> (skip-reduce (lambda (x y) (+ x y)) '(1 2 3))
4  ; 4 = 1 + 3
scm> (skip-reduce (lambda (x y) (+ x y)) '(1 2 3 4 5 6))
9  ; 9 = 1 + 3 + 5

(define (skip-reduce func iter)







)
\end{lstlisting}

\begin{solution}
\begin{lstlisting}
(define (skip-reduce func iter)
    (cond
        ((null? iter) 0)
        ((null? (cdr iter)) (car iter))
        (else
            (func (car iter) 
                (skip-reduce (cdr (cdr iter)))))
    )
)
\end{lstlisting}
\end{solution}