\question
Implement \lstinline{argmax}, a function that takes in a list, \lstinline{lst}, and returns the index of the largest element in \lstinline{lst}. If there are two or more elements which are the largest element, return the index of the one that appears first in \lstinline{lst}.

You can assume all elements of \lstinline{lst} are non-negative integers, and \lstinline{lst} has at least 1 element and no nested lists.

\begin{lstlisting}
(define (argmax lst)
    (define (max-helper lst max-so-far max-index curr-index)
        (cond

            ((__________________) _________________________)

            ((__________________) ________________________ \

                ________________________________)

            (else _________________________________________)
        )
    )

    (max-helper _______________________)
)
\end{lstlisting}

\begin{solution}
\begin{lstlisting}
(define (argmax lst)
    (define (max-helper lst max-so-far max-index curr-index)
        (cond
            ((null? lst) max-index)
            ((> (car lst) max-so-far) 
                (max-helper (cdr lst) (car lst) curr-index (+ curr-index 1)))
            (else
                (max-helper (cdr lst) max-so-far max-index (+ curr-index 1)))
        )
    )
    (max-helper lst 0 0 0)
)
\end{lstlisting}
\end{solution}