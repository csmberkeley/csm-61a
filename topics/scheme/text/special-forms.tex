Special forms \textit{look} like call expressions but aren't -- they implement Scheme language features and follow special evaluation rules (e.g., short-circuiting). 

(Aside: Note that you're free to use a special form name as a variable name, but the name will be looked up \textit{only} in a non-operator position; when used as an operator, it will always refer to the original special form.)

\textbf{Notable Special Forms:}
\begin{center}
\begin{tabular}{ |c|c| }
    \hline
    behavior & syntax \\
    \hline
    variable assignment   & \texttt{(define <variable-name> <value>)} \\
    \hline
    function definition & 
\begin{lstlisting}
(define (<function> <op1> ... <opN>) 
    <body>)
\end{lstlisting}  \\
    \hline
    if / else & \texttt{(if <condition> <true-expr> <else-expr>)} \\
    \hline
    if / elif / else  
& \begin{lstlisting}
(cond 
    (<cond1> <expr1>) 
    ... 
    (else <else-expr>))
\end{lstlisting} \\
    \hline
    and & \texttt{(and <operand1> ... <operandN>)} \\
    \hline
    or & \texttt{(or <operand1> ... <operandN>)} \\
    \hline
    quote & \texttt{(quote <operand1>)} \\
    \hline
    begin & 
\begin{lstlisting}
(begin 
    <expr1>
    ... 
    <exprN>) 
\end{lstlisting} \\
    \hline
    lambdas & 
\begin{lstlisting} 
(lambda (<operand1> ... <operandN>) 
    <body>) 
\end{lstlisting} \\
    \hline
    execute many lines & 
\begin{lstlisting} 
(let ((<var1> <val1>) 
      ... 
      (<varN> <valN>)) 
    body) 
\end{lstlisting}\\
    \hline
\end{tabular}
\end{center}