Scheme is a programming language, much like Python. In fact, many of Python's design features were inspired by Scheme. The point of learning this language is twofold: one, we're looking into what parts of Python generalize to other languages. Two, we want to start thinking about how to design and build (an interpreter  for) a programming language, and it turns out Scheme is a nice one to build. In fact, we'll show you enough of the language in this hour to write recursive procedures. This section covers the basics. You'll learn the rest in lab and discussion. It's pretty awesome that we'll be picking up a whole new programming language within an hour.

Visit \url{scheme.cs61a.org} to try the online interpreter. Type \lstinline$(autodraw)$
and the interpreter will automatically draw box-and-pointer diagrams whenever an expression
evalutes to a Scheme pair.
