Unlike Python, all Scheme lists are linked lists. Recall that, in Python, a linked list is made up of \lstinline{Link}s that each have a \lstinline{first} and a \lstinline{rest}, where the \lstinline{rest} is another \lstinline{Link}. Similarly, each Scheme list is a ``pair'' where the first element of the pair is the first element of the list, and the second element of the pair is the rest of the list (also a pair).

We use the \lstinline{cons} procedure to construct Scheme lists, and \lstinline{nil} to represent empty lists. The sequence $1, 2, 3$ may then be represented as follows: 
\begin{lstlisting}
scm> (cons 1 (cons 2 (cons 3 nil)))
(1 2 3)
\end{lstlisting}

\begin{meta}
It's worth pointing out to your students that, unlike with the \lstinline{Link} class, the \lstinline{nil} must be explicitly provided at the end of the linked list. 
\end{meta}

The \lstinline{car} and \lstinline{cdr} procedures are used to access the elements of a Scheme list. \lstinline{car} gets the first element of a list, while \lstinline{cdr} gets the rest of the list: 

\begin{lstlisting}
scm> (define lst (cons 1 (cons 2 (cons 3 nil))))
lst
scm> (car lst)
1
scm> (cdr lst)
(2 3)
\end{lstlisting}

You can make the following analogy between linked lists in Python and Scheme: 
\begin{center}
\begin{tabular}{ |l|l| }
\hline
 \texttt{Link(1, Link.empty)} & \texttt{(cons 1 nil)} \\
 \texttt{a = Link(1, Link(2, Link.empty))} & \texttt{(define a (cons 1 (cons 2 nil)))}  \\
 \texttt{a.first} & \texttt{(car a)} \\
 \texttt{a.rest} & \texttt{(cdr a)} \\
 \hline
\end{tabular}
\end{center}

The \lstinline{list} procedure and quotation give us additional convenient ways to construct lists: 
\begin{lstlisting}
scm> (list 1 2 3)
(1 2 3)
scm> '(1 2 3)
(1 2 3)
scm> (list 1 (+ 1 1) 3)
(1 2 3)
scm> '(1 (+ 1 1) 3)
(1 (+ 1 1) 3)
\end{lstlisting}
Note that quotation will prevent any of the list items from being evaluated, which can occasionally be inconvenient. 

\begin{guide}
\begin{blocksection}
\textbf{Teaching Tips}
\begin{itemize}
  \item Emphasize to students that Scheme lists are linked lists and NOT Python lists
  \begin{itemize}
    \item Discuss the limitations (e.g. no indexing) and capabilities (e.g. recursion)
  \end{itemize}
  \item If you're an old bear\textsuperscript{TM}, keep in mind that dotted lists have been removed from the curriculum, so Scheme lists have the same functionality as linked lists
  \begin{itemize}
    \item \lstinline{(cons 1 2)} will now raise a SchemeError instead of creating a Malformed list
  \end{itemize}
  \item The \href{https://code.cs61a.org/}{61A Scheme Web interpreter} is \textbf{very useful} for visualizing lists!
  \item If you choose to give a mini-lecture on Scheme list syntax, try using each keyword in an example instead of just talking about them!
\end{itemize}
\end{blocksection}
\end{guide}
