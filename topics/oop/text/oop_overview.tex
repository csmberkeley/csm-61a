\textbf{Object oriented programming} is a paradigm that organizes relationships among data into \textbf{objects} and \textbf{classes}. For example, we can write a \lstinline{Car} class to represent the concept of cars in general: 

\begin{lstlisting}
class Car:
    wheels = 4
    def __init__(self):
        self.gas = 100

    def drive(self):
        self.gas -= 10
        print("Current gas level:", self.gas)

my_car = Car()
\end{lstlisting}

To represent an individual car, we can then create a new instance of \lstinline{Car} by ``calling'' the class. Doing so will automatically construct a new object of type \lstinline{Car}, pass it into the \lstinline{__init__} method (also called the \textbf{constructor}), and then return it. Often, the \lstinline{__init__} method will initialize the \textbf{instance attributes} of an object, which represent the state of an individual object. In this case, \lstinline{__init__} method initially sets the \lstinline{gas} instance attribute of each car to $100$. 

Classes can also have \textbf{class attributes}, which are variables shared by all instances of a class. In the above example, \lstinline{wheels} is shared by all instances of the \lstinline{Car} class. 

\textbf{Instance methods} are special functions that act on the instances of a class. We've already seen the \lstinline{__init__} method. We can call instance methods by using the dot notation we use for instance attributes: 
\begin{lstlisting}
>>> my_car.drive()
Current gas level: 90
\end{lstlisting}
In instance methods, \lstinline{self} is the instance from which the method was called.  We don’t have to explicitly pass in \lstinline{self} because, when we call an instance method from an instance, the instance is automatically passed into the first parameter of the method by Python. That is, \lstinline{my_car.drive()} is exactly equivalent to the following: 
\begin{lstlisting}
>>> Car.drive(my_car)
Current gas level: 80
\end{lstlisting}

\begin{meta}
Something I like to emphasize with my students is that you can \textit{only} access class and instance attributes using dot notation from an instance. 
\end{meta}
