\textbf{\lstinline{__str__}} is special method that converts an object to a string meant to be readable by humans. It may be invoked by directly calling \lstinline{str} on an object. Additionally, calling \lstinline{print()} on an object will call the \lstinline{__str__} method of that object and print whatever value the \lstinline{__str__} call returns. 

The \textbf{\lstinline{__repr__}} method also returns a string representation of an object. However, the representation created by \lstinline{repr} is meant to be read by the Python interpreter, not by humans. When we evaluate some object in the Python interpreter, it will automatically call \lstinline{repr} on that object and then print out the string that \lstinline{repr} returns. It should contain all information about the object.

For example, if we had a \lstinline{Person} class with a name instance variable, we can create a \lstinline{__repr__} and \lstinline{__str__} method like so:
\begin{lstlisting}
def __str__(self):
    return "Hello, my name is " + self.name

def __repr__(self):
    return f"Person({repr(self.name)})"

>>> nobel_laureate = Person("Carolyn Bertozzi")
>>> str(nobel_laureate)
'Hello, my name is Carolyn Bertozzi'

>>> print(nobel_laureate)          
Hello, my name is Carolyn Bertozzi

>>> repr(nobel_laureate)
'Person("Carolyn Bertozzi")'

>>> nobel_laureate
Person("Carolyn Bertozzi")

>>> [nobel_laureate]
[Person("Carolyn Bertozzi")]
\end{lstlisting}

(In an \textbf{f-string}, which is a string with an \lstinline{f} in front of it, the expressions in curly braces are evaluated and their values [converted into strings] are inserted into the f-string, allowing us to customize the f-string based on what the expressions evaluate to.)

\lstinline{__str__}, \lstinline{__repr__}, and \lstinline{__init__} are a just a few examples of double-underscored ``magic'' methods that implement all sorts of special built-in and syntactical features of Python. 