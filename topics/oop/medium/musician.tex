\begin{blocksection}
\question \textbf{Musician} What would Python display? Write the result of executing the code and the prompts below. If a function is returned, write "Function". If nothing is returned, write "Nothing". If an error occurs, write "Error".

\begin{lstlisting}
class Musician:
    popularity = 0
    def __init__(self, instrument):
        self.instrument = instrument
    def perform(self):
        print("a rousing " + self.instrument + " performance")
        self.popularity = self.popularity + 2
    def __repr__(self):
        return self.instrument

class BandLeader(Musician):
    def __init__(self):
        self.band = []
    def recruit(self, musician):
        self.band.append(musician)
    def perform(self, song):
        for m in self.band:
            m.perform()
        Musician.popularity += 1
        print(song)
    def __str__(self):
        return "Here's the band!"
    def __repr__(self):
        band = ""
        for m in self.band:
            band += str(m) + " "
        return band[:-1]

miles = Musician("trumpet")
goodman = Musician("clarinet")
ellington = BandLeader()
\end{lstlisting}
\end{blocksection}

\newpage

\begin{blocksection}
\begin{solution}
Some Quick Refreshers \\
\textbf{Defining attributes:} Instance attributes are defined with the \lstinline{self.attr_name} notation (usually in \lstinline{__init__} but could be elsewhere like in this problem).
Class attributes are defined outside of methods in the body of the class definition, like the variable \lstinline{popularity} in the class \lstinline{Musician}. \\
\textbf{Accessing attributes:}
Instance attributes are referred to using \lstinline{self.attr_name} Class attributes can be referred to using \lstinline{classname.attr_name} or \lstinline{self.attr_name} (Note: using the latter will only work if there are no instance attributes bound with the name \lstinline{attr_name}). \\

Before running any of the code below, \lstinline{miles} and \lstinline{goodman} are set to the musicians created as a result of calling the \lstinline{__init__} constructor method in \lstinline{Musician}. \lstinline{ellington} uses \lstinline{BandLeader's} \lstinline{__init__} method, since \lstinline{BandLeader} is the subclass and has \lstinline{__init__} defined.
\end{solution}

\begin{lstlisting}
>>> ellington.recruit(goodman)
>>> ellington.perform()
\end{lstlisting}
\begin{solution}[.2in]
Error

\lstinline{ellington.recruit(goodman)} adds \lstinline{goodman} to the end of \lstinline{ellington's} instance attribute, \lstinline{band}. Then, \lstinline{ellington} checks its class (\lstinline{BandLeader}) for the \lstinline{perform()} method. But this \lstinline{perform()} is expecting an argument, so this errors.
\end{solution}

\vspace{1\baselineskip}

\begin{lstlisting}
>>> ellington.perform("sing, sing, sing")
\end{lstlisting}
\begin{solution}[.2in]
a rousing clarinet performance \\
sing, sing, sing

Using the same \lstinline{perform()} method, now providing the correct number of arguments. First, going through the band list, \lstinline{goodman} calls its \lstinline{perform()} method, which is defined in \lstinline{Musician}. Here, we print \lstinline{"a rousing"} + \lstinline{goodman}'s instrument + \lstinline{" performance"}, and then \lstinline{goodman}'s \lstinline{self.popularity = self.popularity + 2} happens. The \lstinline{self.popularity} on the right of the equal sign is \lstinline{Musician.popularity} because \lstinline{goodman} doesn't have its own instance attribute named \lstinline{popularity} yet; then it becomes \lstinline{self.popularity = 0 + 2}, and this creates the instance attribute \lstinline{popularity} for \lstinline{goodman}. Then \lstinline{Musician.popularity}, the class attribute, in incremented by 1.
\end{solution}
\end{blocksection}

\vspace{2\baselineskip}

\begin{blocksection}
\begin{lstlisting}
>>> goodman.popularity, miles.popularity
\end{lstlisting}
\begin{solution}[.2in]
(2, 1)

First, we try to get the value of \lstinline{goodman.popularity}. In our environment diagram, we see that \lstinline{goodman} has the instance variable \lstinline{popularity} already defined. Therefore, we get that value, 2, back. Then, we try to access \lstinline{miles.popularity}. In this case, \lstinline{miles} doesn’t have a \lstinline{popularity} instance variable defined, so we default to the class variable. There, we see it defined as 1, so we get that value. Finally, since commas in Python define a tuple, we return the two values as \lstinline{(2, 1)}.
\end{solution}

\vspace{1\baselineskip}

\begin{lstlisting}
>>> ellington.recruit(miles)
>>> ellington.perform("caravan")
\end{lstlisting}
\begin{solution}[.2in]
a rousing clarinet performance\\
a rousing trumpet performance\\
caravan

First, we call \lstinline{ellington.recruit(miles)}. This appends \lstinline{miles} to \lstinline{ellington}'s instance variable, \lstinline{band}. After that, we call \lstinline{ellington.perform("caravan")}. Similar to the previous call on perform, we will loop through all of the values in \lstinline{ellington.band}, calling their perform methods in order. This causes the first two lines to be printed. Next, we increment \lstinline{Musician.popularity} (the class variable of \lstinline{Musician} called \lstinline{popularity}). Lastly, we print the \lstinline{song} variable that was passed in, completing the last line.
\end{solution}

\vspace{2\baselineskip}

\begin{lstlisting}
>>> ellington.popularity, goodman.popularity, miles.popularity
\end{lstlisting}
\begin{solution}[.2in]
(2, 4, 3)
\end{solution}

\vspace{2\baselineskip}

\begin{lstlisting}
>>> print(ellington)
\end{lstlisting}
\begin{solution}[.2in]
Here's the band!

\lstinline{print()} expects the string representation of \lstinline{ellington}, which is given by calling the \lstinline{__str__()} method of \lstinline{ellington}. \lstinline{ellington} checks to see if \lstinline{BandLeader} has a \lstinline{__str__()} method, which it does. So, \lstinline{print(ellington)} then becomes \lstinline{print("Here's the band!")}.
\end{solution}

\vspace{2\baselineskip}
\end{blocksection}
\begin{blocksection}
\begin{lstlisting}
>>> ellington
\end{lstlisting}
\begin{solution}[.2in]
clarinet trumpet

When prompting for \lstinline{ellington}’s value, we return the representation of ellington given by \lstinline{__repr__()}. So, we call \lstinline{BandLeader}’s \lstinline{__repr__()} method.

\end{solution}

\end{blocksection}
