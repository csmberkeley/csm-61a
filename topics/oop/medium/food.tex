\question 
What would Python display? The questions continue on the next page.

\begin{lstlisting}
class Food:
    def __init__(self, name, spoiled = False):
        self.name = name
        self.num_days = 0
        self.spoiled = spoiled

    def can_eat(self):
        self.num_days += 1
        if self.num_days >= 3:
            self.spoiled = True
            print("Oh no! Your food is spoiled!")
        return not self.spoiled

    def mix_food(self, other_food):
        self.num_days = self.num_days + other_food.num_days
        self.name += " " + other_food.name
        self.spoiled = self.spoiled and other_food.spoiled

class Salad(Food):
    def __init__(self, ingredients):
        super().__init__("salad", False)
        self.ingredients = ingredients
		
    def add_ingredients(self, ingredient):
        self.ingredients.append(ingredient)
        print(ingredient.name + " has been added")

    def mix_ingredients(self):
        for ingredient in self.ingredients:
            self.mix_food(ingredient)
        print("Your salad has been mixed.")

lettuce = Food("lettuce")
tomatoes = Food("tomatoes")
chicken = Food("chicken")
ingredients = [lettuce, tomatoes]
my_salad = Salad(ingredients)
\end{lstlisting}

\vspace{9\baselineskip}
\begin{solution}
See visualizations for solutions: \url{https://docs.google.com/presentation/d/1t1yE9DuT8a2ij_QszLOxzUu6-unN46PY1SA_Q48fLz4/edit?usp=sharing}
\end{solution}

\begin{lstlisting}
>>> lettuce.can_eat()
\end{lstlisting}
\begin{solution}[.2in]
True
\end{solution}

\begin{lstlisting}
>>> my_salad.can_eat()
\end{lstlisting}
\begin{solution}[.2in]
True
\end{solution}

\begin{lstlisting}
>>> my_salad.mix_ingredients()
\end{lstlisting}
\begin{solution}[.2in]
Your salad has been mixed.
\end{solution}

\begin{lstlisting}
>>> my_salad.name
\end{lstlisting}
\begin{solution}[.3in]
"salad lettuce tomatoes"
\end{solution}
