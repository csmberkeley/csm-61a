\twocolumn
\begin{blocksection}

\question \textbf{Flying the cOOP} What would Python display? \\
Write the result of executing the code and the prompts below. \\
If a function is returned, write "Function". If nothing is \\
returned, write "Nothing". If an error occurs, write "Error". \\

\vspace{2\baselineskip}

\begin{lstlisting}
class Bird:
    def __init__(self, call):
        self.call = call
        self.can_fly = True
    def fly(self):
        if self.can_fly:
            return "Don't stop me now!"
        else:
            return "Ground control to Major Tom..."
    def speak(self):
        print(self.call)

class Chicken(Bird):
    def speak(self, other):
        Bird.speak(self)
        other.speak()

class Penguin(Bird):
    can_fly = False
    def speak(self):
        call = "Ice to meet you"
        print(call)

andre = Chicken("cluck")
gunter = Penguin("noot")
\end{lstlisting}
\end{blocksection}

\newpage
\begin{blocksection}
\vspace{9\baselineskip}
\begin{lstlisting}
>>> andre.speak(Bird("coo"))
\end{lstlisting}
\begin{solution}[.2in]
cluck \\
coo
\end{solution}

\vspace{3\baselineskip}
\begin{lstlisting}
>>> andre.speak()
\end{lstlisting}
\begin{solution}[.2in]
Error
\end{solution}

\vspace{3\baselineskip}
\begin{lstlisting}
>>> gunter.fly()
\end{lstlisting}
\begin{solution}[.2in]
"Don't stop me now!"
\end{solution}

\vspace{3\baselineskip}
\begin{lstlisting}
>>> andre.speak(gunter)
\end{lstlisting}
\begin{solution}[.2in]
cluck \\
Ice to meet you
\end{solution}

\vspace{3\baselineskip}
\begin{lstlisting}
>>> Bird.speak(gunter)
\end{lstlisting}
\begin{solution}[.2in]
noot
\end{solution}

\end{blocksection}
\onecolumn
