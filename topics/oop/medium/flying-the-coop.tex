\twocolumn
\begin{blocksection}

\question \textbf{Flying the cOOP} What would Python display? Write the result of executing the code and the prompts below. If a function is returned, write "Function". If nothing is returned, write "Nothing". If an error occurs, write "Error".
\newline Hint: You may find it helpful to make an environment diagram tracking your objects and \textit{each} new instance of an object (whether it's assigned to a variable or not!).

\vspace{2\baselineskip}

\begin{lstlisting}
class Bird:
    def __init__(self, call):
        self.call = call
        self.can_fly = True
    def fly(self):
        if self.can_fly:
            return "Don't stop me now!"
        else:
            return "Ground control to Major Tom..."
    def speak(self):
        print(self.call)

class Chicken(Bird):
    def speak(self, other):
        Bird.speak(self)
        other.speak()

class Penguin(Bird):
    can_fly = False
    def speak(self):
        call = "Ice to see you"
        print(call)

andre = Chicken("cluck")
gunter = Penguin("noot")
\end{lstlisting}
\end{blocksection}

\newpage
\begin{blocksection}
\vspace{9\baselineskip}
\begin{lstlisting}
>>> andre.speak(Bird("coo"))
\end{lstlisting}
\begin{solution}[.2in]
cluck \\
coo
\end{solution}

\vspace{3\baselineskip}
\begin{lstlisting}
>>> andre.speak()
\end{lstlisting}
\begin{solution}[.2in]
Error
\end{solution}

\vspace{3\baselineskip}
\begin{lstlisting}
>>> gunter.fly()
\end{lstlisting}
\begin{solution}[.2in]
"Don't stop me now!"
\end{solution}

\vspace{3\baselineskip}
\begin{lstlisting}
>>> andre.speak(gunter)
\end{lstlisting}
\begin{solution}[.2in]
cluck \\
Ice to see you
\end{solution}

\vspace{3\baselineskip}
\begin{lstlisting}
>>> Bird.speak(gunter)
\end{lstlisting}
\begin{solution}[.2in]
noot
\end{solution}
\end{blocksection}
\onecolumn

% Fix spacing for new page
\begin{solution}[-32pt]
\begin{blocksection}
\textbf{Explanations}

\begin{itemize}
    \item \lstinline{andre.speak(Bird("coo"))}
\end{itemize}
The \lstinline{Bird} object is created and immediately passed in as the parameter for \lstinline{Bird}. Even though we don't assign it to a variable, the object still exists and has all the features of a \lstinline{Bird} object.
\end{blocksection}
\end{solution}

\begin{guide}
\begin{blocksection}
This might be difficult to conceptualize to students, so stop to make sure students understand how this works, since this same create-and-use method will be used often in midterm / exam level environment diagram questions.
\end{blocksection}
\end{guide}

\begin{solution}
\begin{blocksection}
\begin{itemize}
    \item \lstinline{andre.speak()}
\end{itemize}
Python expects two parameters but in this case we are only assigning \lstinline{self}.
\end{blocksection}
\end{solution}

\begin{guide}
\begin{blocksection}
It might be good to emphasize to students that this is how regular functions work as well. We have to pass in the correct amount of values.
\end{blocksection}
\end{guide}

\begin{solution}
\begin{blocksection}
\begin{itemize}
    \item \lstinline{gunter.fly()}
\end{itemize}
Note that the \lstinline{Penguin} class will use the constructor for the \lstinline{Bird} class, which sets \lstinline{gunter.can_fly} for the particular instance.
\end{blocksection}
\end{solution}

\begin{guide}
\begin{blocksection}
Step slowly through this part to emphasize how the \lstinline{Penguin} class variable differs from the \lstinline{gunter} instance variable for \lstinline{can_fly}.
\end{blocksection}
\end{guide}

\begin{solution}
\begin{blocksection}
\begin{itemize}
    \item \lstinline{andre.speak(gunter)}
\end{itemize}
This question is really similar to the first one, but instead of \lstinline{Bird("coo")} we use the \lstinline{gunter} object instead.

\begin{itemize}
    \item \lstinline{Bird.speak(gunter)}
\end{itemize}
\lstinline{Bird.speak} looks within the \lstinline{Bird} class to find the speak method. 
\end{blocksection}
\end{solution}

\begin{guide}
\begin{blocksection}
    Some students may think that \lstinline{Bird} is passed in for \lstinline{self}. This is not the case because \lstinline{Bird} is not an instance; it’s a class.
    \end{blocksection}
\end{guide}

