\begin{blocksection}
\question Let's build a Bear using OOP!

Bear instances should have an attribute \lstinline{name} that holds the name of the bear.
The Bear class should have an attribute \lstinline{bears}, a list that stores the name of each bear.
\vspace{1\baselineskip}
\begin{lstlisting}
>>> oski = Bear('Oski')
>>> oski.name
'Oski'
>>> Bear.bears
['Oski']
>>> winnie = Bear('Winnie')
>>> Bear.bears
['Oski', 'Winnie']

class Bear:
\end{lstlisting}

\begin{nonsol}
\vspace{4\baselineskip}
\end{nonsol}

\begin{solution}
  \vspace{-0.5\baselineskip}
\begin{lstlisting}
    bears = []
    def __init__(self, name):
        self.name = name
        Bear.bears.append(self.name)
\end{lstlisting}

Note that \lstinline{self.bears.append(self.name)} also works, but just doing \lstinline{bears.append(self.name)} will result in an error!
\\There is no \lstinline{bears} variable in the \lstinline{__init__} function frame.
\end{solution}
\end{blocksection}