\question Let's slowly build a Bear from start to finish using OOP!

\begin{parts}
\begin{blocksection}
\part
First, let's build a \lstinline{Bear} class for our basic bear.
Bear instances should have an attribute \lstinline{name} that holds the name of the bear and an attribute \lstinline{organs}, an initially empty list of the bear's organs.
The Bear class should have an attribute \lstinline{bears}, a list that stores the name of each bear.
\begin{lstlisting}
class Bear:
    """
    >>> oski = Bear('Oski')
    >>> oski.name
    'Oski'
    >>> oski.organs
    []
    >>> Bear.bears
    ['Oski']
    >>> winnie = Bear('Winnie')
    >>> Bear.bears
    ['Oski', 'Winnie']
    """
\end{lstlisting}

\begin{solution}[3in]
  \vspace{-0.5\baselineskip}
\begin{lstlisting}
    bears = []
    def __init__(self, name):
        self.name = name
        self.organs = []
        Bear.bears.append(self.name)
\end{lstlisting}

Note that just doing \lstinline{bears.append(self.name)} will result in an error!
\\There is no \lstinline{bears} variable in the \lstinline{__init__} function frame.
\end{solution}
\end{blocksection}

\begin{blocksection}
\part Next, let's build an \lstinline{Organ} class to put in our bear.
\lstinline{Organ} instances should have an attribute \lstinline{name} that holds the name of the organ and an attribute \lstinline{bear} that holds the bear it belongs to. The \lstinline{Organ} class should also have an instance method \lstinline{discard(self)} that removes the organ from \lstinline{Organ.organ_count} and the bear's organs list.

The \lstinline{Organ} class should contain a dictionary \lstinline{organ_count} that maps the name of each bear to the number of organs it has. 

Hint: We may need to change the representation of this object for our doc tests to be correct.


\begin{lstlisting}
class Organ:
    """
    >>> oski, winnie = Bear('Oski'), Bear('Winnie')
    >>> oski_liver = Organ('liver', oski)
    >>> Organ.organ_counts
    {'Oski': 1}
    >>> winnie_stomach = Organ('stomach', winnie)
    >>> winnie_liver = Organ('liver', winnie)
    >>> winnie.organs
    [stomach, liver]
    >>> winnie_liver.discard()
    >>> Organ.organ_counts
    {'Oski': 1, 'Winnie': 1}
    >>> winnie.organs
    [stomach]
    """
\end{lstlisting}
\begin{solution}[4 in]
\begin{lstlisting}
    organ_counts = {}

    def __init__(self, name, bear):
        self.name = name
        self.bear = bear
        if bear.name in Organ.organ_counts:
            Organ.organ_counts[bear.name] += 1
        else:
            Organ.organ_counts[bear.name] = 1
        bear.organs.append(self)
    \end{lstlisting}
\end{solution}
\end{blocksection}

\begin{solution}
\begin{lstlisting}
    def discard(self):
        Organ.organ_counts[self.bear.name] -= 1
        self.bear.organs.remove(self)

    def __repr__(self):
        return self.name
\end{lstlisting}

Without the \lstinline{__repr__}, an Organ returns \lstinline{<__main__.Organ object>} instead of its name in \lstinline{Organ.organs}.

Organs do not inherit from Bear, nor should they. Inheritance is used in \textbf{is a} relationships, not \textbf{has a}.
\end{solution}


\begin{blocksection}
\part Now, let's design a \lstinline{Heart} class that inherits from the \lstinline{Organ} class.
When a heart is created, if its bear does not already have a heart, it creates a \lstinline{heart} attribute for that bear. If a bear already has a heart, the old \lstinline{heart} is discarded and replaced with the new one. The bear's organs list and \lstinline{Organ.organ_count} should be updated appropriately. 

Hint: you can use \lstinline{hasattr} to check if a bear has a heart attribute.

\begin{lstlisting}
class Heart(Organ):
    """
    >>> oski, winnie = Bear('Oski'), Bear('Winnie')
    >>> hasattr(oski, 'heart')
    False
    >>> oski_heart = Heart('small heart', oski)
    >>> oski.heart
    small heart
    >>> oski.organs
    [small heart]
    >>> new_heart = Heart('big heart', oski)
    >>> oski.heart
    big heart
    >>> oski.organs
    [big heart]
    >>> Organ.organ_counts["Oski"]
    1
    """
\end{lstlisting}

\begin{solution}[2in]
\begin{lstlisting}
    def __init__(self, name, bear):
        if hasattr(bear, 'heart'):
            bear.heart.discard()
        bear.heart = self
        Organ.__init__(self, name, bear)
\end{lstlisting}
Since Hearts are Organs, we can use Organ's \lstinline{discard} method to remove an old heart easily, without breaking any abstraction barriers. We also can use \lstinline{Organ.__init__} instead of repeating code.
\end{solution}
\end{blocksection}


% \begin{blocksection}
% \part Finally, let's design a Brain class that inherits from the Organ class.
% Bears want to rise up against the Bourgeoise, and a Brain gives them the power to do so.
% \\The appearance of a brain gives the Bear class an attribute \lstinline{rebels}, a \textbf{set} containing the names of all rebelling bears.
% \\A bear with a brain is added to \lstinline{rebels} by default.
% \\It also gives \textbf{all} bears, even ones without brains, a \lstinline{rebel} method that, when called, adds the caller to \lstinline{rebels}.

% \vspace{1\baselineskip}
% \begin{lstlisting}
% >>> big_brain = Brain('big brain', oski)
% >>> Bear.rebels
% {'Oski'}
% >>> smokey = Bear('Smokey')
% >>> smokey.rebel()
% >>> Bear.rebels
% {'Smokey', 'Oski'}
% >>> small_brain = Brain('brain', winnie)
% >>> Bear.rebels
% {'Smokey', 'Oski', 'Winnie'}
% >>> Organ.organs
% {liver: 'Oski', stomach: 'Winnie', big heart: 'Oski', big brain: 'Oski', brain: 'Winnie'}

% class Brain(Organ):
% \end{lstlisting}

% \begin{nonsol}
% \vspace{7\baselineskip}
% \end{nonsol}

% \begin{solution}
%   \vspace{-0.5\baselineskip}
% \begin{lstlisting}
%     def __init__(self, name, bear):
%         if hasattr(Bear, 'rebels'):
%             Bear.rebels.add(bear.name)
%         else:
%             Bear.rebels = set([bear.name])
%         Bear.rebel = lambda self: Bear.rebels.add(self.name)
%         Organ.__init__(self, name, bear)
% \end{lstlisting}

% Be careful not to reset \lstinline{Bear.rebels} upon the creation of a second brain.
% \end{solution}
% \end{blocksection}
\end{parts}
