\begin{blocksection}
\question Let's slowly build a Bear from start to finish using OOP!

First, let's build a Bear class for our basic bear.
Bear instances should have an attribute \lstinline{name} that holds the name of the bear.
The Bear class should have an attribute \lstinline{bears}, a list that stores the name of each bear.
\vspace{1\baselineskip}
\begin{lstlisting}
>>> oski = Bear('Oski')
>>> oski.name
'Oski'
>>> Bear.bears
['Oski']
>>> winnie = Bear('Winnie')
>>> Bear.bears
['Oski', 'Winnie']

class Bear:
\end{lstlisting}

\begin{nonsol}
\vspace{4\baselineskip}
\end{nonsol}

\begin{solution}
  \vspace{-0.5\baselineskip}
\begin{lstlisting}
    bears = []
    def __init__(self, name):
        self.name = name
        Bear.bears.append(self.name)
\end{lstlisting}

Note that just doing \lstinline{bears.append(self.name)} will result in an error!
\\There is no \lstinline{bears} variable in the \lstinline{__init__} function frame.
\end{solution}
\end{blocksection}

\vspace{1\baselineskip}

\begin{blocksection}
\question Next, let's build an Organ class to put in our bear.
Organ instances should have an attribute \lstinline{name} that holds the name of the organ.
The Organ class should contain a dictionary \lstinline{organs} that holds (organ, bear name) key-value pairs, mapping each organ object to the name of its bear.
\\Note that this maps \textbf{objects} to strings! We may need to change the representation of this object for our doc tests to be correct.
\\The Organ class should also have a method \lstinline{discard(self)} that removes the organ from Organ.organs.

\vspace{1\baselineskip}
\begin{lstlisting}
>>> oski_liver = Organ('liver', oski)
>>> winnie_stomach = Organ('stomach', winnie)
>>> winnie_liver = Organ('liver', winnie)
>>> Organ.organs
{liver: 'Oski', stomach: 'Winnie', liver: 'Winnie'}
>>> winnie_liver.discard()
>>> Organ.organs
{liver: 'Oski', stomach: 'Winnie'}

class Organ:
\end{lstlisting}

\begin{nonsol}
\vspace{8\baselineskip}
\end{nonsol}

\begin{solution}
  \vspace{-0.5\baselineskip}
\begin{lstlisting}
    organs = {}
    def __init__(self, name, bear):
        self.name = name
        Organ.organs[self] = bear.name
    def discard(self):
        Organ.organs.pop(self)
    def __repr__(self):
        return self.name
\end{lstlisting}
Note if we used \lstinline{Organ.organs[self.name]} instead of \lstinline{Organ.organs[self]}, it would be impossible to store 2 Organs with the same name.
\\Without the \lstinline{__repr__}, an Organ returns \lstinline{<__main__.Organ object>} instead of its name in \lstinline{Organ.organs}.
\\Organs do not inherit from Bear, nor should they. Inheritance is used in \textbf{is a} relationships, not \textbf{has a}.
\end{solution}
\end{blocksection}


\begin{blocksection}
\question Now, let's design a Heart class that inherits from the Organ class.
When a heart is created, if its bear does not have a heart, it creates a \lstinline{heart} attribute for that bear.
If a bear already has a heart, the old \lstinline{heart} is discarded and replaced with the new one.
\\Hint: you can use \lstinline{hasattr} to check if a bear has a heart attribute.

\vspace{1\baselineskip}
\begin{lstlisting}
>>> hasattr(oski, 'heart')
False
>>> oski_heart = Heart('small heart', oski)
>>> oski.heart
small heart
>>> Organ.organs
{liver: 'Oski', stomach: 'Winnie', small heart: 'Oski'}
>>> new_heart = Heart('big heart', oski)
>>> oski.heart
big heart
>>> Organ.organs
{liver: 'Oski', stomach: 'Winnie', big heart: 'Oski'}

class Heart(Organ):
\end{lstlisting}

\begin{nonsol}
\vspace{5\baselineskip}
\end{nonsol}

\begin{solution}
  \vspace{-0.5\baselineskip}
\begin{lstlisting}
    def __init__(self, name, bear):
        if hasattr(bear, 'heart'):
            bear.heart.discard()
        bear.heart = self
        Organ.__init__(self, name, bear)
\end{lstlisting}
Since Hearts are Organs, we can use Organ's \lstinline{discard} method to remove an old heart easily, without breaking any abstraction barriers.
\\We also can use \lstinline{Organ.__init__} instead of repeating code.
\end{solution}
\end{blocksection}


\begin{blocksection}
\question Finally, let's design a Brain class that inherits from the Organ class.
Bears want to rise up against the Bourgeoise, and a Brain gives them the power to do so.
\\The appearance of a brain gives the Bear class an attribute \lstinline{rebels}, a \textbf{set} containing the names of all rebelling bears.
\\A bear with a brain is added to \lstinline{rebels} by default.
\\It also gives \textbf{all} bears, even ones without brains, a \lstinline{rebel} method that, when called, adds the caller to \lstinline{rebels}.

\vspace{1\baselineskip}
\begin{lstlisting}
>>> big_brain = Brain('big brain', oski)
>>> Bear.rebels
{'Oski'}
>>> smokey = Bear('Smokey')
>>> smokey.rebel()
>>> Bear.rebels
{'Smokey', 'Oski'}
>>> small_brain = Brain('brain', winnie)
>>> Bear.rebels
{'Smokey', 'Oski', 'Winnie'}
>>> Organ.organs
{liver: 'Oski', stomach: 'Winnie', big heart: 'Oski', big brain: 'Oski', brain: 'Winnie'}

class Brain(Organ):
\end{lstlisting}

\begin{nonsol}
\vspace{7\baselineskip}
\end{nonsol}

\begin{solution}
  \vspace{-0.5\baselineskip}
\begin{lstlisting}
    def __init__(self, name, bear):
        if hasattr(Bear, 'rebels'):
            Bear.rebels.add(bear.name)
        else:
            Bear.rebels = set([bear.name])
        Bear.rebel = lambda self: Bear.rebels.add(self.name)
        Organ.__init__(self, name, bear)
\end{lstlisting}

Be careful not to reset \lstinline{Bear.rebels} upon the creation of a second brain.
\end{solution}
\end{blocksection}
