\begin{blocksection}
    \question \text{[Exam Level]} Implement the \lstinline{Poll} class and the \lstinline{tally} function, which takes a choice \lstinline{c} and returns a list describing the number of votes for \lstinline{c}. 
    This list contains pairs, each with a name and the number of times \lstinline{vote} was called on that \lstinline{choice} at the \lstinline{Poll} with that name. Pairs can be in any order. Assume all \lstinline{Poll}
    instances have distinct names. Hint: the dictionary \lstinline{get(key, default)} method (MT 2 guide, page 1
    top-right) returns the value for a \lstinline{key} if it appears in the dictionary and \lstinline{default} otherwise. [Adapted from CS61A Fa18 Midterm 2 Q5(a)]

    \begin{lstlisting}
class Poll:
    s = []

    def __init__(self, n):

        self.name = _________________________________________________

        self.votes = {}

        _____________________________________________________________

    def vote(self, choice):

        self._________________________ = ________________________________
    


def tally(c):
    """Tally all votes for a choice c as a list of (poll name, vote count) pairs.
    >>> a, b, c = Poll('A'), Poll('B'), Poll('C')
    >>> c.vote('dog')
    >>> a.vote('dog')
    >>> a.vote('cat')
    >>> b.vote('cat')
    >>> a.vote('dog')
    >>> tally('dog')
    [('A', 2), ('C', 1)]
    >>> tally('cat')
    [('A', 1), ('B', 1)]
    """

    return____________________________________________________________________
    \end{lstlisting}
    
    
    \begin{solution}

    \begin{lstlisting}
class Poll:
    s = []
    def __init__(self, n):
        self.name = n
        self.votes = {}
        Poll.s.append(self)
    def vote(self, choice):
        self.votes[choice] = self.votes.get(choice, 0) + 1

def tally(c):
    return [(p.name, p.votes[c]) for p in Poll.s if c in p.votes]
    \end{lstlisting}
    
    \end{solution}
    \end{blocksection}