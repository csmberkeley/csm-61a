\begin{blocksection}
    \question The Reprtilia class stores the common and scientific names of various reptiles.
    
    \textbf{Brief introduction to \lstinline{__repr__}} \\ The \lstinline{__repr__} magic method of objects returns the "official" string representation of an object. You can invoke it directly by calling \lstinline{repr(<some object>)}. However, \lstinline{__repr__} doesn't always return something that is easily readable, that is what \lstinline{__str__} is for. Rather, \lstinline{__repr__} ensures that all information about the object is present in the representation. When you ask Python to represent an object in the Python interpreter, it will automatically call \lstinline{repr} on that object and then print out the string that \lstinline{repr} returns.
    
    \vspace{1.5\baselineskip}
    
    \begin{lstlisting}
        """
        >>> python = Reprtilia("python", "pythonidae")
        >>> iguana = Reprtilia("iguana", "iguana")
        >>> python
        Reptilia('python', 'pythonidae')
        >>> f'Did you know that {python}?"
        'Did you know that a python is a "pythonidae"?'
        """
    
        class Reprtilia:
            def __init__(self, name, scientific):
                self.name = name
                self.scientific = scientific
    \end{lstlisting}
\end{blocksection}
    
\begin{blocksection}
    \question How do we define the __repr__ method?

    \vspace{1.5\baselineskip}
    
    \begin{lstlisting}
            def __repr__(self):
    \end{lstlisting}
    \begin{solution}[0.25in]
    \begin{lstlisting}
            return f"Reprtilia({repr(self.name)}, {repr(self.scientific)})"
    \end{lstlisting}
    \end{solution}
\end{blocksection}

\begin{blocksection}
    \question How do we define the __str__ method?

    \vspace{1.5\baselineskip}

    \begin{lstlisting}
            def __str__(self):
    \end{lstlisting}
    \begin{solution}[0.25in]
    \begin{lstlisting}
            return f"a {self.name} is a \"{self.scientific}\""
    \end{lstlisting}
    \end{solution}
\end{blocksection}

\begin{blocksection}
    \question Your friend, Julian, decides to create a new object class based off of Reprtilia called ReprtiliaNoises, and as such, decides to overhaul your __repr__ and __str__! How would the following be implemented following the doctests?
    
    \vspace{1.5\baselineskip}

    \begin{lstlisting}
        """
        >>> python = ReprtiliaNoises("python", "pythonidae", "hiss")
        >>> python
        ReprtiliaNoises('python', 'pythonidae', 'hiss')
        >>> f"Today's reptile of the day is a {python}!"
        "Today's reptile of the day is a python (species: 'pythonidae', noise: 'hiss')!"
        """
    \end{lstlisting}

    \begin{blocksection}
        \question How does Julian redefine the __init__ method?
    
        \vspace{1.5\baselineskip}
        
        \begin{lstlisting}
            class ReprtiliaNoises(Reprtilia):
                def __init__(self, name, scientific, noise):
        \end{lstlisting}
        \begin{solution}[0.25in]
        \begin{lstlisting}
                    super().__init__(name, scientific)
                    self.noise = noise
        \end{lstlisting}
        \end{solution}
    \end{blocksection}    
    
    \begin{blocksection}
        \question How does Julian redefine the __repr__ method?
    
        \vspace{1.5\baselineskip}
        
        \begin{lstlisting}
                def __repr__(self):
        \end{lstlisting}
        \begin{solution}[0.25in]
        \begin{lstlisting}
                return f"ReprtiliaNoises({repr(self.name)}, {repr(self.scientific)}, {repr(self.noise)})"
        \end{lstlisting}
        \end{solution}
    \end{blocksection}

    \begin{blocksection}
        \question How does Julian redefine the __str__ method?
    
        \vspace{1.5\baselineskip}
        
        \begin{lstlisting}
                def __str__(self):
        \end{lstlisting}
        \begin{solution}[0.25in]
        \begin{lstlisting}
                return f"{self.name} (species: \'{self.scientific}\', noise: \'{self.noise}\')"
        \end{lstlisting}
        \end{solution}
    \end{blocksection}
\end{blocksection}    
    