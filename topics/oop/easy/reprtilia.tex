\begin{blocksection}
    \question The Reprtilia class stores the common and scientific names of various reptiles.
    
    \textbf{Brief introduction to \lstinline{__repr__}} \\ The \lstinline{__repr__} magic method of objects returns the "official" string representation of an object. You can invoke it directly by calling \lstinline{repr(<some object>)}. However, \lstinline{__repr__} doesn't always return something that is easily readable, that is what \lstinline{__str__} is for. Rather, \lstinline{__repr__} ensures that all information about the object is present in the representation. When you ask Python to represent an object in the Python interpreter, it will automatically call \lstinline{repr} on that object and then print out the string that \lstinline{repr} returns.
    
    \vspace{1.5\baselineskip}
    
    \begin{lstlisting}
        """
        >>> python = Reprtilia("python", "pythonidae")
        >>> iguana = Reprtilia("iguana", "iguana")
        >>> python
        Reptilia('python', 'pythonidae')
        >>> f'Did you know that {python}?"
        'Did you know that a python is a pythonidae?'
        """
    
        class Reprtilia:
            def __init__(self, name, scientific):
                self.name = name
                self.scientific = scientific
    \end{lstlisting}
            def __repr__(self, other):
                _______________________________________________
            def __str__(self):
                _______________________________________________
\end{blocksection}
    
\begin{blocksection}
    \begin{lstlisting}
            def __repr__(self):

    \end{lstlisting}
\end{blocksection}

\begin{blocksection}
    \begin{lstlisting}
            def __str__(self):
                
    \end{lstlisting}
\end{blocksection}

\begin{blocksection}
    \question Your friend, Julian, decides to overhaul 
    \begin{lstlisting}
            def __repr__(self):
                
    \end{lstlisting}
\end{blocksection}    
    