\begin{blocksection}
\question What would Python display? Write the result of executing the following code and prompts. If nothing would happen, write "Nothing". If an error occurs, write "Error".

\textbf{Brief introduction to \lstinline{__repr__}} \\ The \lstinline{__repr__} magic method of objects returns the "official" string representation of an object. You can invoke it directly by calling \lstinline{repr(<some object>)}. However, \lstinline{__repr__} doesn't always return something that is easily readable, that is what \lstinline{__str__} is for. Rather, \lstinline{__repr__} ensures that all information about the object is present in the representation. When you ask Python to represent an object in the Python interpreter, it will automatically call \lstinline{repr} on that object and then print out the string that \lstinline{repr} returns.

\vspace{1.5\baselineskip}

\begin{lstlisting}
class Jedi:
    lightsaber = "blue"
    force = 25
    def __init__(self, name):
        self.name = name
    def train(self, other):
        other.force += self.force / 5
    def __repr__(self):
        # __repr__ will be covered more next week
        # 
        return "Jedi " + self.name
\end{lstlisting}
\end{blocksection}

\begin{blocksection}
\begin{lstlisting}
>>> anakin = Jedi("Anakin")
>>> anakin.lightsaber, anakin.force
\end{lstlisting}
\begin{solution}[.2in]
("blue", 25)
\end{solution}

\begin{lstlisting}
>>> anakin.lightsaber = "red"
>>> anakin.lightsaber 
\end{lstlisting}
\begin{solution}[.2in]
"red"
\end{solution}

\begin{lstlisting}
>>> Jedi.lightsaber 
\end{lstlisting}
\begin{solution}[.2in]
"blue"
\end{solution}

\begin{lstlisting}
>>> obiwan = Jedi("Obi-wan")
>>> anakin.master = obiwan
>>> anakin.master
\end{lstlisting}
\begin{solution}[.2in]
Jedi Obi-wan
\end{solution}

\begin{lstlisting}
>>> Jedi.master
\end{lstlisting}
\begin{solution}[.2in]
Error
\end{solution}

\begin{lstlisting}
>>> obiwan.force += anakin.force
>>> obiwan.force, anakin.force
\end{lstlisting}
\begin{solution}[.2in]
(50, 25)
\end{solution}

\begin{lstlisting}
>>> obiwan.train(anakin)
>>> obiwan.force, anakin.force
\end{lstlisting}
\begin{solution}[.2in]
(50, 35)
\end{solution}

\begin{lstlisting}
>>> Jedi.train(obiwan, anakin)
>>> obiwan.force, anakin.force
\end{lstlisting}
\begin{solution}[.2in]
(50, 45)
\end{solution}
\end{blocksection}
