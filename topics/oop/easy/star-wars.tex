\begin{blocksection}
\question What would Python display? Write the result of executing the following code and prompts. If nothing would happen, write "Nothing". If an error occurs, write "Error".

\begin{lstlisting}
class ForceWielder():
    force = 25

    def __init__(self, name):
        self.name = name

    def train(self, other):
        other.force += self.force / 5

    def __str__(self):
        return self.name

class Jedi(ForceWielder):
    lightsaber = "blue"

    def __str__(self):
        return "Jedi " + self.name

    def __repr__(self):
        return f"Jedi({repr(self.name)})"

class Sith(ForceWielder):
    lightsaber = "red"
    num_sith = 0

    def __init__(self, name):
        super().__init__(name)
        Sith.num_sith += 1
        if self.num_sith != 2:
            print("Two there should be. No more, no less.")
        
    def __str__(self):
        return "Darth " + self.name

    def __repr__(self):
        return f"Sith({repr(self.name)})"
\end{lstlisting}
\end{blocksection}

\newpage
\begin{lstlisting}
>>> anakin = Jedi("Anakin")
>>> anakin.lightsaber, anakin.force
\end{lstlisting}
\begin{solution}[.2in]
\begin{lstlisting}
("blue", 25)
\end{lstlisting}
\end{solution}

\begin{lstlisting}
>>> obiwan = Jedi("Obi-wan")
>>> anakin.master = obiwan
>>> anakin.master
\end{lstlisting}
\begin{solution}[.2in]
\begin{lstlisting}
Jedi("Obi-wan")
\end{lstlisting}
\end{solution}

\begin{lstlisting}
>>> Jedi.master
\end{lstlisting}
\begin{solution}[.2in]
\begin{lstlisting}
AttributeError
\end{lstlisting}
\end{solution}

\begin{lstlisting}
>>> obiwan.force += anakin.force
>>> obiwan.force, anakin.force
\end{lstlisting}
\begin{solution}[.2in]
\begin{lstlisting}
(50, 25)
\end{lstlisting}
\end{solution}

\begin{lstlisting}
>>> obiwan.train(anakin)
>>> obiwan.force, anakin.force
\end{lstlisting}
\begin{solution}[.2in]
\begin{lstlisting}
(50, 35.0)
\end{lstlisting}
\end{solution}

\begin{lstlisting}
>>> Jedi.train(obiwan, anakin)
>>> obiwan.force, anakin.force
\end{lstlisting}
\begin{solution}[.2in]
\begin{lstlisting}
(50, 45.0)
\end{lstlisting}
\end{solution}

\begin{lstlisting}
>>> sidious = Sith("Sidious")
\end{lstlisting}
\begin{solution}[.2in]
\begin{lstlisting}
Two there should be. No more, no less.
\end{lstlisting}
\end{solution}

\begin{lstlisting}
>>> ForceWielder.train(sidious, anakin)
>>> anakin.lightsaber = "red"
>>> anakin.lightsaber, anakin.force
\end{lstlisting}
\begin{solution}[.2in]
\begin{lstlisting}
("red", 50.0)
\end{lstlisting}
\end{solution}

\begin{lstlisting}
>>> Jedi.lightsaber 
\end{lstlisting}
\begin{solution}[.2in]
\begin{lstlisting}
"blue"
\end{lstlisting}
\end{solution}

\begin{lstlisting}
>>> print(Sith("Vader"), Sith("Maul").num_sith)
\end{lstlisting}
\begin{solution}[.2in]
\begin{lstlisting}
Two there should be. No more, no less.
Darth Vader 3
\end{lstlisting}
\end{solution}

\begin{lstlisting}
>>> rey = ForceWielder("Rey")
>>> rey
\end{lstlisting}
\begin{solution}[.2in]
\begin{lstlisting}
<__main__.ForceWielder object>
\end{lstlisting}
\end{solution}

\begin{lstlisting}
>>> rey.lightsaber
\end{lstlisting}
\begin{solution}[.2in]
\begin{lstlisting}
AttributeError
\end{lstlisting}
\end{solution}

\begin{questionmeta}
To address a point brought up in previous semesters: You may realize that as both a ForceWielder and Jedi, \lstinline{anakin} has no attribute named \lstinline{master} at first. Note to students that it is possible in our OOP objects for us to declare object attributes on the fly. Also communicate to students that while you can declare object attributes on the fly, that does not mean that such attributes persist to the class the object exists in as a whole.
\newline
In my opinion, going through an example like this is far more helpful for students than a mini-lecture. Try to foresee some questions and confusions might have and how you might address them, for example:
\begin{itemize}
    \item Why, in the \lstinline{__init__} method of \lstinline{Sith} can we use \lstinline{self.num_sith} instead of \lstinline{Sith.num_sith}? And why can't we write \lstinline{self.num_sith += 1}?
    \item Why does evaluating \lstinline{rey} give us \lstinline{<__main__.ForceWielder object>}, but this is not the case when we evaluated \lstinline{anakin.master}? 
    \item What's going on with \lstinline{ForceWielder.train(sidious, anakin)}? 
    \item Can we write \lstinline{Jedi.train(sidious, rey)}, even though neither \lstinline{rey} nor \lstinline{sidious} are Jedi? 
\end{itemize}
These are also questions you could bring up if students don't ask them. 
\end{questionmeta}