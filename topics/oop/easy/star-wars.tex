\begin{blocksection}
\question What would Python display? Write the result of executing the following code and prompts. If nothing would happen, write "Nothing". If an error occurs, write "Error".

Note: \lstinline{f-strings}, the strings that have an \lstinline{f} in front of them, replace the \lstinline| {variable} | in the string with \lstinline{variable}, allowing us to print out different strings based on what \lstinline{variable} is passed in

\begin{lstlisting}
class ForceWielder():
    force = 25

    def __init__(self, name):
        self.name = name

    def train(self, other):
        other.force += self.force / 5

    def __str__(self):
        return self.name

class Jedi(ForceWielder):
    lightsaber = "blue"

    def __str__(self):
        return "Jedi " + self.name

    def __repr__(self):
        return f"Jedi({repr(self.name)})"

class Sith(ForceWielder):
    lightsaber = "red"
    num_sith = 0

    def __init__(self, name):
        super().__init__(name)
        Sith.num_sith += 1
        if self.num_sith != 2:
            print("Two there should be. No more, no less.")
        
    def __str__(self):
        return "Darth " + self.name

    def __repr__(self):
        return f"Sith({repr(self.name)})"
\end{lstlisting}
\end{blocksection}

\newpage
\begin{lstlisting}
>>> anakin = Jedi("Anakin")
>>> anakin.lightsaber, anakin.force
\end{lstlisting}
\begin{solution}[.2in]
\begin{lstlisting}
("blue", 25)
\end{lstlisting}
\end{solution}

\begin{lstlisting}
>>> obiwan = Jedi("Obi-wan")
>>> anakin.master = obiwan
>>> anakin.master
\end{lstlisting}
\begin{solution}[.2in]
\begin{lstlisting}
Jedi("Obi-wan")
\end{lstlisting}
\end{solution}

\begin{lstlisting}
>>> Jedi.master
\end{lstlisting}
\begin{solution}[.2in]
\begin{lstlisting}
AttributeError
\end{lstlisting}
\end{solution}

\begin{lstlisting}
>>> obiwan.force += anakin.force
>>> obiwan.force, anakin.force
\end{lstlisting}
\begin{solution}[.2in]
\begin{lstlisting}
(50, 25)
\end{lstlisting}
\end{solution}

\begin{lstlisting}
>>> obiwan.train(anakin)
>>> obiwan.force, anakin.force
\end{lstlisting}
\begin{solution}[.2in]
\begin{lstlisting}
(50, 35.0)
\end{lstlisting}
\end{solution}

\begin{lstlisting}
>>> Jedi.train(obiwan, anakin)
>>> obiwan.force, anakin.force
\end{lstlisting}
\begin{solution}[.2in]
\begin{lstlisting}
(50, 45.0)
\end{lstlisting}
\end{solution}

\begin{lstlisting}
>>> sidious = Sith("Sidious")
\end{lstlisting}
\begin{solution}[.2in]
\begin{lstlisting}
Two there should be. No more, no less.
\end{lstlisting}
\end{solution}

\begin{lstlisting}
>>> ForceWielder.train(sidious, anakin)
>>> anakin.lightsaber = "red"
>>> anakin.lightsaber, anakin.force
\end{lstlisting}
\begin{solution}[.2in]
\begin{lstlisting}
"red", 
\end{lstlisting}
\end{solution}

\begin{lstlisting}
>>> Jedi.lightsaber 
\end{lstlisting}
\begin{solution}[.2in]
\begin{lstlisting}
"blue"
\end{lstlisting}
\end{solution}

\begin{lstlisting}
>>> print(Sith("Vader"), Sith("Maul").num_sith)
\end{lstlisting}
\begin{solution}[.2in]
\begin{lstlisting}
Two there should be. No more, no less.
Darth Vader 3
\end{lstlisting}
\end{solution}

\begin{lstlisting}
>>> rey = ForceWielder("Rey")
>>> rey
\end{lstlisting}
\begin{solution}[.2in]
\begin{lstlisting}
<__main__.ForceWielder object>
\end{lstlisting}
\end{solution}

\begin{lstlisting}
>>> rey.lightsaber
\end{lstlisting}
\begin{solution}[.2in]
\begin{lstlisting}
AttributeError
\end{lstlisting}
\end{solution}