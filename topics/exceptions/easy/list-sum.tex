\begin{blocksection}
\question \lstinline{list_sum} is a function that adds up the elements of a list and returns the sum. However, we have mixed types in our list! Write \lstinline{list_sum} such that rather than using any \lstinline{if} or \lstinline{for} statements, it uses exceptions to catch these errors.

\end{blocksection}

\ifprintanswers\else
\begin{lstlisting}
def list_sum(lst):
    """
    >>> list_sum([1, 2, 3, 4, 5])
    15
    >>> list_sum([1, "2", 3, "4", 5])
    9
    >>> list_sum(["1", "2", 3, 4, 5])
    12
    """
    i = 0
    total = 0
    while True:
        try:
            ____________
        except ____________:
            return ____________
        except ____________:
            ____________
            continue
        i += 1
\end{lstlisting}
\fi

\begin{solution}
\begin{blocksection}
\begin{lstlisting}
def list_sum(lst):
    """
    >>> list_sum([1, 2, 3, 4, 5])
    15
    >>> list_sum([1, "2", 3, "4", 5])
    9
    >>> list_sum(["1", "2", 3, 4, 5])
    12
    """
    i = 0
    total = 0
    while True:
        try:
            total = total + lst[i]
        except IndexError:
            return total
        except TypeError:
            i += 1
            continue
        i += 1
\end{lstlisting}
\end{blocksection}
\end{solution}

\begin{guide}
\begin{blocksection}
\textbf{Teaching Tips}
\begin{itemize}
    \item Write out a version with exception handling and ask students what type of errors will be raised:
    \begin{itemize}
        \item Not allowing \lstinline{if} statements: checking for \lstinline{TypeError} allows us to determine when we have a non-integer value.
        \item Not allowing \lstinline{for} statements: checking for \lstinline{IndexError} allow us to determine when we've reached the end of a list (and are going out of bounds). 
    \end{itemize}
    \item Remind students on how to use \lstinline{try} and \lstinline{except} statements, as well as how to catch only a specific type of error. 
    
\end{itemize}
\end{blocksection}
\end{guide}