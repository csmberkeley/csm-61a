\begin{blocksection}
\question A \textit{digraph} is any pair of immediately adjacent letters; for example, ``otto'' contains three digraphs: ``ot'', ``tt'', and ``to''. Write a function \lstinline{count_digraphs}, which takes a string, \lstinline{text} and a list of letters, \lstinline{alphabet} and analyzes the frequency of digraphs in \lstinline{text} pertaining to the specific letters in \lstinline{alphabet}. Specifically, \lstinline{count_digraphs} returns a dictionary whose keys are the valid digraphs of \lstinline{text} and whose values are the number of times each digraph appears.

\begin{lstlisting}
def count_digraphs(text, alphabet):
    """
    >>> count_digraphs("otto", ['o', 't'])
    {'ot': 1, 'tt': 1, 'to': 1}
    >>> count_digraphs("otto", ['t'])
    {'tt': 1}
    >>> count_digraphs("6161 6", ['6', '1'])
    {'61': 2, '16': 1}
    >>> count_digraphs("lalala", ['l', 'a'])
    {'la': 3, 'al': 2}
    """
\end{lstlisting}
\end{blocksection}


\begin{solution}
\begin{lstlisting}
def count_digraphs(text, alphabet):
    freq = {}
    for i in range(len(text) - 1):
        if text[i] in alphabet and text[i + 1] in alphabet:
            digraph = text[i] + text[i + 1]
            if digraph in freq:
                freq[digraph] += 1
            else: 
                freq[digraph] = 1
    return freq
\end{lstlisting}
\end{solution}

\begin{questionmeta}
What's the point of this question? Well, analyzing the frequency of digraphs can be valuable in all sorts of situations, including cryptanalysis. More broadly, using dictionaries to count occurrences is a common programming task, and this problem aims to provide practice with that concept.

Students will probably need to look at the doc tests to fully understand the problem, including what we mean by `valid` and whether spaces should count or not. 

If your students are completely lost, walk through the doctests with them, asking how they would find digraphs by hand. This can lead them to identify which two characters they should check in each iteration.

Students may struggle with differentiating actions based on whether a digraph is already in the dictionary. Here are some guiding questions to help:
\begin{itemize}
    \item What happens when we encounter a new digraph that isn't already in the dictionary?
    \item How should we update the count for a digraph that is already present?
    \item What conditions must be met to add to the count of an existing digraph?
\end{itemize}

**Tips and Tricks**:
\begin{itemize}
    \item **Iterate Over Indices**: Use a loop that iterates through indices of the string, checking each character and the next one to form the digraph.
    \item **String Concatenation**: Remember that you can easily create a digraph by concatenating two characters with `text[i] + text[i + 1]`.
    \item **Dictionary Methods**: Utilize the `get` method for dictionaries, which can simplify checking if a key exists and providing a default value if it doesn't. For example, `freq[digraph] = freq.get(digraph, 0) + 1`.
    \item **Edge Cases**: Discuss how to handle cases where the text is very short or contains no valid letters. Ensure students consider input validation.
    \item **Testing**: Encourage students to add their own test cases, especially edge cases, to ensure their function behaves as expected.
\end{itemize}

By addressing these questions and employing these tips, students will better understand the logic behind updating the dictionary and how to implement the function effectively.
\end{questionmeta}
