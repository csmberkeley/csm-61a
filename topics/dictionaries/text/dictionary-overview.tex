\begin{blocksection}
Dictionaries are another useful Python data structure that store a collection of items. However, instead of assigning each item a numerical index, each \textbf{value} in a dictionary is mapped to by some \textbf{key}. 

Dictionaries are denoted with curly braces and use much of the syntax---including item selection with square brackets, membership testing with \lstinline{in}, and length checking with \lstinline{len}---is the same as that of sequences. Consider the following ``Big'' example:
\begin{lstlisting}
>>> big_game_wins = {"Cal": 48, "Stanford": 65}
>>> big_game_wins 
{"Cal": 48, "Stanford": 65}
>>> big_game_wins["Stanford"]
65
>>> big_game_wins["Cal"]
48
>>> big_game_wins["Cal"] += 1
>>> big_game_wins["Cal"]
49
\end{lstlisting}
\end{blocksection}

\begin{blocksection}
\begin{lstlisting}
>>> list(big_game_wins.keys())
["Cal", "Stanford"]
>>> list(big_game_wins.values())
[49, 65]

>>> "Cal" in big_game_wins
True
>>> "Tie" in big_game_wins
False
>>> 65 in big_game_wins
False
\end{lstlisting}
\end{blocksection}

\begin{blocksection}
\begin{lstlisting}
>>> big_game_wins["Tie"]
KeyError: Tie
>>> big_game_wins["Tie"] = 11
>>> big_game_wins["Tie"]
11
\end{lstlisting}
\end{blocksection}

\begin{meta}
Here, I decided to not list out everything a dictionary can do but rather teach by giving a long example. I find that students tend to glaze over when they're asked to look at something that long, so I really recommend walking through the whole process. 

From a technical standpoint, dictionaries are ordered in Python. But your students don't need to know that.
\end{meta}