\begin{blocksection}
\question Given the following code block, what is outputted by the lines that follow?

\begin{lstlisting}
def foo():
    a = 0
    if a == 0:
        print("Hello")
        yield a
        print("World")

>>> foo()
\end{lstlisting}
\begin{solution}[0.25in]
\begin{lstlisting}
<generator object>
\end{lstlisting}
\end{solution}

\begin{lstlisting}
>>> foo_gen = foo()
>>> next(foo_gen)
\end{lstlisting}

\begin{solution}[0.5in]
\begin{lstlisting}
Hello
0
\end{lstlisting}
\end{solution}

\begin{lstlisting}
>>> next(foo_gen)
\end{lstlisting}

\begin{solution}[0.5in]
\begin{lstlisting}
World
StopIteration
\end{lstlisting}
\end{solution}
\end{blocksection}

\begin{lstlisting}
>>> for i in foo():
...   print(i)
\end{lstlisting}

\begin{solution}[0.5in]
\begin{lstlisting}
Hello
0
World
\end{lstlisting}
\end{solution}
\begin{blocksection}
\question Write a new function \lstinline$goo$ that satisfies the following doctests:
\\Optional: then write the body in one line using \lstinline{range(1, 11)}.
\begin{lstlisting}
>>> a = list(goo())
>>> a
[1, 2, 3, 4, 5, 6, 7, 8, 9, 10]
\end{lstlisting}

\begin{solution}[0.50in]
We can yield multiple times with a \lstinline$while$ statement and incrementing
\lstinline$a$:

\begin{lstlisting}
def goo():
    a = 0
    while a < 10:
        a += 1
        yield a
\end{lstlisting}

Alternatively, we can write it in one line using yield from:
\begin{lstlisting}
def goo():
    yield from range(1, 11)
\end{lstlisting}
\end{solution}
\end{blocksection}

\begin{blocksection}
\begin{guide}
\textbf{Teaching Tips}
\begin{itemize}
\item Emphasize heavily the fact that when generators are called, they return a generator object. They do NOT start executing their function body until after \texttt{next} is called! (So what does that first line return? A generator object!)
\item Remind students that generator objects are independent from one another; if you create a new one from calling the same function again, it starts from the beginning again. Each generator on its own, however, remembers where it stopped after the previous \texttt{next} call, so that it can resume the next time you call \texttt{next}.
\item What happens when there are no more \texttt{yield} statements, like in the second \texttt{call} on \texttt{foo\_gen}? The generator has reached the end of all possible values to iterate over, and so it returns a StopIteration error.
\item If you stick a generator object inside a for loop (or a list, for that matter), it will go all the way through from start to finish, outputting each yield value after another.
\begin{itemize}
\item Careful, however: `start' doesn't necessarily mean the very first lines of the function or the first yield call; if you feed in a generator on which you've already called \texttt{next}, its "start" will be where it last left off.
\end{itemize}
\item goo() can serve as a good introduction to the difference between yield and yield from! Go over both solutions and compare them.
\end{itemize}
\end{guide}
\end{blocksection}
