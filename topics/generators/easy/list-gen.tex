\question
\begin{parts}
\part
Implement \texttt{n\_apply}, which takes in 3 inputs \texttt{f}, \texttt{n}, \texttt{x}, and outputs the result of applying \texttt{f}, a function, \texttt{n} times to \texttt{x}. For example, for \texttt{n = 3}, output the result of \texttt{f(f(f(x)))}.

\begin{lstlisting}
def n_apply(f, n, x):
  """
  >>> n_apply(lambda x: x + 1, 3, 2)
  5
  """

  for __________________________:

    x = ________________________

  return _______________________
\end{lstlisting}

\begin{solution}
\begin{lstlisting}
def n_apply(f, n, x):
  for i in range(n):
    x = f(x)
  return x
\end{lstlisting}
\end{solution}

\begin{guide}
Run through a quick example problem ( the doc test is a good one), and ask the students how they get call a function on a value n number of times, and slowly lead them towards iteration
\end{guide}

\part
Now implement \texttt{list\_gen}, which takes in some list of integers \texttt{lst} and a \texttt{function f}. For the element at index \texttt{i} of \texttt{lst}, \texttt{list\_gen} should apply \texttt{f} to the element \texttt{i} times and yield this value \texttt{lst[i]} times. You may use \texttt{n\_apply} from the previous part.

\begin{lstlisting}
def list_gen(lst, f):
  """
  >>> a = list_gen([1, 2, 3], lambda x: x + 1)
  >>> list(a)
  [1, 3, 3, 5, 5, 5]
  """

  for __________________________:

    yield from [_______________________________________]
\end{lstlisting}

\begin{solution}
\begin{lstlisting}
def list_gen(lst, f):
  for i in range(len(lst)):
    yield from [n_apply(f, i, lst[i]) for j in range(lst[i])]
\end{lstlisting}
\end{solution}
\begin{guide}
Go over what \texttt{yield from} does (and give a very simple example of how yield from can be implemented with just yield)

Use one example and then go over what each value should be used for.
For example, for the doctest, we have the number 2, we want to call \texttt{f} on 2 exactly 1 time (because it’s the ith index) and we want to yield it 2 times (because lst[i] is 2)

This helps identify what value should go where (eg i should go in the \texttt{n\_apply} call, and lst[i] is used inside the yield from to yield the return from \texttt{n\_apply} that many times)
If students are still struggling, fill in the parts like range and \texttt{n\_apply}, but without the arguments and have students fill in the values
Do have them understand the logic of why we use a for loop inside the yield from (and what happens when we do this)

\end{guide}
\end{parts}
