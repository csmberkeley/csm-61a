\question
\begin{parts}
\part
Implement \texttt{n\_apply}, which takes in 3 inputs \texttt{f}, \texttt{n}, \texttt{x}, and outputs the result of applying \texttt{f}, a function, \texttt{n} times to \texttt{x}. For example, for \texttt{n = 3}, output the result of \texttt{f(f(f(x)))}.

\begin{lstlisting}
def n_apply(f, n, x):
  """
  >>> n_apply(lambda x: x + 1, 3, 2)
  5
  """
  for __________________:
    x = ______________
  return _______________
\end{lstlisting}

\begin{solution}
\begin{lstlisting}
def n_apply(f, n, x):
  for i in range(n):
    x = f(x)
  return x
\end{lstlisting}
\end{solution}

\part
Now implement \texttt{list\_gen}, which takes in some list of integers \texttt{lst} and a \texttt{function f}. For the element at index \texttt{i} of \texttt{lst}, \texttt{list\_gen} should apply \texttt{f} to the element \texttt{i} times and yield this value \texttt{lst[i]} times.

\begin{lstlisting}
def list_gen(lst, f):
  """
  >>> a = list_gen([1, 2, 3], lambda x: x + 1)
  >>> list(a)
  [1, 3, 3, 5, 5, 5]
  """
  for _________________:
    yield from [___________________________________]
\end{lstlisting}

\begin{solution}
\begin{lstlisting}
def list_gen(lst, f):
  for i in range(len(lst)):
    yield from [n_apply(f, i, lst[i]) for j in range(lst[i])]
\end{lstlisting}
\end{solution}
\end{parts}
