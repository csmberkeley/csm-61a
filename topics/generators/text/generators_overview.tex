An \textbf{iterable} is any container that can be processed sequentially. Think of an iterable as anything you can loop over, such as lists or strings.  You can see this in \lstinline{for} loops, which sequentially loop through each element of a sequence. The anatomy of the for loop can be described as:
\begin{lstlisting}
for some_var in iterable:
    <do something with some_var>
\end{lstlisting}
An \textbf{iterator} remembers where it is during its iteration. Though an iterator is an iterable, the reverse is not necessarily true. Think of an iterable as a book whereas an iterator is a bookmark. We can use the \lstinline{iter} function to create an iterator from an object that is iterable (such as a list). 

\textbf{Generators}, which are a specific type of \textbf{iterators}, are created using the traditional function definition syntax in Python (\lstinline{def}) with the body of the function containing one or more \lstinline{yield} statements. When a generator (a function that has \lstinline{yield} in the body) is called, it returns a generator object. When we call the generator object, we evaluate the body of the function until we have yielded a value. The \lstinline{yield} statement pauses the function, yields the value, saves the local state so that evaluation can be resumed right where it left off.  \lstinline{yield} operates similarly to a return statement. \lstinline{yield from} is another way to yield values. When we \lstinline{yield from} a list, it yields each element in our list one at a time. 