On a conceptual level, \textbf{iterables} are simply objects whose elements can be iterated over. Think of an iterable as anything you can use in a \lstinline{for} loop, such as ranges, lists, strings, or dictionaries.

On a technical level, iterables are a bit more complicated. An \textbf{iterator} is an object on which you can (repeatedly) call \lstinline{next}, which will return the next element of a sequence. For example, if \lstinline{it} is an iterator representing the sequence $1, 2, 3$, then we could do the following: 
\begin{lstlisting}
>>> next(it)
1
>>> next(it)
2
>>> next(it)
3
>>> next(it)
StopIteration
\end{lstlisting}

\lstinline{StopIteration} is an exception that is raised when an iterator has no more elements to produce; it's how we know we've reached the end of an iterator. Iterators that will never produce a \lstinline{StopIteration} exception are called \textit{infinite}. 

Under this regime, an iterable is formally defined as an object that can be turned into an iterator by passing it into the \lstinline{iter} function. When you iterate over an iterable, Python first uses \lstinline{iter} to create an iterator from the iterable and then iterates over the iterator. The simple \lstinline{for} loop syntax abstracts away this fact. 
f
There are a few useful functions that act on iterables that are particularly useful: 
\begin{itemize}
    \item \lstinline{map(f, it)}: Returns an iterator that produces each element of \lstinline{it} with the function \lstinline{f} applied to it.
    \item \lstinline{filter(pred, it)}: Returns an iterator that includes only the elements of \lstinline{it} where the predicate function \lstinline{pred} returns true. 
    \item \lstinline{reduce(f, it, init)}: Reduces \lstinline{it} to a single value by repeatedly calling the two-argument function \lstinline{f} on the elements of \lstinline{it}: \lstinline{reduce(add, [1, 2, 3])} $\rightarrow$ \lstinline{6}. Optionally, an initializer may be provided: \lstinline{reduce(add, [1], 5)} $\rightarrow$ \lstinline{6}. 
\end{itemize}

\begin{meta}
Technically, \lstinline{map} and \lstinline{filter} are not functions but classes, but that is not a distinction we need to make. 
\end{meta}

\textbf{Generators}, which are a specific type of iterator, are created using the traditional function definition syntax in Python (\lstinline{def}) with the body of the function containing one or more \lstinline{yield} statements. When a generator function (a function that has \lstinline{yield} in the body) is called, it returns a generator object; the body of the function is not executed. Only when we call \lstinline{next} on the generator object is the body executed until we hit a \lstinline{yield} statement. The \lstinline{yield} statement yields the value and pauses the function. \lstinline{yield from} is another way to yield values. When we \lstinline{yield from} another iterable, it yields each element from that other iterable one at a time. 

The following generators all represent the sequence $1,2,3$: 

\begin{minipage}[t]{0.2\textwidth}
\begin{lstlisting}
def a():
    yield 1
    yield 2
    yield 3
\end{lstlisting}
\end{minipage}
\begin{minipage}[t]{0.4\textwidth}
\begin{lstlisting}
def b():
    for x in range(1, 4):
        yield x
\end{lstlisting}
\end{minipage}
\begin{minipage}[t]{0.4\textwidth}
\begin{lstlisting}
def c():
    yield from b()
\end{lstlisting}
\end{minipage}

\begin{meta}
Something to really emphasize here is the difference between regular function execution and generator function execution. When you call a generator function, you do not begin executing the function body! You only begin executing the function body when \lstinline{next} is called on the generator object. You then pause when you hit a \lstinline{yield} statement. I like to tell my students that this is an ``abuse of notation'': they're coopting function syntax to do something completely different from what a function normally does. 

Another thing I like to emphasize is that it is impossible to go ``backward'' with iterators and generators. After all, we only have a \lstinline{next}, not a \lstinline{prev}!

You might find it advantageous to go over some of the examples more in depth. 

You may or may not find it useful to present students with an example of how iteration works behind the scene: 

\begin{center}
    \begin{minipage}[t]{0.4\textwidth}
    \begin{lstlisting}
    for x in "Hello":
        print(x)
    \end{lstlisting}
    \end{minipage}
    \begin{minipage}[t]{0.4\textwidth}
    \begin{lstlisting}
    it = iter("Hello")
    while True:
        try: 
            x = next(it)
            print(x)
        except StopIteration:
            pass
    \end{lstlisting}
    \end{minipage}
    \end{center}

It's possible this may confuse some students, so be cautious if you attempt to use this or a similar example. In particular, students may be confused by the infinite looping and the \lstinline{try} and \lstinline{except} blocks. While error handling isn't something super important in CS 61A, they should be able to use it specifically for dealing with iterators, so it might be a good idea to go over this a bit with your students. 
\end{meta}