\begin{blocksection}
\question Define \lstinline$all_sums$, a generator that iterates through all the sums that can be formed by adding the elements in \lstinline$lst$.
\begin{lstlisting}
def all_sums(lst):
    """
    >>> list(all_sums([]))
    [0]                      #sum nothing
    >>> list(all_sums([1, 2]))
    [3, 2, 1, 0]             #1 + 2, 2, 1, 0
    >>> list(all_sums([1, 2, 3]))
    [6, 5, 4, 3, 3, 2, 1, 0] #repeat sums are ok! (3 and 2+1)
    """
\end{lstlisting}

\begin{solution}[1.5in]
\begin{lstlisting}
    if len(lst) == 0:
        yield 0
    else:
        for sum_rest in all_sums(lst[1:]):
            yield sum_rest + lst[0]
            yield sum_rest
\end{lstlisting}
\end{solution}
\end{blocksection}

\begin{blocksection}
    \begin{guide}
        \textbf{Teaching Tips}
        \begin{itemize}
            \item This is a classic tree recursion problem but now in generator form!
            \item A tree diagram of how the list splits is a good visualization to draw
            \item If students are struggling to understand the problem, start from the base case of an empty list and work your way up with the sums of a length-1 list, length-2, etc.
            \item As always, the recursive leap of faith is helpful in understanding what \lstinline{all_sums(lst[1:])} returns.
            \item Even though this is a generator problem, we iterate over the call in a for loop so we can treat the function like it returns a list!
        \end{itemize}
    \end{guide}
\end{blocksection}
