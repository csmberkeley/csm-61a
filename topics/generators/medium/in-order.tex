\begin{blocksection}
\question
Define a generator function \lstinline{in_order}, which takes in a tree \lstinline{t}, and assume that \lstinline{t} has either 0 or 2 branches only. Fill in \lstinline{in_order} so that it returns a generator that yields the labels of \lstinline{t} in the following order: first branch node, parent node, second branch node.

\begin{lstlisting}
def in_order(t):
    """
    >>> t = Tree(0, [Tree(1), Tree(2, [Tree(3), Tree(4)])])
    >>> list(in_order(t))
    [1, 0, 3, 2, 4]
    """

    if _____________________:

        yield ______________________________
    else:

        lst = ______________________________

        for ________________________________:

            _________________________________
\end{lstlisting}
\end{blocksection}

\begin{blocksection}
\begin{solution}
\begin{lstlisting}
def in_order(t):
    if t.is_leaf():
        yield t.label
    else:
        lst = [in_order(t.branches[0]), [t.label], in_order(t.branches[1])]
        for elem in lst:
            yield from elem
\end{lstlisting}
\end{solution}
\end{blocksection}

\begin{guide}
\begin{blocksection}
\textbf{Teaching Tips}
    \begin{itemize}
    \item Trees are meant to be implemented recursively, and this should be emphasized to students.
    \item What is the base case of the problem? With trees it is typically the leaf, and it works out in this case, where there is only one item to yield.
    \item Draw out an example of a tree (maybe the doctest). What do we expect the recursive call on each of the branches to return (note that trees either have 0 or 2 branches)?
    \item After seeing what the recursive calls do, figure out how you combine the label, the left tree recursive call, and the right tree recursive call to get the desired result. Yielding the left recursive call's values, then the label, and then the right recursive call will give the inorder traversal.
    \item Since there is skeleton code, students will have to use the "yield from" keyword. Use examples to show how it should be used in comparison to "yield", and consider how the solution can be fit in the given lines.
    \end{itemize}
\end{blocksection}
\end{guide}