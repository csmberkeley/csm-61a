\question
Write a generator function \lstinline{num_elems} that takes in a possibly nested list of numbers \lstinline{lst} and yields the number of elements in each nested list before finally yielding the total number of elements (including the elements of nested lists) in \lstinline{lst}. For a nested list, yield the size of the inner list before the outer, and if you have multiple nested lists, yield their sizes from left to right.

\begin{lstlisting}
def num_elems(lst):
    """
    >>> list(num_elems([3, 3, 2, 1]))
    [4]
    >>> list(num_elems([1, 3, 5, [1, [3, 5, [5, 7]]]]))
    [2, 4, 5, 8]
    """

    count = _______________

    for ___________________________:

        if ________________________:

            for _________________________________:

                yield ___________________________

            _______________________________
        else:

            _______________________________

    yield ___________________________
\end{lstlisting} 
\begin{solution}
\begin{lstlisting}
def num_elems(lst):
    count = 0
    for elem in lst:
        if type(elem) is list:
            for c in num_elems(elem):
                yield c
            count += c
        else:
            count += 1
    yield count
\end{lstlisting}

\lstinline{count} refers to the number of elements in the current list \lstinline{lst} (including the number of elements inside any nested list). Determine the value of \lstinline{count} by looping through each element of the current list \lstinline{lst}. If we have an element \lstinline{elem} which is of type \lstinline{list}, we want to yield the number of elements in each nested list of \lstinline{elem} before finally yielding the total number of elements in \lstinline{elem}. We can do this with a recursive call to \lstinline{num_elems}. Thus, we yield all the values that need to be yielded using the inner for loop. The last number yielded by this inner loop is the total number of elements in \lstinline{elem}, which we want to increase \lstinline{count} by. Otherwise, if \lstinline{elem} is not a list, then we can simply increase \lstinline{count} by 1. Finally, yield the total count of the list.
\end{solution}

\begin{guide}
    \textbf{Teaching Tips}
    \begin{itemize}
       \item Double check with your students to make sure they understand the differences between iterables and iterators.
       \item When we call next(), we pick up from where the last yield statement ran.
       \item Try walking through one of the doctests if students are confused by what the problem is asking for.
    \end{itemize}
 \end{guide}
