\begin{blocksection}
\question Write a function that takes in a list \lstinline{nums} and returns a
new list with only the primes from \lstinline{nums}. Assume that
\lstinline{is_prime(n)} is defined. You may use a \lstinline{while} loop, a
\lstinline{for} loop, or a list comprehension.


\begin{lstlisting}
def all_primes(nums):
\end{lstlisting}

\begin{solution}[2in]
\begin{lstlisting}
    result = []
    for i in nums:
        if is_prime(i):
            result = result + [i]
    return result

    List comprehension:
    return [x for x in nums if is_prime(x)]
\end{lstlisting}
\end{solution}
\end{blocksection}

\begin{blocksection}
    \begin{guide}
    \textbf{Teaching Tips}
    \begin{enumerate}
            \item Students may be very familiar with list comprehensions at this point, so don’t spend too much time going over this question if your students don’t need it
            \item If they are stuck doing it with list comprehensions, state that one can approach it with a for loop
            \item Might help to go through how to convert from a for/while loop to a list comprehension
            \item If students are struggling with list comprehensions, it might be a good idea to explain that the entire list comprehension returns a list, instead of just acting like a for loop.
    \end{enumerate}
    \end{guide}
\end{blocksection}