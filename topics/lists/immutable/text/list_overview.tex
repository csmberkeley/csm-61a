\textbf{Lists Introduction:}

Lists are type of sequence, which means they are an ordered collection of values that has both length and the ability to select elements. 

\begin{lstlisting} 
>>> lst = [1, False, [2, 3], 4] # a list can contain anything
>>> len(lst) 	
4
>>> lst[0] 	
1
>>> lst[4] 	# Indices of a list only go up to the len(lst)
Error: list index out of range
\end{lstlisting}

We can iterate over lists using their index, or iterate of elements directly

\begin{lstlisting}
for index in range(len(lst)):
	# do things
for item in lst:
	# do things
\end{lstlisting}

\textbf{List comprehensions} are a useful way to iterate over lists when your desired result is a list.
\begin{lstlisting}
new_list2 = [<expression> for <element> in <sequence> if <condition>]
\end{lstlisting}

We can use \textbf{list splicing} to create a copy of a certain portion or all of a list 

\begin{lstlisting}
new_list = lst[<starting index>:<ending index>]
\end{lstlisting}

