\textbf{Lists Introduction:}

Lists are an ordered collection of values that can contain any type of data.

\textbf{List comprehensions}, which only work on lists, are a concise and powerful method to create a new list from another sequence. The syntax for a list comprehension is
   \begin{lstlisting}
   [<expression> for <element> in <sequence> if <condition>]
   \end{lstlisting}

   We could equivalently write the following: 
\begin{lstlisting}
	lst = []
	for <element> in <sequence>:
		if <condition>:
			lst = lst + [<expression>] # add current element to the list
\end{lstlisting}

The \lstinline{if <condition>} filter statement is optional. The following list comprehension doubles each odd element of [1, 2, 3, 4]:


\begin{lstlisting}
>>> [i * 2 for i in [1, 2, 3, 4] if i % 2 != 0] 
[2, 6]
\end{lstlisting}

Equivalent in \lstinline{for} loop syntax:

\begin{lstlisting}
	lst = []
	for i in [1, 2, 3, 4]:
		if i % 2 != 0:
			lst = lst + [i * 2] # add current element to the list
\end{lstlisting}

\begin{blocksection}
	\begin{guide}
	\textbf{Teaching Tips}
	\begin{itemize}
			\item \textbf{Box and Pointer Diagrams}: make sure students know how to draw them. Try explaining lists using box and pointer diagrams from the start so they can visualize how they interact with lists
			\begin{itemize}
				\item Be very careful with showing and explaining how pointers work. You can treat pointers as a distinct object in the list (e.g. this element of the list is a pointer), so that they have a solid foundation once we get to mutation (like how a shallow copy copies what’s in the box as is, which would be just the pointer)
				\item Pay attention to where students draw their pointers (e.g. if a pointer points to another list, make sure it points to the beginning of the list)
			\end{itemize}
			\item For students that have not had experience with lists or arrays, it might help to write up some common operations and keep it on one part of the board. Here are some tricky points:
			\begin{itemize}
				\item Zero indexing - might help to label indices with the first couple of box and pointer diagrams
				\item List slicing - what elements are included or not included?
				\begin{itemize}
					\item Inclusive start, exclusive end
					\item Probably don’t need to get into every little thing (like negative indices, skipping elements, etc.)
				\end{itemize}
				\item Deep copy vs shallow copy: when you copy a nested list, copy pointers to the nested list (a shallow copy) rather than make a new copy of the nested list (a deep copy).
					
			\end{itemize}
	\end{itemize}
	\end{guide}
\end{blocksection}
