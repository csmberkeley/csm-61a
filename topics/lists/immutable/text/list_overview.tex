\textbf{Lists Introduction:}

Lists are a data structure, an ordered collection of values that contain any type of data, even more lists itself!
This may be a new concept for you if you have background in other languages, like Java, where lists can only be made of the same type elements.

In order to write a list, we wrap our elements with square brackets, and separate elements with commas.

We can access specific properties of a list, such as the length, 
and the specific index of an item in the list, although it is important to note that
when we index into lists, we start at 0, and not 1.

\begin{lstlisting}
>>> lst = [1, False, [2, 3], 4] # a list can contain any data type
>>> len(lst)
4
>>> lst[0] # Indexing starts at 0
1
>>> type(lst[2]) # Can have lists within lists!
<class 'list'>
>>> lst[4] # Indexing ends at len(lst) - 1
Error: list index out of range
\end{lstlisting}

We can iterate over lists using their index, or iterate over elements directly

\begin{lstlisting}
a = [1, 2, 3 ,4]

for index in range(len(lst)):
	print(a[index])
for item in lst:
	print(item)
\end{lstlisting}

Both for loops will output:

1
2
3
4

\textbf{List comprehensions} are a concise and powerful method to create a new list out of sequences. The general syntax of list comprehensions follows:
\begin{lstlisting}
[<expression> for <element> in <sequence> if <condition>]
\end{lstlisting}

The \lstinline{if <condition>} is optional, and an equivalent for loop for a list comprehension is as follows:
\begin{lstlisting}
	lst = []
	for <element> in <sequence>:
		if <condition>:
			lst = lst + [<expression>] # add current element to the list
\end{lstlisting}

An example of list comprehensions in use:

\begin{lstlisting}
>>> [i * 2 for i in [1, 2, 3, 4] if i % 2 != 0] # iterate over numbers from [1, 2, 3, 4] and if they are odd, multiply them by 2
[2, 6]
\end{lstlisting}

Equivalence in for loop syntax:

\begin{lstlisting}
	lst = []
	for i in [1, 2, 3, 4]:
		if i % 2 != 0:
			lst = lst + [i * 2] # add current element to the list
\end{lstlisting}

We can use \textbf{list slicing} to create a copy of a certain portion or all of a list.

The general syntax for slicing a list \lstinline{lst} is as follows:

\begin{lstlisting}
	lst[<start index>:<end index>:<step size>]
\end{lstlisting}

Where the portion of the list we want to keep starts at \lstinline{<start index>} and ends one before
\lstinline{<end index>}. Once we have that portion of the list, if we want to only keep some elements, we can
add the optional parameter \lstinline{<step size>}, which will allow us skip or reverse those elements.

If we do not supply \lstinline{<start index>} or \lstinline{<end index>}, the default parameter, will be the
beginning, and the end of the list, respectively. 

\begin{lstlisting}
>>> lst = [1, 2, 3, 4, 5]
>>> lst[2:] # Create a new list with only the elements starting from the 2nd index
[3, 4, 5]
>>> lst[:3] # Create a new list with only the elements up until the 4th index
[1, 2, 3]
>>> lst[::-1] # Reverse the entire list
[5, 4, 3, 2, 1]
>>> lst[::2] # Create a new list while skipping every other element
[1, 3, 5]
\end{lstlisting}

\begin{blocksection}
	\begin{guide}
	\textbf{Teaching Tips}
	\begin{itemize}
			\item \textbf{Box and Pointer Diagrams}: make sure students know how to draw them. Try explaining lists using box and pointer diagrams from the start so they can visualize how they interact with lists
			\begin{itemize}
				\item Be very careful with showing and explaining how pointers work. You can treat pointers as a distinct object in the list (e.g. this element of the list is a pointer), so that they have a solid foundation once we get to mutation (like how a shallow copy copies what’s in the box as is, which would be just the pointer)
				\item Pay attention to where students draw their pointers (e.g. if a pointer points to another list, make sure it points to the beginning of the list)
			\end{itemize}
			\item For students that have not had experience with lists or arrays, it might help to write up some common operations and keep it on one part of the board. Here are some tricky points:
			\begin{itemize}
				\item Zero indexing - might help to label indices with the first couple of box and pointer diagrams
				\item List slicing - what elements are included or not included?
				\begin{itemize}
					\item Inclusive start, exclusive end
					\item Probably don’t need to get into every little thing (like negative indices, skipping elements, etc.)
				\end{itemize}
				\item Deep copy vs shallow copy: when you copy a nested list, copy pointers to the nested list (a shallow copy) rather than make a new copy of the nested list (a deep copy).
			\end{itemize}
	\end{itemize}
	\end{guide}
\end{blocksection}
