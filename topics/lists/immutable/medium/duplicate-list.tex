\begin{blocksection}
\question Write a function \lstinline{duplicate_list}, which takes in a list of positive integers and returns a new list with each element \lstinline{x} in the original list duplicated \lstinline{x} times.

\begin{lstlisting}
def duplicate_list(lst):
    """
    >>> duplicate_list([1, 2, 3])
    [1, 2, 2, 3, 3, 3]
    >>> duplicate_list([5])
    [5, 5, 5, 5, 5]
    """
    _______________________________
    
    for ____________________________:

         for ____________________________:

              __________________________________

    _______________________________

\end{lstlisting}

\begin{solution}
\begin{lstlisting}
new_list = []
for x in lst:
     for i in range(x):
          new_list = new_list + [x]
return new_list
\end{lstlisting}
\end{solution}
\end{blocksection}

\begin{meta}
\textbf{Teaching Tips}
    \begin{enumerate}
            \item If students have trouble arriving at the solution, walk through the intuition of nested for loops
            and discuss what each loop represents. For example, the first loop represents iterating over each element
            of the list and the second one represents repeating that element.
            \item This is a good problem to emphasize how we can format our logic and approach to problems based on 
            the skeleton code
    \end{enumerate}
\end{meta}

