\begin{blocksection}
\question Given some list \texttt{lst}, possibly a deep list, mutate \texttt{lst} to have the accumulated sum of all elements so far in the list. If there is a nested list, mutate it to similarly reflect the accumulated sum of all elements so far in the nested list. Return the total sum of the original \texttt{lst}.

\emph{Hint:} The \lstinline$isinstance$ function returns True for \lstinline$isinstance(l, list)$ if \texttt{l} is a list and False otherwise.

\begin{lstlisting}
def accumulate(lst):
    """
    >>> l = [1, 5, 13, 4]
    >>> accumulate(l)
    23
    >>> l
    [1, 6, 19, 23]
    >>> deep_l = [3, 7, [2, 5, 6], 9]
    >>> accumulate(deep_l)
    32
    >>> deep_l
    [3, 10, [2, 7, 13], 32]
    """
    sum_so_far = 0
    for ________________________________________:
	________________________________________
        if isinstance(___________________, list):
	    inside = ___________________________
            ____________________________________
        else:
            ____________________________________
            ____________________________________
    return ___________________________________
\end{lstlisting}
\end{blocksection}

\begin{blocksection}
\begin{solution}[1in]
\begin{lstlisting}
def accumulate(lst):
    sum_so_far = 0
    for i in range(len(lst)):
        item = lst[i]
        if isinstance(item, list):
            inside = accumulate(item)
            sum_so_far += inside
        else:
            sum_so_far += item
            lst[i] = sum_so_far
    return sum_so_far
\end{lstlisting}
\end{solution}
\end{blocksection}


\begin{guide}
\begin{blocksection}
\textbf{Teaching Tips}

\begin{itemize}
\item To keep track of the accumulated sum we need to create a variable in the function that keeps track of the overall sum of the list so we can mutate it.
\item Make sure to emphasize the distinction between \lstinline{for item in lst} and \lstinline{for i in range(len(lst))}. (We need \lstinline{i} in order to mutate the list.) Perhaps allow your students to first make the mistake of using the former, so that they may realize this difference on their own. Granted, if they aren't able to catch this on their own, do nudge them in the right direction. 
\item Why do we need a conditional in the for loop? What do we do when we have a nested list?
\begin{enumerate}
\item Integers:  For integers we just add the value to the ongoing sum and then mutate the current index of the list to be the cumulative sum
\item Lists: We need to break down the list and get the values, both so that we can update them and so that we can add it into our sum.  However, we don’t know how many levels of nesting we have in our list
\begin{itemize} 
\item We could have something like \lstinline{[1, [2, [3]]]}, so we need a function that will sum up the 
values from a potentially nested list.  Do we have a function that does this?
\item \textbf{Emphasize the recursive leap of faith when calling accumulate on the inner list}
\end{itemize}
\end{enumerate}
\item We return the accumulated sum of the list which includes all values, even the nested ones because of the recursive call.
\end{itemize}
\end{blocksection}
\end{guide}