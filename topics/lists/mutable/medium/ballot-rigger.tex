\begin{blocksection}
\question
\textbf{a. Ballot Rigger} In the 2020 election, Donald Trump's campaign advisors have decided to push back against the election fraud by rigging ballots themselves. Let's see how many ballots they were able to rig!

\begin{lstlisting}
class Ballot:
    choices = ___________________________
    def __init__(self, name, bubble):
        self.name = name
        self.vote = _________________________

california = [Ballot('nancy', 'c'), Ballot('jeremy', 'd'), Ballot('joe', 'b')]
white_house = [Ballot('donald', 'a'), Ballot('melania', 'a'), Ballot('mike', 'a')]
postman = []
postman.extend(white_house)
postman.append(california)
\end{lstlisting}
In the ballot, bubble ‘a’ corresponds to ‘Trump’, ‘b’ corresponds to ‘Biden’, ‘c’ corresponds to ‘West’, and ‘d’ corresponds to ‘Other’. Fill out the blanks above to categorize each voter's choice.
\begin{solution}[1in]
\begin{lstlisting}
class Ballot:
    choices = {'a': 'Trump','b':'Biden', 'c':'West', 'd':'Other'}
    def __init__(self, name, bubble):
        self.name = name
        self.vote = Ballot.choices[bubble]
california = [Ballot('nancy', 'c'), Ballot('jeremy', 'd'), Ballot('joe', 'b')]
white_house = [Ballot('donald', 'a'), Ballot('melania', 'a'), Ballot('mike', 'a')]
postman = []
postman.extend(white_house)
postman.append(california)
\end{lstlisting}
\end{solution}
\end{blocksection}

\begin{blocksection}
\vspace{1\baselineskip}
\textbf{b. Changing Ballots}
The Ballot Rigger takes all ballots and rigs them, changing votes for Biden and Other to Trump. Implement this in the lines of code above.
\vspace{1\baselineskip}
\begin{lstlisting}
class BallotRigger:
    def rig(self, ballot):
        ___________________________________
        	_______________________________

bad_guy = __________________
for ballot in california:
    bad_guy.__________________
for ballot in white_house:
    bad_guy.__________________


\end{lstlisting}
\begin{solution}[1in]
\begin{lstlisting}
class BallotRigger:
    def rig(self, ballot):
        if ballot.vote == 'Biden' or ballot.vote == 'Other':
            ballot.vote = 'Trump'
bad_guy = BallotRigger()
for ballot in california:
    bad_guy.rig(ballot)
for ballot in white_house:
    bad_guy.rig(ballot)
\end{lstlisting}
\end{solution}
\vspace{1\baselineskip}
\textbf{c. Tallying Votes}
After the Ballot Rigger has rigged the ballots, return the number of votes for each candidate, implement the \lstinline{count_votes} method, which adds votes to the count dictionary.
\emph{Hint:} How can we deal with nested lists of ballots?
\vspace{1\baselineskip}
\begin{lstlisting}
choices = {'Trump': 0 ,'Biden': 0, 'West': 0, 'Other': 0}
def count_votes(ballots):
    for b in ballots:
        if isinstance(________,________):
            _______________
        else:
            ____________ += 1

ballot_counter = postman[:]
count_ballots(ballot_counter)


\end{lstlisting}
\begin{solution}[1in]
\begin{lstlisting}
choices = {'Trump': 0 ,'Biden': 0, 'West': 0, 'Other': 0}
def count_votes(ballots):
    for b in ballots:
        if isinstance(b, list):
            count_votes(b)
        else:
            choices[b.vote] += 1

ballot_counter = postman[:]
count_votes(ballot_counter)
\end{lstlisting}
\end{solution}
\end{blocksection}



\begin{blocksection}
\vspace{1\baselineskip}
\textbf{d. Final Count}
Fill in the final tallies of votes for each candidate.
\emph{Hint:} It might help to draw box and pointer diagrams for \lstinline{postman} abd \lstinline{ballot_counter}.
\vspace{1\baselineskip}
\begin{lstlisting}
choices = {'Trump': __ ,'Biden': __, 'West': __, 'Other': __}

\end{lstlisting}
\begin{solution}[1in]
\begin{lstlisting}
choices = {'Trump': 5 ,'Biden': 0, 'West': 1, 'Other': 0}
\end{lstlisting}
\end{solution}

\end{blocksection}
