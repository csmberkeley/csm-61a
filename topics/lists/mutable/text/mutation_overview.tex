Let’s imagine it’s your first year at Cal, and you have signed up for your first classes!
\begin{lstlisting}
>>> classes = ["CS61A", "Math 53", "R1B", "Chem 1A"]
>>> classes_ptr = classes
>>> classes_copy = classes[:]
\end{lstlisting}

\begin{center}
\includegraphics[scale=0.75]{pointers.PNG}
\end{center}
After a few weeks, you realize that you cannot keep up with the workload and you need to drop a class. You’ve chosen to drop Chem 1A. Based on what we know so far, to change our classes list, we would have to create a new list with all the same elements as the original list except for Chem 1A. But that is silly, since all we really need to do is remove the Chem 1A element from our list.

We can fix this issue with list mutation. In Python, some objects, such as lists and dictionaries, are mutable, meaning that their contents or state can be changed over the course of program execution. Other objects, such as numeric types, tuples, and strings are immutable, meaning they cannot be changed once they are created. Therefore, instead of creating a new list, we can just call \lstinline{classes.pop()}, which removes the last element from the list.
\newpage
\begin{lstlisting}
>>> classes.pop() # pop returns whatever item it removed
"Chem 1A"
\end{lstlisting}

\begin{center}
\includegraphics[scale=0.75]{pointers_mutate.PNG}
\end{center}

Here are more list methods that mutate:
\newline
% Note to future semseters: replace this text overview with this table. SP23 UPDATE; table doesn't exist in repo :( WAIT LOL NVM IT DOES just not pathed correctly
\includegraphics[width=.9\textwidth]{list-mutation.png}

\text{(credits: Mihira Patel)}

\begin{guide}
	\textbf{Teaching Tips}
	\begin{itemize}
			\item Common Misconceptions:
			\begin{itemize}
				\item Students may be confused about the return value of mutation functions
				\begin{itemize}
					\item Try contrasting \lstinline{pop} with \lstinline{remove}, showing them how only \lstinline{pop} returns the element
					\item Tell them to reference the list mutability table
				\end{itemize}
			\end{itemize}
			\item The objectives for students are to:
			\begin{itemize}
				\item Distinguish between mutable and non-mutable objects
				\item The effects and return values of mutation functions
				\item Become comfortable with pointers and how to copy objects
			\end{itemize}
	\end{itemize}
\end{guide}
