\begin{blocksection}

\question Implement \lstinline{filter-lst}, which takes in a one-argument function \lstinline{f} and
a list \lstinline{lst}, and returns a new list containing only the elements in \texttt{lst}
for which \texttt{f} returns true. Your function must use a constant number of active frames.

\textit{Hint: recall that the built-in \lstinline{append} procedure concatenates two lists together. }

\begin{lstlisting}
;Doctests
scm> (filter-lst (lambda (x) (> x 2)) '(1 2 3 4 5))
(3 4 5)

(define (filter-lst f lst)










)
\end{lstlisting}

\begin{solution}[0.5in]
\begin{lstlisting}
(define (filter-lst f lst)
  (define (filter-tail lst so-far)
     (cond ((null? lst) so-far)
           ((f (car lst)) (filter-tail (cdr lst)
                          (append so-far (list (car lst)))))
           (else (filter-tail (cdr lst) so-far))))
  (filter-tail lst nil))
\end{lstlisting}
\end{solution}

\end{blocksection}

\begin{guide}
\begin{blocksection}
\textbf{Teaching Tips}
  Recall that normal recursion uses many frames, so when this function requires a constant number of active frames, it is referring to tail recursion. Ensure that your students know this before approaching the problem.

  When explaining this problem I would once again use a general formula to approaching tail-recursive problems and apply it to this problem
  \begin{enumerate}
    \item What variables do you need to keep track of in the helper function that are not present in the outer function. You will almost always need an additional parameter in the helper function to keep track of your progress, in this case, \lstinline{sofar} which contains only the elements of the list that satisfy the function \lstinline{f}
    \item What is our base case? In this case we know we want to look at all of the elements \lstinline{lst}, so once we are done we return our stored list, \lstinline{sofar}
    \item How are we updating our result as we go on? Are we adding to a list, keeping track of a number? In this case we are adding to a list \lstinline{sofar}, so we are only adding an element to \lstinline{sofar} in our recursive call if it satisfies \lstinline{f}, otherwise we are going to call our function to move onto the next element without appending anything to \lstinline{sofar}
    \item What should we initialize our helper function with? In this case we are returning a list, so \lstinline{sofar} is initialized to an empty list in the first helper function call
    
  \end{enumerate}
\end{blocksection}
\end{guide}
