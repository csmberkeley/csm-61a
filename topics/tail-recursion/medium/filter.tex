\begin{blocksection}

\question Implement \lstinline{filter-lst}, which takes in a one-argument function \lstinline{f} and
a list \lstinline{lst}, and returns a new list containing only the elements in \texttt{lst}
for which \texttt{f} returns true. Your function must use a constant number of active frames.

\textit{Hint: recall that the built-in \lstinline{append} procedure concatenates two lists together. }

\begin{lstlisting}
;Doctests
scm> (filter-lst (lambda (x) (> x 2)) '(1 2 3 4 5))
(3 4 5)

(define (filter-lst f lst)










)
\end{lstlisting}

\begin{solution}[0.5in]
\begin{lstlisting}
(define (filter-lst f lst)
  (define (filter-tail lst so-far)
     (cond ((null? lst) so-far)
           ((f (car lst)) (filter-tail (cdr lst)
                          (append so-far (list (car lst)))))
           (else (filter-tail (cdr lst) so-far))))
  (filter-tail lst nil))
\end{lstlisting}
\end{solution}

\end{blocksection}

\begin{guide}
\begin{blocksection}
\textbf{Teaching Tips}
  \begin{itemize}
    \item What does a constant number of active frames mean? (It means use tail recursion.) Make sure your students get this before they start writing the problem. 
    \item Again, we use a helper function to store the variable that we want to return
    \item Note that f is not a necessary argument for the helper function because it does not change between calls
    \item In this case, we use \lstinline{so-far} to keep track of which elements pass our filter
    \item In each recursive call, we add the first element of \lstinline{lst} to \lstinline{so-far} if it passes the filter, otherwise we keep it the same, and in both cases we advance \lstinline{lst}
    \item Thus, at each step of the recursion, \lstinline{so-far} contains a list of the elements that have passed \lstinline{f} that we have seen so far
  \end{itemize}
\end{blocksection}
\end{guide}
