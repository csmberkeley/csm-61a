\question Slice and Dice

\begin{parts}
\begin{blocksection}

\part Implement \lstinline{slice}, which takes in a list \lstinline{lst}, a starting index \lstinline{i}, and an ending index \lstinline{j},
and returns a new list containing the elements of \lstinline{lst} from index \lstinline{i} to \lstinline{j - 1}. 

\begin{lstlisting}
;Doctests
scm> (slice '(0 1 2 3 4) 1 3)
(1 2)
scm> (slice '(0 1 2 3 4) 3 5)
(3 4)
scm> (slice '(0 1 2 3 4) 3 1)
()

(define (slice lst i j)
    






)
\end{lstlisting}

\end{blocksection}
\begin{blocksection}

\begin{solution}
\begin{lstlisting}
(define (slice lst i j)
      (cond ((or (null? lst) (>= i j)) nil)
               ((= i 0) (cons (car lst) (slice (cdr lst) i (- j 1))))
               (else (slice (cdr lst) (- i 1) (- j 1)))))
\end{lstlisting}
\end{solution}

\end{blocksection}

\begin{blocksection}
\begin{guide}
\textbf{Teaching Tips}
\begin{itemize}
	\item Begin by exploring base cases. Ask them to consider the case where $i \geq j$.
	\item Consider how we would find a single element instead of a slice.
	\item Ask them about the significance of $i=0$ and $j=0$.
\end{itemize}
\end{guide}
\end{blocksection}


\begin{blocksection}

\part Now implement \lstinline{slice} tail recursively!

\begin{lstlisting}
(define (slice lst i j)
    










)
\end{lstlisting}
\end{blocksection}

\begin{blocksection}
\begin{solution}
\begin{lstlisting}
(define (slice lst i j)
  (define (slice-tail lst i j lst-so-far)
      (cond ((or (null? lst) (>= i j)) lst-so-far)
               ((= i 0) (slice-tail (cdr lst) i (- j 1) (append lst-so-far (list (car lst)))))
               (else (slice-tail (cdr lst) (- i 1) (- j 1) lst-so-far))))
  (slice-tail lst i j nil))
\end{lstlisting}
Alternate Solution:
\begin{lstlisting}
(define (slice lst i j)
  (define (slice-tail lst index lst-so-far)
      (cond ((or (null? lst) (= index j)) lst-so-far)
               ((<= i index) (slice-tail (cdr lst) (+ index 1) (append lst-so-far (list (car lst)))))
               (else (slice-tail (cdr lst) (+ index 1) lst-so-far))))
  (if (< i j) (slice-tail lst 0 nil) nil))
\end{lstlisting}
\end{solution}

\end{blocksection}
\end{parts}

\begin{blocksection}
\begin{guide}
The purpose of this problem is to show the different considerations that go into non-tail-recursive
and tail-recursive procedure design. This develops the critical reasoning skills that can help
them design tail-recursive functions in the future. 

We expect this problem to take a very long time. If you are strapped for time, you may skip straight to the second
half of the problem and simply give your students the solution to the first half (by perhaps writing it on the board). 

\textbf{Teaching Tips}
\begin{itemize}
	\item Walk through why the previous solution was not tail recursive.
	\item Ask the students to consider how they can store and build the entire solution using the append function.
	\item If time allows, show them the alternate solution.
\end{itemize}
\end{guide}
\end{blocksection}