\begin{blocksection}

\question Implement \texttt{slice}, which takes in a a list \texttt{lst}, a starting index \texttt{i}, and an ending index \texttt{j},
and returns a new list containing the elements of \texttt{lst} from index \texttt{i} to \texttt{j - 1}.

\begin{lstlisting}
;Doctests
scm> (slice '(0 1 2 3 4) 1 3)
(1 2)
scm> (slice '(0 1 2 3 4) 3 5)
(3 4)
scm> (slice '(0 1 2 3 4) 3 1)
()

(define (slice lst i j)
    







)
\end{lstlisting}

\end{blocksection}
\begin{blocksection}

\begin{solution}
\begin{lstlisting}
(define (slice lst i j)
      (cond ((or (null? lst) (>= i j)) nil)
               ((= i 0) (cons (car lst) (slice (cdr lst) i (- j 1))))
               (else (slice (cdr lst) (- i 1) (- j 1)))))
\end{lstlisting}
\end{solution}

\end{blocksection}
\begin{blocksection}

\question Now implement slice with the same specifications, but make you implementation tail recurisve.
\newline
You may wish to use the built-in append function, which takes in two lists and returns a 
new list containing the elements of the two lists concatenated together.

\begin{lstlisting}
(define (slice lst i j)
    










)
\end{lstlisting}
\end{blocksection}

\begin{blocksection}
\begin{solution}
\begin{lstlisting}
(define (slice lst i j)
  (define (slice-tail lst i j lst-so-far)
      (cond ((or (null? lst) (>= i j)) lst-so-far)
               ((= i 0) (slice-tail (cdr lst) i (- j 1) (append lst-so-far (list (car lst)))))
               (else (slice-tail (cdr lst) (- i 1) (- j 1) lst-so-far))))
  (slice-tail lst i j nil))
\end{lstlisting}
Alternate Solution:
\begin{lstlisting}
(define (slice lst i j)
  (define (slice-tail lst index lst-so-far)
      (cond ((or (null? lst) (= index j)) lst-so-far)
               ((<= i index) (slice-tail (cdr lst) (+ index 1) (append lst-so-far (list (car lst)))))
               (else (slice-tail (cdr lst) (+ index 1) lst-so-far))))
  (if (< i j) (slice-tail lst 0 nil) nil))
\end{lstlisting}
\end{solution}

\end{blocksection}