\begin{blocksection}
\question Consider the following function:

\begin{lstlisting}
(define (sum-list lst)
  (if (null? lst)
    0
    (+ (car lst) (sum-list (cdr lst)))
  )
)
\end{lstlisting}
\end{blocksection}
\begin{parts}
  \begin{blocksection}
  \part What are all of expressions of \lstinline{sum-list} that are in tail contexts? (Hint: there are three.) Is the call to \lstinline{sum-list} tail recursive?  
  \begin{solution}[1.5in]
  The tail context expressions are: 
  \begin{itemize}
    \item The entire \lstinline{if} expression is in a tail context because it is the last operand of the body of a function. 
    \item The \lstinline{0} is in a tail context because it is the second operand of a tail-context \lstinline{if} expression. 
    \item The expression \lstinline{(+ (car lst) (sum-list (cdr lst)))} is in a tail context because it is the third operand of a tail-context \lstinline{if} expression. 
  \end{itemize}
  The call to \lstinline{sum-list} is not tail recursive because it is not in a tail context. On a more conceptual level, it is not the last expression we evaluate; after the recursive call returns, we still have to perform the addition operation. 
  \end{solution}
\end{blocksection}
\begin{blocksection}
  \part As we increase the length of \lstinline{lst}, how does the total amount of space used by \lstinline{sum-list} change? Why? 
  \begin{solution}[1.5in]
    Space usage increases linearly with the length of \lstinline{lst}. The recursive call to \lstinline{sum-list} is not in a tail context, so Scheme is not able to optimize it. That means that each time \lstinline{sum-list} is recursively called, another active frame is opened, taking up more space. 
    \end{solution}
  \end{blocksection}
\begin{blocksection}
\part Rewrite \lstinline{sum-list} to be tail recursive.

\begin{lstlisting}
(define (sum-list-tail lst)











)
\end{lstlisting}

\begin{solution}[0.5in]
\begin{lstlisting}
(define (sum-list-tail lst)
  (define (sum-list-helper lst sofar)
    (if (null? lst)
      sofar
      (sum-list-helper (cdr lst) (+ sofar (car lst)))
    )
  )
  (sum-list-helper lst 0)
)
\end{lstlisting}
\end{solution}
\begin{meta}
    Creating tail-recursive functions may initially be tricky for students, so I would try and organize your approach into a few steps
    \begin{enumerate}
      \item What variables do you need to keep track of in the helper function that are not present in the outer function. You will almost always need an additional parameter in the helper function to keep track of your progress, in this case, \lstinline{sofar}, which keeps track of the sum of all of the numbers
      \item What is our base case? In this case we know we want to look at all of the elements \lstinline{lst}, so once we are done we return our stored sum, \lstinline{sofar}
      \item How are we updating our result as we go on? Are we adding to a list, keeping track of a number? In this case we are keeping a running sum with each \lstinline{(car lst)}
      \item What should we initialize our helper function with? In this case our sum starts at 0, so \lstinline{sofar} is initialized to 0 in the first helper function call
    \end{enumerate}
\end{meta}
\end{blocksection}
\begin{blocksection}
\part As we increase the length of \lstinline{lst}, how does the total amount of space used by our optimized version of \lstinline{sum-list} change? Why? 
\begin{solution}[0.8in]
  Space usage is constant due to tail call optimization. 
  \end{solution}
\end{blocksection}
\end{parts}
