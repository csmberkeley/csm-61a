\begin{blocksection}
\question Consider the following function:

\begin{lstlisting}
(define (sum-list lst)
  (if (null? lst)
    0
    (+ (car lst) (sum-list (cdr lst)))
  )
)
\end{lstlisting}
\end{blocksection}
\begin{parts}
  \begin{blocksection}
  \part What are all of expressions of \lstinline{sum-list} that are in tail contexts? (Hint: there are three.) Is the call to \lstinline{sum-list} tail recursive?  
  \begin{solution}[1.5in]
  The tail context expressions are: 
  \begin{itemize}
    \item The entire \lstinline{if} expression is in a tail context because it is the last operand of the body of a function. 
    \item The \lstinline{0} is in a tail context because it is the second operand of a tail-context \lstinline{if} expression. 
    \item The expression \lstinline{(+ (car lst) (sum-list (cdr lst)))} is in a tail context because it is the third operand of a tail-context \lstinline{if} expression. 
  \end{itemize}
  The call to \lstinline{sum-list} is not tail recursive because it is not in a tail context. On a more conceptual level, it is not the last expression we evaluate; after the recursive call returns, we still have to perform the addition operation. 
  \end{solution}
\end{blocksection}
\begin{blocksection}
  \part As we increase the length of \lstinline{lst}, how does the total amount of space used by \lstinline{sum-list} change? Why? 
  \begin{solution}[1.5in]
    Space usage increases linearly with the length of \lstinline{lst}. The recursive call to \lstinline{sum-list} is not in a tail context, so Scheme is not able to optimize it. That means that each time \lstinline{sum-list} is recursively called, another active frame is opened, taking up more space. 
    \end{solution}
  \end{blocksection}
\begin{blocksection}
\part Rewrite \lstinline{sum-list} to be tail recursive.

\begin{lstlisting}
(define (sum-list-tail lst)











)
\end{lstlisting}

\begin{solution}[0.5in]
\begin{lstlisting}
(define (sum-list-tail lst)
  (define (sum-list-helper lst sofar)
    (if (null? lst)
      sofar
      (sum-list-helper (cdr lst) (+ sofar (car lst)))
    )
  )
  (sum-list-helper lst 0)
)
\end{lstlisting}
\end{solution}
\end{blocksection}
\begin{blocksection}
\part As we increase the length of \lstinline{lst}, how does the total amount of space used by our optimized version of \lstinline{sum-list} change? Why? 
\begin{solution}[0.8in]
  Space usage is constant due to tail call optimization. 
  \end{solution}
\end{blocksection}
\end{parts}
