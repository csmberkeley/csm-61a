\begin{blocksection}
\question Consider the following function:

\begin{lstlisting}
(define (sum-list lst)
  (if (null? lst)
    0
    (+ (car lst) (sum-list (cdr lst)))
  )
)
\end{lstlisting}

\vspace{2\baselineskip}

Why is sum-list not a tail call? Optional: draw out the environment diagram of this sum-list with list: (1 2 3
). When do you add 2 and 3?

\begin{solution}[0.5in]
Sum list is not the last call we make, it's actually the other addition which we do after we evaluate sum-list.
Sum list is not the last expression we evaluate.
\end{solution}
\end{blocksection}

\newpage

\begin{blocksection}
\question Rewrite sum-list in a tail recursive context.

\begin{lstlisting}
(define (sum-list-tail lst)











)
\end{lstlisting}

\begin{solution}[0.5in]
\begin{lstlisting}
(define (sum-list-tail lst)
  (define (sum-list-helper lst sofar)
    (if (null? lst)
      sofar
      (sum-list-helper (cdr lst) (+ sofar (car lst)))
    )
  )
  (sum-list-helper lst 0)
)
\end{lstlisting}
\end{solution}

\end{blocksection}
