\begin{blocksection}
\question Consider the following function:

\begin{lstlisting}
(define (count-instance lst x)
  (cond ((null? lst) 0)
        ((equal? (car lst) x) (+ 1 (count-instance
                                          (cdr lst) x)))
        (else (count-instance (cdr lst) x))))
\end{lstlisting}

\vspace{2\baselineskip}

What is the purpose of \texttt{count-instance}? Is it tail recursive? Why or why not?
\newline
Optional: draw out the environment diagram of this \texttt{count-instance} with \texttt{lst = (1 2 1)} and \texttt{x = 1}.

\begin{solution}[0.5in]
\texttt{count-instance} returns the number of times \texttt{x} appears in \texttt{lst}.
\\It is not tail recursive. The call to \texttt{count-instance} is an arguments to a function call, so it will not be the final thing we do in every frame (we will have to apply \texttt{+} after evaluating it.)
\end{solution}


\question Rewrite count-instance to be tail recursive. (Hint: helper functions are often useful in implementing Tail Recursion.)

\begin{lstlisting}
(define (count-tail lst x)










)
\end{lstlisting}

\begin{solution}[0.5in]
\begin{lstlisting}
(define (count-tail lst x)
  (define (count-helper lst instances)
    (cond ((null? lst) instances)
      ((equal? (car lst) x) (count-helper (cdr lst) (+ instances 1)))
      (else (count-helper (cdr lst) instances))))
  (count-helper lst 0))
\end{lstlisting}
\end{solution}
\end{blocksection}

\begin{guide}
\begin{blocksection}
\textbf{Teaching Tips}
  \begin{itemize}
    \item A common approach to converting non tail-recursive functions into tail-recursive functions is to keep track of the return value as a parameter
    \item To keep track of an extra parameter in recursion, we learned earlier in the class to use helper functions, so helper functions are commonly used for tail recursion
    \item In this case, we're trying to keep track of the count of \lstinline{x} we see inside \lstinline{lst}
    \item Rather than adding 1 to the recursive call of \lstinline{(cdr lst)}, we can keep track of this return value as a parameter--in this case \lstinline{instances}
    \item We let \lstinline{instances} represent the count of \lstinline{x}. In each recursive call, we increment \lstinline{instances} if \lstinline{(car lst)} is equal to \lstinline{x}, otherwise we keep it the same
    \item Thus, at each step of the recursion, the parameter \lstinline{instances} should represent the
    count of \lstinline{x} that we have seen up to that point
    \item The logic between the tail-recursive version and the original function is mostly the same; we just need to change the return values to account for the parameter \lstinline{instances}
  \end{itemize}
\end{blocksection}
\end{guide}
