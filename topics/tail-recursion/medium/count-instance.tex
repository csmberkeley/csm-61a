\begin{blocksection}
\question Consider the following function:

\begin{lstlisting}
(define (count-instance lst x)
  (cond ((null? lst) 0)
        ((equal? (car lst) x) (+ 1 (count-instance 
                                          (cdr lst) x)))
        (else (count-instance (cdr lst) x))))
\end{lstlisting}

\vspace{2\baselineskip}

What is the purpose of \texttt{count-instance}? Is it tail recursive? Why or why not?
\newline
Optional: draw out the environment diagram of this sum-list with \texttt{lst = (1 2 1)} and \texttt{x = 1}.

\begin{solution}[0.5in]
\texttt{count-instance} returns the number of time \texttt{x} appears in \texttt{lst}. It is not tail recursive. The call to \texttt{count-instance} appears as one of the arguments to a function call, so it will not be the final thing we do in every frame (we will have to apply \texttt{+} after evaluating it.)
\end{solution}


\question Rewrite count-instance to be tail recursive.

\begin{lstlisting}
(define (count-tail lst x)










)
\end{lstlisting}

\begin{solution}[0.5in]
\begin{lstlisting}
(define (count-tail lst x)
  (define (count-helper lst instances)
    (cond ((null? lst) instances)
      ((equal? (car lst) x) (count-helper (cdr lst) (+ instances 1)))
      (else (count-helper (cdr lst) instances))))
  (count-helper lst 0))
\end{lstlisting}
\end{solution}
\end{blocksection}

\begin{blocksection}
\begin{solution}
A common approach to converting non tail-recursive functions into 
tail-recursive functions is to keep track of the return value as a 
parameter.
To keep track of an extra parameter in recursion, we learned earlier 
in the class to use helper functions. In this case, we're trying to 
keep track of the count of \lstinline{x} we see inside \lstinline{lst}. Rather than adding 1 
to the recursive call of \lstinline{(cdr lst)}, we can keep track of this return 
value as a parameter--in this case \lstinline{instances}. We let \lstinline{instances} 
represent the count of \lstinline{x}. In each recursive call, we increment 
\lstinline{instances} if \lstinline{(car lst)} is equal to \lstinline{x}, otherwise we keep it the same. 
Thus, at each step of the recursion, \lstinline{instances} should represent the 
count of \lstinline{x} that we have seen up to that point. In our initial call 
to helper, we default \lstinline{instances} to 0 because we haven't seen any \lstinline{x}'s 
yet.

The logic between the tail-recursive version and the original 
function is mostly the same; we just need to change the return 
values to account for the parameter \lstinline{instances}. For the base case, 
rather than returning 0, we should instead return \lstinline{instances}, which 
represents the count of \lstinline{x} in \lstinline{lst} (that we've seen up to that point). 
This makes sense because if \lstinline{lst} is empty, then the default value of 
\lstinline{instances} is just 0. Otherwise, we make recursive calls and change 
\lstinline{instances} accordingly.

\end{solution}

\end{blocksection}
