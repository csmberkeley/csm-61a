\begin{blocksection}
\question Consider the following function:

\begin{lstlisting}
(define (count-instance lst x)
  (cond ((null? lst) 0)
        ((equal? (car lst) x) (+ 1 (count-instance 
                                          (cdr lst) x)))
        (else (count-instance (cdr lst) x))))
\end{lstlisting}

\vspace{2\baselineskip}

What is the purpose of \texttt{count-instance}? Is it tail recursive? Why or why not?
\newline
Optional: draw out the environment diagram of this sum-list with \texttt{lst = (1 2 1)} and \texttt{x = 1}.

\begin{solution}[0.5in]
\texttt{count-instance} returns the number of time \texttt{x} appears in \texttt{lst}. It is not tail recursive. The call to \texttt{count-instance} appears as one of the arguments to a function call, so it will not be the final thing we do in every frame (we will have to apply \texttt{+} after evaluating it.)
\end{solution}
\end{blocksection}

\newpage

\begin{blocksection}
\question Rewrite count-instance to be tail recursive.

\begin{lstlisting}
(define (count-tail lst x)










)
\end{lstlisting}

\begin{solution}[0.5in]
\begin{lstlisting}
(define (count-tail lst x)
    (define (count-helper lst x instances)
		    (cond ((null? lst) instances)
              ((equal? (car lst) x) (count-helper (cdr lst) x (+ instances 1)))
              (else (count-helper (cdr lst) x instances))))
    (count-helper lst x 0))
\end{lstlisting}
\end{solution}

\end{blocksection}
