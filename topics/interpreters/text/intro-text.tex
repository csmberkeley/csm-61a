An \textbf{interpreter} is a computer program that understands, processes, and executes other programs. The Scheme interpreter we will cover in CS 61A is built around the \textbf{Read-Eval-Print Loop}, which consists of the following steps:
\begin{enumerate}
    \item \textbf{Read} the raw input and parse it into a data structure we can easily handle. 
    \item \textbf{Evaluate} the parsed expression. 
    \item \textbf{Print} the result to output. 
\end{enumerate}

One of the challenges of designing interpreters is to represent the input in a way that the interpreter's language can understand.
For example, since our Scheme interpreter is written in Python, we need to parse Scheme tokens into a usable Python representation.
Conveniently, every Scheme call expression and special form is represented in Scheme as a linked list. 
Therefore, we will represent Scheme lists with the \lstinline{Pair} class, which is a type of linked list. 

Just like \lstinline{Link}, a \lstinline{Pair} instance has two attributes, \lstinline{first} and \lstinline{second}, which contain the first element and rest of the linked list, respectively. Instead of using \lstinline{Link.empty} to represent an empty list, \lstinline{Pair} uses \lstinline{nil}.

For example, during the read step, the Scheme expression \lstinline{(+ 1 2)} would be \textbf{tokenized} into \lstinline{'(', '+', '1', '2', ')'} and then organized into a \lstinline{Pair} instance as \lstinline{(Pair('+', Pair(1, Pair(2, nil))))}. 

Once we have parsed our input, we evaluate the expression by calling \lstinline{scheme_eval} on it. If it's a procedure call, we recursively call \lstinline{scheme_eval} on the operator and the operands. Then we return the result of calling \lstinline{scheme_apply} on the evaluated operator and operands, which computes the procedure call. If it's a special form, the relevant evaluation rules are followed in a similar matter.

For example, when we provide \lstinline{(+ 1 (+ 2 3))} as input to the interpreter, the following happen: 
\begin{itemize}
    \item \lstinline{(+ 1 (+ 2 3))} is parsed to \lstinline{Pair('+', Pair(1, Pair(Pair('+', Pair(2, Pair(3, nil))), nil)))}
    \item The interpreter recognizes this is a procedure call. 
    \item \lstinline{scheme_eval} is called on the operator, \lstinline{'+'}, and returns the addition procedure.
    \item \lstinline{scheme_eval} is called on the operand \lstinline{1} and returns \lstinline{1}. 
    \item \lstinline{scheme_eval} is called on the operand \lstinline{Pair('+', Pair(2, Pair(3, nil)))}. 
    \begin{itemize}
        \item The interpreter recognizes this is a procedure call. 
        \item \lstinline{scheme_eval} is called on the operator, \lstinline{'+'}, and returns the addition procedure.
        \item \lstinline{scheme_eval} is called on the operand \lstinline{2} and returns \lstinline{2}. 
        \item \lstinline{scheme_eval} is called on the operand \lstinline{3} and returns \lstinline{3}. 
        \item \lstinline{scheme_apply} is called on the evaluated procedure and parameters (\lstinline{Pair(2, Pair(3, nil))}) and returns \lstinline{5}. 
    \end{itemize}
    \item \lstinline{scheme_apply} is called on the evaluated procedure and parameters (\lstinline{Pair(1, Pair(5, nil))}) and returns \lstinline{6}. 
    \item \lstinline{6} is printed to output. 
\end{itemize}

\begin{meta}
Explaining interpreters can be quite tricky. You will probably need to field a lot of student questions
in order to ensure that they are understanding things. 

Everything in Scheme is lists!!!! I think this is something tricky that a lot of students don't really understand. 
Understanding this once and for all is what finally made Scheme make sense to me, but your mileage may
vary. 

Here are a few questions that you probably want to be answered by your mini-lecture: 

\end{meta}