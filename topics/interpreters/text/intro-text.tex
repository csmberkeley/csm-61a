An \textbf{interpreter} is a computer program that understands, processes, and executes other programs. The Scheme interpreter we will cover in CS 61A is built around the \textbf{Read-Eval-Print Loop}, which consists of the following steps:
\begin{enumerate}
    \item \textbf{Read} the raw input and parse it into a data structure we can easily handle. 
    \item \textbf{Evaluate} the parsed expression. 
    \item \textbf{Print} the result to output. 
\end{enumerate}

One of the challenges of designing interpreters is to represent the input in a way that the interpreter's language can understand.
For example, since our Scheme interpreter is written in Python, we need to parse Scheme tokens into a usaable Python representation.
To achieve this, we will represent Scheme lists with the \texttt{Pair} class, which is essentially a linked list that uses \texttt{nil} instead of \texttt{Link.empty}.

Once we have parsed our input, we evaluate the expression by calling \lstinline{scheme_eval} on it. If it's a procedure call, we recursively call \lstinline{scheme_eval} on the operator and the operands. Then we return the result of calling \lstinline{scheme_apply} on the evaluated operator and operands, which computes the procedure call. If it's a special form, the relevant evaluation rules are followed in a similar matter.

For example, when we provide \lstinline{(+ 1 (+ 2 3))} as input to the interpreter, the following happen: 
\begin{itemize}
    \item \lstinline{(+ 1 (+ 2 3))} is parsed to \lstinline{Pair('+', Pair(1, Pair(Pair('+', Pair(2, Pair(3, nil))), nil)))}
    \item The interpreter recognizes this is a procedure call. 
    \item \lstinline{scheme_eval} is called on the operator, \lstinline{'+'}, and returns the addition procedure.
    \item \lstinline{scheme_eval} is called on the operand \lstinline{1} and returns \lstinline{1}. 
    \item \lstinline{scheme_eval} is called on the operand \lstinline{Pair('+', Pair(2, Pair(3, nil)))}. 
    \begin{itemize}
        \item The interpreter recognizes this is a procedure call. 
        \item \lstinline{scheme_eval} is called on the operator, \lstinline{'+'}, and returns the addition procedure.
        \item \lstinline{scheme_eval} is called on the operand \lstinline{2} and returns \lstinline{2}. 
        \item \lstinline{scheme_eval} is called on the operand \lstinline{3} and returns \lstinline{3}. 
        \item \lstinline{scheme_apply} is called on the evaluated procedure and parameters (\lstinline{Pair(2, Pair(3, nil))}) and returns \lstinline{5}. 
    \end{itemize}
    \item \lstinline{scheme_apply} is called on the evaluated procedure and parameters (\lstinline{Pair(1, Pair(5, nil))}) and returns \lstinline{6}. 
    \item \lstinline{6} is printed to output. 
\end{itemize}
\begin{meta}
Everything in Scheme is lists!!!! 
\end{meta}