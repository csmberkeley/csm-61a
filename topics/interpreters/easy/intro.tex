\question The following questions refer to the Scheme interpreter. Assume we're using the implementation seen in lecture and 
in the Scheme project.
\begin{parts}
\begin{blocksection}
\part What's the purpose of the read stage in a Read-Eval-Print Loop? For our Scheme interpreter, what does it 
take in, and what does it return?
\begin{solution}[1in] 
The read stage returns a representation of the code that is easier to process later in the interpreter by putting it in a new data structure. In our interpreter,
it takes in a string of code, and outputs a Pair representing an expression (which is really just the same as a Scheme list).
\end{solution}
\end{blocksection}

\begin{blocksection}
\part What are the two components of the read stage? What do they do?
\begin{solution}[1in]
The read stage consists of
\begin{enumerate}[1.]
\item The lexer, which breaks the input string and breaks it up into tokens (individual characters or symbols)
\item The parser, which takes that string of tokens and puts it into the data structure that the read stage outputs (in our case, a Pair).
\end{enumerate}
\end{solution}
\end{blocksection}

\begin{blocksection}
\part Write out the constructor for the \lstinline{Pair} object that the read stage creates from the input string \texttt{(define (foo x) (+ x 1))}
\begin{solution}[1in]
Pair("define", Pair(Pair("foo", Pair("x", nil)), Pair(Pair("+", Pair("x", Pair(1, nil))), nil)))
\end{solution}
\end{blocksection}

\begin{blocksection}
\part For the previous example, imagine we saved that Pair object to the variable \texttt{p}. How could we check that the expression 
is a \texttt{define} special form? How would we access the name of the function and the body of the function?
\begin{solution}[1in]
We could check to see that it's a define special form by checking if \lstinline$p.first == "define"$. 

We could get its name by accessing \lstinline$p.second.first.first$ and get the body of the function with \lstinline$p.second.second.first$.
\end{solution}
\end{blocksection}
\end{parts}

\begin{guide}
\textbf{Teaching Tips}
\begin{itemize}
	\item A great way to go about these short answer type questions is to have a mini lecture prepared and then go through the answer of each question in your lecture.
	\item Often the words read, eval, and print may not make the most intuitive sense to students right away. Encourage them to think about them in different angles and make analogies to listening to someone talk or following an instruction: you always process what you receive first, then you actually do the thing, and then you show that you understood or were able to produce results.
	\item If you can think of a clever way to remember lexer and parser that would be really helpful to students!
	\item Remind students that Pairs are nothing more than linked lists to lessen the possible apprehension at hand-creating the Pairs. This will also save a lot of headaches with .seconds and .firsts in the project. It may even be helpful to draw out an environment diagram of the Pair structure as a linked list.
\end{itemize}
\end{guide}