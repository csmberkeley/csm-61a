\begin{blocksection}
\question Implement \lstinline{pow_of_two}, which prints all the positive integer powers of two less than or equal to \lstinline{n} in ascending order. This function should return \lstinline{None}.

\emph{Follow up question: What would you change about your solution if the question asked to print all the powers of two \textbf{strictly less than} \lstinline{n}?}

\begin{lstlisting}
def pow_of_two(n):
    """
    >>> pow_of_two(6)
    1 
    2 
    4 
    >>> result = pow_of_two(16)
    1
    2
    4
    8
    16
    >>> result is None
    True
    """
\end{lstlisting}
\begin{solution}[2in]
\begin{lstlisting}
    curr = 1
    while curr <= n:
        print(curr)
        curr *= 2 # equivalent to curr = curr * 2
\end{lstlisting}
Since we are multiplying \lstinline{curr} by 2 on each iteration of the while loop, \lstinline{curr} holds values that are powers of 2.
Notice that since there is no return statement in this function, when Python reaches the end of the function, it automatically returns \lstinline{None}.

The answer to the follow up question is that the condition of our while loop would change to \lstinline{curr < n}.
Walk through the code for \lstinline$pow_of_two(16)$ with both of the conditions to see why they produce different outputs!

Another way you could have written this function is by using \lstinline{pow} or the \lstinline{**} operator. That solution would look something like this where you would keep track of the exponent itself:

\begin{lstlisting}
    exponent = 0
    while (2 ** exponent) <= n:
        print(2 ** exponent)
        exponent += 1
\end{lstlisting}
\end{solution}
\end{blocksection}