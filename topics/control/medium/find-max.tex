%%%%%%%%%%%%%%%%%%%%%%%%%%%%%%%%%%%%%%%%%%%%%%%%%%%%%%%%
% REVIEW OF PROBLEM S23
% REVIEWER: ALYSSA SMITH
% I'm gonna be so fr with y'all here, I think the way CS 61A tries to instill structure of code is absolutely stupid
% WHO'S GONNA NEED MASSIVE ONE-LINER RETURN STATEMENTS
% Nonethless, this makes sense, encouraging students to think in "cases" and establish clauses and conditions based upon that.
% Not really a great problem in my opinion, but instills the classic CS 61A structure of problems in students.
% For future semesters, this problem was in S22, F22. I took it out S23 because fizzbuzz brings in while loops.
% Additionally, it's a better "case-thinking" problem than this.
% Also it's the HackerRank practice question, so more utility asdkljghasdklg.
%%%%%%%%%%%%%%%%%%%%%%%%%%%%%%%%%%%%%%%%%%%%%%%%%%%%%%%%
\begin{blocksection}
\question Write a function \lstinline{find_max} that takes in three numbers, \lstinline{x}, \lstinline{y}, \lstinline{z}, and returns the maximum of the provided values. Assume that \lstinline{x}, \lstinline{y}, and \lstinline{z} are unique. Do not use Python's built-in \texttt{max} function.

\begin{lstlisting}
def find_max(x, y, z):
\end{lstlisting}
\begin{solution}[2in]
\begin{lstlisting}
def find_max(x, y, z):
    if x > y and x > z:
        return x
    elif y > x and y > z:
        return y
    else:
    	return z
\end{lstlisting}
\end{solution}
\end{blocksection}