\question Fill out the following table. Write N/A if the conversion is not possible. Some entries have already been filled out for you.

\begin{center}
  \begin{tabular}{|c|c|c|}
    \hline
    Decimal & Binary (unsigned) & Binary (two's complement) \\
    \hline\hline
    25 &  &  \\
    \hline
    & 0010 1010 & \\
    \hline
    &  & 0011 1100 \\
    \hline
    -66 &  &  \\
    \hline
    &  & 1010 1000 \\
    \hline
  \end{tabular}
\end{center}

\begin{itemize}
  \item To convert from binary to decimal, write out which powers of two correspond to a 1, and sum them together.
  \item To convert from decimal to binary, it can be useful to create a binary place value table.
  \begin{center}
  \begin{tabular}{|c|c|c|c|c|c|c|c|}
    \hline
    128 & 64 & 32 & 16 & 8 & 4 & 2 & 1 \\
    \hline
    0 & 0 & 0 & 1 & 1 & 0 & 0 & 1 \\
    \hline
  \end{tabular}
  \end{center}
  \begin{itemize}
    \item Starting with the closest power of two \textbf{smaller} than our decimal, set it to 1 in the table and subtract this from the decimal. Repeat until the decimal is zero.
  \end{itemize}
  \item To convert from a negative decimal to two's complement binary, begin by finding the closest power of two \textbf{larger} than the decimal. For 66, this is 128.
  \begin{center}
  \begin{tabular}{|c|c|c|c|c|c|c|c|}
    \hline
    -128 & 64 & 32 & 16 & 8 & 4 & 2 & 1 \\
    \hline
    1 & 0 & 1 & 1 & 1 & 1 & 1 & 0 \\
    \hline
  \end{tabular}
  \end{center}
  \begin{itemize}
    \item Set this value to 1 in the table and \textbf{add} this value to the decimal (since it is really a negative number) which should result in a positive decimal.
    \item Then, we use normal binary conversion to shrink the decimal to zero.
  \end{itemize}
\end{itemize}

\begin{solution}[0.5in]
\begin{center}
  \begin{tabular}{|c|c|c|}
    \hline
    Decimal & Binary (unsigned) & Binary (two's complement) \\
    \hline\hline
    25 & 0001 1001 & 0001 1001 \\
    \hline
    42 & 0010 1010 & 0010 1010 \\
    \hline
    60 & 0011 1100 & 0011 1100 \\
    \hline
    -66 & N/A & 1011 1110 \\
    \hline
    -88 & N/A & 1010 1000 \\
    \hline
  \end{tabular}
\end{center}
\end{solution}

\begin{blocksection}
\begin{guide}
\textbf{Teaching Tips}
\begin{itemize}
  \item Students may be confused by the addition of binary numbers and circuits to course material.
  \item You can reassure them that they will likely not be tested on this material, but they are good subjects to know for computer scientists regardless.
  \item It may be a good idea to walk students through rows 1 and 2 first, then have them try the rest on their own.
  \item To convert from binary to decimal, write out which powers of two correspond to a 1, and sum them together.
  \item To convert from decimal to binary, it can be useful to create a binary place value table.
  \begin{center}
  \begin{tabular}{|c|c|c|c|c|c|c|c|}
    \hline
    128 & 64 & 32 & 16 & 8 & 4 & 2 & 1 \\
    \hline
    0 & 0 & 0 & 1 & 1 & 0 & 0 & 1 \\
    \hline
  \end{tabular}
  \end{center}
  \begin{itemize}
    \item Starting with the closest power of two \textbf{smaller} than our decimal, set it to 1 in the table and subtract this value from the decimal. Repeat until the decimal is zero.
  \end{itemize}
  \item To convert from a negative decimal to two's complement binary, begin by finding the closest power of two \textbf{larger} than the decimal. For 66, this is 128.
  \begin{center}
  \begin{tabular}{|c|c|c|c|c|c|c|c|}
    \hline
    -128 & 64 & 32 & 16 & 8 & 4 & 2 & 1 \\
    \hline
    1 & 0 & 1 & 1 & 1 & 1 & 1 & 0 \\
    \hline
  \end{tabular}
  \end{center}
  \begin{itemize}
    \item Set this value to 1 in the table and \textbf{add} this value to the decimal (since it is really a negative number) which should result in a positive decimal.
    \item Then, we use normal binary conversion to shrink the decimal to zero.
  \end{itemize}
  \item If students are interested, consider explaining why 1000 and 1111 1000 are equal using Two's Complement (why negative numbers have leading ones).
\end{itemize}
\end{guide}
\end{blocksection}
