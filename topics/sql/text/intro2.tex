Last week we went over SQL and all the main keywords in SQL. This week we are going to go over more SQL problems.

\begin{center}
    
    \begin{tabular}{|c|c|}
     \hline
     \textbf{Clause} & \textbf{Function} \\
     \hline
     \textbf{SELECT} & Specifies the columns to retrieve or calculate or used in creation of a new row \\
     \hline
     \textbf{UNION} & Joins rows together \\
     \hline
     \textbf{CREATE TABLE} & Creates a new named table \\
     \hline
     \textbf{FROM} & Specifies the table from which to retrieve data \\
     \hline
     \textbf{WHERE} & Chooses the rows that meet the condition \\
     \hline
     \textbf{GROUP BY} & Groups rows that share a property \\
     \hline
     \textbf{HAVING} & Filters groups based on conditions, applied after grouping \\
     \hline
     \textbf{ORDER BY} & Sorts the result set by specified columns \\
     \hline
     \textbf{LIMIT} & Specifies the maximum number of rows to return \\
     \hline
    \end{tabular}
\end{center}

In SQL, the order of execution is as follows:
\begin{enumerate}
    \item FROM
    \item WHERE
    \item GROUP BY
    \item HAVING
    \item SELECT
    \item ORDER BY
    \item LIMIT
\end{enumerate}