\begin{blocksection}
\part Write a query that returns, for each species, the difference between our hatchery's revenue versus the competitor's revenue for one whole fish. Assume that number of pieces per one whole fish of a certain species of fish stays the same across both our hatchery and our comptitor's hatchery. \\
\begin{meta}
  \textbf{Note:} It may be helpful to remind students of the formula for revenue: price per unit * number of units
\end{meta}
% This is because we make 30 pieces at \$4 a piece for \$120, whereas the competitor will make 30 pieces at \$2 a piece for \$60. Therefore, the difference is 60.

\begin{solution}[1.5in]
\begin{lstlisting}
SELECT fish.species, (fish.price - competitor.price) * pieces
    FROM fish, competitor
    WHERE fish.species = competitor.species;
\end{lstlisting}
\end{solution}
\end{blocksection}

\begin{blocksection}
\begin{guide}
\textbf{Teaching Tips}
\begin{itemize}
  \item Make sure students know the basics of understanding/looking through a table
  \begin{itemize}
    \item It may help to write a basic example using SELECT, FROM, and WHERE
  \end{itemize}
  \item Have students clearly define which columns they will need from the table (species, price, pieces) before coding
  \item Remind students they are able to do arithmetic on cells they select
  \item While not in the solution, you can use this problem to explain the benefits of aliasing
  \begin{itemize}
    \item Show students they can write \lstinline{FROM fish as __, competitor as __}
  \end{itemize}
\end{itemize}
\end{guide}
\end{blocksection}
