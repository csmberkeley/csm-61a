\documentclass{exam}
\usepackage{../../commonheader}
\lstset{language=Scheme}

%%% CHANGE THESE %%%%%%%%%%%%%%%%%%%%%%%%%%%%%%%%%%%%%%%%%%%%%%%%%%%%%%%%%%%%%%
\discnumber{9}
\title{\textsc{SQL and Tail Recursion}}
\date{April 11 to April 15, 2016}
%%%%%%%%%%%%%%%%%%%%%%%%%%%%%%%%%%%%%%%%%%%%%%%%%%%%%%%%%%%%%%%%%%%%%%%%%%%%%%%

\begin{document}
\maketitle
\rule{\textwidth}{0.15em}
\fontsize{12}{15}\selectfont

%%% INCLUDE TOPICS HERE %%%%%%%%%%%%%%%%%%%%%%%%%%%%%%%%%%%%%%%%%%%%%%%%%%%%%%%

\section{Tail Recursion}
\begin{questions}
\begin{blocksection}
\question What is a tail context/tail call? What is a tail recursive function?
\begin{solution}[0.5in]
A tail call is a call expression in a tail context. A tail context is usually the final action of a procedure/function.\newline
\newline
A tail recursive function is where all the recursive calls of the function are in tail contexts. \newline
\newline
An ordinary recursive function is like building up a long chain of domino pieces, then knocking down the last one. A tail recursive function is like putting a domino piece up, knocking it down, putting a domino piece up again, knocking it down again, and so on. This metaphor helps explain why tail calls can be done in constant space, whereas ordinary recursive calls need space linear to the number of frames (in the metaphor, domino pieces are equivalent to frames).
\end{solution}
\question Why are tail calls useful for recursive functions?
\begin{solution}[0.5in]
When a function is tail recursive, it can effectively discard all the past recursive frames and only keep the current frame in memory. This means we can use a constant amount of memory with recursion, and that we can deal with an unbounded number of tail calls with our Scheme interpreter.
\end{solution}
\end{blocksection}

\begin{blocksection}
Answer the following questions with respect to the following function: 
\begin{lstlisting}
(define (sum-list lst)
	(if (null? lst)
		0
		(+ (car lst) (sum-list (cdr lst)))
	)
) 
\end{lstlisting}
\question Why is sum-list not a tail call? Optional: draw out the environment diagram of this sum-list with list: (1 2 3). When do you add 2 and 3?
\begin{solution}[0.5in]
Sum list is not the last call we make, it’s actually the other addition which we do after we evaluate sum-list. Sum list is not the last expression we evaluate.
\end{solution}
\question Rewrite sum-list in a tail recursive context.
\begin{solution}[0.5in]
\begin{lstlisting}
(define (sum-list-tail lst)
	(define (sum-list-helper lst sofar)
		(if (null? lst)
			sofar
			(sum-list-helper 
				(cdr lst) 
				(+ sofar (car lst))
			)
		)
	)
	(sum-list-helper lst 0)
)
\end{lstlisting}
\end{solution}
\end{blocksection}

\section{SQL}
\begin{blocksection}
\begin{center}
\begin{tabular}{ |c|c|c|c|c| } 
 \hline
 Name & Food & Color & Editor & Language \\ 
 \hline
 Tiffany & Thai & Purple & Notepad++ & Java \\ 
 \hline
 Diana & Pie & Green & Sublime & Java \\ 
 \hline
  Allan & Sushi & Orange & Emacs & Ruby \\ 
 \hline
 Alfonso & Tacos & Blue & Vim & Python \\ 
 \hline
 Kelly & Ramen & Green & Vim & Python \\ 
 \hline
\end{tabular}
\end{center}
\question Create a new table \textbf{mentors} that contains all the information above.
\begin{solution}[1in]
\begin{lstlisting}
create table mentors as
    select 'Tiffany' as name, 'Thai' as food, 'Purple' as color, 'Notepad++' 
 as editor, 'Java' as language union
    select 'Diana', 'Pie', 'Green', 'Sublime', 'Java' union
    select 'Allan', 'Sushi', 'Orange', 'Emacs', 'Ruby' union
    select 'Alfonso', 'Tacos', 'Blue', 'Vim', 'Python' union
    select 'Kelly', 'Ramen', 'Green', 'Vim', 'Python';
\end{lstlisting}
\end{solution}
\end{blocksection}

%%% Question %%%
\begin{blocksection}
\question Write a query that lists all the mentors along with their favorite food if their favorite color is green.\newline
Output: 
\begin{lstlisting}
Diana|Pie 
Kelly|Ramen
\end{lstlisting}
\begin{solution}[1in]
\begin{lstlisting}
SELECT m.name, m.food
  FROM mentors as m
  WHERE m.color = 'Green';
  
WITHOUT ALIASING: 

SELECT name, food
  FROM mentors
  WHERE color = 'Green';
\end{lstlisting}
\end{solution}

\question Write a query that lists all the mentors along with their favorite food if their favorite color is green.\newline
Output: 
\begin{lstlisting}
Sushi|Orange
Pie|Green
Thai|Purple
\end{lstlisting}
\begin{solution}[1in]
\begin{lstlisting}
SELECT m.food, m.color
  FROM mentors as m
  WHERE m.language <> 'Python';

WITHOUT ALIASING

SELECT food, color
  FROM mentors
  WHERE language != 'Python';
\end{lstlisting}
\end{solution}
\end{blocksection}

%%% Question %%%
\begin{blocksection}
\question Write a query that lists all the pairs of mentors who like the same language. (How can we make sure to remove duplicates?) \newline
Output:
\begin{lstlisting}
Kelly|Alfonso
Tiffany|Diana
\end{lstlisting}
\begin{solution}[2in]
\begin{lstlisting}
SELECT m1.name, m2.name 
    FROM mentors as m1, mentors as m2 
    WHERE m1.language = m2.language and m1.name > m2.name;
\end{lstlisting}
\end{solution}

\question Write a query that has the same data, but alphabetizes the rows by name. (Hint: Use order by.) \newline
Output:
\begin{lstlisting}
Alfonso|Tacos|Blue|Vim|Python
Allan|Sushi|Orange|Emacs|Ruby
Diana|Pie|Green|Sublime|Java
Kelly|Ramen|Green|Vim|Python
Tiffany|Thai|Purple|Notepad++|Java
\end{lstlisting}
\begin{solution}[1in]
\begin{lstlisting}
SELECT * 
  FROM mentors
  ORDER BY name;
\end{lstlisting}
\end{solution}
\end{blocksection}


%%% Question %%%
\begin{blocksection}
\section{Fish Population} 
The 61A mentors want to start a fish hatchery, and they need your help to analyze the data they've collected for the fish populations! Also, running a hatchery is expensive -- they'd like to make some money on the side by selling some seafood (only older fish of course) to make delicious sushi. \newline
\newline
The following table contains a subset of the data that has been collected. The SQL column names are listed in brackets. Note: we must be able to extend your queries to larger tables! (i.e, don't hard code your answers) \newline
\newline
fish*
\begin{center}
\begin{tabular}{ |c|c|c|c|c| } 
 \hline
 Species [species] & Population [pop] & Breeding Rate [rate] & \$/piece [price] & \# of pieces [pieces] \\ 
 \hline
 Salmon & 500 & 3.3 & 4 & 30 \\ 
 \hline
 Eel & 100 & 1.3 & 4 & 15 \\ 
 \hline
  Yellowtail & 700 & 2.0 & 3 & 30 \\ 
 \hline
 Tuna & 600 & 1.1 & 3 & 20 \\ 
 \hline
\end{tabular}
\end{center}
*(This was made with fake data, do not actually sell fish at these rates)
\question \textbf{Aggregation} 
\begin{parts}
\part Profit is good, but more profit is better. Write a query to select the species that yields the most number of pieces for each price. Your output should include the species, price, and pieces.
\begin{solution}[0.5in]
\begin{lstlisting}
SELECT species, price, MAX(pieces) FROM fish GROUP BY price;
\end{lstlisting}
\end{solution}

\part Write a query to find the three most populated fish species.
\begin{solution}[0.5in]
\begin{lstlisting}
SELECT species FROM fish ORDER BY -pop LIMIT 3;
\end{lstlisting}
\end{solution}

\part Write a query to find the total number of fish in the "ocean." Additionally, include the number of species we summed. Your output should have the number of species and the total population.
\begin{solution}[0.5in]
\begin{lstlisting}
SELECT COUNT(species), SUM(pop) FROM fish;
\end{lstlisting}
\end{solution}
\end{parts}
\end{blocksection}

\begin{blocksection} 
(d) Business is good, but a bunch of competition has sprung up! Through some cunning corporate espionage, we have determined that one such competitor plans to open shop with the following rates:\newline
\newline
competitor
\begin{center}
\begin{tabular}{ |c|c| } 
 \hline
 Species [species] & \$/piece [price] \\ 
 \hline
 Salmon & 2 \\ 
 \hline
 Eel & 3.4 \\ 
 \hline
  Yellowtail & 3.2  \\ 
 \hline
 Tuna & 2.6  \\ 
 \hline
\end{tabular}
\end{center}
Write a query that compares how much our hatchery will charge per fish versus the competitor. For example, the table should contain the following row: \newline
\texttt{Salmon | 60} \newline \newline
Because we make 30 pieces at \$4 a piece for \$120, whereas the competitor will make 30 pieces at \$2 a piece for \$60. Finally, the difference is 60. Remember to do this for every species!
\begin{solution}[1in]
\begin{lstlisting}
SELECT fish.species, (fish.price - competitor.price) * pieces 
    FROM fish, competitor 
    WHERE fish.species = competitor.species;
\end{lstlisting}
\end{solution}


\end{blocksection}

\question \textbf{Recursive Select} Suppose these fish breed every day. The population of each fish gets multiplied by its breeding rate every year. Write a recursive select function that creates a table of fish 10 years from now.

\begin{solution}[0.5in]
\begin{lstlisting}
WITH 
  yearly_pop(yearly_species, yearly_pop, N) AS (
     SELECT species, pop, 0 FROM fish UNION
     SELECT yearly_species, yearly_pop * rate, N + 1 
         FROM yearly_pop, fish 
         WHERE yearly_species = species AND N <= 10
  )
SELECT yearly_species, yearly_pop FROM yearly_pop WHERE N = 10;

\end{lstlisting}
\end{solution}


\end{questions}

%%%%%%%%%%%%%%%%%%%%%%%%%%%%%%%%%%%%%%%%%%%%%%%%%%%%%%%%%%%%%%%%%%%%%%%%%%%%%%%

\end{document}
