%Linked List --> Class --> Medium

\begin{blocksection}
    \question DNA carries the genetic instructions that enable the functioning of many living creatures, including us. The bases of a DNA sequence include adenine (A), guanine (G), cytosine (C), and thymine (T). Adenine (A) pairs with thymine (T), and guanine (G) pairs with cytosine (C).

    Let us represent DNA as a linked list with values representing A, G, C, and T.
        
    Implement \textit{reverse\_dna}, which takes in a linked list \textit{strand} that represents a DNA strand. It destructively alters the linked list to reverse it. This function does not return anything.
    
    \begin{lstlisting}
        def reverse_dna(strand):
        """Reverses a DNA strand 
        >>> d = Link(“C”, Link(“A”, Link(“C”, Link(“G”)))) # <C A C G>
        >>> reverse_dna(d)
        >>> print(d)
        <G C A C>
        """
        assert isinstance(strand, Link)
            if ___________:
                return ____
            reverse_dna(_____)
            _________________
            _________________
            return strand        
    \end{lstlisting}
    
    \begin{solution}
        def reverse_dna(strand):
        """Reverses a DNA strand 
        >>> d = Link(“C”, Link(“A”, Link(“C”, Link(“G”)))) # <C A C G>
        >>> reverse_dna(d)
        >>> print(d)
        <G C A C>
        """
        assert isinstance(strand, Link)
            if strand is Link.empty or strand.rest is Link.empty:
                return strand
            reverse_dna(strand.rest)
            strand.rest.rest = strand
            strand.rest = Link.empty
            return strand
    \end{solution}
    \end{blocksection}
    
    \begin{questionmeta}
        Through this question. I hope to first introduce the theme of the next few questions of the worksheet. I used DNA as I thought it is a nice way of modelling real-world phenomena with a concept that people have learned. 

        This question requires students to have a good understanding of working with linked list pointers, for example the difference between .rest.rest and rest or when you can set a Link.empty. 

        This also shows how recursion can be integrated into a linked list question. I tried to add blanks in a manner that mirrors the 61a exams, as the blanks try to guide the student towards a particular solution. 
    \end{questionmeta}
    