\documentclass{exam}
\usepackage{../commonheader}
\lstset{language=Scheme}

%%% CHANGE THESE %%%%%%%%%%%%%%%%%%%%%%%%%%%%%%%%%%%%%%%%%%%%%%%%%%%%%%%%%%%%%%
\discnumber{13}
\title{\textsc{Introduction to Scheme}}
\date{April 10--April 14, 2023}
%%%%%%%%%%%%%%%%%%%%%%%%%%%%%%%%%%%%%%%%%%%%%%%%%%%%%%%%%%%%%%%%%%%%%%%%%%%%%%%

\begin{document}
\maketitle
\rule{\textwidth}{0.15em}

\begin{meta}
\begin{blocksection}
    \textbf{Recommended Timeline}
    \begin{itemize}
        \item Scheme Guided Minilecture: 25 mins (esther and i will handle this)
        \item Practice Problem #1: 5 mins
        \item Practice Problem #2: 5 mins
        \item Practice Problem #3: 7 mins
        \item Practice Problem #4: 8 mins
    \end{itemize}
\end{blocksection}
\end{meta}
You'll notice that the times on the meta for this worksheet actually add up to 50 minutes this time! This is because this section is meant to act as your students' first introduction to Scheme, as they haven't even been exposed to it during lecture (we're just giving them a head start ;-)).
\begin{meta}
\textbf{Teaching Tips}
\begin{itemize}
    \item To ease in Scheme, it can help to start by comparing and contrasting with Python
    \item Have students write a basic function in Python (like an iterative countdown), then replicate it in Scheme
    \item Have students list language features of Python (variable assignments, conditional statements, logic operators, etc.), and explain how Scheme implements those features
    \item Make sure to give a disclaimer that while high level features may be analogous, the internals are different!
    \item Scheme features break into three broad categories: Primitives, Call Expressions, and Special Forms (the latter two are called Compound Expressions)
    \item Primitives evaluate to themselves (4 evaluates to 4, \#t to \#t, etc.)
    \item Call Expressions begin with a function name and are followed by arguments- evaluate function name, evaluate arguments, and apply function to arguments
    \item Special Forms begin with a keyword and are followed by subexpressions, which are evaluated in a way based on the specific keyword
\end{itemize}
\end{meta}

\section{Scheme}
\subimport{../../topics/scheme/text/}{scheme-guided-overview.tex}
\begin{questions}
\subimport{../../topics/scheme/easy/}{hailstone.tex}
\subimport{../../topics/macros/easy/}{meta-apply.tex}
\subimport{../../topics/macros/easy/}{nand.tex}
\subimport{../../topics/macros/medium/}{apply-twice.tex}
\subimport{../../topics/macros/medium/}{python-if.tex}
\subimport{../../topics/macros/easy/}{combine-num.tex}
\subimport{../../topics/macros/medium/}{censor.tex}
\end{questions}
%%%%%%%%%%%%%%%%%%%%%%%%%%%%%%%%%%%%%%%%%%%%%%%%%%%%%%%%%%%%%%%%%%%%%%%%%%%%%%%

\end{document}
