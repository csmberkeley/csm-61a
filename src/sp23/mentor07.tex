\documentclass{exam}
\usepackage{../../commonheader}
\usepackage{setspace}

%%% CHANGE THESE %%%%%%%%%%%%%%%%%%%%%%%%%%%%%%%%%%%%%%%%%%%%%%%%%%%%%%%%%%%%%%
\discnumber{7}
\title{\textsc{Data Abstraction, Function-based Trees, and Mutability}}
\date{February 27--March 03, 2022}
%%%%%%%%%%%%%%%%%%%%%%%%%%%%%%%%%%%%%%%%%%%%%%%%%%%%%%%%%%%%%%%%%%%%%%%%%%%%%%%

\begin{document}
\maketitle
\rule{\textwidth}{0.15em}
\fontsize{12}{15}\selectfont

%%% INCLUDE TOPICS  HERE %%%%%%%%%%%%%%%%%%%%%%%%%%%%%%%%%%%%%%%%%%%%%%%%%%%%%%%

\begin{meta}
\textbf{Recommended Timeline}
\begin{itemize}
    \item ADT mini-lecture (5 min)
    \item Q1: Pokemon selectors (5 min)
    \item Q2: Are friends (3 min)
    \item Q3: Cross type friends (7 min)
    \item Q4: Pokemon constructor (8 min)
    \item Q5: Even Square Tree (8 min)
    \item Q6: All Paths (12 min)
    \item Q7: Contains N (12 min)
\end{itemize}
The times in the recommended timeline do not add up to 50 minutes because no mentor
is expected to get through all the problems during section. The worksheet is a skeleton
around which you should structure your section to best meet the needs of your students.
If you get stressed out about covering a lot of content, I encourage you to be open with
your students about the way these sessions are structured.  

You should probably ask your students at the beginning of section what they would rather go
over---ADTs or trees---and then allocate time appropriately. 
\end{meta}


\section{Abstraction}
\subimport{../../topics/data-abstraction/text/}{abstraction-summary.tex}
\begin{meta}
    If data abstraction is new to your students or they don't feel very confident in the topic, \textbf{consider walking them through the following problems}.
    
    Emphasize the \textbf{importance of selectors} -- useful for 2).
    
    A good visualization is to draw the data abstraction out using box and pointer diagrams. \textbf{Make sure not to get caught up on any specific representation of the data abstraction}, as they should be easy to change 3) is an alternate representation].
    
    Talk about what it means to \textbf{break the abstraction barrier. How do you make sure that you are not breaking the abstraction barrier?}
    \end{meta}
\begin{questions}
    \subimport{../../topics/data-abstraction/easy/}{pokemon.tex}
    \subimport{../../topics/lists/mutable/medium/}{accumulate.tex}
\end{questions}

\newpage
\section{Tree ADT}
\subimport{../../topics/trees/adt/text/}{tree_overview.tex}
\begin{questions}
    \subimport{../../topics/trees/adt/medium}{even-square-tree.tex}
    \subimport{../../topics/trees/adt/medium}{all-paths.tex}
    \subimport{../../topics/trees/adt/hard}{contains-n.tex}
\end{questions}

\end{document}