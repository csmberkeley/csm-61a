\documentclass{exam}
\usepackage{../commonheader}

\discnumber{3}
\title{Environmental Diagrams \titlebreak and Higher-Order Functions}
\date{January 30--February 3, 2023}

\begin{document}
\maketitle

\begin{blocksection}
\begin{guide}
\textbf{Recommended Timeline}

Times do not add to one hour because it is not expected that any mentor will get through all questions in a single hour. 

\begin{itemize}
    \item Environment Diagrams Mini-Lec + New Frame Written Question - 12 mins
    \item Swap - 8 mins
    \item Joke - 7 mins
    \item Higher Order Functions Mini-Lec + Written Question - 3 mins
    \item Foobar or Apple (Q2 or Q3): these two problems are pretty similar - 10 mins
    \item xyz - 8 mins
    \item whole\_sum or mystery (Q5 or Q6+7) - if there isn't enough time, pick depending and how students feel;
    5 is a pretty standard HOF question that involves digit manipulation, 6+7 for some more practice with lambdas - 10 mins
\end{itemize}
\end{guide}
\end{blocksection}


\section{Environment Diagrams}

\begin{questions}
\subimport{../../topics/environments/easy/}{new-frame.tex}
\subimport{../../topics/functions-and-expressions/medium/env-diagram/}{swap.tex}
\subimport{../../topics/functions-and-expressions/medium/env-diagram/}{joke.tex}
\end{questions}

\section{Higher-Order Functions}
\begin{questions}
\subimport{../../topics/hof/easy/}{why.tex}
\subimport{../../topics/hof/easy/env-diagram/}{foobar.tex}
\begin{questionmeta}
    It's probably not super important for you to do both \lstinline{foobar} and \lstinline{apple}, as these questions are pretty similar. 
\end{questionmeta}
\subimport{../../topics/hof/medium/env-diagram/}{apple.tex}
\begin{questionmeta}
    It's probably not super important for you to do both \lstinline{foobar} and \lstinline{apple}, as these questions are pretty similar. 
\end{questionmeta}

\subimport{../../topics/hof/medium/}{xyz.tex}
\subimport{../../topics/hof/medium/}{check-sum.tex}

\subimport{../../topics/hof/medium/}{mystery.tex}
\subimport{../../topics/hof/medium/wwpd/}{mystery.tex}
\clearpage

\end{questions}

\end{document}
