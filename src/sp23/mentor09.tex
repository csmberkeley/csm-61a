\documentclass{exam}
\usepackage{../../commonheader}
\usepackage{outlines}

%%% CHANGE THESE %%%%%%%%%%%%%%%%%%%%%%%%%%%%%%%%%%%%%%%%%%%%%%%%%%%%%%%%%%%%%%
\discnumber{9}
\title{\textsc{Object Oriented Programming}}
\date{March 13--March 17, 2023}
%%%%%%%%%%%%%%%%%%%%%%%%%%%%%%%%%%%%%%%%%%%%%%%%%%%%%%%%%%%%%%%%%%%%%%%%%%%%%%%

\begin{document}
	\maketitle
	\rule{\textwidth}{0.15em}
	\fontsize{12}{15}\selectfont

	%%% INCLUDE TOPICS  HERE %%%%%%%%%%%%%%%%%%%%%%%%%%%%%%%%%%%%%%%%%%%%%%%%%%%%%%%
\begin{meta}
	\textbf{Recommended Timeline}
	\begin{itemize}
		\item Section 1: Object Oriented Programming
		\begin{itemize}
		\item OOP Mini Lecture: 8 minutes
		\item Q1: WWPD (Star Wars): 10 minutes
		\item Q2: Build a Bear: 15 minutes
		\end{itemize}
	\end{itemize}
Yeah, I know. Both OOP and linked lists. On the same worksheet. Unfortunately, this is just how the scheduling for the course worked out. Sorry about that. As you might imagine, this worksheet might be a little longer than usual, so don't worry if you don't get to all the problems. The worksheets are a question bank around which you can structure your section to best meet the needs of your students. The times do not add up to 50 minutes for this reason. 

A little note on the above: When I was a JM, I would often feel bad when I didn't get to all (or most) of the problems on the worksheet. I think I saw the worksheet as a thing to be conquered. Since then, I've learned that this is not a good way to look at the world. I've said this on every single meta, and I'll say it again: your goal is not to get through every problem on the worksheet. It's to help your students. Go at the appropriate pace for you and your students, and you'll be golden :) Anything you don't get to can be extra practice for them on their own time. 

Again, I highly recommend that you not spend too much time in mini-lecture. Only go over what your students need you to go over, because the active learning involved in problem solving is far more instructive. 

\end{meta}

\section{Object Oriented Programming}
\subimport{../../topics/oop/text/}{oop_overview.tex}

\subimport{../../topics/oop/text}{inheritance_overview.tex}

\newpage
\begin{questions}
\subimport{../../topics/oop/easy/}{star-wars.tex}
\newpage
\subimport{../../topics/oop/medium/}{photosynthesis.tex}
\end{questions}

\end{document}