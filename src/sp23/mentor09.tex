\documentclass{exam}
\usepackage{../../commonheader}
\usepackage{outlines}

%%% CHANGE THESE %%%%%%%%%%%%%%%%%%%%%%%%%%%%%%%%%%%%%%%%%%%%%%%%%%%%%%%%%%%%%%
\discnumber{9}
\title{\textsc{Object Oriented Programming}}
\date{March 13--March 17, 2023}
%%%%%%%%%%%%%%%%%%%%%%%%%%%%%%%%%%%%%%%%%%%%%%%%%%%%%%%%%%%%%%%%%%%%%%%%%%%%%%%

\begin{document}
	\maketitle
	\rule{\textwidth}{0.15em}

	%%% INCLUDE TOPICS  HERE %%%%%%%%%%%%%%%%%%%%%%%%%%%%%%%%%%%%%%%%%%%%%%%%%%%%%%%
\begin{meta}
	\textbf{Recommended Timeline}
	\begin{itemize}
		\item OOP Mini Lecture: 8 minutes
		\item Q1: WWPD (Star Wars): 10 minutes
		\item Q2: PingPongTracker: 10 minutes
		\item Q3: TeamBaller: 8 minutes
		\item Q4: Photosynthesis: 15 minutes ***HIGHLY RECOMMENDED*** 
	\end{itemize}
With this semester being lighter in terms of the scheduling of course content, this week is purely dedicated to OOP. As such is the case, this worksheet is probably among the lighter ones this semester, and is dedicated to making sure your students are comfortable with the basics of OOP. You'll notice that this week, the times add to pretty much 50 minutes. This is not to say that you should do the whole worksheet, however -- gauge student weak points and feel free to review topics/clarify accordingly.

As an additional note: students will be fairly new to the concept of inheritance, feel free to take more time explaining a high-level overview of inheritance. 
\end{meta}

\section{Object Oriented Programming}
\subimport{../../topics/oop/text/}{oop_overview.tex}
\subimport{../../topics/oop/text/}{inheritance_overview.tex}
\newpage
\begin{questions}
\subimport{../../topics/oop/easy/}{star-wars.tex}
\newpage
\subimport{../../topics/oop/medium/}{pingpong.tex}
\newpage
\subimport{../../topics/oop/easy/}{team-baller-fa21.tex}
\newpage
\subimport{../../topics/oop/medium/}{photosynthesis.tex}
\end{questions}

\end{document}