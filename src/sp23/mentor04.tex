\documentclass{exam}
\usepackage{../commonheader}

%%% CHANGE THESE %%%%%%%%%%%%%%%%%%%%%%%%%%%%%%%%%%%%%%%%%%%%%%%%%%%%%%%%%%%%%%
\discnumber{4}
\title{Higher-Order Environments, Currying, and Function Scope}
\date{February 6--February 10, 2023}
%%%%%%%%%%%%%%%%%%%%%%%%%%%%%%%%%%%%%%%%%%%%%%%%%%%%%%%%%%%%%%%%%%%%%%%%%%%%%%%

\begin{document}
\maketitle
\rule{\textwidth}{0.15em}
\fontsize{12}{15}\selectfont

%%% INCLUDE TOPICS HERE %%%%%%%%%%%%%%%%%%%%%%%%%%%%%%%%%%%%%%%%%%%%%%%%%%%%%%%
\begin{meta}
\textbf{Recommended Timeline}
\begin{itemize}
    \item Higher-order environment and currying mini-lecture
    \item Compound OR Partial Summer: 15 minutes
    \item Selective Sum: 10 minutes
    \item is\_sorted and/or collapse: 10--20 minutes
    \item Mandelbrot: 20+ minutes, only do if your students are very comfortable with recursion
\end{itemize}
As a reminder, there is no expectation that you get through all problems in a section. Pick the most pertinent problems for your section. 
\end{meta}

\begin{questions}
    \section{Higher-Order Functions cont.}
    \subimport{../../topics/hof/hard/}{compound.tex}
    \subimport{../../topics/hof/hard/}{partial_summer.tex}
    \begin{questionmeta}
        Generally only choose one of the two problems to work on, since they cover similar ground in terms of HOFs. These are on the harder side so also consider doing the rest of the worksheet and coming back.
    \end{questionmeta}

    \subimport{../../topics/environments/medium/}{abde.tex}

    \section{Recursion I}
    \subimport{../../topics/recursion/text/}{recursion_overview.tex}
    \subimport{../../topics/trees/easy/}{wrong_factorial.tex}
\end{questions}

\end{document}
