\documentclass{exam}
\usepackage{../commonheader}

%%% CHANGE THESE %%%%%%%%%%%%%%%%%%%%%%%%%%%%%%%%%%%%%%%%%%%%%%%%%%%%%%%%%%%%%%
\discnumber{5}
\title{Recursion, Tree Recursion}
\date{February 13--February 17, 2022}
%%%%%%%%%%%%%%%%%%%%%%%%%%%%%%%%%%%%%%%%%%%%%%%%%%%%%%%%%%%%%%%%%%%%%%%%%%%%%%%

\begin{document}
\maketitle
\rule{\textwidth}{0.15em}
\fontsize{12}{15}\selectfont

%%% INCLUDE TOPICS HERE %%%%%%%%%%%%%%%%%%%%%%%%%%%%%%%%%%%%%%%%%%%%%%%%%%%%%%%
\begin{meta}
\textbf{Recommended Timeline}
\begin{itemize}
    \item Tree recursion mini lecture: 8 minutes
    \item Gibonacci: 15 minutes -- skippable if your students feel especially comfortable with tree recursion; otherwise please do this.
    \item FizzBuzz: 8 minutes -- could be used as a step between Gibonacci and Selective Sum.
    \item Selective Sum: 10 minutes
    \item Has Sum: 10 minutes
    \item Mario Number: 10 minutes
    \item Collapse: 10--20 minutes
\end{itemize}
As a reminder, there is no expectation that you get through all problems in a section. Choose questions that will best benefit your students.
\end{meta}

\subimport{../../topics/recursion/text/}{tree_recursion_overview.tex}
\begin{questions}
    \subimport{../../topics/recursion/easy/}{gibonacci.tex}
    \subimport{../../topics/recursion/medium/}{fizzbuzz.tex}
    \subimport{../../topics/recursion/medium/}{selective-sum.tex}
    \subimport{../../topics/recursion/medium/}{discussions1.tex} 
    \subimport{../../topics/recursion/medium/}{mario-number.tex}
    \subimport{../../topics/recursion/medium/}{collapse.tex}
\end{questions}

\end{document}
