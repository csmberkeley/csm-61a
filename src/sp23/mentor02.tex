\documentclass{exam}
\usepackage{../commonheader}

\discnumber{2}
\title{Python, Functions, Expressions, and Control}
\date{January 23--January 27, 2022}

\begin{document}
\maketitle
\begin{meta}
\textbf{Recommended Timeline}
\begin{itemize}
  \item Introductions/Expectations/Icebreaker [10 minutes]
  \item Q1: What Would Python Display + Minilecture/explanations [15 minutes]
  \item Q2: Order of evaluation [5 min]
    \subitem{Note: don't need to do all parts if students get the idea}
  \item Code Writing [20 minutes]
    \subitem{- Again, no need to do all questions}
    \subitem{- 2 and 3 are medium difficulty \& the last one is a bit more challenging, pick based on how your students are feeling}
  \item Tips and Tricks for succeeding in CS61A [leftover time]
\end{itemize}
\end{meta}


\section{Intro to Python}
\begin{questions}
\subimport{../../topics/basics/}{potpurri.tex}
\subimport{../../topics/basics/}{funky-funcs.tex}
\begin{questionmeta}
  It's probably not necessary to get through all the parts of this problem if your students get the idea. The last subpart is probably the most instructive. 
  For question 2, feel free to remind students of the general ``order of operations'' of functions, in that they start inward and expand outward. Feel free to use any analogies, that help students understand.
  If your students are still confused, it's advisable to walk through the first few problems with them step by step.
\end{questionmeta}
\end{questions}

\section{Control}
\begin{questions}
\subimport{../../topics/control/easy/}{divisible-by-4.tex}
\begin{questionmeta}
  The purpose of this problem is essentially to ensure that students are familiar with Python syntax. Its solution involves little complex thought. 
\end{questionmeta}
\subimport{../../topics/control/medium/}{fizzbuzz.tex} 
\begin{questionmeta}
  Students may overthink this problem, trying to find a ``clever'' way to do it with a smaller number of comparisons.
  There are many variations of this problem. If students solve this easily, encourage them to find a way to solve this with a for loop, or some other variation.
\end{questionmeta}
\subimport{../../topics/control/medium/}{pow-of-2.tex}
\begin{questionmeta}
  If your students are not familiar with factorials, you may want to give them a brief overview before going over this problem. 
\end{questionmeta}
\subimport{../../topics/control/challenge/}{min_fact.tex}
\begin{questionmeta}
  This question differs significantly from a similar question that was used in past years. Please review it carefully before teaching it.
  One area in which students may need help is addressing the edge case in which n is greater than the limit.
\end{questionmeta}
\end{questions}

\end{document}
