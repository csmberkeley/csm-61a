\documentclass{exam}
\usepackage{../commonheader}
\usepackage{outlines}

%%% CHANGE THESE %%%%%%%%%%%%%%%%%%%%%%%%%%%%%%%%%%%%%%%%%%%%%%%%%%%%%%%%%%%%%%
\discnumber{10}
\title{\textsc{Scheme Data Abstraction}}
\date{Aug 1 - 3, 2022}
%%%%%%%%%%%%%%%%%%%%%%%%%%%%%%%%%%%%%%%%%%%%%%%%%%%%%%%%%%%%%%%%%%%%%%%%%%%%%%%

\begin{document}
\maketitle
\rule{\textwidth}{0.15em}
\fontsize{12}{15}\selectfont

\begin{guide}
\begin{blocksection}
\textbf{Recommended Timeline}
\begin{outline}[enumerate]
    \1 Note: this worksheet might be on the shorter end, if you have extra time - feel free to review interpreters or go over the exam level question
    \1 Scheme Data Abstraction Intro - 5 minutes
    \1 Elephants - 10 minutes
    \2 introduction to creator and selector methods
    \2 building scheme lists + selector methods
    \2 finding data abstraction violations
    \2 emphasize the point of data abstraction
    \1 Summation - 20 minutes
    \2 fill in the selector methods
    \2 students might not have seen reduce, feel free to introduce/reference it in the built-in procedures
    \2 for the final task, ask students how they can utilize the summation constructor
    \1 Past Exam Question: Sp22 Final Q12 - optional
    \2 scheme data abstraction question! combines knowledge of constructors, selectors, and not violating the abstraction barrier
\end{outline} 
\end{blocksection}
\end{guide}

\section{Scheme Data Abstraction}
\subimport{../../topics/data-abstraction/text/}{abstraction-summary-scheme.tex}

\section{Elephants}
\begin{questions}
\subimport{../../topics/data-abstraction/easy/}{elephants-scheme.tex}
\end{questions}
\newpage

\section{Summation}
\begin{questions}
\subimport{../../topics/data-abstraction/medium/}{summation.tex}
\end{questions}

% \newpage
% \begin{questions}
% \subimport{../../topics/data-abstraction/medium/}{bookshelf.tex}
% \end{questions}

% \newpage
\section{Past Exam Questions}
\begin{outline}[enumerate]
    \1 \href{https://cs61a.org/exam/sp22/final/61a-sp22-final.pdf#page=25}{Sp22 Final Q12}
\end{outline}

%%%%%%%%%%%%%%%%%%%%%%%%%%%%%%%%%%%%%%%%%%%%%%%%%%%%%%%%%%%%%%%%%%%%%%%%%%%%%%%

\end{document}