\documentclass{exam}
\usepackage{../../commonheader}

%%% CHANGE THESE %%%%%%%%%%%%%%%%%%%%%%%%%%%%%%%%%%%%%%%%%%%%%%%%%%%%%%%%%%%%%%
\discnumber{7}
\title{\textsc{Trees and Linked Lists}}
\date{July 23, 2021}
%%%%%%%%%%%%%%%%%%%%%%%%%%%%%%%%%%%%%%%%%%%%%%%%%%%%%%%%%%%%%%%%%%%%%%%%%%%%%%

\begin{document}
\maketitle
\rule{\textwidth}{0.15em}
\fontsize{12}{15}\selectfont

%%% INCLUDE TOPICS  HERE %%%%%%%%%%%%%%%%%%%%%%%%%%%%%%%%%%%%%%%%%%%%%%%%%%%%%%%
\begin{guide}
\textbf{Recommended Timeline}
\begin{itemize}
    \item Notes
    \begin{itemize}
        \item In past semesters, 61a introduced the tree data abstraction and then mutable trees 
        (the Tree class) afterwards. Now, we're starting straight away with mutable trees!
        \item Even for students who have prior coding experience, trees and linked lists can be difficult to work with. 
        If your students were comfortable moving more quickly before, be more careful about pacing to ensure students have 
        a solid conceptual understanding of these topics.
        \item This worksheet covers two tricky topics and is pretty long so don't expect to cover everything!
    \end{itemize}
    \item Trees Review/Mini-Lecture - 5 min
    \item Q1 Tree Sum: the remaining questions are a bit more difficult than this one so make sure students have
    understand this question before moving on - 7 min
    \item The next three tree questions are in increasing order of difficulty.
    Q2 and Q3 both use helper functions to keep track of some additional information so I'd suggest choosing one of the two.
    Q4 is a bit more challenging so feel free to skip!
    \begin{itemize}
        \item Q2 Delete Path Duplicates - 7 min
        \item Q3 Replace Leaves Sum - 8 min
        \item (If extra time and/or challenge needed): Q4 contains\_n - 10 min
    \end{itemize}
    \item Linked Lists Review/Mini-Lecture - 5 min
    \begin{itemize}
        \item knowledge of the str and repr methods are not required for any of the worksheet questions. Students may also not be super 
        familiar with representation since it was right before the midterm and never explicitly covered in discussion. No need to  
        really emphasize those methods when reviewing the Linked List class.
        \item linked lists are like trees that can only have one branch
    \end{itemize}
    \item Q1 WWPD: drawing box and pointers here is really helpful for students to visualize linked lists! - 7 min
    \item Q2 Skip - 6 min
    \item Q3 Skip (no mutate) - 4 min: You may choose to skip this problem to spend more time on mutable tree/other questions
     (or, quickly walk through the solution so students are aware it exists).
    \item Q4 Reverse - 10 min
\end{itemize}
\end{guide}

\newpage
\section{Trees}
\subimport{../../topics/trees/class/text/}{implementation.tex}
\subimport{../../topics/trees/class/text/}{class_overview.tex}
\begin{questions}
    \subimport{../../topics/trees/class/easy/}{tree-sum.tex}
    \newpage
    \subimport{../../topics/trees/class/medium/}{delete-path-duplicates.tex}
    \newpage
    \subimport{../../topics/trees/class/medium/}{replace-leaves-sum.tex}
    \subimport{../../topics/trees/class/hard/}{contains-n.tex}
\end{questions}

\section{Linked Lists}
\subimport{../../topics/linked-lists/class/text/}{linked_list_overview.tex}
\subimport{../../topics/linked-lists/class/text/}{implementation.tex}
\newpage
\begin{questions}
\subimport{../../topics/linked-lists/class/easy/}{wwpd.tex}
\newpage
\subimport{../../topics/linked-lists/class/easy/}{skip.tex}
\subimport{../../topics/linked-lists/class/easy/}{skip-no-mutate.tex}
\subimport{../../topics/linked-lists/class/medium/}{reverse.tex}
\end{questions}

\end{document}
