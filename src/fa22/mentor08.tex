\documentclass{exam}
\usepackage{../../commonheader}

%%% CHANGE THESE %%%%%%%%%%%%%%%%%%%%%%%%%%%%%%%%%%%%%%%%%%%%%%%%%%%%%%%%%%%%%%
\discnumber{8}
\title{\textsc{Mutability, Iterators, and Generators}}
\date{October 10--October 14, 2022}
%%%%%%%%%%%%%%%%%%%%%%%%%%%%%%%%%%%%%%%%%%%%%%%%%%%%%%%%%%%%%%%%%%%%%%%%%%%%%%%

\begin{document}
\maketitle
\rule{\textwidth}{0.15em}
\fontsize{12}{15}\selectfont

%%% INCLUDE TOPICS  HERE %%%%%%%%%%%%%%%%%%%%%%%%%%%%%%%%%%%%%%%%%%%%%%%%%%%%%%%

\begin{guide}
    \newline
    \textbf{Recommended Timeline}
    \begin{itemize}
        \item Mutability
        \begin{itemize}
            \item Mutability Mini Lecture - 5 minutes
            \item wwpd-mutation - 5 minutes
            \item mystery - 5 minutes
            \item nice-ice-cream - 7 minutes
            \item append-to-all - 5 minutes
            \item accumulate - 10 minutes*
            \item There are a lot of questions and topics, so pick and choose between these problems based on your students comfortability.
            \item Choose at least one of the environment diagram problems. Mystery is an easier choice, nice-ice-cream is more complex.
            \item wwpd-mutation and append-to-all are easier problems if your students are struggling with these concepts.
            \item Accumulate is tougher so only do this one if your students are understanding the concepts.
        \end{itemize}
        \item Iterator \& Generator Focus:
        \begin{itemize}
            \item Iterators \& Generators Mini Lecture - 6 minutes
            \item Interleave - 10 minutes
            \item List Gen - 8 minutes
            \item In Order - 8 minutes
        \end{itemize}
    \end{itemize}
\end{guide}

\section{Mutability}
	\textbf{Mutation}

	\subimport{../../topics/lists/mutable/text/}{mutation_overview.tex}
	\textbf{Mutable vs immutable}

	\subimport{../../topics/lists/mutable/text/}{mutable-vs-immutable.tex}
	\begin{questions}
		\subimport{../../topics/lists/mutable/medium/}{wwpd-mutation.tex}
		\newpage
		\subimport{../../topics/lists/mutable/easy/env-diagram/}{mystery.tex}
		\newpage
    \subimport{../../topics/lists/mutable/medium/env-diagram/}{nice-ice-cream.tex}
    \newpage
    \subimport{../../topics/lists/mutable/easy/}{append-to-all.tex}
    \subimport{../../topics/lists/mutable/medium/}{accumulate.tex}
	\end{questions}

\section{Iterators \& Generators}
\subimport{../../topics/generators/text/}{generators_overview.tex}
\begin{questions}
\subimport{../../topics/iterators/easy/}{interleave.tex}
\subimport{../../topics/generators/easy/}{foo.tex}
\begin{questionmeta}
    Between foo and list-gen, pick one that would best fit your students' needs. They both do a pretty good job of introducing general generator concepts.
\end{questionmeta}
\subimport{../../topics/generators/easy/}{list-gen.tex}
\subimport{../../topics/generators/medium/}{hailstone.tex}

\end{questions}

\end{document}
