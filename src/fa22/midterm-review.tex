\documentclass{exam}
\usepackage{../../commonheader}

%%% CHANGE THESE %%%%%%%%%%%%%%%%%%%%%%%%%%%%%%%%%%%%%%%%%%%%%%%%%%%%%%%%%%%%%%
\discnumber{?}
\title{Midterm Review}
\date{October 23rd, 2022}
%%%%%%%%%%%%%%%%%%%%%%%%%%%%%%%%%%%%%%%%%%%%%%%%%%%%%%%%%%%%%%%%%%%%%%%%%%%%%%%

\begin{document}
\maketitle
\rule{\textwidth}{0.15em}
\fontsize{12}{15}\selectfont

%%% INCLUDE TOPICS  HERE %%%%%%%%%%%%%%%%%%%%%%%%%%%%%%%%%%%%%%%%%%%%%%%%%%%%%%%

\section{Recursion}
    \begin{blocksection}
        A skip list is defined as a sublist of a list such that each element in the sublist is non adjacent in the
        original list. For original list \lstinline{[5, 6, 8, 2]}, the lists \lstinline{[5, 8]}, \lstinline{[5, 2]}, \lstinline{[6, 2]}, \lstinline{[5]}, \lstinline{[6]}, \lstinline{[8]}, \lstinline{[2]}, \lstinline{[]} are
        all skip lists of the original list. The empty list is always a skip list of any list.

        Given a list \lstinline{int_lst} of unique integers, return a list of all unique skip lists of \lstinline{int_lst} where each skip
        list contains integers in strictly increasing order. The order in which the skip lists are returned does not matter.
        
        \begin{lstlisting}
        def list_skipper(int_list):
            """
            >>> list_skipper([5,6,8,2])
            [[5, 8], [5], [6], [8], [2], []]
            >>> list_skipper([1,2,3,4,5])
            [[1, 3, 5], [1, 3], [1, 4], [1, 5], [1], [2, 4], [2, 5], [2], [3, 5], [3], [4], [5], []]
            >>> list_skipper([])
            [[]]

            """
            if len(int_lst) == 0:
                return ________________
            with_first = ___________________
            without_first = ____________________
            with_first = [ _________ for x in with_first if x == [] or ____________]
            return with_first + without_first
        \end{lstlisting}
    \end{blocksection}
    \begin{solution}
        \begin{lstlisting}
            def list_skipper(int_list):
                if len(int_lst) == 0:
                    return [[]]
                with_first = list_skipper(int_lst[2:])
                without_first = list_skipper(int_lst[1:])
                with_first = [ [int_lst[0]] + x for x in with_first if x == [] or x[0] > int_lst[0]]
                return with_first + without_first
        \end{lstlisting}
    \end{solution}
    \begin{blocksection}
        \begin{lstlisting}
            def maxkd(meteor, k):
                """
                Given a number `meteor`, finds the largest number of length `k` or fewer,
                composed of digits from `meteor`, in order.
            
                >>> maxkd(1234, 1)
                4
                >>> maxkd(32749, 2)
                79
                >>> maxkd(1917, 2)
                97
                >>> maxkd(32749, 18)
                32749
                """
                if ______:
                    return ______
                a = ______
                b = ______
                return ______
        \end{lstlisting}
    \end{blocksection}
    \begin{solution}
        \begin{lstlisting}
            def maxkd(meteor, k):
                """
                Given a number `meteor`, finds the largest number of length `k` or fewer,
                composed of digits from `meteor`, in order.

                >>> maxkd(1234, 1)
                4
                >>> maxkd(32749, 2)
                79
                >>> maxkd(1917, 2)
                97
                >>> maxkd(32749, 18)
                32749
                """
                if meteor == 0 or k == 0:
                    return 0
                a = maxkd(meteor // 10, k - 1) * 10 + meteor % 10
                b = maxkd(meteor // 10, k)
                return max(a, b)
        \end{lstlisting}
    \end{solution}
\section{Lists, Mutability and Dictionaries}
\begin{questions}
    \subimport{../../topics/growth/easy/}{jumpstart.tex}
    \subimport{../../topics/growth/easy/}{belgian-waffle.tex}
\end{questions}

\section{Higher Order Functions}
\begin{questions}
    \subimport{../../topics/hof/medium/}{make-digit-remover.tex}
\end{questions}

\section{Recursion}
\begin{questions}
    \subimport{../../topics/recursion/medium/}{add-up.tex}
\end{questions}

\section{Lists}
\begin{questions}
    \subimport{../../topics/lists/mutable/medium/}{incredibles.tex}
    \subimport{../../topics/recursion/medium/}{subsets.tex}
\end{questions}

\newpage

\section{Iterators and Generators}
\begin{questions}
    \subimport{../../topics/generators/medium/}{num-elems.tex}
\end{questions}

\section{Object Oriented Programming}
\begin{questions}
    \subimport{../../topics/oop/medium/}{pingpong.tex}
\end{questions}

\section{Linked Lists}
\begin{questions}
    \subimport{../../topics/linked-lists/class/medium/}{combine-two.tex}
    \subimport{../../topics/linked-lists/class/hard/}{insert-at.tex}
\end{questions}
\end{document}
