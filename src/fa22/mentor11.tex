\documentclass{exam}
\usepackage{../../commonheader}
\usepackage{outlines}

%%% CHANGE THESE %%%%%%%%%%%%%%%%%%%%%%%%%%%%%%%%%%%%%%%%%%%%%%%%%%%%%%%%%%%%%%
\discnumber{11}
\title{\textsc{Scheme}}
\date{October 31--November 4, 2022}
\lstset{language=Scheme}
%%%%%%%%%%%%%%%%%%%%%%%%%%%%%%%%%%%%%%%%%%%%%%%%%%%%%%%%%%%%%%%%%%%%%%%%%%%%%%%

\begin{document}
\maketitle
\rule{\textwidth}{0.15em}
\fontsize{12}{15}\selectfont

\begin{meta}
\begin{blocksection}
\textbf{Recommended Timeline}
\begin{itemize}
    \item Scheme mini-lecture: 15 min
    \item What Would Scheme Do? (Basics): 7 min
    \item Apply Multiple: 10 min
    \item Hailstone: 13 min 
    \item Scheme lists mini-lecture: 10 min
    \item What Would Scheme Do? (Lists): 7 min
    \item Is Prefix?: 10 min
    \item Argmax: 15 min
\end{itemize}
\end{blocksection}
\end{meta}

\section{Scheme}
\begin{meta}
The text explanations on this worksheet are very lengthy. \textbf{Do not go over all of this material in mini-lecture.} That would be an incredible waste of time. Instead, ask your students what they would like you to go over and focus on that material in mini-lecture. They will have already been exposed to much of this material in lecture and discussion, so they shouldn't need to be completely retaught it again. 
\textbf{Teaching Tips}
\begin{outline}[enumerate]
    \1 To ease in Scheme, it can help to start by comparing and contrasting with Python
    \2 Have students write a basic function in Python (like an iterative countdown), then replicate it in Scheme
    \2 Have students list language features of Python (variable assignments, conditional statements, logic operators, etc.), and explain how Scheme implements those features
    \2 Make sure to give a disclaimer that while high level features may be analogous, the internals are different!
    \1 Scheme features break into three broad categories: Primitives, Call Expressions, and Special Forms (the latter two are called Compound Expressions)
    \2 Primitives evaluate to themselves (4 evaluates to 4, \#t to \#t, etc.)
    \2 Call Expressions begin with a function name and are followed by arguments- evaluate function name, evaluate arguments, and apply function to arguments
    \2 Special Forms begin with a keyword and are followed by subexpressions, which are evaluated in a way based on the specific keyword
    \1 Useful Links
    \2 \href{https://cs61a.org/articles/scheme-spec/}{Scheme Specification} (for overfiew, types, and special forms)
    \2 \href{https://cs61a.org/articles/scheme-builtins/}{Scheme Built-In Procedure Reference} (for built-in procedures)
\end{outline}
\end{meta}

\subimport{../../topics/scheme/text/}{scheme-basics-overview.tex}
\begin{questions}
\subimport{../../topics/scheme/easy/wwsd/}{basics.tex}
\subimport{../../topics/scheme/medium/}{apply-n.tex}
\subimport{../../topics/scheme/easy/}{hailstone.tex}
\end{questions}

\section{Scheme Lists}
\subimport{../../topics/scheme/text/}{scheme-lists-overview.tex}
\newpage
\begin{questions}
\subimport{../../topics/scheme/easy/wwsd/}{lists_nodots.tex}
\subimport{../../topics/scheme/medium/}{is-prefix.tex}
\subimport{../../topics/scheme/medium/}{argmax.tex}


\end{questions}

\end{document}