\documentclass{exam}
\usepackage{../commonheader}
\lstset{language=Scheme}

%%% CHANGE THESE %%%%%%%%%%%%%%%%%%%%%%%%%%%%%%%%%%%%%%%%%%%%%%%%%%%%%%%%%%%%%%
\discnumber{13}
\title{\textsc{Programs as Data \& Macros}}
\date{November 14--November 18, 2022}
%%%%%%%%%%%%%%%%%%%%%%%%%%%%%%%%%%%%%%%%%%%%%%%%%%%%%%%%%%%%%%%%%%%%%%%%%%%%%%%

\begin{document}
\maketitle
\rule{\textwidth}{0.15em}

\begin{meta}
\begin{blocksection}
    \textbf{Recommended Timeline}
    \begin{itemize}
        \item Macros Intro - 10 minutes
        \item Q1 (\lstinline{macros-quasi-wwsd}): 8 minutes
        \item Q2 (\lstinline{meta-apply}): 5 minutes
        \item Q3 (\lstinline{NAND}): 5 minutes
        \item Q4 (\lstinline{apply-twice}): 10 minutes
        \item Q5 (\lstinline{python-if}): 10 minutes
        \item Q6 (\lstinline{combine-num}): 10 minutes
        \item Q7 (\lstinline{censor}): 15 minutes 
    \end{itemize}
\end{blocksection}
\end{meta}
We have provided a very large number of problems with this worksheet. You should pick and choose the ones you want to best help your students! Take a healthy sampling of the different difficulty levels and have fun!

As always, no mentor is expected to get through every problem on this worksheet. 
\begin{meta}
\textbf{Teaching Tips}
\begin{itemize}
    \item Before we jump into Macros, it is very important to ensure that your students understand quasiquotation.
    \item Take your time with the quasiquotation WWSD and drawing parallels with Python's f-strings may make it easier for your students to understand 
    \item Draw out box and pointer diagrams to show how the expressions in macros are being stored when the operands are unevaluated
    \item If you would like a quick refresher on how to think about macros, please refer to this link! \href{https://docs.google.com/document/d/1JSbvtJ5bYUEhovDZd_gQnBvkG_WDcafmX-4B3QeIXZU/edit}{guide}
\end{itemize}
\end{meta}

\section{Macros}
\subimport{../../topics/macros/text/}{intro.tex}
\begin{questions}
\subimport{../../topics/macros/easy/}{macros-quasi.tex}
\subimport{../../topics/macros/easy/}{meta-apply.tex}
\subimport{../../topics/macros/easy/}{nand.tex}
\subimport{../../topics/macros/medium/}{apply-twice.tex}
\subimport{../../topics/macros/medium/}{python-if.tex}
\subimport{../../topics/macros/easy/}{combine-num.tex}
\subimport{../../topics/macros/medium/}{censor.tex}
\end{questions}
%%%%%%%%%%%%%%%%%%%%%%%%%%%%%%%%%%%%%%%%%%%%%%%%%%%%%%%%%%%%%%%%%%%%%%%%%%%%%%%

\end{document}
