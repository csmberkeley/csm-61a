\documentclass{exam}
\usepackage{../commonheader}

\discnumber{2}
\title{Python and Control}
\date{August 31--September 2, 2022}

\begin{document}
\maketitle
\begin{meta}
\textbf{Recommended Timeline}
\begin{itemize}
  \item Introductions/Expectations/Icebreaker [10 minutes]
  \item What Would Python Display + Minilecture/explanations [15 minutes]
  \item Operators and Operands [5 minutes]
    \subitem{- Note: don't need to do all parts if students get the idea}
  \item Code Writing [20 minutes]
    \subitem{- Again, no need to do all questions}
    \subitem{- 2 and 3 are medium difficulty \& the last one is a bit more challenging, pick based on how your students are feeling}
  \item Tips and Tricks for succeeding in CS61A [leftover time]
\end{itemize}
\end{meta}


\section{Intro to Python}
\begin{questions}
\subimport{../../topics/basics/}{potpurri.tex}
\subimport{../../topics/basics/}{funky-funcs.tex}
\end{questions}

\section{Control}
\begin{questions}
\subimport{../../topics/control/easy/}{divisible-by-4.tex}
\subimport{../../topics/control/medium/}{find-max.tex}
\subimport{../../topics/control/medium/}{pow-of-2.tex}
\begin{questionmeta}
  If your students are not familiar with factorials, you may want to give them a brief overview before going over this problem. 
\end{questionmeta}
\subimport{../../topics/control/challenge/}{min_fact.tex}
\end{questions}

\end{document}
