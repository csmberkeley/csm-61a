\documentclass{exam}
\usepackage{../../commonheader}
\usepackage{setspace}

%%% CHANGE THESE %%%%%%%%%%%%%%%%%%%%%%%%%%%%%%%%%%%%%%%%%%%%%%%%%%%%%%%%%%%%%%
\discnumber{6}
\title{\textsc{Sequences and Containers}}
\date{October 3--October 007, 2022}
%%%%%%%%%%%%%%%%%%%%%%%%%%%%%%%%%%%%%%%%%%%%%%%%%%%%%%%%%%%%%%%%%%%%%%%%%%%%%%%

\begin{document}
\maketitle
\rule{\textwidth}{0.15em}
\fontsize{12}{15}\selectfont

%%% INCLUDE TOPICS  HERE %%%%%%%%%%%%%%%%%%%%%%%%%%%%%%%%%%%%%%%%%%%%%%%%%%%%%%%

\begin{meta}
    \textbf{Recommended Timeline}
    \begin{itemize}
        %%% MODIFY %%%%
        \item Lists Minilecture: 10 minutes
        \item Lists WWPD: 5 minutes
        \item Lists Environment Diagram: 15 minutes
        \item Comprehensions: 12 minutes
        \item Duplicate List: 7 minutes
        \item Dictionaries Minilecture/example: 5 minutes/0 minutes
        \item Snapshot - 4 minutes
        \item Digraph - 10 minutes
    \end{itemize}
    This week, we're providing slides! Check them out in the content team folder. Feel free to use these while you mini-lecture, and make a copy and modify them if you'd like. 

    Teaching sequences can be tricky because there's a lot of material to cover, but it tends to be pretty boring ``this is how this works''-type content instead of deeper conceptual, information. So please, for the love of God, do not do a detailed mini-lecture on every aspect of sequences. This will take up far too much time and will probably not be a valuable experience for your students. Before mini-lecturing, it's valuable to ask your students what they would like you to go over specifically so that you're not repeating a bunch of information they already know. 
\end{meta}

\section{Sequences}
\subimport{../../topics/sequences/text/}{sequences_overview.tex}
\begin{questions}
    \subimport{../../topics/lists/immutable/easy/wwpd/}{simple-modified-2.tex}
    \begin{questionmeta}
        An alternative to doing a super detailed mini-lecture is going over these problems with your students and ``learning by doing.'' 
    \end{questionmeta}
    \subimport{../../topics/lists/immutable/easy/env-diagram/}{reverse.tex}
    \begin{questionmeta}
        This problem apparently is very tricky and takes a lot of time. So beware of this information while doing it with your students. 
    \end{questionmeta}
    \newpage % To give space for students to draw out the ED
    \subimport{../../topics/lists/immutable/medium/}{comprehensions.tex}
    \begin{questionmeta}
        I think it's probably not necessary to go over all of these with your students, just do enough to where they're comfortable with things. 
    \end{questionmeta}
    \subimport{../../topics/lists/immutable/medium/}{duplicate-list.tex}
\end{questions}

\section{Dictionaries}
\subimport{../../topics/dictionaries/text}{dictionary-overview.tex}
\begin{questions}
    \subimport{../../topics/dictionaries/}{snapshot.tex}
    \begin{questionmeta}
        This problem apparently is rather easy, so you can probably skip over it if your students are strapped for time or if you think they probably don't need it. 
    \end{questionmeta}
    \subimport{../../topics/dictionaries/}{digraph.tex}
\end{questions}
\end{document}
