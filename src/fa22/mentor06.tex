\documentclass{exam}
\usepackage{../../commonheader}
\usepackage{setspace}

%%% CHANGE THESE %%%%%%%%%%%%%%%%%%%%%%%%%%%%%%%%%%%%%%%%%%%%%%%%%%%%%%%%%%%%%%
\discnumber{6}
\title{\textsc{Containers, Data Abstraction, and Sequences}}
\date{October 3--October 007, 2022}
%%%%%%%%%%%%%%%%%%%%%%%%%%%%%%%%%%%%%%%%%%%%%%%%%%%%%%%%%%%%%%%%%%%%%%%%%%%%%%%

\begin{document}
\maketitle
\rule{\textwidth}{0.15em}
\fontsize{12}{15}\selectfont

%%% INCLUDE TOPICS  HERE %%%%%%%%%%%%%%%%%%%%%%%%%%%%%%%%%%%%%%%%%%%%%%%%%%%%%%%

\begin{guide}
    \textbf{Recommended Timeline}
    \begin{itemize}
        \item Lists Mini Lecture - 5 minutes
        \item Lists WWPD - 7 minutes
        \item Lists Environment Diagram - 8 minutes
        \item All Primes - 8 minutes
        \item Comprehensions - 7 minutes
        \item Abstraction Mini Lecture - 5 minutes/0 minutes
        \item Pokemon/Bookshelf - 10 minutes/15 minutes
    \end{itemize}
\end{guide}

\section{Lists}
\subimport{../../topics/lists/immutable/text/}{list_overview.tex}
\begin{questions}
    \subimport{../../topics/lists/immutable/easy/wwpd/}{simple-modified-2.tex}
    \subimport{../../topics/lists/immutable/easy/env-diagram/}{reverse.tex}
    \newpage % To give space for students to draw out the ED
    \subimport{../../topics/lists/immutable/easy/}{all-primes.tex}
    \subimport{../../topics/lists/immutable/medium/}{comprehensions.tex}
\end{questions}

\newpage
\section{Sequences}


\newpage
\section{Abstraction}
\subimport{../../topics/data-abstraction/text/}{abstraction-summary.tex}
\begin{questions}
    \subimport{../../topics/data-abstraction/easy/}{pokemon.tex}
    \begin{questionmeta}
        Students should only do the Pokemon problem if they are uncomfortable with understanding the structure of data abstraction.
    \end{questionmeta}
    \begin{questionmeta}
        When introducing the Pokemon problem, feel free to create your own analogy in addition to the one given in the summary.
    \end{questionmeta}
    \begin{questionmeta}
        A common analogy for data abstraction are: cars (the inside variables are engine, inner workings, outside are steering wheel and gas pedal).
    \end{questionmeta}
    \subimport{../../topics/data-abstraction/medium/}{bookshelf.tex}
    \begin{questionmeta}
        This is a hard question! Only do it if your students are absolutely assured in their definition of data abstraction and the abstraction barrier.
    \end{questionmeta}
    \begin{questionmeta}
        Feel free to spend even more time on this. Lists are more important for students generally, but understanding data abstraction deeply is a great setup for OOP!
    \end{questionmeta}
\end{questions}

\newpage
\section{Dictionaries}
\begin{questions}
    \subimport{../../topics/misc/dictionaries/}{count-t.tex}
\end{questions}


\end{document}
