\documentclass{exam}
\usepackage{../../commonheader}
\usepackage{outlines}

%%% CHANGE THESE %%%%%%%%%%%%%%%%%%%%%%%%%%%%%%%%%%%%%%%%%%%%%%%%%%%%%%%%%%%%%%
\discnumber{9}
\title{\textsc{Midterm 2 Review}}
\date{We'll find out the date and put it here }
%%%%%%%%%%%%%%%%%%%%%%%%%%%%%%%%%%%%%%%%%%%%%%%%%%%%%%%%%%%%%%%%%%%%%%%%%%%%%%%

\begin{document}
	\maketitle
	\rule{\textwidth}{0.15em}
	\fontsize{12}{15}\selectfont

\begin{questions}

\begin{blocksection}
\section{Recursion}
	\question Skipping Around
    \begin{blocksection}
        A skip list is defined as a sublist of a list such that each element in the sublist is non adjacent in the
        original list. For original list \lstinline{[5, 6, 8, 2]}, the lists \lstinline{[5, 8]}, \lstinline{[5, 2]}, \lstinline{[6, 2]}, \lstinline{[5]}, \lstinline{[6]}, \lstinline{[8]}, \lstinline{[2]}, \lstinline{[]} are
        all skip lists of the original list. The empty list is always a skip list of any list.

        Given a list \lstinline{int_lst} of unique integers, return a list of all unique skip lists of \lstinline{int_lst} where each skip
        list contains integers in strictly increasing order. The order in which the skip lists are returned does not matter.
        
        \begin{lstlisting}
        def list_skipper(int_list):
            """
            >>> list_skipper([5,6,8,2])
            [[5, 8], [5], [6], [8], [2], []]
            >>> list_skipper([1,2,3,4,5])
            [[1, 3, 5], [1, 3], [1, 4], [1, 5], [1], [2, 4], [2, 5], [2], [3, 5], [3], [4], [5], []]
            >>> list_skipper([])
            [[]]

            """
            if len(int_lst) == 0:
                return ________________
            with_first = ___________________
            without_first = ____________________
            with_first = [ _________ for x in with_first if x == [] or ____________]
            return with_first + without_first
        \end{lstlisting}
    \end{blocksection}
    \begin{solution}
        \begin{lstlisting}
        def list_skipper(int_list):
            if len(int_lst) == 0:
                return [[]]
            with_first = list_skipper(int_lst[2:])
            without_first = list_skipper(int_lst[1:])
            with_first = [ [int_lst[0]] + x for x in with_first if x == [] or x[0] > int_lst[0]]
            return with_first + without_first
        \end{lstlisting}
    \end{solution}
	\question maxkd
    \begin{blocksection}
        \begin{lstlisting}
        def maxkd(meteor, k):
            """
            Given a number `meteor`, finds the largest number of length `k` or fewer,
            composed of digits from `meteor`, in order.
        
            >>> maxkd(1234, 1)
            4
            >>> maxkd(32749, 2)
            79
            >>> maxkd(1917, 2)
            97
            >>> maxkd(32749, 18)
            32749
            """
            if ______:
                return ______
            a = ______
            b = ______
            return ______
        \end{lstlisting}
    \end{blocksection}
    \begin{solution}
        \begin{lstlisting}
        def maxkd(meteor, k):
            """
            Given a number `meteor`, finds the largest number of length `k` or fewer,
            composed of digits from `meteor`, in order.

            >>> maxkd(1234, 1)
            4
            >>> maxkd(32749, 2)
            79
            >>> maxkd(1917, 2)
            97
            >>> maxkd(32749, 18)
            32749
            """
            if meteor == 0 or k == 0:
                return 0
            a = maxkd(meteor // 10, k - 1) * 10 + meteor % 10
            b = maxkd(meteor // 10, k)
            return max(a, b)
        \end{lstlisting}
    \end{solution}
\section{Lists, Mutability and Dictionaries}
	\question Fall 2019 Final, Q2
    \begin{blocksection}
        Draw the environment diagram that results from running the following code.
        \begin{lstlisting}
        def cold(d):
            day = rain[:1]
            night = lambda: len(day)
            cold = rain.pop()
            day = d
            return night
        rain = [3, 4]
        d, day = [5], [lambda: d]
        cold(rain + [day[0](), rain])()
        \end{lstlisting}
    \end{blocksection}
    \begin{solution}
        \begin{lstlisting}
            https://pythontutor.com/cp/composingprograms.html#code=def%20cold%28d%29%3A%0A%20%20%20%20day%20%3D%20rain%5B%3A1%5D%0A%20%20%20%20night%20%3D%20lambda%3A%20len%28day%29%0A%20%20%20%20cold%20%3D%20rain.pop%28%29%0A%20%20%20%20day%20%3D%20d%0A%20%20%20%20return%20night%0Arain%20%3D%20%5B3,%204%5D%0Ad,%20day%20%3D%20%5B5%5D,%20%5Blambda%3A%20d%5D%0Acold%28rain%20%2B%20%5Bday%5B0%5D%28%29,%20rain%5D%29%28%29&cumulative=true&curInstr=17&mode=display&origin=composingprograms.js&py=3&rawInputLstJSON=%5B%5D
        \end{lstlisting}
    \end{solution}
	\question memory_store

    \begin{blocksection}
        Implement \lstinline{memory_store}, a function that will return two functions: \lstinline{add_val} and \lstinline{times_seen}. When called on a number \lstinline{var1}, \lstinline{times_seen} will return the number of times \lstinline{add_val} has been called on that particular value \lstinline{var1}.
        
        \begin{lstlisting}
        def memory_store():
            """
            >>> add_val, times_seen = memory_store()
            >>> add_val(4)
            >>> for _ in range(3):
            ... add_val(3)
            >>> times_seen(3)
            3
            >>> times_seen(4)
            1
            >>> times_seen(2)
            0
            >>> add_val(3)
            >>> times_seen(3)
            4
            """
            memory = {}
            def add_val(var1):
                if var1 in memory:
                    ________________
                else:
                    __________________
            def times_seen(var1):
                return _________________
            return add_val, times_seen
        \end{lstlisting}
    \end{blocksection}
    \begin{solution}
        \begin{lstlisting}
            def memory_store():
            """
            >>> add_val, times_seen = memory_store()
            >>> add_val(4)
            >>> for _ in range(3):
            ... add_val(3)
            >>> times_seen(3)
            3
            >>> times_seen(4)
            1
            >>> times_seen(2)
            0
            >>> add_val(3)
            >>> times_seen(3)
            4
            """
            memory = {}
            def add_val(var1):
                if var1 in memory:
                    memory[var1] = memory[var1] + 1
                else:
                    memory[var1] = 1
            def times_seen(var1):
                return memory.get(var1, 0)
            return add_val, times_seen
        \end{lstlisting}
    \end{solution}

\question You can do it like this. 

\begin{parts}
	\part asdf
	\part adgfadfdasf
\end{parts}
\end{questions}

\end{document}