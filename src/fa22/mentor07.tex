\documentclass{exam}
\usepackage{../../commonheader}
\usepackage{setspace}

%%% CHANGE THESE %%%%%%%%%%%%%%%%%%%%%%%%%%%%%%%%%%%%%%%%%%%%%%%%%%%%%%%%%%%%%%
\discnumber{7}
\title{\textsc{Abstract Data Types}}
\date{October 3--October 7, 2022}
%%%%%%%%%%%%%%%%%%%%%%%%%%%%%%%%%%%%%%%%%%%%%%%%%%%%%%%%%%%%%%%%%%%%%%%%%%%%%%%

\begin{document}
\maketitle
\rule{\textwidth}{0.15em}
\fontsize{12}{15}\selectfont

%%% INCLUDE TOPICS  HERE %%%%%%%%%%%%%%%%%%%%%%%%%%%%%%%%%%%%%%%%%%%%%%%%%%%%%%%


\section{Abstraction}
\subimport{../../topics/data-abstraction/text/}{abstraction-summary.tex}
\begin{questions}
    \subimport{../../topics/data-abstraction/easy/}{pokemon.tex}
    \begin{questionmeta}
        Students should only do the Pokemon problem if they are uncomfortable with understanding the structure of data abstraction.
    \end{questionmeta}
    \begin{questionmeta}
        When introducing the Pokemon problem, feel free to create your own analogy in addition to the one given in the summary.
    \end{questionmeta}
    \begin{questionmeta}
        A common analogy for data abstraction are: cars (the inside variables are engine, inner workings, outside are steering wheel and gas pedal).
    \end{questionmeta}
    \subimport{../../topics/data-abstraction/medium/}{bookshelf.tex}
    \begin{questionmeta}
        This is a hard question! Only do it if your students are absolutely assured in their definition of data abstraction and the abstraction barrier.
    \end{questionmeta}
    \begin{questionmeta}
        Feel free to spend even more time on this. Lists are more important for students generally, but understanding data abstraction deeply is a great setup for OOP!
    \end{questionmeta}
\end{questions}

\newpage
\section{ADT Trees}
\subimport{../../topics/trees/adt/text/}{tree_overview.tex}
\subimport{../../topics/trees/adt/text}{draw-tree-modified.tex}
\subimport{../../topics/trees/adt/text/}{implementation.tex}

\end{document}