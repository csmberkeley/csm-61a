\documentclass{exam}
\usepackage{../../commonheader}
\lstset{language=Scheme}

%%% CHANGE THESE %%%%%%%%%%%%%%%%%%%%%%%%%%%%%%%%%%%%%%%%%%%%%%%%%%%%%%%%%%%%%%
\discnumber{12}
\title{Tail Recursion \& Interpreters}
\date{November 7--November 11, 2022}
%%%%%%%%%%%%%%%%%%%%%%%%%%%%%%%%%%%%%%%%%%%%%%%%%%%%%%%%%%%%%%%%%%%%%%%%%%%%%%%

\begin{document}
\maketitle
\rule{\textwidth}{0.15em}

\begin{meta}
    Interpreters and tail recursion are the most brutal topics in CS 61A.
    Students are typically very confused by the content, and mentors find them difficult to
    teach as well. You probably will not be able to get through as many problems this
    week as you usually do, and that's OK. Go at the requisite pace so that students
    can properly follow what you're teaching. It's likely that they'll leave your 
    section still feeling uneasy, which is to be expected given the difficulty
    of these topics. Just know that they will have much more time and practice with
    interpreters and tail recursion, so you don't have to get them to the finish line.
    You will make a difference, and in my book, that's amazing. 

    As usual, there are many more problems on this worksheet than any reasonable mentor
    could cover in section. Please treat the worksheet as a problem bank around which
    you can structure your section to best meet the needs of you students. 

    \textbf{Recommended Timeline}
    \begin{itemize}
        \item Tail Recursion Mini Lecture: 10 minutes
        \item Q1 (Sum List): 14 minutes
        \item Q2 (Filter List): 10 minutes
        \item Q3 (Slice and Dice): 20 minutes
        \item Interpreters Mini Lecture: 12 minutes
        \item Q1 (Intro): 14 minutes
        \item Q2 (Eval Apply): 10 minutes
        \item Q3 (Hack Pi): 20 minutes
        \item Q4 (Scheme Challenge): 20 minutes
s    \end{itemize}
\newpage
\end{meta}

\section{Tail Recursion}
\subimport{../../topics/tail-recursion/text/}{intro.tex}
\vspace{1in}
\begin{questions}
    \subimport{../../topics/tail-recursion/medium/}{sum-list.tex}
    \subimport{../../topics/tail-recursion/medium/}{filter.tex}
    \subimport{../../topics/tail-recursion/medium/}{slice.tex}
\end{questions}

\section{Interpreters}
\subimport{../../topics/interpreters/text/}{intro-text.tex}
\begin{questions}
    \subimport{../../topics/interpreters/easy/}{intro.tex}

    \begin{meta}
        You will have likely gone through a lot of this stuff in your mini-lecture, so you
        can either gloss over/skip this, or incorporate it into your instruction (i.e. go
        over it together).
    \end{meta}

    \subimport{../../topics/interpreters/medium/}{eval-apply.tex}

    \begin{questionmeta}
        This problem and the one that follow are very similar, so you probably do not have to 
        do both of them if you are strapped for time. In fact, you should probably just do as many of these
        ``count the number of times \lstinline{scheme_eval} is called'' problems as your students need
        to understand what is going on. 
    \end{questionmeta}
    
    \subimport{../../topics/interpreters/medium/}{hack-pi.tex}
\end{questions}

\section{Scheme Challenge}
\begin{questions}
\subimport{../../topics/scheme/challenge/}{max-depth.tex}
\end{questions}
%%%%%%%%%%%%%%%%%%%%%%%%%%%%%%%%%%%%%%%%%%%%%%%%%%%%%%%%%%%%%%%%%%%%%%%%%%%%%%%
\end{document}
