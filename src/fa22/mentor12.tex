\documentclass{exam}
\usepackage{../../commonheader}
\lstset{language=Scheme}

%%% CHANGE THESE %%%%%%%%%%%%%%%%%%%%%%%%%%%%%%%%%%%%%%%%%%%%%%%%%%%%%%%%%%%%%%
\discnumber{12}
\title{Tail Recursion \& Interpreters}
\date{November 7--November 11, 2022}
%%%%%%%%%%%%%%%%%%%%%%%%%%%%%%%%%%%%%%%%%%%%%%%%%%%%%%%%%%%%%%%%%%%%%%%%%%%%%%%

\begin{document}
\maketitle
\rule{\textwidth}{0.15em}
\fontsize{12}{15}\selectfont

\begin{guide}
\begin{blocksection}
    \textbf{Recommended Timeline}
    \begin{itemize}
        \item Tail Recursion Intro - 8 minutes
        \begin{itemize}
            \item Feel free to work questions 1 and 2 into your mini-lecture/intro
        \end{itemize}
        \item Q3, Q4 \lstinline{count-instance} - 10 minutes
        \item Q5 \lstinline{filter} - 10 minutes
        \item Interpreters Intro (10-15 min): Consider walking through Q1 to Q4 as 
        you give a mini-lecture/review to leave more time for later problems. If you do this give students time to try 
        Q3 and Q4 on their own though
        \item Eval Apply (10 min): You may want to do one in its entirety before leaving students to try the rest
        (and explain the rules in a step-by-step manner), since students often find these types of problems unintuitive or challenging.
        \item There's a "Scheme Challenge" problem at the end if your students would prefer to get more practice with Scheme code writing
        \item There are also two past exam questions on tail calls at the end for students to reference later!
    \end{itemize}
\end{blocksection}
\newpage
\end{guide}

\section{Tail Recursion}
\subimport{../../topics/tail-recursion/text/}{intro.tex}
\newpage
\begin{questions}
    \subimport{../../topics/tail-recursion/medium/}{sum-list.tex}
    \subimport{../../topics/tail-recursion/medium/}{filter.tex}
    \newpage
    \subimport{../../topics/tail-recursion/medium/}{slice.tex}
\end{questions}

\section{Interpreters}
\subimport{../../topics/interpreters/text/}{intro-text.tex}
\newpage
\begin{questions}
    \subimport{../../topics/interpreters/easy/}{intro.tex}
    \subimport{../../topics/interpreters/medium/}{eval-apply.tex}
\newpage
    \subimport{../../topics/interpreters/medium/}{hack-pi.tex}
\end{questions}

\section{Scheme Challenge}
\begin{questions}
\subimport{../../topics/scheme/challenge/}{max-depth.tex}
\end{questions}

%%%%%%%%%%%%%%%%%%%%%%%%%%%%%%%%%%%%%%%%%%%%%%%%%%%%%%%%%%%%%%%%%%%%%%%%%%%%%%%

\end{document}
