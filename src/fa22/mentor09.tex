\documentclass{exam}
\usepackage{../../commonheader}
\usepackage{outlines}

%%% CHANGE THESE %%%%%%%%%%%%%%%%%%%%%%%%%%%%%%%%%%%%%%%%%%%%%%%%%%%%%%%%%%%%%%
\discnumber{9}
\title{\textsc{OOP}}
\date{October 17--21, 2022}
%%%%%%%%%%%%%%%%%%%%%%%%%%%%%%%%%%%%%%%%%%%%%%%%%%%%%%%%%%%%%%%%%%%%%%%%%%%%%%%

\begin{document}
	\maketitle
	\rule{\textwidth}{0.15em}
	\fontsize{12}{15}\selectfont

	%%% INCLUDE TOPICS  HERE %%%%%%%%%%%%%%%%%%%%%%%%%%%%%%%%%%%%%%%%%%%%%%%%%%%%%%%
\begin{guide}
	\textbf{Recommended Timeline}
	\begin{outline}[enumerate]
		\1 OOP Mini Lecture - classes, calling methods, class vs instance variables - 5 minutes
		\1 \lstinline{wwpd-starwars} - check your undersanding - 2-3 minutes
		\1 \lstinline{build-a-bear} - nice introduction it implementing a class - 5 minutes 
		\1 \lstinline{pingpong-tracker} - students will see ping pong on hw3, this question is the OOP version - 12 minutes
		\1 Optional Past Exam Questions
		\2 Mutability: Su19 MT Q3 and Sp19 Final Q2 - getting comfortable with environment diagrams, drawing box and pointers, visualizing list mutation operations
		\2 OOP: Fa19 Final Q5 - implementing classes, Su15 MT2 Q3 - implementing the init method, question works in dictionaries
		\1 Notes
		\2 this worksheet might be a little longer, so don't worry if you don't get to all the problems. The worksheets are a question bank!
		\2 students are most likely feeling very stressed about the midterm, give them some study and last min prep tips + comfort them
	\end{outline}
\end{guide}


\section{OOP}
\subimport{../../topics/oop/text/}{oop_overview.tex}
\newpage
\begin{questions}
\subimport{../../topics/oop/easy/}{star-wars.tex}
\newpage
\subimport{../../topics/oop/medium/}{build-a-bear-su21.tex}
\subimport{../../topics/oop/medium/}{pingpong-su21.tex}
\end{questions}

\end{document}