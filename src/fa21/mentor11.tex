\documentclass{exam}
\usepackage{../../commonheader}

%%% CHANGE THESE %%%%%%%%%%%%%%%%%%%%%%%%%%%%%%%%%%%%%%%%%%%%%%%%%%%%%%%%%%%%%%
\discnumber{8}
\title{\textsc{OOP Design, Review, Scheme}}
\date{November 1, 2021 to November 5, 2021}
%%%%%%%%%%%%%%%%%%%%%%%%%%%%%%%%%%%%%%%%%%%%%%%%%%%%%%%%%%%%%%%%%%%%%%%%%%%%%%%

\begin{document}
\maketitle
\rule{\textwidth}{0.15em}
\fontsize{12}{15}\selectfont

\begin{guide}
\begin{blocksection}
\textbf{Recommended Timeline}
Do one of the optional starred problems if you have time. OOP content may seem
out of place post-MT2.
\begin{itemize}
  \item OOP WWPD - 5 min*
  \item OOP Design - 10 min*
  \item Scheme Mini-Lecture - 10 min
  \item WWSD - 5 min
  \item Hailstone - 8 min
  \item Scheme Lists Mini-Lecture - 6 min
  \item Scheme List WWSD - 8 min
  \item Waldo - 8 min
  \item Challenge problem (if time) - 6 min*
\end{itemize}
\end{blocksection}
\end{guide}

\section{OOP Review}
\begin{questions}
\subimport{../../topics/oop/medium/}{flying-the-coop.tex}
\end{questions}


\section{OOP Design}
\begin{questions}
\subimport{../../topics/oop/medium/}{supermarket-design.tex}
\end{questions}
\newpage

\section{Scheme}
\begin{guide}
\begin{blocksection}
\textbf{Teaching Tips}
\begin{itemize}
  \item To ease in Scheme, it can help to start by comparing and contrasting with Python
  \begin{itemize}
    \item Have students write a basic function in Python (like an iterative countdown), then replicate it in Scheme
    \item Have students list language features of Python (variable assignments, conditional statements, logic operators, etc.), and explain how Scheme implements those features
    \item Make sure to give a disclaimer that while high level features may be analogous, the internals are different!
  \end{itemize}
  \item Scheme features break into three broad categories: Primitives, Call Expressions, and Special Forms (the latter two are called Compound Expressions)
  \begin{itemize}
    \item Primitives evaluate to themselves (4 evaluates to 4, \#t to \#t, etc.)
    \item Call Expressions begin with a function name and are followed by arguments- evaluate function name, evaluate arguments, and apply function to arguments
    \item Special Forms begin with a keyword and are followed by subexpressions, which are evaluated in a way based on the specific keyword
  \end{itemize}
  \item Useful Links
  \begin{itemize}
    \item \href{https://cs61a.org/articles/scheme-spec/}{Scheme Specification} (for overfiew, types, and special forms)
    \item \href{https://cs61a.org/articles/scheme-builtins/}{Scheme Built-In Procedure Reference} (for built-in procedures)
  \end{itemize} 
\end{itemize}
\end{blocksection}
\end{guide}

\subimport{../../topics/scheme/text/}{scheme-basics-overview.tex}
\begin{questions}
\newpage
\section{What Would Scheme Print?}
\subimport{../../topics/scheme/easy/wwsd/}{basics.tex}

\section{Code Writing in Scheme}
\subimport{../../topics/scheme/easy/}{hailstone.tex}

\section{Scheme Lists}
\subimport{../../topics/scheme/text/}{scheme-lists-overview.tex}
\newpage
\subimport{../../topics/scheme/easy/wwsd/}{lists_nodots.tex}

\section{More Code Writing in Scheme}
\subimport{../../topics/scheme/medium/}{waldo.tex}
\end{questions}

\end{document}
