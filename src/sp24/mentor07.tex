\documentclass{exam}
\usepackage{../../commonheader}
\usepackage{setspace}

%%% CHANGE THESE %%%%%%%%%%%%%%%%%%%%%%%%%%%%%%%%%%%%%%%%%%%%%%%%%%%%%%%%%%%%%%
\discnumber{7}
\title{\textsc{Function-based Trees, Mutability, Iterators + Generators}}
\date{February 19--February 23, 2023}

%%%%%%%%%%%%%%%%%%%%%%%%%%%%%%%%%%%%%%%%%%%%%%%%%%%%%%%%%%%%%%%%%%%%%%%%%%%%%%%

\begin{document}
\maketitle
\rule{\textwidth}{0.15em}

%%% INCLUDE TOPICS  HERE %%%%%%%%%%%%%%%%%%%%%%%%%%%%%%%%%%%%%%%%%%%%%%%%%%%%%%%

\begin{meta}
\textbf{Recommended Timeline}
\begin{itemize}
    \item ADT Tree mini-lecture (8 min)
    \item Q1: Draw Tree ADT (4 min)
    \item Q2: Prune (7 min)
    \item Q3: Replace x (10 min)
    \item Mutation mini-lecture (6 min)
    \item Q4: Mutative WWPD (15 min)
    \item Q5: Accumulate (15 min)
    \item Iterators + Generators Mini-lecture (8 min)
    \item Q6: Foo (5 min)
    \item Q7: All Sums (10 min)
\end{itemize}
The times in the recommended timeline do not add up to 50 minutes because no mentor
is expected to get through all the problems during section. The worksheet is a skeleton
around which you should structure your section to best meet the needs of your students.
If you get stressed out about covering a lot of content, I encourage you to be open with
your students about the way these sessions are structured. This is an extremeley packed worksheet, with three extremely
important topics for the midterm being covered in a surface-level way. The problems in this worksheet, with the exception of Accumulate 
and All Sums, are meant to be a gentle introduction to all of the necessary topics. We will be more in-depth in the midterm review worksheet.

You should probably ask your students at the beginning of section what they would rather go
over---Trees, mutability, or iterators + generators---and then allocate time appropriately.
Feel free to go slow and go over your methods of solving tree questions with students, emphasizing how the abstraction barrier can give clues 
as to how to solve problems. For mutability, don't spend too long on the mini-lecture, as it's moreso realizing what methods are mutative and what aren't.
Iterators + generators will be gone over in greater difficulty + depth in the next worksheet so feel free to spend more time on trees and mutability.
\end{meta}

\section{Tree ADT}
\subimport{../../topics/trees/adt/text/}{tree_overview.tex}
\newpage
\begin{questions}
    \subimport{../../topics/trees/adt/draw-tree}{draw-tree.tex}
    \newpage

    % BEGIN PRUNE %
    \begin{blocksection}
    \textbf{BEFORE \lstinline{prune(1)}:}
    \subimport{../../topics/trees/adt/prune}{before.tex}
    \hspace{0.5in}
    \textbf{AFTER \lstinline{prune(1)}:}
    \subimport{../../topics/trees/adt/prune}{after.tex}
    \end{blocksection}
    

    \subimport{../../topics/trees/adt/prune}{problem.tex}
    % END PRUNE %

    \subimport{../../topics/trees/adt/medium}{replace-x.tex}
\end{questions}

\newpage
\section{Mutability}
\subimport{../../topics/lists/mutable/text/}{mutation_overview.tex}
	\begin{questions}
	    \subimport{../../topics/lists/mutable/medium/}{wwpd-mutation.tex}
        \subimport{../../topics/lists/mutable/medium/}{accumulate.tex}
	\end{questions}


\newpage
\section{Iterators \& Generators}
\subimport{../../topics/generators/text/}{generators_overview.tex}
\begin{questions}
    \subimport{../../topics/generators/easy/}{foo.tex}
    \subimport{../../topics/generators/medium/}{all-sums.tex}
\end{questions}

\end{document}