\documentclass{exam}
\usepackage{../commonheader}

\discnumber{3}
\title{Environment Diagrams \& Higher-Order Functions}
\date{September 9 -- September 13, 2024}

\begin{document}
\maketitle

\begin{meta}
    \textbf{Example Timeline}
    \begin{itemize}
        \item Environment Diagrams Mini Lecture + Q1 [15 Mins]
            \subitem 1. You can use Q1 as an example for the mini-lecture
            \subitem 2. A lot of students (even those with significant programming background) get confused by env. diagrams. It's probably worth doing the minilecture even if you have advanced students.
            \subitem 3. You should address when we need to make a new frame in an environment diagram [This was the old Q1].
        \item Problems: Environment Diagrams [7 Mins]
            \subitem Q2 - Joke
        \item Higher Order Functions Overview + Reasoning [3 Mins]
            \subitem Make sure you explain why \& where to use lambda functions \& HOF functions. Give students a practical example if needed.
        \item Problems (You pick which ones): HOF [25 Mins]
        \begin{enumerate}
            \item Q3 - Foobar
            \item Q4 - xyz
            \item Q5 - whole\_sum
            \item Q6 - mystery
            \item Q7 - lambda-wwpd
        \end{enumerate}
        
    \end{itemize}
\end{meta}

\section{Environment Diagrams}
\begin{questions}
    \subimport{../../topics/functions-and-expressions/medium/env-diagram/}{swap.tex}
    \begin{questionmeta}
        Suggested Time: 5 min; Difficulty: Medium
        \begin{itemize}
            \item This question stresses variables in different scopes. 
            \subitem Show difference between \lstinline{x} and \lstinline{y} in both global and local frames.
            \subitem Also note to students that a call to \lstinline{swap(x, y)} will not actually swap the values of \lstinline{x} and \lstinline{y} in the frame where it is called.
            \item It might also be good to recap what \lstinline{x, y = y, x} does in python -- ensure students know that this is a special feature of python and that switching happens in 1 line, by order of how the values are listed. 
        \end{itemize}
      \end{questionmeta}
    \subimport{../../topics/functions-and-expressions/medium/env-diagram/}{joke.tex}
    \begin{questionmeta}
        Suggested Time: 7 min; Difficulty: Medium
        \begin{itemize}
            \item Make sure that the students understand how Python looks for a value of a variable, from local (to parent(s)) to global.
            \item Make sure your students understand the difference between an intrinsic name and a bound name
            \subitem Intrinsic: For user defined functions, this intrinsic name is the name used in the \lstinline{def} statement 
            \subitem Bound: Names of variables that point to the function object. A function can have many bound names, and the bound names of a function can often change. 
            \item It may be good to remind your students to evaluate the functions on the right hand side first, then assign to variables on the left hand side.
        \end{itemize}    
      \end{questionmeta}
          
\end{questions}


\section{Higher-Order functions}
\begin{questions}
    \subimport{../../topics/hof/easy/env-diagram/}{foobar.tex}
    \begin{questionmeta}
        Suggested Time: 10 min; Difficulty: Easy
    \end{questionmeta}

    \subimport{../../topics/hof/medium/}{xyz.tex}
    \begin{questionmeta}
        Suggested Time: 6 min; Difficulty: Medium
        \begin{itemize}
            \item Give your students advice on how to break down these nested fill in the blank/skeleton questions.
            \subitem You can tell them about the typical make an educated guess based on intuition, then testing it by plugging it in and running the function manually.
        \end{itemize}
    \end{questionmeta}

    \subimport{../../topics/hof/medium/}{check_sum.tex}
    \begin{questionmeta}
        Suggested Time: 8 Mins; Difficulty: Medium
        \begin{itemize}
            \item Remind your students that for HOFs, you must \lstinline{return} the inner function (ie we must \lstinline{return check} to use it).
            \item Also depending on the skill level of students in your section, a recap of digit manipulation may be needed (ie x // 10, x \% 10, etc.)
        \end{itemize}
    \end{questionmeta}

    \subimport{../../topics/hof/medium/}{mystery.tex}
    \begin{questionmeta}
        Suggested Time: 5 Min - This problem is probably optional; Difficulty: Medium;

        Using doctests to understand how a function should work is a fundamental part of CS 61A. The goal of this question is to force students to exercise that muscle by removing any other description of \lstinline{mystery}.
    \end{questionmeta}

    \subimport{../../topics/hof/medium/wwpd/}{lambda-wwpd.tex}
    \begin{questionmeta}
        Suggested Time: 5 - 7 Min; Difficulty: Medium - This problem is probably also optional unless your students want extra lambda or HOF practice.
    \end{questionmeta}

\end{questions}

\end{document}