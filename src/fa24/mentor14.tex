\documentclass{exam}
\usepackage{../../commonheader}
\lstset{language=Scheme}

\discnumber{14}
\title{\textsc{Macros Review \& Intro to SQL}}
\date{November 25--November 29, 2024}

\begin{document}
\maketitle
\rule{\textwidth}{0.15em}

\begin{meta}
\begin{blocksection}
    \textbf{Recommended Timeline}
    \begin{itemize}
        \item Macros Intro: 10 minutes
        \item Q1. \lstinline{and-odds-thanksgiving}: 8 minutes
        \item SQL mini-lecture: 10 minutes
        \item Q1 (Mentors): 10 minutes
        \item Q2 (Circus): 20 minutes
        \item Q3 (Fish): 20 minutes
    \end{itemize}
\end{blocksection}
\vspace{5mm}
This is a short worksheet given that this will be a short week, so take advantage of this time to get to know your mentees a bit more and clarify any questions they might have regarding the topics covered this week :)

As a reminder, no mentor is expected to get through every problem on this worksheet.
\end{meta}

\vspace{3mm}

\begin{meta}
\textbf{Teaching Tips}
\begin{itemize}
    \item Before we jump into macros, it is very important to ensure that your students understand quasiquotation.
    \item Draw out box-and-pointer diagrams to show how the expressions in macros are being stored when the operands are unevaluated, if needed.
    \item If you would like a quick refresher on how to think about macros, please refer to this \href{https://docs.google.com/document/d/1JSbvtJ5bYUEhovDZd_gQnBvkG_WDcafmX-4B3QeIXZU/edit}{guide}.
\end{itemize}
\end{meta}

\vspace{2mm}

\section{Macros Review}
\begin{questions}
\small
\subimport{../../topics/macros/medium/}{and-odds-thanksgiving.tex}
\normalsize
\newpage
\end{questions}
\section{SQL}
\subimport{../../topics/sql/text/}{intro.tex}
\newpage

\begin{questions}
  \question
  \subimport{../../topics/sql/mentors/}{table.tex}
  \begin{parts}
    \subimport{../../topics/sql/mentors/easy/}{create-table.tex}
    \subimport{../../topics/sql/mentors/easy/}{alphabetical.tex}
    \subimport{../../topics/sql/mentors/easy/}{food-and-color-if-language-not-python.tex}
  \end{parts}
\end{questions}

\end{document}
