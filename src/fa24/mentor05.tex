\documentclass{exam}
\usepackage{../commonheader}

\discnumber{5}
\title{Recursion, Tree Recursion}
\date{September 23 -- September 27, 2024}

\begin{document}
\maketitle

\begin{meta}
    \textbf{Example Timeline}
    \begin{itemize}
        \item Tree Recursion Mini Lecture [8 Mins]
            \subitem 1. Consider creating a simple example to demonstrate recursion. Example: A recursive accumulation function (ie: next \# = $\sum$ previous \#s)
        \item Problems: Recursion [20 Mins]
        \begin{itemize}
            \item Q1 - FizzBuzz
            \item Q2 - Sum Prime digits
            % \item Q3 - Near
        \end{itemize}
        \item Problems: Tree Recursion [20 Mins]
        \begin{itemize}
            \item Q4 - Copy Machine
            \item Q5 - Mario Number
            \item Q6 - [Conceptual \& Optional] Fast Modular Exponentiation
        \end{itemize}
        
    \end{itemize}
\end{meta}

\section{Recursion}
\begin{questions}
    \subimport{../../topics/recursion/medium/}{fizzbuzz.tex}
    \begin{questionmeta}
        This is an intro example of recursion - Feel free to skip if your students are clear on the concept of recursion.\\
        $\Rightarrow$ Suggested Time: 3 min; Difficulty (Official): Medium; Difficulty (Adjusted): Easy/Intro
      \end{questionmeta}

    \subimport{../../topics/recursion/medium/}{sum-prime-digits.tex}
    \begin{questionmeta}
        \begin{itemize}
            \item Again, this problem is likely skippable if you have advanced students.
            \item Make sure your students understand why you are calling \lstinline{sum_prime_digits} twice.
        \end{itemize}    
        Suggested Time: 5 min; Difficulty (Official): Medium; Difficulty (Adjusted): Easy $\rightarrow$ Medium Mezzanine
      \end{questionmeta}
    
    % \subimport{../../topics/recursion/hard/}{near.tex}
    % \begin{questionmeta}
        % $\Rightarrow$ Suggested Time: 20 min; Difficulty (Official): Hard -- Exam level; Difficulty (Adjusted): Hard -- Exam level
    % \end{questionmeta}
\end{questions}


\section{Tree Recursion}
\subimport{../../topics/recursion/text/}{tree_recursion_overview.tex}
\begin{questions}
    \subimport{../../topics/recursion/medium/}{copy-machine1.tex}
    \begin{questionmeta}
        \begin{itemize}
            \item This is a good mini-lecture problem to use as a demo since this is probably the closest thing the students have seen to what they demo-ed in lecture.
        \end{itemize}
        $\Rightarrow$ Suggested Time: 6 min; Difficulty (Official): Medium; Difficulty (Adjusted): Easy
    \end{questionmeta}
    \newpage
    \subimport{../../topics/recursion/medium/}{mario-number.tex}
    \begin{questionmeta}
        $\Rightarrow$ Suggested Time: 10 min; Difficulty (Official): Medium; Difficulty (Adjusted): Easy $\rightarrow$ Medium (Mezzanine) \\ \\
        The classic mario-number problem from CSM. A worksheet wouldn't be complete without it!
    \end{questionmeta}
    \newpage
    \subimport{../../topics/recursion/medium/}{fast_modular_exponentiation.tex}
    \begin{questionmeta}
        Suggested Time: 10 min; Difficulty (Official): Medium; Difficulty (Adjusted): Medium
    \end{questionmeta}

\end{questions}

\end{document}