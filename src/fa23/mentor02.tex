\documentclass{exam}
\usepackage{../commonheader}

\discnumber{2}
\title{Python, Functions, Expressions, and Control}
\date{August 28--September 1, 2023}

\begin{document}
\maketitle
\begin{meta}
\textbf{Recommended Timeline}
\begin{itemize}
  \item Introductions/Expectations/Icebreaker [10 minutes]
  \item Q1: What Would Python Display + Minilecture/explanations [15 minutes]
  \item Q2: Order of evaluation [5 minutes]
    \subitem{Note: you don't need to do all parts if students get the idea}
  \item Code Writing [20 minutes]
    \subitem{- Again, no need to do all questions}
    \subitem{- The questions ramp up in difficulty from easy -- medium -- medium/hard -- hard. The last one is a bit more challenging; pick based on how your students are feeling}
  \item Tips and Tricks for succeeding in CS61A [leftover time]
\end{itemize}
\end{meta}


\section{Intro to Python}
\begin{questions}
\subimport{../../topics/basics/}{potpurri.tex}
\subimport{../../topics/basics/}{funky-funcs.tex}
\begin{questionmeta}
  It's probably not necessary to get through all the parts of this problem if your students get the idea. The last subpart is probably the most instructive. 
  For question 2, feel free to remind students of the general ``order of operations'' of functions, in that they start inward and expand outward. Feel free to use any analogies that help students understand.
  If your students are still confused, it's advisable to carefully walk through the first few problems with them step by step. Encourage them to underline/annotate portions of the line to keep track of what parentheses go where!
\end{questionmeta}
\end{questions}

\section{Control}
\begin{questions}
\subimport{../../topics/control/easy/}{divisibility-check.tex}
\begin{questionmeta}
  This question has been somewhat altered since last semester, now including an elif statement since we got rid of fizzbuzz. This problem is primarily an introduction to Python syntax.
  It's recommended to go over the alternate solution to this question as it touches upon how to use if statements in the ever-dreaded one-liners 61A loves to put on exams. 
\end{questionmeta}
\subimport{../../topics/control/medium/}{pow-of-2.tex}
\begin{questionmeta}
  With this question, one talking point can be when to use while vs. for loops in Python. It is possible to have a for loop implementation, but it involves continue statements and realistically is not optimal run-time wise.
  Make sure your students understand that print returns None and that it is valid for a Python function to not have a return statement.
\end{questionmeta}
\subimport{../../topics/control/medium/}{leap-year.tex}
\begin{questionmeta}
  This question is similar to Divisibility Check (Q1), except it tests for students' knowledge of how ``and'' and ``or'' work.
  Only do this question if you decide to skip Divisibility Check, as the concepts it exercises are very similar.
\end{questionmeta}
\subimport{../../topics/control/challenge/}{min_fact.tex}
\begin{questionmeta}
  If your students are not familiar with factorials, you may want to give them a brief overview before going over this problem.
  One area in which students may need help is addressing the edge case in which n is greater than the limit. Feel free to go over it as a hint while they solve, or emphasize it when going over the solution.
\end{questionmeta}
\end{questions}

\end{document}
