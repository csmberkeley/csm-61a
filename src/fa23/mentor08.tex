\documentclass{exam}
\usepackage{../../commonheader}

%%% CHANGE THESE %%%%%%%%%%%%%%%%%%%%%%%%%%%%%%%%%%%%%%%%%%%%%%%%%%%%%%%%%%%%%%
\discnumber{8}
\title{\textsc{Mutability, Iterators, and Generators}}
\date{October 9--October 13, 2023}
%%%%%%%%%%%%%%%%%%%%%%%%%%%%%%%%%%%%%%%%%%%%%%%%%%%%%%%%%%%%%%%%%%%%%%%%%%%%%%%

\begin{document}
\maketitle
\rule{\textwidth}{0.15em}

%%% INCLUDE TOPICS  HERE %%%%%%%%%%%%%%%%%%%%%%%%%%%%%%%%%%%%%%%%%%%%%%%%%%%%%%%

\begin{guide}
    \textbf{Recommended Timeline}
    \begin{itemize}
        \item Mutability
        \begin{itemize}
            \item Mutability Mini Lecture: 6 minutes
            \item What Would Python Do: 5 minutes
            \item Accumulate: 10 minutes
            \item Nice Ice Cream: 15 minutes (beware that this is a relatively longer environment diagram question!)
            \item There are a lot of questions and topics, so pick and choose between these problems based on your students comfortability.
        \end{itemize}
        \item Iterators \& Generators:
        \begin{itemize}
            \item Iterators \& Generators Mini Lecture: 7 minutes
            \item In Order: 5 minutes
            \item Foo: 8 minutes
            \item All Sums: 10 minutes (a question that combines tree recursion and generators)
            \item Skip Machine: 16 minutes (worth to go though, in content's humble opinion, but will take some time)
        \end{itemize}
    \end{itemize}

    As a reminder, these times do not add up to 50 minutes because no one is expected 
    to get through all questions in a section. This is especially true this week, 
    because this worksheet is rather long. You should use the worksheet as a problem bank
     around which you can structure your section to best accommodate the needs of your 
     students. Both before and during section, consider which questions would be most 
     instructive and how you should budget your time. As always, we highly recommend directly asking your students what they would like to focus on.
     
    In general, we recommend planning how long you're going to spend on each section (say, 25 minutes on mutability and 25 minutes on iterators and generators)
    and then restrict yourself to that time budget so that you can get a good coverage of different topics. You can budget the time according to how your students feel on the different topics.
\end{guide}

\section{Mutability}
	\subimport{../../topics/lists/mutable/text/}{mutation_overview.tex}
	\begin{questions}
	    \subimport{../../topics/lists/mutable/medium/}{wwpd-mutation.tex}
        \subimport{../../topics/lists/mutable/medium/}{accumulate.tex}
        \subimport{../../topics/lists/mutable/medium/env-diagram/}{nice-ice-cream.tex}
	\end{questions}

\newpage
\section{Iterators \& Generators}
\subimport{../../topics/generators/text/}{generators_overview.tex}
\begin{questions}
    \subimport{../../topics/generators/medium/}{in-order-adt.tex}
    \subimport{../../topics/generators/easy/}{foo.tex}
    \subimport{../../topics/generators/medium/}{all-sums.tex}
    \subimport{../../topics/iterators/easy/}{skipmachine.tex}
\end{questions}

\end{document}
