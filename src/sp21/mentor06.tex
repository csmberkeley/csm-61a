\documentclass{exam}
\usepackage{../../commonheader}

%%% CHANGE THESE %%%%%%%%%%%%%%%%%%%%%%%%%%%%%%%%%%%%%%%%%%%%%%%%%%%%%%%%%%%%%%
\discnumber{4}
\title{\textsc{Trees and Mutation}}
\date{February 22 - February 14, 2021}
%%%%%%%%%%%%%%%%%%%%%%%%%%%%%%%%%%%%%%%%%%%%%%%%%%%%%%%%%%%%%%%%%%%%%%%%%%%%%%%

\begin{document}
\maketitle
\rule{\textwidth}{0.15em}
\fontsize{12}{15}\selectfont

%%% INCLUDE TOPICS  HERE %%%%%%%%%%%%%%%%%%%%%%%%%%%%%%%%%%%%%%%%%%%%%%%%%%%%%%%

\begin{guide}
    \textbf{Recommended Timeline}
    \begin{itemize}
        \item Midterm Debrief - 5 minutes
        \item Trees Mini Lecture and WWPD - 7 minutes
        \item \lstinline{sum_of_nodes} - 7 minutes
        \item Choose one of the following (note \lstinline{all_paths} is harder)
            \begin{itemize}
            \item \lstinline{replace_x} - 8 minutes
            \item \lstinline{all_paths} - 10 minutes
            \end{itemize}
        \item Mutation Mini Lecture - 6 minutes
        \item Mutable List WWPD - 7 minutes
        \item \lstinline{accumulate} - 8 minutes
    \end{itemize}
    \vspace{.5cm}
    \textbf{Recommended Timeline for NPE Sections}
    \begin{itemize}
        \item Section 1:
            \begin{itemize}
            \item Midterm Debrief - 5 minutes
            \item Trees Mini Lecture and WWPD - 10 minutes
            \item \lstinline{sum_of_nodes} - 10 minutes
            \item \lstinline{replace_x} - 10 minutes
            \item \lstinline{all_paths} - 15 minutes
            \end{itemize} 
        \item Section 2:
            \begin{itemize}
            \item Mutation Mini Lecture - 7 minutes
            \item Mutable List WWPD - 10 minutes
            \item \lstinline{accumulate} - 10 minutes
            \item \lstinline{contains-n} - 20 minutes
            \end{itemize}
    \end{itemize}
\end{guide}

\section{Trees}
\subimport{../../topics/trees/adt/text/}{tree_overview.tex}
\subimport{../../topics/trees/adt/text/}{implementation.tex}
\subimport{../../topics/trees/adt/text/}{draw-tree.tex}
\vspace{.5cm}
\begin{questions}
\subimport{../../topics/trees/adt/construct-tree/}{wwpd.tex}
\newpage
\subimport{../../topics/trees/adt/medium/}{tree-sum.tex}
\newpage
\subimport{../../topics/trees/adt/medium/}{replace-x.tex}
\subimport{../../topics/trees/adt/medium/}{all-paths.tex}
\end{questions}

\newpage
\section{Mutation}
\subimport{../../topics/lists/mutable/text/}{mutation_overview.tex}
\begin{questions}
\subimport{../../topics/lists/mutable/medium/}{wwpd-mutation-expanded.tex}
\subimport{../../topics/lists/mutable/medium/}{accumulate.tex}
\end{questions}


\newpage
\section{Challenge Problems}
\textbf{Note:} These problems are meant to be challenging and may take a long time. Please attempt the previous questions on the worksheet first.
\begin{questions}
\subimport{../../topics/trees/adt/hard/}{contains-n.tex}
\end{questions}



\begin{guide}
	\begin{blocksection}
	\textbf{Teaching Tips}
	\begin{itemize}
			\item Draw out a tree and ask them to play out the algorithm
			\begin{itemize}
	                \item If you were a computer, how would you replace all the \lstinline{x}'s?
	                \item You would check the value of your current tree, and then move on to each of the branches
	                \item Can we somehow “simplify” all of this repeated work?
            \end{itemize}
		\item Make sure they respect abstraction barriers 
            \begin{itemize}
                \item If there isn't a \lstinline{set_value} function, how can we return a tree with an updated value?
                \item Answer: create a new tree with \lstinline{0} and the new branches
            \end{itemize}
		\item Warn them against trying to evaluate branches
            \begin{itemize}
                \item What is the simplest replacement we can do?
                \item How can we delegate branch replacements to recursive calls?
            \end{itemize}
            \item If we have multiple branches, how do we make the recursive call on each branch? (Answer: a for loop)
            \begin{itemize}
                \item What happens in the for loop if there aren’t any branches? Nothing.
                \begin{itemize}
                    \item This is why we don’t need an explicit base case (ex. if len(branches) == 0)
                \end{itemize}
            \end{itemize}
	\end{itemize}
	\end{blocksection}
\end{guide}


\end{document}
