\documentclass{exam}
\usepackage{../../commonheader}
\usepackage{setspace}

%%% CHANGE THESE %%%%%%%%%%%%%%%%%%%%%%%%%%%%%%%%%%%%%%%%%%%%%%%%%%%%%%%%%%%%%%
\discnumber{4}
\title{\textsc{Python Lists, Dictionaries, and Data Abstraction}}
\date{July 7, 2021}
%%%%%%%%%%%%%%%%%%%%%%%%%%%%%%%%%%%%%%%%%%%%%%%%%%%%%%%%%%%%%%%%%%%%%%%%%%%%%%%

\begin{document}
\maketitle
\rule{\textwidth}{0.15em}
\fontsize{12}{15}\selectfont

%%% INCLUDE TOPICS  HERE %%%%%%%%%%%%%%%%%%%%%%%%%%%%%%%%%%%%%%%%%%%%%%%%%%%%%%%

\begin{guide}
You may not get to all the problems on the worksheet, so make sure you know which topics your students would like to prioritize.
\newline
    \textbf{Recommended Timeline}
    \begin{itemize}
        \item Lists Mini Lecture - 5 minutes
        \item Lists WWPD - 8 minutes
        \item Lists ED - 7 minutes
        \item is\_prime - 5 minutes
        \item List Comprehension - 10 minutes
        \item Abstraction Mini Lecture - 5 minutes
        \item Elephants - 10 minutes
        \item Dictionaries Mini Lecture - 2 minutes
        \item replace\_all - 5 minutes
        \item counter - 6 minutes
    \end{itemize}
\end{guide}

\section{Lists}
\subimport{../../topics/lists/immutable/text/}{list_overview.tex}
\begin{questions}
\subimport{../../topics/lists/immutable/easy/wwpd/}{simple-modified.tex}
\subimport{../../topics/lists/immutable/easy/env-diagram/}{reverse.tex}
% \subimport{../../topics/lists/immutable/medium/wwpd/}{assignment-and-slicing.tex}
\subimport{../../topics/lists/immutable/easy/}{all-primes.tex}
\subimport{../../topics/lists/immutable/medium/}{comprehensions.tex}
\end{questions}

\newpage
\section{Abstraction}
\subimport{../../topics/data-abstraction/text/}{abstraction-summary.tex}
\begin{questions}
\subimport{../../topics/data-abstraction/easy/}{elephants.tex}
\end{questions}

\newpage
\section{Dictionaries}
Dictionaries are data structures that map keys to values. In Python, the key-value pairs in a dictionary are unordered.
\begin{guide}
\newline
Make sure to emphasize the similarities and differences between dictionaries and lists. For example, like lists, you can 
add and retrieve items in dictionaries using indexing syntax. However, unlike lists, dictionaries do not have 
indices, so you must add and retrieve things using keys, like this: \texttt{dict[key] = value}.
\end{guide}
\begin{questions}
\subimport{../../src/su21/}{replace_all.tex}
\newpage
\subimport{../../src/su21/}{counter.tex}
\end{questions}

\end{document}