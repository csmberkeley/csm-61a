\documentclass{exam}
\usepackage{../commonheader}

\discnumber{1}
\title{Control, Environment Diagrams}
\date{June 25, 2021}

\begin{document}
\maketitle


\begin{blocksection}
\begin{guide}
  This worksheet primarily covers functions, control, and environment diagrams, which have been covered and introduced in Tuesday and Wednesday's lecture, as well as Disc01 and Lab01.
  
  There is also a challenge problem (i.e. exam prep) at the bottom for if you finish early or your students are looking for something more difficult to try.
  
This is a fairly long worksheet so don't feel pressured to do all of the problems! Make sure to poll your students to see which problems they are more interested in covering.

\textbf{Estimated Problem Times}
\newline
Again, you probably won't be able to do all of them- but you can choose a few based on how long they are.
\begin{itemize}
    \item Handle Overflow - 5 min
    \item Power of Two - 8 min
    \item Min Factorial - 5 min
    \item New Frame Written Question - 2 mins
    \item Swap - 8 mins
    \item Joke - 7 mins
    \item digit\_div (challenge problem) - 15 mins
\end{itemize}
\end{guide}
\end{blocksection}

\section{Control}
\begin{blocksection}
\define{Control} is the use of boolean expressions to prevent or allow a block of code to run based on a condition.
We use \texttt{if} statements and \texttt{while} loops in conjunction with these boolean expressions depending on how we want the code to behave.
\texttt{if/elif/else} blocks will execute the first block of code for which the condition is \texttt{True}, whereas \texttt{while} loops will repeatedly
execute a block of code while the condition is \texttt{True}.
\end{blocksection}
\begin{questions}
\subimport{../../topics/control/easy/}{handle-overflow.tex}
\subimport{../../topics/control/medium/}{pow-of-2.tex}
\newpage
\subimport{../../topics/control/challenge/}{min_fact.tex}
\end{questions}

\newpage

\section{Environment Diagrams}
\subimport{../../src/su21/}{env-diagram-paragraph.tex}
\begin{questions}
\subimport{../../topics/environments/easy/}{new-frame.tex}
\subimport{../../topics/functions-and-expressions/medium/env-diagram/}{swap.tex}
\newpage
\subimport{../../topics/functions-and-expressions/medium/env-diagram/}{joke.tex}
\end{questions}

\newpage
\section{Challenge Control Problem}
\begin{questions}
\subimport{../../topics/control/challenge/}{digit_div.tex}

\clearpage

\end{questions}

\end{document}
