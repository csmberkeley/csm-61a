\documentclass{exam}
\usepackage{../../commonheader}

%%% CHANGE THESE %%%%%%%%%%%%%%%%%%%%%%%%%%%%%%%%%%%%%%%%%%%%%%%%%%%%%%%%%%%%%%
\discnumber{10}
\title{\textsc{Interpreters and Macros}}
\date{July 30, 2021}
%%%%%%%%%%%%%%%%%%%%%%%%%%%%%%%%%%%%%%%%%%%%%%%%%%%%%%%%%%%%%%%%%%%%%%%%%%%%%%%

\begin{document}
\maketitle
\rule{\textwidth}{0.15em}
\fontsize{12}{15}\selectfont


\begin{guide}
\textbf{Recommended Timeline}
\begin{itemize}
  \item Interpreters Intro (10-15 min): Consider walking through these problems as 
  you give a mini-lecture/review to leave more time for later problems.
  \item Eval Apply (10 min): You may want to do one in its entirety before leaving students to try the rest
  (and explain the rules in a step-by-step manner), since students often find these types of problems unintuitive or challenging.
  \item Macros Intro / mini-lecture (5 min)
  \item \lstinline{macros-quasi} (5 min): A WWSD-type problem.
  \item \lstinline{if-macro} (10 min): If running short on time, prioritize explaining the second part about differences between macros and functions instead of moving on to \lstinline{apply-twice}.
  \item \lstinline{apply-twice} (10 min): Combines macros with list operations.
\end{itemize}
\newpage
\end{guide}

%%% INCLUDE TOPICS HERE %%%%%%%%%%%%%%%%%%%%%%%%%%%%%%%%%%%%%%%%%%%%%%%%%%%%%%%
\section{Interpreters}
\subimport{../../topics/interpreters/text/}{intro-text.tex}
\begin{questions}
\subimport{../../topics/interpreters/easy/}{intro.tex}
\subimport{../../topics/interpreters/medium/}{eval-apply.tex}
\end{questions}

\newpage
\section{Macros}
\subimport {../../topics/macros/text/}{intro.tex}
\begin{questions}
\subimport{../../topics/macros/easy/}{macros-quasi.tex}
\subimport{../../topics/macros/medium/}{if-macro.tex}
\subimport{../../topics/macros/medium/}{apply-twice.tex}
\end{questions}

\end{document}
