\documentclass{exam}
\usepackage{../../commonheader}
\usepackage{setspace}

%%% CHANGE THESE %%%%%%%%%%%%%%%%%%%%%%%%%%%%%%%%%%%%%%%%%%%%%%%%%%%%%%%%%%%%%%
\discnumber{6}
\title{\textsc{Orders of Growth and Midterm Review}}
\date{July 14, 2021}
%%%%%%%%%%%%%%%%%%%%%%%%%%%%%%%%%%%%%%%%%%%%%%%%%%%%%%%%%%%%%%%%%%%%%%%%%%%%%%%

\begin{document}
\maketitle
\rule{\textwidth}{0.15em}
\fontsize{12}{15}\selectfont

%%% INCLUDE TOPICS  HERE %%%%%%%%%%%%%%%%%%%%%%%%%%%%%%%%%%%%%%%%%%%%%%%%%%%%%%%

\begin{guide}
    The following worksheet consists of two sections: Orders of Growth (efficiency) and Midterm Review.
    It is recommended to spend about half of the section on each, but this may be adjusted based on how your
    students feel about orders of growth. The objective is to get them feeling as prepared as possible for the midterm.
    \newline\newline
    Make sure to poll your students in the beginning of the section to see which topics they would like to cover for midterm review.
    This worksheet includes the following topics: (reverse) environment diagrams, higher order functions, recursion, tree recursion, lists, trees, and abstraction/dictionaries.
    Iterators and Generators will NOT be in scope for the midterm. You may also want to spend some time at the end of the section sharing midterm advice and answering any questions
    your students might have about the midterm.
    \textbf{Recommended Timeline}
    \begin{itemize}
        \item Orders of Growth Mini Lecture - 5 minutes
        \item Jumpstart - 5 minutes
        \item Belgian Waffle - 5 minutes
        \item Fast Exponentiation - 8 minutes
        \item Midterm Review (gauge what topics students want to brush up on and do those problems) - 25 minutes
        \item Share midterm tips / Q&A - 5 minutes
    \end{itemize}
\end{guide}

\section{Orders of Growth}
\subimport{../../topics/growth/text/}{oog_overview.tex}
\begin{questions}
\subimport{../../topics/growth/easy/}{jumpstart.tex}
\subimport{../../topics/growth/easy/}{belgian-waffle.tex}
\subimport{../../topics/growth/medium/}{fast-exponentiation.tex}
\end{questions}

\newpage
\section{Midterm Review - Environment Diagrams}
\begin{questions}
\subimport{../../topics/hof/medium/env-diagram/}{zebra.tex}
\subimport{../../topics/recursion/medium/env-diagram/}{foobar1.tex}
\subimport{../../topics/recursion/medium/env-diagram/}{recurses.tex}
\end{questions}

\newpage
\section{Midterm Review - Higher Order Functions}
\begin{questions}
\subimport{../../topics/hof/easy/}{make-interval.tex}
\subimport{../../topics/hof/medium/}{alternator.tex}
\subimport{../../topics/hof/hard/}{partial_summer.tex}
\end{questions}

\section{Midterm Review - Recursion}
\begin{questions}
\subimport{../../topics/recursion/medium/}{sum-prime-digits.tex}
\end{questions}

\section{Midterm Review - Tree Recursion}
\begin{questions}
\subimport{../../topics/recursion/medium/}{midterm.tex}
\subimport{../../topics/recursion/medium/}{add-up.tex}
\end{questions}

\newpage
\section{Midterm Review - Lists}
\begin{questions}
\subimport{../../topics/lists/immutable/medium/wwpd/}{assignment-and-slicing.tex}
\subimport{../../topics/lists/immutable/medium/}{duplicate-list.tex}
\end{questions}

\newpage
\section{Midterm Review - Trees}
\begin{questions}
\subimport{../../topics/trees/adt/medium/}{count.tex}
\subimport{../../topics/trees/adt/medium/}{even-square-tree.tex}
\subimport{../../topics/trees/adt/hard/}{scrabble.tex}
\end{questions}

\newpage
\section{Midterm Review - Abstraction}
\subimport{../../topics/data-abstraction/medium/}{bookshelf.tex} 
\end{document}
