\documentclass{exam}
\usepackage{../commonheader}

\discnumber{2}
\title{Higher-Order Functions, Self Reference, and Lambdas}
\date{June 29, 2021}

\begin{document}
\maketitle


\begin{blocksection}
\begin{guide}

This worksheet covers HOF's, self reference, and lambdas, which were covered in Discussion 2 on Tuesday. 
All problems are HOF problems, but they are broken down into the topics they emphasize.

\textbf{Estimated Problem Times}
\newline
You might not finish the worksheet, so prioritize problems that cover topics your students could use more practice with.
\begin{itemize}
    \item Higher Order Functions Mini-Lec - 5 mins
    \item Foobar - 6 mins
    \item Apple - 4 mins
    \item Mystery - 5 mins
    \item Mystery WWPD - 5 mins
    \item xyz - 8 mins
    \item self-reference - 10 mins
\end{itemize}
\end{guide}
\end{blocksection}


\section{Higher-Order Functions: Environment Diagrams}
Higher-order functions are functions that take in one or more functions as arguments, return a function, or do both. 
Since environment diagrams are a good tool to keep track of variables, values, and frames, we can use them to illustrate and 
understand the behavior of HOF's.
\begin{questions}
\subimport{../../topics/hof/easy/env-diagram/}{foobar.tex}
\subimport{../../topics/hof/medium/env-diagram/}{apple.tex}
\end{questions}

\newpage
\section{Higher-Order Functions: Lambda Expressions}
A lambda expression evaluates to a function without binding it to a name. Just like with a def statement, the body 
of a lambda expression is not evaluated until the function is called. Note that some of the lambda expressions in these problems 
are also higher-order functions.

\begin{questions}
\subimport{../../topics/hof/medium/}{mystery.tex}
\subimport{../../topics/hof/medium/wwpd/}{mystery.tex}

\begin{guide}
Note that the mystery function referred to in this problem is the same function we just wrote in the previous problem.
\end{guide}
\newpage
\subimport{../../topics/hof/medium/}{xyz.tex}
\end{questions}


\newpage
\section{Higher-Order Functions: Self Reference}
We use the term "self reference" to categorize a particular type of HOF in which the function returns itself, 
either directly or with the help of another function.

\begin{questions}
\subimport{../../topics/hof/hard/}{self-referencing.tex}
\end{questions}

\clearpage
\end{document}
