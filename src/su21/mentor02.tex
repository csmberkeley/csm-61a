\documentclass{exam}
\usepackage{../commonheader}

\discnumber{1}
\title{Python and Control}
\date{June 25, 2021}

\begin{document}
\maketitle


\begin{blocksection}
\begin{guide}

  This worksheet primarily covers environment diagrams and HOF's, both of which should have been covered in Wednesday/Thursday lectures.
There is also a challenge problem (i.e. exam prep) at the bottom for if you finish early or your students are looking for something more difficult to try.
This is a fairly long worksheet so don't feel pressured to do all of the problems! Make sure to poll your students to see which problems they are more interested in covering.

\textbf{Estimated Problem Times}
\newline
Again, you probably won't be able to do all of them- but you can choose a few based on how long they are.
\begin{itemize}
    \item Higher Order Functions Mini-Lec + Written Question - 2 mins
    \item Foobar - 6 mins
    \item Apple - 4 mins
    \item xyz - 8 mins
    \item self-referencing - 10 mins
\end{itemize}
\end{guide}
\end{blocksection}




\section{Higher-Order Functions}
\begin{questions}
\subimport{../../topics/hof/easy/}{why.tex}
\subimport{../../topics/hof/easy/env-diagram/}{foobar.tex}
\subimport{../../topics/hof/medium/env-diagram/}{apple.tex}

\newpage
\subimport{../../topics/hof/medium/}{xyz.tex}
\subimport{../../topics/hof/hard/}{self-referencing.tex}

\newpage
\subimport{../../topics/hof/medium/}{mystery.tex}
\subimport{../../topics/hof/medium/wwpd/}{mystery.tex}

\clearpage

\end{questions}

\end{document}
