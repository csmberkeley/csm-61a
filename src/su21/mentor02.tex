\documentclass{exam}
\usepackage{../commonheader}

\discnumber{2}
\title{Higher-Order Functions, Self Reference, and Lambdas}
\date{June 30, 2021}

\begin{document}
\maketitle


\begin{blocksection}
\begin{guide}

This worksheet covers HOF's, self reference, and lambdas, which were covered in Discussion 2 on Tuesday. 
All problems are HOF problems, but they are broken down into the topics they emphasize.

\textbf{Estimated Problem Times}
\newline
You might not finish the worksheet, so prioritize problems that cover topics your students could use more practice with.
\begin{itemize}
    \item Higher Order Functions Mini-Lec - 5 mins
    \item Foobar - 6 mins
    \item Apple - 4 mins
    \item Mystery - 5 mins
    \item Mystery WWPD - 5 mins
    \item xyz - 8 mins
    \item self-reference - 10 mins
\end{itemize}
\end{guide}
\end{blocksection}


\section{Higher-Order Functions: Environment Diagrams}
\begin{guide}
There are several characteristics of an HOF environment diagram that differ from environment diagrams students may 
be used to seeing. The first and most important is the use of functions as parameters and return values, since 
that is unique to HOF's. Make sure your students understand, for example, that a function defined in Global and then 
passed to f1 as a parameter is still the same function, and the parent will be Global even if the function is called in 
f1. The second thing students may not have seen before is functions whose parent frames aren't Global. This is not unique to HOF's, 
but it's likely a new concept, and can some practice to understand. Third, students may need a reminder about notation for 
lambda functions in environment diagrams, so remind them of the correct syntax.
\newline
\end{guide}
Higher-order functions are functions that take in one or more functions as arguments, return a function, or do both. 
Since environment diagrams are a good tool to keep track of variables, values, and frames, we can use them to illustrate and 
understand the behavior of HOF's.
\begin{questions}
\subimport{../../topics/hof/easy/env-diagram/}{foobar.tex}
\subimport{../../topics/hof/medium/env-diagram/}{apple.tex}
\end{questions}

\newpage
\section{Higher-Order Functions: Lambda Expressions}
\begin{guide}
It might be helpful to convert an HOF lambda to its corresponding def statements so students 
can understand the connection. Additionally, students often struggle with understanding how to call lambda functions on the same 
line as they were created, especially if there is a series of calls like in the third problem of this section. Consider 
demonstrating how such calls are made, possibly with the corresponding def statements for comparison.
\newline
\end{guide}
A lambda expression evaluates to a function without binding it to a name. Just like with a def statement, the body 
of a lambda expression is not evaluated until the function is called. Note that some of the lambda expressions in these problems 
are also higher-order functions.

\begin{questions}
\subimport{../../topics/hof/medium/}{mystery.tex}
\subimport{../../topics/hof/medium/wwpd/}{mystery.tex}

\begin{guide}
Note that the mystery function referred to in this problem is the same function we just wrote in the previous problem.
\end{guide}
\newpage
\subimport{../../topics/hof/medium/}{xyz.tex}
\end{questions}


\newpage
\section{Higher-Order Functions: Self Reference}
\begin{guide}
Self reference is when a function eventually returns itself. Students will have learned about recursion 
in lecture today, so it is important to clarify that the return value of a self referencing function is a function and not 
a function call. Self referencing functions will often (but not always) use a helper function, especially when the value 
of parameters in the outside function need to be changed with each call.
\newline
\end{guide}
We use the term "self reference" to categorize a particular type of HOF in which the function returns itself, 
either directly or with the help of another function.

\begin{questions}
\subimport{../../topics/hof/hard/}{self-referencing.tex}
\end{questions}

\clearpage
\end{document}
