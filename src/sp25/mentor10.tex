\documentclass{exam}
\usepackage{../commonheader}

\discnumber{10}
\title{\textsc{Introdution to Scheme}}
\date{March 31 -- April 4, 2025}

\begin{document}
\maketitle
\rule{\textwidth}{0.15em}

\begin{meta}
\begin{blocksection}
    \textbf{Recommended Timeline}
    \begin{itemize}
        \item Midterm \#2 Check-in: 5 mins
        \item Scheme Guided Minilecture: 30 mins
        \item If, and, or WWSD: 6 mins
        \item Define Eval WWSD: 7 mins
        \item Hailstone: 15 mins
    \end{itemize}
\end{blocksection}
\end{meta}
\begin{meta}
    This section is meant to act as your students' first introduction to Scheme, as they haven't even been exposed to it during lecture (we're just giving them a head start ;-)). You'll notice that the minilecture this time around is long. This is because we've added check-in WWSD questions to go along with explanations.
\textbf{Teaching Tips}
\begin{itemize}
    \item Please introduce the Scheme Built-In Procedure Reference and the CS 61A Scheme Specification pages located under the Resources dropdown in the navigation bar on the course website.
    \item To ease in Scheme, show how Scheme implements features of Python like variable assignments, conditional statements, logic operators, etc.
    \item Have students write a basic function in Python (like an iterative countdown), then replicate it in Scheme
    \item Make sure to give a disclaimer that while high level features may be analogous, the internals are different!
    \item Scheme features break into three broad categories: Primitives, Call Expressions, and Special Forms (the latter two are called Compound Expressions)
    \item Primitives evaluate to themselves (4 evaluates to 4, \#t to \#t, etc.)
    \item Call Expressions begin with a function name and are followed by arguments- evaluate function name, evaluate arguments, and apply function to arguments
    \item Special Forms begin with a keyword and are followed by subexpressions, which are evaluated in a way based on the specific keyword
    \item Don't feel like you have to cover everything as a reminder. Better to go slow than do the whole worksheet where your students at the end of the section learn nothing because the section was too rushed.
\end{itemize}
\end{meta}

\section{Scheme}
\subimport{../../topics/scheme/text/}{scheme-guided-overview.tex}
\begin{questions}
\subimport{../../topics/scheme/easy/wwsd/}{if-and-or.tex}
\subimport{../../topics/scheme/easy/wwsd/}{define-eval.tex}
\subimport{../../topics/scheme/easy/}{hailstone.tex}
\end{questions}

\end{document}