\documentclass{exam}
\usepackage{../commonheader}

%%% CHANGE THESE %%%%%%%%%%%%%%%%%%%%%%%%%%%%%%%%%%%%%%%%%%%%%%%%%%%%%%%%%%%%%%
\discnumber{4}
\title{Higher-Order Environments, Currying, and Introductory Recursion}
\date{February 10 --February 14, 2025}
%%%%%%%%%%%%%%%%%%%%%%%%%%%%%%%%%%%%%%%%%%%%%%%%%%%%%%%%%%%%%%%%%%%%%%%%%%%%%%%

\begin{document}
\maketitle
\rule{\textwidth}{0.15em}
\fontsize{12}{15}\selectfont

%%% INCLUDE TOPICS HERE %%%%%%%%%%%%%%%%%%%%%%%%%%%%%%%%%%%%%%%%%%%%%%%%%%%%%%%
\begin{meta}
\textbf{Recommended Timeline}
\begin{itemize}
    \item HOFs and Environment Diagrams mini-lecture/review - 5 mins
    \item Inception OR ABDE - 10 mins (check in with your students to see how they feel about the general structure of higher order functions and drawing environment diagrams; do this if they feel a bit shaky)
    \item General recursion mini-lecture - 10 mins (Since this week is a bit weird, this will likely be your students' first interaction with recursion! Take more time on this if needed.)
    \item Identify Fibonnacci - 7 mins
    \item Wrong factorial - 5 to 10 mins
    \item num\_digits - 5 mins
\end{itemize}
Please remember, there is no expectation that you get through all problems in a section. Pick the most pertinent problems for your section. 
Also note that  this week's worksheet is intentionally shorter to accommodate our students, as they are taking their first midterm this week.
Lastly,  page 5 of the student-facing worksheet has been left blank in case they need extra scratch paper to work on any of the problems.
\end{meta}

\begin{questions}
    \section{Higher-Order Functions in Environment Diagrams cont.}
    \subimport{../../topics/hof/medium/env-diagram/}{inception.tex}
    \subimport{../../topics/environments/medium/}{abde.tex}
    \begin{questionmeta}
        Inception and ABDE are very similar in terms of the skills they test. If your students are finding this concept challenging, focus on these problems during your section before moving on to the HOF challenge problems. It's important to work on either Compound or Partial Summer, as they both cover essential aspects of higher-order functions. Since these problems are more challenging, consider working on the rest of the worksheet first and then dedicate time to thoroughly working through one of these challenge problems.
        % Generally only choose either Compound or Partial Summer for your challenge problem, since they cover similar ground in terms of HOFs. These are on the harder side so also consider doing the rest of the worksheet and coming back.
    \end{questionmeta}

    \section{Recursion}
    \subimport{../../topics/recursion/text/}{recursion_overview_sp24.tex}
    \subimport{../../topics/recursion/easy/}{identify-fib.tex}
    \subimport{../../topics/recursion/easy/}{wrong_factorial.tex}
    \subimport{../../topics/recursion/easy/}{num-digits.tex}
\end{questions}

\end{document}
