\documentclass{exam}
\usepackage{../commonheader}
\usepackage{listings} 
\lstset{language=Scheme}
\discnumber{12}
\title{\textsc{Interpreters, Programs as Data \& Macros}}
\date{April 14 -- April 18, 2025}

\begin{document}
\maketitle
\rule{\textwidth}{0.15em}
\fontsize{12}{15}\selectfont


\begin{guide}
\textbf{Recommended Timeline}
\begin{itemize}
  \item Interpreters Intro (10-15 min): Consider walking through these problems as 
  you give a mini-lecture/review to leave more time for later problems.
  \item Eval Apply (10 min): You may want to do one in its entirety before leaving students to try the rest
  (and explain the rules in a step-by-step manner), since students often find these types of problems unintuitive or challenging.
  \item Macros Intro / mini-lecture (5 min)
  \item \lstinline{macros-quasi} (5 min): A WWSD-type problem.
  \item Macro Quasi [10 Mins]
  \item Meta Apply [5 Mins]
  \item NAND [5 Mins]
  \item Apply Twice [5 Mins]
  \item Censor [10 Mins]
\end{itemize}
\newpage
\end{guide}


\section{Interpreters}
\subimport{../../topics/interpreters/text/}{intro-text.tex}
\begin{questions}
\subimport{../../topics/interpreters/easy/}{intro.tex}
\subimport{../../topics/interpreters/medium/}{eval-apply.tex}
\end{questions}

\newpage
\section{Macros}
\subimport {../../topics/macros/text/}{intro.tex}
\begin{questions}
\subimport{../../topics/macros/easy/}{macros-quasi.tex}
 \subimport{../../topics/macros/easy/}{meta-apply.tex}
  \subimport{../../topics/macros/easy/}{nand.tex}
  \subimport{../../topics/macros/medium/}{apply-twice.tex}
  \subimport{../../topics/macros/medium/}{censor.tex}
\end{questions}

\end{document}
