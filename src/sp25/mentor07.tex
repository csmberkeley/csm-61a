\documentclass{exam}
\usepackage{../commonheader}

\discnumber{7}
\title{Function-Based Trees, Mutability, \& Iterators}
\date{March 3 -- March 7, 2025}

\begin{document}
\maketitle

\begin{meta}
    \textbf{Example Timeline}\\
    General: There is a lot of problems in this worksheet. The timings below are suggested times and those are a note for you in case you do choose to do that problem. And do not try to do all of them; better to go slow with a couple of problems than rush through all of them.
    \begin{enumerate}
        \item Tree ADT [30 Minutes] {\begin{itemize} 
        \item Mini-Lecture [7-8 Mins]
        \item Q1 - Tree Drawing - Difficulty: Easy [2-3 Mins]
        \item Q2 - Prune - Difficulty: Medium [7 Mins]
        \item Q3 - Replace x - Difficulty: Medium (Skeleton Code) [10 Mins]
        \item Q4 - Contains N - Difficulty Hard, Exam Level [13 Mins]
        \end{itemize}}
        \item Mutability [25 Minutes] {\begin{itemize} 
            \item Mini-Lecture [7-8 Mins]
            \item Q1 - WWPD - Difficulty: Medium: [10 Mins]
            \item Q2 - Accumulate - Difficulty: Medium [7 Mins]
            \item Q3 - Scrabble! - Difficulty: Medium+, Exam Level [10 Mins]
        \end{itemize}}
        \item Iterators [15 Minutes] {\begin{itemize} 
            \item Mini-Lecture [5 Mins]
            \item Q1 - WWPD - Difficulty: Medium [7-8 Mins]
        \end{itemize}}
    \end{enumerate}
\end{meta}

\section{Function-Based Trees}
\subimport{../../topics/trees/adt/text/}{tree_overview.tex}
\begin{questions}
    \subimport{../../topics/trees/adt/draw-tree/}{draw-tree.tex}
    \begin{questionmeta}
        Suggested Time: 2-3 min; Difficulty: Easy
        This question is just there for them to practice and solidify their understanding of the relationship between the code and the tree diagram. 
        If you decide to do this, have them do this on their own.
    \end{questionmeta}
    \newpage
    % BEGIN PRUNE %
    \begin{blocksection} 
        \textbf{BEFORE \lstinline{prune(1)}:}
        \subimport{../../topics/trees/adt/prune}{before.tex}
        \hspace{0.5in}
        \textbf{AFTER \lstinline{prune(1)}:}
        \subimport{../../topics/trees/adt/prune}{after.tex}
    \end{blocksection}
    \subimport{../../topics/trees/adt/prune}{problem.tex}
    \subimport{../../topics/trees/adt/medium/}{replace-x.tex} 
    \subimport{../../topics/trees/adt/hard/}{contains-n.tex}
\end{questions}

\newpage
    
\section{Mutability}
\subimport{../../topics/lists/mutable/text/}{mutation_overview.tex}
\begin{questions}
    \subimport{../../topics/lists/mutable/medium/env-diagram/}{spooky-list.tex}
    \subimport{../../topics/lists/mutable/medium/}{accumulate.tex}
    \newpage
    \subimport{../../topics/lists/mutable/hard/}{scrabbler.tex}
\end{questions}

\section{Iterators}
\subimport{../../topics/iterators/}{iterators_overview.tex}
\begin{questions}
    \subimport{../../topics/iterators/easy/}{wwpd.tex}
    \begin{questionmeta}
        Emphasize that calling iter on an iterator will return an iterator whose starting position is the current location the iterator which was passed in is at. Also, this new iterator is not influenced by the iterator which was passed in anymore. Calling next on the previous iterator does not move this iterator's position.
    \end{questionmeta}
\end{questions}

\end{document}