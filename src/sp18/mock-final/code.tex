\q{12}{Clever Pun No. 5}

Suppose Samo the dog needs to make his way across an n x n grid to get back to Professor DeNero. 
Samo is a very loyal dog and wants to reach the Professor in the fewest moves possibles, but at 
the same time, Samo is an opportunist, and notices treats scattered throughout the grid.
\newline
Suppose Samo starts at location (0, 0) on the grid G and Professor DeNero is at location (n, n) 
on the grid; that is, they are on opposite corners. Our input grid G tells us how many treats are 
at any location - for a location (x, y), the number of treats in that location can be found with G[x][y]. 
Given that Samo can move up, down, left, or right (no diagonals), and that Samo will eat all the treats 
in a location as he leaves it, fill in the function trail\textunderscore of\textunderscore treats to return 
the maximum amount of treats Samo can eat if he takes the minimum moves to get to Professor DeNero. 
(Samo will also eat all the treats at Professor DeNeros location when he reaches it.)



\vfill
\makebox{\hspace*{-1.8cm}\solutionimage{\lstinputlisting{clever_pun_no5.py}}{\lstinputlisting{clever_pun_no5_sol.py}}}
\vfill


% \newpage
%
% \q{6}{Binary Trees}
%
% \textbf{Definition}. A \emph{binary search tree} is a \lstinline$BTree$ instance
% for which the label of each node is larger than all labels in its left branch
% and smaller than all labels in its right branch.
%
% \begin{enumerate}[leftmargin=1em]
%
% \subq{4} Implement \lstinline$largest$, which takes a binary search tree
% \lstinline$t$ and a number \lstinline$x$. It returns the largest label in
% \lstinline$t$ that is smaller than \lstinline$x$. If no such label exists, it
% returns 0. \textbf{Assume that \lstinline$t$ contains only positive numbers as
% labels.}
%
% \makebox{\hspace*{-1.8cm}\solutionimage{\lstinputlisting{largest.py}}{\lstinputlisting{largest_sol.py}}}
% \makebox{\hspace*{-2.5cm}\includegraphics[width=1.35in]{btreeex.png}}
% \vfill
%
% \subq{2} Implement \lstinline$second$, which takes a binary search tree
% \lstinline$t$ \textbf{containing only positive numbers}, and a number
% \lstinline$x$. It returns the \textbf{second largest} label in \lstinline$t$
% that is smaller than \lstinline$x$.
%
% \makebox{\hspace*{-1.8cm}\solutionimage{\lstinputlisting{second.py}}{\lstinputlisting{second_sol.py}}}
%
% \end{enumerate}
