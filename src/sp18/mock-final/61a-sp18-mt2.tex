%%%%%%%%%%%%%%%%%%%%%%%%%%%%%%%%%%%%%%%%%%%%%%%%%%%%%%%%%%%%%%%%%%%%%%%%%%%%%
%  _______  _______  ___      ____   _______
% |       ||       ||   |    |    | |   _   |
% |       ||  _____||   |___  |   | |  |_|  |
% |       || |_____ |    _  | |   | |       |
% |      _||_____  ||   | | | |   | |       |
% |     |_  _____| ||   |_| | |   | |   _   |
% |_______||_______||_______| |___| |__| |__|
%
% TeX file for CS61A Exams
%%%%%%%%%%%%%%%%%%%%%%%%%%%%%%%%%%%%%%%%%%%%%%%%%%%%%%%%%%%%%%%%%%%%%%%%%%%%%%%

\documentclass[twoside]{article}
\usepackage{exam}

\pagestyle{myheadings}
\markboth{}{{\large Name: }\sp{4in}}

% For Python code
\lstset{%
  language=Python,
  basicstyle=\ttfamily,
  showstringspaces=false
  keywordstyle=\color{black},
  commentstyle=\color{black},
  stringstyle=\color{black},
}

\def\semester{Spring 2018}

%%% Showing solutions %%%%%%%%%%%%%%%%%%%%%%%%%%%%%%%%%%%%%%%%%%%%%%%%%%%%%%%%%

\def\showsolution{0}
\def\doshow{0}
\ifx\doshow\showsolution%
\newcommand{\solution}[1]{{\color{red}#1}}
\newcommand{\solutioncircle}[1]{{\color{red}#1}}
\newcommand{\solutionimage}[2]{#2}  % First arg is question, second is solution
\else%
%\newcommand\solution[1]{} % excludes
\newcommand{\solutioncircle}[1]{#1} % don't color text but still display it
\newcommand{\solutionimage}[2]{#1} % First arg is question, second is solution
\fi

%%% Document %%%%%%%%%%%%%%%%%%%%%%%%%%%%%%%%%%%%%%%%%%%%%%%%%%%%%%%%%%%%%%%%%%

\title{\sc Mock Exam \solution{}}

\begin{document}
\thispagestyle{empty}
\maketitle

\medskip

\textbf{INSTRUCTIONS}

\begin{itemize}
\item You have 3 hours to complete the exam.

\item The exam is closed book, closed notes, closed computer, closed calculator,
except three hand-written 8.5"~$\times$~11" crib sheet of your own creation and
the official CS 61A study guides.

\item Mark your answers \textbf{on the exam itself}. We will \emph{not} grade
answers written on scratch paper.
\end{itemize}

\medskip

\begin{center}
\begin{tabular}{|m{6cm}|m{8cm}|}
\hline
Last name & \\ [0.8cm]
\hline
First name & \\ [0.8cm]
\hline
Student ID number & \\ [0.8cm]
\hline
CalCentral email (\href{http://berkeley.edu}{\nolinkurl{\_@berkeley.edu}}) & \\ [0.8cm]
\hline
TA & \\ [0.8cm]
\hline
Name of the person to your left & \\ [0.8cm]
\hline
Name of the person to your right & \\ [0.8cm]
\hline
\emph{All the work on this exam is my own.} \textbf{(please sign)} & \\ [0.8cm]
\hline
\end{tabular}
\end{center}

\textbf{POLICIES \& CLARIFICATIONS}

\begin{itemize}

\item You may use built-in Python functions that do not require import, such as
\lstinline$min$, \lstinline$max$, \lstinline$pow$, \lstinline$len$, and
\lstinline$abs$.

\item You \textbf{may not} use example functions defined on your study guides
unless clearly specified by the question.

\item For fill-in-the blank coding problems, we will only grade work written in
the provided blanks. You may only write one Python statement per blank line, and
it must be indented to the level that the blank is indented. you may not need every 
blank, however.

\item Unless otherwise specified, you are allowed to reference functions
defined in previous parts of the same question.

\end{itemize}


\newpage
\begin{enumerate}
    \q{WWPD}

For each of the expressions in the table below, write the output displayed by
the interactive Python interpreter when the expression is evaluated. The output
may have multiple lines.
The interactive interpreter displays the repr string of the value of a successfully
evaluated expression, unless it is \lstinline$None$.
Write ``FUNC'' to indicate a functional value.

The first two rows have been provided as an example.

Assume that you have started \lstinline$python3$ and executed all the code to the
left of the table first.


\newcommand{\mlst}{\begin{tabular}{l}}\newcommand{\emlst}{\end{tabular}}
\hspace*{-0.5in}
\begin{minipage}[t]{0.45\textwidth}
\lstinputlisting{call.py}
\end{minipage} %
\begin{tabular}[t]{|m{5cm}|m{4.2cm}|}
\hline
\textbf{Expression} & \textbf{Interactive Output} \\
\hline
\lstinline$[2, 3]$ & [2, 3] \\
\hline
\lstinline$print((2, 3))$ & (2, 3) \\
\hline
\lstinline$a.a $   &
   \solution{\mlst\lstinline$FUNC$\\
   \lstinline$[FUNC, None]$\emlst}  \\ [\solutionimage{18ex}{17ex}] \hline
\lstinline$c.a $   &
     \solution{\mlst
                \lstinline$ FUNC$ \\
                \lstinline$ 17 $ \\
                \lstinline$ 1 $ \\
                \lstinline$ 1 $\emlst} \\ [\solutionimage{18ex}{17ex}] \hline
\lstinline$a.update() \\ b.update() \\ A.a$  & \solution{\lstinline$[0, -1, 1, -1] $} \\ [\solutionimage{18ex}{18ex}] \hline
\lstinline$A2.a$   & \solution{\lstinline$[2, [4, [9, 2]]]$} \\ [\solutionimage{18ex}{17ex}] \hline
\mlst\lstinline$z=4$ \\ \lstinline$mx(z)$ \\ \lstinline$print(z)$\emlst   & \solution{\lstinline$ 4$} \\ [\solutionimage{18ex}{17ex}] \hline
\lstinline$a.lst$ & \solution{\lstinline$[0, -1, 1, -1] $} \\ [\solutionimage{18ex}{18ex}] \hline
\lstinline$B(a).a.a$ & \solution{\lstinline$[0, -1, 1, -1] $} \\ [\solutionimage{18ex}{18ex}] \hline
\lstinline$B(A).a.update()$ & \solution{\lstinline$[0, -1, 1, -1] $} \\ [\solutionimage{18ex}{18ex}] \hline
\end{tabular}


    \clearpage
    \q{Environment Diagram}

Fill in the environment diagram that results from executing the
code below until the entire program is finished, an error occurs, or all frames
are filled.  \emph{You may not need to use all of the spaces or frames.}

A complete answer will:

\begin{itemize}
    \item Add all missing names and parent annotations to frames.
    \item Add all missing values created or referenced during execution.
    \item Show the return value for each local frame.
    \item Use box-and-pointer notation for list values. You do not need to write
    index numbers or the word ``list''.
\end{itemize}

\vspace{0.3in}


\hspace{-0.7in}\solutionimage{\input{env1}}{\includegraphics[scale=1]{env1-soln-image}}

    \clearpage
    \begin{blocksection}
\question Similarly to ``Accumulate'' with mutable lists, we want to find the accumulated sum of any iterable! Write an iterator class that takes in a list and returns the sum of the list thus far.

\begin{lstlisting}
>>> accu = Accumulator([1, 2, 3, 4, 5, 6])
>>> for a in accu:
...     print(a)
1
3
6
10
15
21
\end{lstlisting}

\begin{solution}[1.5in]
\begin{lstlisting}
class Accumulator:
    def __init__(self, lst):
        self.lst = lst
        self.index = 0
        self.sum = 0

    def __next__(self):
        if self.index >= len(self.lst):
            raise StopIteration()
        self.sum += self.lst[self.index]
        self.index += 1
        return self.sum

    def __iter__(self):
        return self
\end{lstlisting}
\end{solution}
\end{blocksection}

    \clearpage
\end{enumerate}

% \newpage
% \begin{center}
%   \textbf{No more questions.}
% \end{center}
% \newpage
%

\end{document}
