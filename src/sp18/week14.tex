\documentclass{exam}
\usepackage{../commonheader}
\lstset{language=Scheme}

%%% CHANGE THESE %%%%%%%%%%%%%%%%%%%%%%%%%%%%%%%%%%%%%%%%%%%%%%%%%%%%%%%%%%%%%%
\discnumber{10}
\title{\textsc{SQL and Final Review}}
\date{April 23 to April 25, 2018}
%%%%%%%%%%%%%%%%%%%%%%%%%%%%%%%%%%%%%%%%%%%%%%%%%%%%%%%%%%%%%%%%%%%%%%%%%%%%%%%

\begin{document}
\maketitle
\rule{\textwidth}{0.15em}
\fontsize{12}{15}\selectfont

\section{Creating Tables, Querying Data}
\subimport{../../topics/sql/mentors/}{table.tex}
\begin{questions}
\subimport{../../topics/sql/mentors/}{create-table.tex}
\newpage
\subimport{../../topics/sql/mentors/easy/}{food-if-color-green.tex}
\subimport{../../topics/sql/mentors/easy/}{food-and-color-if-language-not-python.tex}
\subimport{../../topics/sql/mentors/medium/}{pairs-same-language.tex}
\end{questions}

\newpage
\section{Aggregation}
\subimport{../../topics/sql/fish/text/}{intro.tex}
\subimport{../../topics/sql/fish/tables/}{fish.tex}

Hint: The aggregate functions \texttt{MAX}, \texttt{MIN}, \texttt{COUNT}, and \texttt{SUM} return the maximum, minimum, number, and sum of the values in a column. The  \texttt{GROUP BY} clause of a select statement is used to partition rows into groups.

\begin{questions}
\subimport{../../topics/sql/fish/easy/}{most-populated-species.tex}
\subimport{../../topics/sql/fish/easy/}{most-pieces-per-price.tex}
\subimport{../../topics/sql/fish/easy/}{number-fish.tex}

\newpage
\subimport{../../topics/sql/fish/text/}{competitor.tex}
\subimport{../../topics/sql/fish/tables/}{competitor.tex}
\subimport{../../topics/sql/fish/medium/}{competitor.tex}
\end{questions}

\newpage
{\huge \vspace*{0.5cm} \textsc{Final Review}}

\rule{\textwidth}{0.15em}

\section{Environment Diagrams}
\begin{questions}
\subimport{../../topics/lists/mutable/medium/env-diagram/}{one-two-three.tex}
\end{questions}

\newpage
\section{Recursive Data Structures}
\begin{questions}
\subimport{../../topics/trees/class/medium/}{double-tree.tex}
\newpage
\subimport{../../topics/linked-lists/class/medium/}{double-link.tex}
\subimport{../../topics/linked-lists/class/medium/}{shuffle.tex}
\end{questions}

\newpage
\section{Scheme}
\begin{questions}
\subimport{../../topics/scheme/medium/}{insert.tex}
\end{questions}

\newpage
\section{Iterators, Generators, and Streams}
\begin{questions}
\subimport{../../topics/iterators/easy/}{skipmachine.tex}
\newpage
\subimport{../../topics/streams/scheme/medium/}{puns.tex}
\end{questions}




%%%%%%%%%%%%%%%%%%%%%%%%%%%%%%%%%%%%%%%%%%%%%%%%%%%%%%%%%%%%%%%%%%%%%%%%%%%%%%%

\end{document}
