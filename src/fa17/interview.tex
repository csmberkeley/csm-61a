\documentclass{exam}
\usepackage{../commonheader}

%%% CHANGE THESE %%%%%%%%%%%%%%%%%%%%%%%%%%%%%%%%%%%%%%%%%%%%%%%%%%%%%%%%%%%%%%
\discnumber{10}
\title{\textsc{Spring 2018 Interview Questions}}
\date{November 27 to December 5, 2017}
%%%%%%%%%%%%%%%%%%%%%%%%%%%%%%%%%%%%%%%%%%%%%%%%%%%%%%%%%%%%%%%%%%%%%%%%%%%%%%%

\begin{document}
\maketitle
\rule{\textwidth}{0.15em}
\fontsize{12}{15}\selectfont

\begin{questions}

\begin{blocksection}
\question Identify what the Scheme interpreter will display, and give the number of calls made to \lstinline$scheme_eval$ and \lstinline$scheme_apply$.
\begin{lstlisting}

scm> (define (compute x) 
        (cond 
          ((= (- x 13) 40) x) 
          ((equal? x '(cons 7 4)) 7) 
          ((= (remainder x 2) 0) true)
        )
      )

scm> (compute 54)

\end{lstlisting}

\begin{solution}
\lstinline$compute$ \linebreak
\lstinline$true$ \linebreak
24 \lstinline$scheme_eval$, 6 \lstinline$scheme_apply$
\end{solution}

\end{blocksection}

\begin{blocksection}
\question Write a function \lstinline$all_subsets$, which iteratively finds all possible subsets of a list (including the trivial and null subsets).
\begin{lstlisting}

def all_subsets(lst):
    """
    >>> all_subsets([1,2,3])
    [[], [1], [2], [1, 2], [3], [1, 3], [2, 3], [1, 2, 3]]
    """

\end{lstlisting}

\begin{solution}
\begin{lstlisting}
def all_subsets(lst):
    results = [[]]
    while lst:
        results += [q + [lst[0]] for q in results]
        lst.pop(0)
    return results
\end{lstlisting}    
\end{solution}

\end{blocksection}



\end{questions}
\end{document}