\documentclass{exam}
\usepackage{../commonheader}
\lstset{language=Scheme}

%%% CHANGE THESE %%%%%%%%%%%%%%%%%%%%%%%%%%%%%%%%%%%%%%%%%%%%%%%%%%%%%%%%%%%%%%
\discnumber{9}
\title{\textsc{Generators and Streams}}
\date{November 13 to November 17, 2017}
%%%%%%%%%%%%%%%%%%%%%%%%%%%%%%%%%%%%%%%%%%%%%%%%%%%%%%%%%%%%%%%%%%%%%%%%%%%%%%%

\begin{document}
\maketitle
\rule{\textwidth}{0.15em}
\fontsize{12}{15}\selectfont



\section{Generators}
\begin{questions}
    \subimport{../../topics/generators/easy/}{foo.tex}
    \newpage
    \subimport{../../topics/generators/medium/}{hailstone.tex}
    \subimport{../../topics/generators/medium/}{tree-sequence.tex}
\end{questions}


\newpage
%%% Question %%%
\section{Streams}
\begin{questions}
\begin{blocksection}
\question What's the advantage of using a stream over a linked list?
\begin{solution}[0.5in] 
Lazy evaluation. We only evaluate up to what we need.
\end{solution}

\question What's the maximum size of a stream?
\begin{solution}[0.5in]
Infinity
\end{solution}

\question What's stored in first and rest? What are their types? 
\begin{solution}[0.5in]
First is a value, rest is another stream (either a method to calculate it, or an already calculated stream). In the case of Scheme, this is called a promise.
\end{solution}

\question When is the next element actually calculated?
\begin{solution}[.5in]
Only when it's requested (and hasn't already been calculated)
\end{solution}
\end{blocksection}


\section{What Would Scheme Print?}
\begin{blocksection}
\question For each of the following lines of code, write what Scheme would output.

\begin{lstlisting}
scm> (define x 1)
\end{lstlisting}
\begin{solution}[.45in]
\texttt{x}
\end{solution}

\begin{lstlisting}
scm> (if 2 3 4)
\end{lstlisting}
\begin{solution}[.45in]
\texttt{3}
\end{solution}

\begin{lstlisting}
scm> (define p (delay (+ x 1)))
\end{lstlisting}
\begin{solution}[.45in]
\begin{lstlisting}
p
\end{lstlisting}
\end{solution}

\begin{lstlisting}
scm> p
\end{lstlisting}
\begin{solution}[.45in]
\begin{lstlisting}
#[promise]
\end{lstlisting}
\end{solution}

\begin{lstlisting}
scm> (force p)
\end{lstlisting}
\begin{solution}[.45in]
\begin{lstlisting}
2
\end{lstlisting}
\end{solution}

\begin{lstlisting}
scm> (define (foo x) (+ x 10))
\end{lstlisting}
\begin{solution}[.45in]
\texttt{foo}
\end{solution}

\begin{lstlisting}
scm> (define bar (cons-stream (foo 1) (cons-stream (foo 2) bar)))
\end{lstlisting}
\begin{solution}[.45in]
\texttt{bar}
\end{solution}
\end{blocksection}

\begin{blocksection}
\begin{lstlisting}
scm> (car bar)
\end{lstlisting}
\begin{solution}[.45in]
\texttt{11}
\end{solution}

\begin{lstlisting}
scm> (cdr bar)
\end{lstlisting}
\begin{solution}[.45in]
\begin{lstlisting}
#[promise]
\end{lstlisting}
\end{solution}

\begin{lstlisting}
scm> (define (foo x) (+ x 1))
\end{lstlisting}
\begin{solution}[.35in]
\texttt{foo}
\end{solution}

\begin{lstlisting}
scm> (cdr-stream bar)
\end{lstlisting}
\begin{solution}[.35in]
\begin{lstlisting}
(3 . #[promise])
\end{lstlisting}
\end{solution}

\begin{lstlisting}
scm> (define (foo x) (+ x 5))
\end{lstlisting}
\begin{solution}[.35in]
\texttt{foo}
\end{solution}

\begin{lstlisting}
scm> (car bar)
\end{lstlisting}
\begin{solution}[.35in]
\texttt{11}
\end{solution}

\begin{lstlisting}
scm> (cdr-stream bar)
\end{lstlisting}
\begin{solution}[.35in]
\begin{lstlisting}
(3 . #[promise])
\end{lstlisting}
\end{solution}

\end{blocksection}

\section{Code Writing for Streams}

%%% Question %%%
\begin{blocksection}
\question Write out \texttt{double\_naturals}, which is a stream that evaluates to the sequence 1, 1, 2, 2, 3, 3, etc.
\begin{lstlisting}
(define (double_naturals)
    (double_naturals_helper 1 0)
)

(define (double_naturals_helper first go-next)










)

\end{lstlisting}

\begin{solution}
\begin{lstlisting}
(define (double_naturals_helper first go-next)
    (if (= 1 go-next)
        (cons-stream first (double_naturals_helper (+ 1 first) 0))
        (cons-stream first (double_naturals_helper first 1))
    )
)

;Alternative Solutions
(define (double_naturals_helper first go-next)
    (cons-stream first (double_naturals_helper (+ go-next first) (- 1 go-next)))
)
\end{lstlisting}
\end{solution}
\end{blocksection}

\begin{blocksection}
\question Write out \texttt{interleave}, which returns a stream that alternates between the values in stream1 and stream2. Assume that the streams are infinitely long.
\begin{lstlisting}
(define (interleave stream1 stream2)













)
\end{lstlisting}

\begin{solution}
\begin{lstlisting}
(define (interleave stream1 stream2)
(cons-stream (car stream1) 
 (interleave stream2 (cdr-stream stream1)))

;Alternative solution
(define (interleave stream1 stream2)
(cons-stream (car stream1) 
 (cons-stream (car stream2)
  (interleave (cdr-stream stream1)
  (cdr-stream stream2))))
)
\end{lstlisting}
\end{solution}
\end{blocksection}


\newpage

\begin{blocksection}
\section{Challenge Question}
\question \textbf{(Optional)} Write a generator that takes in a tree and yields each possible path from root to leaf, represented as a list of the values in that path. Use the object-oriented representation of trees in your solution.
\newline

\begin{lstlisting}
def all_paths(t):
    """
    >>> t = Tree(1, [Tree(2, [Tree(5)]), Tree(3, [Tree(4)])])
    >>> print(list(all_paths(t)))
        [[1, 2, 5], [1, 3, 4]]
    """    
\end{lstlisting}

\begin{solution}[0.5in]
\begin{lstlisting}
    if t.is_leaf():
        yield [t.label]
    for b in t.branches:
        for subpath in all_paths(b):
            yield [t.label] + subpath
\end{lstlisting}

\end{solution}
\end{blocksection}
\end{questions}




%%%%%%%%%%%%%%%%%%%%%%%%%%%%%%%%%%%%%%%%%%%%%%%%%%%%%%%%%%%%%%%%%%%%%%%%%%%%%%%

\end{document}