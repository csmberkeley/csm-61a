\begin{blocksection}
    \question Write a generator that, given m (the amount of money you have), pc (the cost of one pear), and ac (the cost of one apple)
    yields all possible combinations of fruit that you can buy that uses up the most of your money. In other words, each combination of fruits
    should not result in enough money left over to buy another fruit. Combinations of the same number of each fruit in different orders is okay.
    It is also okay if each combination has an extra space at the end.
    
    \begin{lstlisting}
    def fruit_options(m, pc, ac):
        """
        >>> print(list(fruit_options(10, 2, 5)))
        ['pear pear pear pear pear ', 'pear pear apple ', 'pear apple pear ',
         'apple pear pear ', 'apple apple ']
        """
        if __________________________________:
            yield ""
        if m >= pc:
            for ______________________________________:
                ________________________________
        if m >= ac:
            for ______________________________________:
                ____________________________________
                
    \end{lstlisting}
    
    \begin{solution}[1in]
    \begin{lstlisting}
        def fruit_options(m, pc, ac): 
            if m < pc and m < ac:
                yield ""
            if m >= pc:
                for p in fruit_options(m-pc, pc, ac):
                    yield "pear " + p;
            if m >= ac:
                for a in fruit_options(m-ac, pc, ac):
                    yield "apple " + a;    
    \end{lstlisting}
    \end{solution}
    \end{blocksection}
    
    \begin{guide}
    \begin{blocksection}
        \textbf{Teaching Tips}
        \begin{itemize}
        \item This problem shows that some generator problems can be approached in a similar way as recursion problems, except with yield instea of return.
        \item Another goal of this problem is to show how iterating through a generator can be done through a for ___ in ___ loop.
        \end{itemize}
    \end{blocksection}
    \end{guide}